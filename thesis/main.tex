\documentclass[a4paper, twoside]{thesis}

\usepackage{StreamingPriority}
\usepackage[utf8]{inputenc}
\usepackage{amsmath}
\usepackage{amsfonts}
\usepackage{amssymb}
\usepackage{graphicx}
\usepackage{blindtext}
\usepackage{subfiles}
\usepackage[italian]{babel}
\usepackage{algorithm2e, setspace}

\graphicspath{{images/}}

\pgfplotsset{compat=1.18}
\begin{document}
\title{
\huge{Caricamento di scene 3D complesse: strategie di prioritizzazione degli asset e loro valutazione quantitativa basata su immagini}
}
% Proposte titoli 
%Valutazione quantitativa basata su immagini di strategie di prioritizzazione 
% di asset per il caricamento di scene 3D complesse.
%Image-based quantitative assessment of loading prioritization for assets in complex 3D scenes.

% Strategie di prioritizzazione 
% di asset per il caricamento di scene 3D complesse,
% e loro valutazione quantitativa basata su immagini.


\author{Andrea Borghesi}
\matr{952944}
\logo{images/unimi_logo}
\university{Università degli Studi di Milano}
\dept{Dipartimento di Informatica}
\degr{Corso di Laurea in Informatica}
\superv{Prof. Marco Tarini}
\date{Anno Accademico 2021/2022}

\pagestyle{plain}
\maketitle
\thispagestyle{empty}
\cleardoublepage
\chapter*{Sommario}
    \thispagestyle{empty}
    % Nel contesto del rendering di scene 3D di grandi dimensioni i limiti imposti dalla memoria sono sempre un ostacolo da affrontare con cautela. Il passaggio dei dati dalla memoria di massa o da remoto fino alla VRAM è un problema recentemente affrontato a livello hardware. Questa tesi si pone l'obiettivo di proporre strategie per la gestione dell'ordine di caricamento delle istanze interne ad una scena 3D, approcciando il problema a livello software. Si pone particolare attenzione nel rendere l'operazione di caricamento il meno percettibile possibile dall'utente. Questa necessità nasce principalmente nel mondo videoludico, dove è uso comune strutturare un mondo virtuale in sotto-scene di cui, con gli standard odierni, si richiede che il passaggio tra queste sia impercettibile. Per fare ciò, le strategie tratteranno molto da vicino la gestione delle istanze che compongono la vista finale, con l'obiettivo di comporla in tempo utile comprensiva di asset critici al funzionamento dell'applicativo. La misura dell'efficacia di queste politiche viene stimata con uno strumento basato su immagini che approssima la distanza percettiva tra le due. Delle strategie implementate, le più efficaci secondo lo strumento valutativo progettato, sono quelle analizzano principalmente la sezione occupata nella vista da ogni istanza di asset e la provenienza della fonte luminosa per predire la presenza di ombre.
    
    Nel contesto del rendering di scene 3D complesse, le limitazioni della VRAM impongono che queste vengano caricate solo quando necessarie, anziché risiedere in memoria per l'intera durata dell'applicazione. Quando una nuova scena viene caricata, numerosi asset competono per la larghezza di banda nel trasferimento tra la memoria di massa e la VRAM. I ritardi percepiti dall'utente finale vengono ridotti visualizzando e interagendo con una scena parziale durante il caricamento. Questo risultato dipende fortemente dall'ordine in cui vengono caricati gli asset. In questo lavoro, (1) si dimostra l'impatto della scelta dell'ordine degli asset; (2) viene proposto un piccolo numero di strategie euristiche per determinare automaticamente un ordine degli asset, ottimizzato per una determinata scena e punto di inizio; (3) si presenta uno strumento agnostico basato su immagini per valutare l'efficacia di un determinato ordine di caricamento degli asset, che può essere utilizzato per classificare a posteriori ulteriori alternative; (4) si collauda lo strumento di valutazione prodotto sulle euristiche di ordinamento proposte su un piccolo set di scene di prova tratte da istanze di video giochi reali e non. Lo studio suggerisce che i criteri per guidare l'ordine del caricamento degli asset dovrebbero includere la previsione delle ombre portate e la dimensione della sezione dello schermo occupata da ciascuna istanza.
\chapter*{Abstract}
\thispagestyle{empty}
In typical 3D video games, limitations in VRAM dictate that large 3D scenes are loaded on demand, rather than reside in memory for the entire duration of the application. When a new scene is loaded, for example at the start of a new level, numerous assets compete for the bandwidth from storage to VRAM, and delays in the player experience are reduced by displaying and interacting with a partially loaded scene while the load is in progress. The result strongly depends on the order in which assets are loaded. In this work, we (1) demonstrate the drastic impact of the choice of the ordering of the assets; (2) propose a small number of heuristic strategies to automatically determine an ordering of assets, optimized for a given scene and spawning point; (3) present an agnostic, image-based tool to assess the efficacy of any given ordering of assets, that can be used to rank a-posteriori competing alternatives; (4) test our own tool on the ordering produced by our proposed heuristics, over a small set of test scenes taken from instances of real games. Our study suggests that the criteria to drive the order of asset loading should include the prediction of cast shadow and the size of the section of the screen occupied by each instance.

\pagenumbering{roman}

\newgeometry{bottom=3cm, top=2cm}
\tableofcontents
\restoregeometry

\pagestyle{headings}
    \chapter{Introduzione}
\pagenumbering{arabic}
    Nel contesto del rendering di scene di grandi dimensioni i limiti imposti dalla memoria sono sempre un ostacolo da affrontare con cautela; raramente è possibile caricare l'interità degli asset, si necessita dunque di caricare sottoinsiemi fondamentali. A fronte di questo problema interviene il ruolo del designer, che è in grado di strutturare il mondo in più scene di dimensioni gestibili e interconnesse tra di loro. Una grande città, per esempio, divisa in zone connesse da sottopassaggi, vicoli stretti o gallerie che fungono da spazio di transizione tra una scena e un'altra.
    
    Il problema che si pone è riferito alla transizione tra due di queste sotto-scene (da adesso in poi chiamate solo \textit{scene}). L'operazione di scaricare la scena precedente e caricare la successiva richiede un ammontare di tempo non indifferente. La strategia più utilizzata per diversi anni è stata la semplice transizione ad una schermata di attesa, ma gli standard odierni richiedono che queste vengano rimosse per garantire un maggiore senso di fluidità e di percezione dello spazio.
    
    In molte situazioni, specialmente quelle in contesto videoludico, l'utente finale potrebbe non avere una macchina sufficientemente performante per effettuare questa sequenza di scaricamento e caricamento senza che se ne accorga: il tempo impiegato per attraversare lo spazio di transizione non potrebbe bastare e il giocatore potrebbe vedere degli asset ``apparire dal nulla'', rovinando l'esperienza di immersione nell'ambiente.


    \section*{Obiettivi} % 1.2
    Ci si pone l'obiettivo di introdurre politiche di ordinamento degli asset che andranno rispettate durante il caricamento. Queste politiche prioritizzeranno il caricamento delle istanze di asset che compongono la vista finale della scena e le componenti critiche per il corretto funzionamento dell'applicativo, garantendo un migliore margine di tempo per il caricamento della scena anche per le macchine meno performanti senza rovinare l'esperienza percettiva dell'utente finale. 
    
    Per misurare l'efficacia di queste strategie si propone uno strumento basato su immagini che, sotto le dovute ipotesi, fornisce una valutazione della qualità dell'ordinamento utilizzato per il caricamento di una scena.
    \chapter{Lavoro correlato}
Le strategie di prioritizzazione del caricamento degli asset sono spesso proprietarie della casa produttrice e, di conseguenza, non divulgate al di fuori di esse. 
È dunque arduo trovare soluzioni pubbliche e l'argomento non è stato trattato nello specifico da studi scientifici.
Le soluzioni saranno implementate su Unity come richiesto dall'azienda ospitante, 
% nonostante il motore non divulghi le politiche di ordinamento applicate per il caricamento degli asset;
nonostante delle sue criticità che verranno esposte successivamente (vedi capitolo \ref{cap:limitazioni&lavorofuturo}).

Il problema di minimizzare il tempo di attesa fornendo all'utente una preview significativa dei dati 3D che stanno venendo caricati è stato affrontato intensamente nel contesto della semplificazione di modelli 3D \cite{schroeder1992decimation, semplificationenvelopes} e in particolare nelle mesh progressive \cite{hoppe1996progressive, hoppe199827} e nella loro compressione~\cite{progmeshstreaming1}.
La differenza nel contesto di questo elaborato è la granularità: in questo lavoro si sceglie l'ordine di caricamento di interi modelli, mentre nel caso della semplificazione progressiva di una mesh si sceglie quale lato collassare per ottenere una buona mesh approssimante al prossimo passo. Una mesh approssimante viene definita buona secondo dei criteri di distanza dalla mesh originale. Analogamente vedremo come arrivare in tempi utili ad una versione della scena che, per quanto incompleta, comprende tutti gli elementi che compongono la vista finale di una data inquadratura. Si sottolinea che queste politiche non intendono sostituire tecniche come quelle delle mesh progressive, ma vogliono invece porsi ad un livello di astrazione più alto, definendo quali mesh, eventualmente progressive, iniziare a caricare all'interno della scena. La bontà finale di ogni strategia proposta in questo elaborato viene giudicata da un valutatore esterno, basato su immagini. 

La valutazione della similitudine tra immagini è anch'esso un argomento molto trattato in letteratura \cite{corsini2013perceptual} specialmente nell'ambito dell'apprendimento delle macchine \cite{Wang_2014_CVPR, chechik2010large}. Il metodo proposto non sfrutterà queste ultime tecniche di apprendimento ma si limiterà a misurare la differenza percettiva analizzando utilizzando spazi colorimetrici percettivamente quasi omogenei \cite{ciecolorimetry, perceptualsimilarity}.
\\


    \chapter{Strategie di prioritizzazione}
\label{cap:strategie}
In questo capitolo verranno esposte delle politiche di prioritizzazione delle istanze che compongono una \textbf{scena}, uno spazio tridimensionale composto da oggetti chiamati \textbf{istanze di asset} (o anche solo istanze). Tutte le istanze sono caratterizzate da una trasformazione spaziale che li colloca all'interno dello spazio. Inoltre, a ognuna possono essere associate delle componenti che la definiscono come: generici script, audio, mesh ed eventuali tessiture. 
Di queste componenti quella che verrà maggiormente considerata nelle strategie esposte nei successivi paragrafi è la mesh. In particolare considereremo mesh 3D complete.

Si sottolinea che le strategie proposte ipotizzano che, soprattutto nel contesto videoludico, sia possibile interagire con la scena prima del caricamento dell'ultimo oggetto della scena. Dunque queste politiche avranno anche il compito di fornire un sottoinsieme di istanze critico per il funzionamento dell'applicativo (vedi Capitolo \ref{cap:strategie:zoneinteresse&assetprioritari}).

È inoltre opportuno delineare perché si sta dedicando tale attenzione a comporre ordinamenti e sottoinsiemi critici. Nella transizione tra due scene, gli asset che comporranno la scena che deve essere caricata risiederanno, nel caso migliore, nella memoria di massa o, per le applicazioni distribuite, in remoto. Prima di renderizzare delle istanze di questi asset è necessario trasferire questi file nella VRAM, uno spostamento di una quantità di dati non trascurabile. Una scena realistica utilizzata per delle valutazioni nei paragrafi successivi (Figura \ref{fig:scenarealistica}), necessita di renderizzare circa 4000 istanze che fanno uso di 900 mesh e 1365 tessiture. In tutto queste occupano \SI{2.2}{\giga\byte}. Il trasferimento di questi dati dalla memoria di massa o da remoto sino alla VRAM è rallentato non solo dalle limitate velocità di trasferimento della memoria di massa o dalla banda a disposizione, ma anche dalla traduzione dei file in un formato adatto al rendering. Questo può richiedere di effettuare ulteriori elaborazioni "on load" dei modelli, oltre che necessitare il passaggio dei dati nel bus tra la memoria centrale e la GPU, anch'esso soggetto ad altri limiti di banda. Queste elaborazioni "on load" includono: compressione e/o decompressione delle tessiture in un formato adatto alla GPU; creazione di livelli di MIP map; calcolo delle direzioni tangenti per vertici; e altro.

Questo sottolinea come il caricamento della scena raramente è un processo immediato. Identificare quali istanze compongono un sottoinsieme utilizzabile della scena è importante per rendere questo processo poco percettibile.
\\

Un progetto Unity contenente uno strumento che simula le politiche descritte è disponibile all'indirizzo \url{https://github.com/andreaborghesi00/asset-loading-priority}.

\section{Struttura di una strategia e ipotesi}
Le strategie proposte hanno la seguente struttura: 
\begin{itemize}[label={}]
\item \verb|IN:| Istanze componenti la scena, entry point
\item \verb|OUT:| Insieme di coppie $\langle$\verb|istanza, priorità|$\rangle$
\end{itemize}
L'input coincide con tutte le istanze presenti sulla scena e un \textit{entry point}, un'istanza priva di mesh che rappresenta i l punto di entrata della scena; l'output è un insieme di coppie che rappresenta il valore di importanza associato ad ogni istanza. La computazione dell'output è effettuata offline, in gergo \textit{baked}.

Ogni strategia avrà il compito principale di assegnare un valore di importanza ad ogni asset. Successivamente queste valutazioni potranno essere utilizzate a runtime, ma prima di utilizzarle dovranno essere riordinate in ordine crescente rispetto al valore di priorità. Nelle simulazioni effettuate è stato utilizzato Heapsort \cite{williams:64:algorithm} con complessità temporale pari a $O(n\log(n))$.
\\

\paragraph{Note implementative} Su Unity anche asset che non presentano una mesh hanno una posizione di conseguenza anche essi vengono considerati nella valutazione. Inoltre quelli che sono recipienti di routine critiche per l'applicativo necessitano un valore di priorità calcolato differentemente dal resto degli asset, vedi Capitolo  
\ref{cap:strategie:zoneinteresse&assetprioritari}.

% \textit{Specificare la motivazioni (cioè il rationale), cioè 
% cosa ci ha spinto a pensare che sia una strategia ragionevole.}
\section{Closest-First}
\label{cap:closest-first}
    \paragraph{Rationale}
        Gli oggetti inquadrati vicini alla \textit{camera} saranno quelli che occuperanno una sezione importante della vista; analogamente quelli più distanti avranno maggiori possibilità di venire occlusi.
        In aggiunta, nel contesto videoludico è intuitivo dare priorità a istanze di asset vicine al giocatore con la quale potrebbe interagire a breve, afferendo all'asset non solo un valore che rappresenta la distanza ma anche una stima della sua importanza semantica, per quanto grossolana.\\

    Per misurare la distanza tra due asset si utilizza la distanza euclidea tra le origini dei due asset nello spazio mondo, ovvero l'origine dello spazio locale della mesh dell'asset trasposto nello spazio mondo.
    Il valore di priorità coincide dunque con la distanza tra l'origine dell'istanza dell'asset dall'entry point considerato.

    Per quanto questa strategia sia semplice, presenta delle performance molto consistenti in presenza di fonti di luce che generano ombre portate. Si è osservato che questo avviene indipendentemente dall'angolo di provenienza della luce, probabilmente grazie al suo comportamento molto conservativo. 
    
    % specifica il concetto di conservativo utilizzato: utilizza pochissime informazioni dell'asset. es solo la posizione
    

    \begin{algorithm}
        \caption{Closest first}
        \setstretch{1.2}
        \label{alg:closest-first}
        \DontPrintSemicolon
        \SetAlgoLined

        \SetKwData{ev}{evaluated}
        \SetKwData{ep}{entryPoint}
        \SetKwData{a}{assets}
        \SetKwData{s}{sorted}
        
        \SetKwProg{Func}{ClosestFirst}{(\a, \ep)}{end}
            \Func{}{
                $\ev \gets $Distance$(\a, \ep)$\;
                \Return{$\ev$}\;
            }
    \end{algorithm}

    \begin{algorithm}
        \caption{Distance}
        \setstretch{1.2}
        \label{alg:distance}
        \DontPrintSemicolon
        \SetAlgoLined

        \SetKwData{aa}{assets}
        \SetKwData{a}{asset}
        \SetKwData{r}{result}
        \SetKwData{ep}{entryPoint}
        \SetKwData{d}{dist}
        \SetKwProg{Func}{Distance}{(\aa, \ep)}{end}
            \Func{}{
                $\r \gets \{\}$\;
                \ForAll{$\a$ \textup{in} $\aa$}{
                    $\d \gets \sqrt{(a.x - \ep.x)^2 + (\a.y - \ep.y)^2 + (\a.z - \ep.z)^2}$\;
                    $\r \gets \r \cup \langle \a, \d \rangle$\;
                }
                \Return{$\r$}\;
            }
    \end{algorithm}

% descrivi la func Distance

\section{Closest-First on Camera}
\label{cap:closest-first-on-camera}
    \paragraph {Rationale}
    Una scelta naturale è quella di prioritizzare ciò che è inquadrato dalla camera, ovvero tutti quegli asset che intersecano il \emph{view frustum} corrente.\\
    
    Prima di effettuare ogni valutazione si partiziona l'insieme universo delle istanze degli asset $U$ in due sottoinsiemi $A$ e $B$, dove $A$ è l'insieme di tutti gli asset che intersecano il view frustum e $B$ il suo complementare.
    
    La valutazione degli asset in $A$ viene calcolata con la distanza euclidea come visto nella politica descritta al paragrafo precedente. La valutazione degli asset in $B$ coincide invece con la somma tra la distanza euclidea e alla massima valutazione effettuata nell'insieme $A$. A livello implementativo, per considerare questa costante viene aggiunto un parametro \emph{offset} alla funzione di calcolo della distanza. Questa maggiorazione consente di riordinare la lista degli asset e ottenere una lista i cui primi $|A|$ elementi coincidono con gli asset dell'insieme $A$ e i successivi $|B|$ elementi con coincidono con gli asset dell'insieme $B$. 
    
    
    \begin{algorithm}[htb!]
        \caption{Closest first on Camera}
        \label{alg:closest-first-on-camera}
        \DontPrintSemicolon
        \SetAlgoLined
        \setstretch{1.2}
        \SetKwData{aa}{assets}
        \SetKwData{a}{asset}
        \SetKwData{ac}{onFrame}
        \SetKwData{ep}{entryPoint}
        \SetKwData{ec}{onCamera}
        \SetKwData{er}{offCamera}
        \SetKwData{of}{offset}

        
        \SetKwProg{Func}{ClosestFirstOnCamera}{(\aa, \ep)}{end}
        \Func{}{
            $A \gets \{\:\}$\;

            \ForAll(){$\a$ \textup{in} $\aa$}{
                % \If(){$\f.$Intersects(\a.bounds)}{
                \If(){\textup{asset intersects view frustum}}{
                    $A \gets A \cup \{ \a \}$\;
                }
            }
            $A \gets $Distance$(A, \ep)$\;
            $\of \gets \max(A$.value$)$\;
            $B \gets $DistanceOffset$((\aa \setminus A), \ep, \of)$\;
            \Return{$A \cup B$}\;
        }
    \end{algorithm}

        \begin{algorithm}
        \caption{DistanceOffset}
        \setstretch{1.2}
        \label{alg:distance-offset}
        \DontPrintSemicolon
        \SetAlgoLined

        \SetKwData{aa}{assets}
        \SetKwData{a}{asset}
        \SetKwData{r}{result}
        \SetKwData{ep}{entryPoint}
        \SetKwData{d}{dist}
        \SetKwData{of}{offset}
        \SetKwProg{Func}{DistanceOffset}{(\aa, \ep, \of)}{end}
            \Func{}{
                $\r \gets \{\}$\;
                \ForAll{$\a$ \textup{in} $\aa$}{
                    $\d \gets ||a\texttt{.position}-\ep\texttt{.position}||_2$\;
                    %\sqrt{(a.x - \ep.x)^2 + (\a.y - \ep.y)^2 + (\a.z - \ep.z)^2}$\;
                    $\r \gets \r \cup \langle \a, \d + \of \rangle$\;
                }
                \Return{$\r$}\;
            }
    \end{algorithm}
% rinomina Distance con 3 parametri 
\newpage


\section{Sphere Tracing}
Le prossime politiche sfrutteranno dei note strategie di rendering: \emph{Ray Tracing} e \emph{Ray Marching}. In verità queste non denotano dei precisi algoritmi, bensì delle classi di tecniche di rendering che condividono delle similarità fondamentali.

\subsection{Nozioni preliminari}
\subsubsection{Ray Tracing}
Il Ray Tracing è una tecnica di rendering che modella il comportamento della luce. Il concetto di \textit{vista} nel contesto sensoriale è modellato da \emph{raggi}, delle semirette che partono da un punto detto origine (per esempio dal centro di un pixel o dalla camera) che potenzialmente intersecano oggetti nello spazio.
Da un punto di vista algoritmico, si lancia un raggio passante per ogni pixel, con origine il centro della camera, con l'obiettivo di trovare l'oggetto intersecante più vicino. Successivamente si estraggono le informazioni della faccia della mesh colpita e altre informazioni legate all'intera mesh, necessarie per determinare il colore da assegnare al pixel (frammento) in questione. 
Si cerca il primo oggetto intersecante dato che, con l'ipotesi che gli oggetti considerati siano completamente opachi, l'oggetto più vicino occluderà la vista degli oggetti più distanti. È opportuno sottolineare che, dato che l'origine è fissata nel centro della camera, i raggi passanti per ogni pixel hanno una direzione diversa, questo implica che, se la distanza focale è finita, la proiezione è in prospettiva.

Si definisce formalmente un raggio in 3 dimensioni come la funzione
\begin{align*}
    r: \mathbb{R}_+ \to \mathbb{R}^3 \\
    r(t) = r_0 + tr_d
\end{align*}
dove $r_0$ è il punto di partenza, $r_d$ è la direzione, e l'argomento $t$ è la distanza che si vuole percorrere lungo il raggio. Se $r_0$ è il centro della camera e $p$ la generica posizione del centro di un pixel, allora la direzione del raggio per il pixel considerato è $r_d = \frac{r_0-p}{||r_0-p||}$.

Non resta che ottenere le intersezioni dei raggi con gli oggetti. Sia $f$ la funzione che descrive una superficie implicita \cite{bloomenthal1997introduction:implicitsurfaces}: si vuole risolvere analiticamente $F = f \circ r$ tale che $F(t) = 0$. Questa soluzione non è sempre ottenibile in tempi utili (vedi il toroide). Alcune implementazioni sfruttano il metodo di Newton, nel quale viene analizzato lo sviluppo di Taylor fino al secondo ordine della funzione per trovare iterativamente una migliore approssimazione delle radici. Questo metodo però non converge sempre e, nel caso considerato, potrebbe persino convergere in punti non utili (convergenza ad un punto diverso dal più vicino). Un ulteriore requisito di questo metodo è che la funzione sia derivabile, proprietà che non è posseduta da alcun poliedro.

\paragraph{Ombre portate}
I raggi sono soggetti alle leggi note dell'ottica geometrica. Si ipotizza di viaggiare nello stesso mezzo trasmissivo in ogni momento, e dunque, per la natura geometrica del raggio, esso rispetterà il principio di Fermat. Analogamente, la legge di Snell per la rifrazione non verrà applicata dato che si considerano solo oggetti completamente opachi. Resta da considerare la legge della riflessione: dato un raggio incidente su una superficie con angolo $\theta_{in}$, il raggio riflesso, passante per il punto di intersezione, avrà un angolo $\theta_{out} = -\theta_{in}$ rispetto alla superficie. % nemmeno questa viene totalmente rispettata, perché al primo impatto si cerca direttamente la fonte di luce

Il colore di un pixel non è sempre definito solo dall'oggetto che interseca ma quasi sempre anche dall'ambiente circostante. Nel mondo reale la luce è rifratta e rimbalza molteplici volte, alterando il colore e la luminanza proveniente da una superficie. È possibile simulare questi stimoli fisici, ma per limiti computazionali queste simulazioni a volte risultano grossolane approssimazioni della realtà. 

\subsubsection{Ray Marching}
Il Ray Marching è una classe di metodi di rendering per superfici implicite (o parametriche). L'idea alla base di queste strategie è quella di attraversare iterativamente un raggio, avvicinandosi ad ogni passo al punto di intersezione. 
Al posto di avvalersi delle derivate come nel metodo di Newton, si utilizzano delle funzioni dette \emph{Signed-Distance Function} (SDF) che approssimano la distanza dall'oggetto per assicurarsi di non superare la superficie. Queste funzioni descrivono delle superfici dette \emph{Distance Surfaces} \cite{bloomenthal1991convolution}.
Se la lunghezza del passo lungo il raggio è fissata, questo metodo generale risulterebbe troppo oneroso da eseguire per ogni raggio. È invece più opportuno, dato un punto sul raggio $\mathbf{x}$, scegliere una lunghezza che coincide con il valore della SDF calcolata in $\mathbf{x}$, supponendo che questa non sovrastimi mai la distanza dalla \textit{Distance Surface}. Questa tecnica è detta Sphere Tracing o Sphere Marching \cite{hart1996sphere}.
\paragraph{Signed-Distance Function (SDF)}Sia $f: \R^3 \to \R$ una funzione che descrive una superficie implicita (es. una sfera) e $A$ l'insieme dei punti interni e in superficie definito come
$$A=(\mathbf{x} : f(\mathbf{x}) \leq 0)$$
allora una \textit{Distance Surface} è una iso-superficie di una funzione $d: 2^{\R^3} \times \R^3 \to \R$ $$d(A, \mathbf{x}) = \min_{y\in A}||\mathbf{x} - \mathbf{y}||_2$$
dove $d$ è anche detta \emph{point-to-set distance} e definisce implicitamente $A$ dall'esterno.%@@@@@@@@@@@@@@@@@@
Se esiste una funzione $g: \mathbb{R}^3 \to \mathbb{R}$ tale che
$$g(\mathbf{x}) \leq d(f^{-1}(0), \mathbf{x})$$

allora $g$ è detta \emph{Signed-Distance Function}. Ottenere questa funzione non è sempre triviale, ma lo è per alcune superfici primitive come piani, sfere, ellissoidi, coni, tori e cubi.

Un generico algoritmo di Sphere Tracing, considerando un singolo raggio, è descritto dall'Algoritmo \ref{alg:sphere-tracing} dove $t$ è la distanza percorsa sul raggio e $d$ è la lunghezza del passo ad ogni iterazione.
L'algoritmo termina quando la distanza complessiva percorsa supera una costante arbitrariamente grande $D$ oppure quando la distanza dalla superficie è minore di una costante $\epsilon$ arbitrariamente piccola.

\begin{algorithm}
    \caption{Sphere Tracing}
    \label{alg:sphere-tracing}
    \setstretch{1.2}
    \DontPrintSemicolon
    \SetAlgoLined
    
    \While{$t < D$}{
        $d \gets f(r(t))$\;
        \If{$d < \epsilon$}{
            \Return{$t$}\;
        }
        $t \gets t + d$\;
    }
    \Return{$\emptyset$}\;
\end{algorithm}

A livello implementativo i parametri $D$ ed $\epsilon$ influiscono significativamente sulle performance dell'algoritmo. Una scelta troppo grande di $D$ potrebbe far effettuare lunghi passi a vuoto sprecando tempo, e una scelta troppo grande di $\epsilon$ potrebbe restituire una risposta troppo grossolana. È intuitivo pensare che, dato che queste politiche verranno eseguite offline, sia opportuno utilizzare un $\epsilon$ molto piccolo, un $D$ molto grande e lanciare un raggio per ogni pixel per ottenere la massima precisione. 
Questo approccio è opportuno in fase di produzione contrariamente, durante il testing, può essere più conveniente eseguire prove più grossolane ma veloci da computare.
 
\subsection{Valutazione della distanza}
\label{cap:sphere-distanza}
\paragraph{Rationale} 
Non tutti le istanze di asset intersecanti il \textit{view frustum} verranno renderizzate nella vista. Alcune di queste possono occluderne altre, anche solo parzialmente. Ci si pone l'obiettivo di prioritizzare quelle istanze che non sono completamente occluse, le metodologie descritte nel capitolo precedente permettono di identificarle.

Identificato il sottoinsieme di istanze che compone la vista, si procede a valutare i singoli elementi per distanza dall'entry point, come svolto nelle politiche descritte nei paragrafi \ref{cap:closest-first} e \ref{cap:closest-first-on-camera}.
\\

Dato l'insieme $A$ contenente tutte le istanze intersecanti il \textit{view frustum}, $D$ ed $E$ sono una partizione di $A$ contenente tutte le istanze che comporranno la vista e le istanze totalmente occluse rispettivamente.

Gli elementi di $D$ sono ottenuti applicando la tecnica di Sphere Tracing vista nel paragrafo precedente. Per semplicità e limiti implementativi, negli esperimenti effettuati si considerano le AABB (\textit{Axis Aligned Bounding Box}) di ogni asset e non con l'effettiva geometria complessa di ognuno per rilevare la loro presenza nella vista. Questo implica la presenza di falsi positivi nell'insieme $E$, nonostante ciò, valutando le performance di questa strategia, tale partizionamento si è rilevato una buona approssimazione.
% dovrei ripetere che le istanze in $D$ sono valutate per distanza? mi sembra di essere ripetitivo ma è anche la cruciale differenza tra questa politica e la prossima

\begin{algorithm}
    \caption{Sphere Tracing: Distance priority}
    \label{alg:sphere-tracing-distance-priority}
    \DontPrintSemicolon
    \SetAlgoLined
    \setstretch{1.2}

    \SetKwData{occ}{D}
    \SetKwData{oe}{E}
    \SetKwData{aa}{assets}
    \SetKwData{ep}{entryPoint}
    \SetKwData{oc}{offCamera}
    \SetKwData{o}{offset}
    \SetKwData{of}{A}
    \SetKwData{a}{asset}
    
    \SetKwProg{Func}{SphereTracingDistancePriority}{(\aa, \ep)}{end}
    \Func{}{

        $\occ \gets \{\}$\;
        $\of \gets \{\}$\;
        \ForAll(){$\a$ \textup{in} $\aa$}{
            \If(){\textup{asset intersects view frustum}}{
                $\of \gets \of \cup \{ \a \}$\;
            }
        }
        
        \ForAll(){\textup{p in pixels}}{
            hit $\gets$ SphereTracing($p$)\;
            \If{\textup{hit is not null}}{
                $\occ \gets \occ\; \cup \;$hit\;
            }
        }
        $\occ \gets $Distance$(\occ, \ep)$\;
        $\o \gets \max(\occ$.value$)$\;
        $\oe \gets \of \setminus \occ$\;
        $\oe \gets $DistanceOffset$(\oe, \ep, \o)$\;
        $\o \gets \max(\oe$.value$)$\;
        $\oc \gets \aa \setminus \of$\;
        $\oc \gets $DistanceOffset$(\oc, \ep, \o)$\;
        
        \Return{$\occ \; \cup \; \oe \; \cup \; \oc$}\;
    }
\end{algorithm}
\newpage
\subsection{Valutazione della dimensione}
\label{cap:sphere-dimensione} % quante parole per niente
\paragraph{Rationale} 
La politica descritta al paragrafo precedente assegna un valore di importanza agli asset che compongono la vista pari alla loro distanza dall'entry point. In questo contesto la distanza si sta ponendo come approssimazione molto grossolana della dimensione di un asset nella vista. Si propone una variazione nella quale si tiene traccia della \textit{quantità} di raggi colpiscono un oggetto. Lanciando un raggio per pixel, la bontà dell'approssimazione della dimensione è pari alla bontà della AABB rispetto alla mesh considerata.
\\
% Si vuole fornire una variante della politica precedente, nella quale si considera un caso particolare ma molto frequente: si hanno molteplici oggetti piccoli non occlusi vicini alla camera e un asset sfondo distante, occluso in piccola parte dagli oggetti vicini alla camera, si suppone inoltre che l'impatto percettivo creato dalla mancanza dello sfondo è maggiore di quello che verrebbe provocato dalla mancanza di tutti gli asset vicini. Un caso realistico corrisponderebbe al rendering di una stanza o di uno spazio aperto dove l'orizzonte è modellato da pochi grandi asset. 
% Se valutassimo gli asset per distanza, lo sfondo riceverebbe una priorità inferiore malgrado la sua presenza sia critica da un punto di vista percettivo, per porre rimedio si decide di valutare gli asset per dimensione, e per fornirne un'approssimazione si decide di utilizzare nuovamente lo Sphere Tracing e si considera un oggetto come "grande" quanto il numero di raggi che lo colpiscono.


Come già enunciato, questa è una variante della strategia precedente, con l'unica differenza nella funzione di valutazione utilizzata per l'insieme degli asset di $D$. Ogni volta che un raggio interseca un asset viene aggiunta una tupla $\langle \text{asset}, -1\rangle$ all'insieme degli occlusori oppure, se già esistente, viene aggiornata la tupla corrispondente all'asset decrementando il valore di importanza associato. È opportuno evidenziare che, come per la valutazione per la distanza, l'implementazione delle fornita approssima gli asset alle loro AABB, ma è comunque possibile fornire delle approssimazioni migliori persino esatte se disponibili con le rispettive SDF.

I restanti asset vengono come descritto nella politica precedente.

\begin{algorithm}
    \caption{Sphere Tracing: Size Priority}
    \label{alg:sphere-tracing-size-priority}
    \DontPrintSemicolon
    \SetAlgoLined
    \setstretch{1.2}

    \SetKwData{occ}{D}
    \SetKwData{oe}{E}
    \SetKwData{aa}{assets}
    \SetKwData{ep}{entryPoint}
    \SetKwData{oc}{offCamera}
    \SetKwData{o}{offset}
    \SetKwData{of}{A}
    \SetKwData{a}{asset}
    \SetKwData{h}{hit}
    \SetKwProg{Func}{SphereTracingSizePriority}{(\aa, \ep)}{end}
    \Func{}{

        $\occ \gets \{\}$\;
        $\of \gets \{\}$\;
        \ForAll(){$\a$ \textup{in} $\aa$}{
            \If(){\textup{asset intersects view frustum}}{
                $\of \gets \of \cup \{ \a \}$\;
            }
        }
        
        \ForAll(){\textup{p in pixels}}{
            $\h\gets$ SphereTracing($p$)\;
            \If{\textup{hit is not null}}{
                \If{$\exists n\in\mathbb{Z}_- : \langle \h, n\rangle \in \occ$}{
                    $\occ \gets (\occ\; \setminus \langle \h, n\rangle) \cup \langle \h, n-1\rangle $\;
                }\Else{
                    $\occ \gets \occ \;\cup\; \langle \h, -1\rangle$\;
                }
            }
        }
        $\oe \gets \of \setminus \occ$\;
        $\oe \gets $Distance$(\oe, \ep)$\;
        $\o \gets \max(\oe$.value$)$\;
        $\oc \gets \aa \setminus \of$\;
        $\oc \gets $DistanceOffset$(\occ, \ep, \o)$\;
        
        \Return{$\occ \; \cup \; \oe \; \cup \; \oc$}\;
    }
\end{algorithm}

\newpage
\section{Ray Tracing - Ombre portate}
\label{cap:ombreportate}
\paragraph{Rationale} Le istanze sono contenute in AABB, una struttura approssimante molto semplice di cui sono note le soluzioni analitiche per calcolare le intersezioni. È possibile dunque abbandonare le approssimazioni fornite dallo Sphere Tracing e utilizzare il Ray Tracing per trovare le intersezioni esatte.
Il risultato del rendering non è influenzato soltanto dalle istanze presenti nel view frustum, ma anche da quelle che ne influenzano l'illuminazione. Una fonte luminosa non inquadrata, per esempio, potrebbe generare delle ombre portate o dei riflessi che verranno poi rappresentati nella vista finale.
Questa strategia vuole concentrarsi sulle ombre portate, e di come queste possano causare causare una forte differenza percettiva, sebbene in maniera indiretta, anche da asset non inquadrati. Uno scenario tipico di questa occorrenza è quello delle stanze chiuse, che presentano soffitti e pareti; nonostante una parete non possa essere in vista, l'ombra che genera sugli asset in vista è di grande impatto percettivo. Si vuole dunque tenere conto degli asset generatori di ombre portate nella vista.
\paragraph{Ipotesi} Nell'implementazione fornita si assume, per semplicità, la presenza di una singola fonte luminosa nella scena, tuttavia, il metodo è estendibile a molteplici fonti.
\\

La struttura è analoga all'ultima strategia, nella quale però si sfrutta il Ray Tracing per ottenere delle soluzioni analitiche per le intersezioni.

Si lancia un raggio per ogni pixel e, se questo interseca un'istanza $x$ (la sua AABB), o viene aggiunta la coppia $\langle x, -1 \rangle$ all'insieme $D$ oppure viene aggiornata la coppia relativa all'istanza decrementando il contatore associato.
Successivamente, dato il punto $p$ di collisione con $x$, si lancia un secondo raggio dal punto $p$ in direzione della fonte luminosa. Se il secondo raggio interseca un'istanza $y$, allora questo sta proiettando un'ombra su $x$.
Si aggiunge la coppia $\langle y, -1 \rangle$ a $D$, o analogamente, se esiste già una coppia relativa all'istanza $y$, questa viene aggiornata, decrementando il valore associato di $\alpha$. Quest'ultimo parametro $\alpha$ consente di alterare il peso afferito alle istanze proiettori di ombre portate rispetto alle istanze in vista. Negli esperimenti effettuati si è concluso empiricamente che il valore portatore dei migliori risultati è $\alpha = 1$. Si osserva che l'utilizzo di $\alpha = 0$ comporta che le istanze non in vista che generano ombre portate, vengono comunque aggiunte in $D$ ma vengono considerate con priorità minima.

Si decide di inserire le istanze che proiettano ombra in $D$ per due principali motivi
\begin{itemize}
    \item l'effetto percettivo che provoca un'ombra è paragonabile a quello causato da un'istanza in vista
    \item  un'istanza potrebbe essere sia proiettore di un'ombra visibile che in vista. È dunque opportuno cumulare l'importanza di queste due caratteristiche 
\end{itemize}

% si può spostare
Per l'implementazione di questa politica si è fatto uso della libreria di Unity \emph{Physics} \cite{raycastingunitydoc}, che ha ridotto drasticamente i tempi di calcolo dato che queste operazioni avvengono sulla GPU.

\begin{algorithm}
    \caption{CastShadows}
    \label{alg:cast-shadows}
    \DontPrintSemicolon
    \SetAlgoLined
    \setstretch{1.2}

    \SetKwData{occ}{D}
    \SetKwData{oe}{E}
    \SetKwData{aa}{assets}
    \SetKwData{ep}{entryPoint}
    \SetKwData{oc}{offCamera}
    \SetKwData{o}{offset}
    \SetKwData{of}{A}
    \SetKwData{a}{asset}
    \SetKwData{h}{hit}
    \SetKwData{hs}{hitShadow}
    \SetKwData{ls}{lightSource}
    
    \SetKwProg{Func}{CastShadows}{($\aa, \ep, \alpha, \ls$)}{end}
    \Func{}{

        $\occ \gets \{\}$\;
        $\of \gets \{\}$\;
        \ForAll(){$\a$ \textup{in} $\aa$}{
            \If(){\textup{asset intersects view frustum}}{
                $\of \gets \of \cup \{ \a \}$\;
            }
        }
        
        \ForAll(){\textup{p in pixels}}{
            $\h\gets$ RayCast(\textup{origin: }$p$, \textup{direction: }$p - \ep$)\;
            \If{\textup{hit is not null}}{
                \If{$\exists n\in\mathbb{Z}_- : \langle \h, n\rangle \in \occ$}{
                    $\occ \gets (\occ\; \setminus \langle \h, n\rangle) \cup \langle \h, n-1\rangle $\;
                    
                }\Else{
                    $\occ \gets \occ \;\cup\; \langle \h, -1\rangle$\;
                }
                $\hs \gets$ RayCast(origin: $\h$.position, direction: $\ls-\h$.position\;   
                \If{$\exists m\in\mathbb{Z}_- : \langle \hs, m\rangle \in \occ$}{
                    $\occ \gets (\occ\; \setminus \langle \hs, m\rangle) \cup \langle \hs, m-\alpha\rangle $\;
                }\Else{
                    $\occ \gets \occ \;\cup\; \langle \hs, -\alpha\rangle$\;
                }
            }
        }
        $\oe \gets \of \setminus \occ$\;
        $\oe \gets $Distance$(\oe, \ep)$\;
        $\o \gets \max(\oe$.value$)$\;
        $\oc \gets \aa - \of$\;
        $\oc \gets $DistanceOffset$(\occ, \ep, \o)$\;
        
        \Return{$\occ \; \cup \; \oe \; \cup \; \oc$}\;
    }
\end{algorithm}
\newpage
\section{Asset prioritari}
\label{cap:strategie:zoneinteresse&assetprioritari}
La maggioranza delle strategie viste in questo capitolo si concentrano sulla gestione delle istanze nel view frustum, e associando un valore di importanza pari alla distanza dall'entry point per gli asset non in vista. Queste scelte sono dovute all'ignoranza di ulteriori informazioni se non lo stato della scena al momento dell'entrata. Tra le informazioni mancanti vi sono quelle semantiche: alcuni asset nella scena potrebbero avere un'importanza critica, che questa sia narrativa, logica o percettiva. Si può pensare all'impatto percettivo che potrebbe causare il caricamento tardivo di una fonte luminosa, in particolar modo se è l'unica in scena.

Si richiede quindi all'utente di fornire una \textbf{lista di eccezioni}: istanze di asset alla quale dovrà essere fornita priorità assoluta. Queste non verranno considerate dalle politiche di ordinamento e verranno associate direttamente con un valore di importanza massima.
L'ordine di caricamento delle eccezioni è dato dalla loro posizione nella lista, gestibile dall'utente. Le istanze assegnate ad un indice $i$ basso hanno una maggiore priorità. Nell'ordinamento finale, il valore assegnato agli asset d'eccezione sarà il minore possibile codificabile indicato con $\texttt{MAX}$ e, per rispettare l'ordinamento, verrà aggiunto il valore dell'indice di posizione.
La coppia rappresentante un'istanza d'eccezione $\texttt{instance}$ sarà dunque nella forma $\langle \texttt{instance}, -\texttt{MAX} + i\rangle$.

% Zone di interesse
% Un'ulteriore informazione semantica che possiamo richiedere all'utente sono dei \textbf{punti di interesse}.
% Ovvero punti che identificano il centro di una zona di interesse 

% Per trattare lo spostamento della camera si richiede all'utente di fornire punti nella scena che verranno denotati come \emph{di interesse} associati ad un \emph{coefficiente di interesse}. Gli asset che prima venivano valutati solo in funzione della distanza dall'entry point verranno adesso verranno valutati in aggiunta in funzione della distanza da ogni punto di interesse pesato con il relativo coefficiente.
% quest'ultima parte delle zone di interesse è ancora da implementare, ma in ogni caso non verrebbe utilizzata nella valutazione quantitativa delle politiche



        \chapter{Uno strumento per la valutazione quantitativa}
    \label{cap:toolvalutazione}
        %Semplicemente, dovresti dire che, per i tuoi scopi, ti è sufficiente misurare la distanza fra immagini come l'integrale (cioè la somma diviso l'area, cioè la media) 
    % della distanza fra i pixel, e magari anche come MASSIMO della distanza fra i pixel, anche se il calcolo della distanza fra immagini si può avvalere di metodi più sofisticati.
    % E aggiungere che, nel fare questo, come "distanza fra i pixel" usi uno spazio percettivo del colore, cioè converti i colori dalla rappresentazione RGB ad una rappresentazione
    % (Lab) che almeno tenta di correlare la distanza euclidea fra punti con la distanza percettiva.
    % Basta dire questo.        
        %intro: 
        Questo capitolo presenta uno strumento basato su immagini progettato per la valutazione quantitativa della performance di una qualsiasi data strategia di prioritizzazione. Il soggetto della valutazione è la sequenza di \textit{frame} di caricamento della scena che segue un dato ordinamento.
        
        Le performance di caricamento sono state limitate per mettere in evidenza il ruolo dell'ordinamento delle istanze, imponendo $d$ APF (Asset per Frame, quanti istanze di asset caricare per ogni frame) e $n$ FPS (\textit{Frame per Second}). La frequenza di frame analizzata è ottenuta tramite un piccolo modulo che acquisisce $n$ screenshot al secondo durante il caricamento della scena.

        \section{Metodo valutativo}
        La valutazione avviene prelevando l'ultimo frame, rappresentante la vista caricata, e designandolo come \emph{ground truth}, ovvero il frame che vogliamo raggiungere.
        Si suppone che caricare ulteriori asset dopo l'ultimo frame non cambierebbe la vista.
        Successivamente si prendono gli screenshot in ordine di tempo e si confrontano con la ground truth. 
        
        Si osserva che l'ultimo frame non è necessariamente quello in cui tutte le istanze sono caricate. È più corretto affermare invece che è il frame con tutte le istanze in vista caricate.
        \\
        
        Si definisce il concetto di frame: un frame è una matrice $P\in [{A}^3]^{h \times w}$ dove $A = [0, 255] \subset \mathbb{N}$ e $h\times w$ è la risoluzione dei frame considerati. Gli elementi di $P$ sono triple $P_{i,j} \in {A}^3$ che corrispondono con la combinazione sRGB che identifica il colore rappresentato dal pixel in posizione $i,j$. Sia $F = [{A}^3]^{h\times w}$ , il confronto tra due frame è una funzione $f:  F \times F \to \mathbb{R}^+$. Il confronto utilizzato è la media della distanza euclidea pixel a pixel con la ground truth:
        $$
        f(P_t, \texttt{PTruth}) = \frac{1}{wh}\sum_{i=1}^{h}\sum_{j=1}^{w}{||P_{t,i,j}, \texttt{PTruth}_{i,j}||_2}
        $$
        % dove la distanza $d$ è espressa come la differenza componente a componente di ogni pixel in valore assoluto:
        % % notazione con i delta?
        % $$
        % d(P1_{i,j}, P2_{i,j}) =  |P1_{i,j,0} - P2_{i,j,0}| + |P1_{i,j,1} - P2_{i,j,1}| + |P1_{i,j,2} - P2_{i,j,2}|
        % $$
        
        % Che, restando nello spazio colorimetrico sRGB, corrisponde alla somma delle differenze delle quantità di rosso, verde e blu necessario per comporre il colore finale.

        \paragraph{Agnosticità dello strumento}
        Si osserva come non vi sia alcuna dipendenza tra le politiche viste precedentemente e lo strumento proposto, rendendolo utilizzabile in molteplici contesti valutativi nei quali sia necessario comparare un'approssimativa distanza percettiva tra immagini.

        \section{Uniformità percettiva}
        Lo spazio colorimetrico sRGB porta con sé un problema che mina la bontà dell'approssimazione delle valutazioni: non è percettivamente uniforme. Dati due spostamenti $a$ e $b$ della stessa lunghezza (misurata con la distanza euclidea) partendo dallo stesso punto, non corrispondono sempre alla stessa distanza percettiva. Come si può notare nella Figura \ref{fig:chromaticitydiagram}\footnote{La rappresentazione mostrata del diagramma di cromaticità è solo un'approssimazione limitata dallo spazio colorimetrico in uso dello schermo o utilizzato nella stampa di questo elaborato}, colori (\textbf{hue}) differenti occupano aree di grandezze non omogenee. Il motivo di questa problematica risiede nelle lunghezze d'onda con la quale vengono percepiti i colori: la scala dei verdi viene percepiti da circa \SI{495}{\nano\meter} fino a \SI{560}{\nano\meter}, mentre quella dei i gialli da circa \SI{560}{\nano\meter} fino a \SI{580}{\nano\meter}.

        \begin{figure}
            \centering
            \includegraphics[scale=.4]{images/chromaticity-diagram.png}
            \caption{CIE 1931 diagramma di cromaticità \cite{chromaticitydiagram}}
            \label{fig:chromaticitydiagram}
        \end{figure}

        Lo spazio colorimetrico CIE L*a*b*, introdotto nel 1976 dalla Commissione Internazionale per l'Illuminazione \cite[63]{ciecolorimetry}, è stato concepito per essere percettivamente quasi uniforme nell'accezione appena esposta: piccole variazioni nello spazio corrispondono sempre a piccole variazioni percettive e viceversa. In tutte le valutazioni verrà utilizzato l'illuminante standard CIE D65 come punto bianco standard.

                       La differenza (o distanza) percettiva tra due pixel espressi nello spazio colorimetrico CIE L*a*b* è definita da 
        $$
            \Delta E = ((\Delta L^*)^2 + (\Delta a^*)^2 + (\Delta b^*)^2)^\frac{1}{2}
        $$

        \section{Lettura dell'output}
        L'output fornito da queste analisi è una sequenza di differenze percettive tra ogni frame in ordine cronologico dalla ground truth. Per una migliore visualizzazione si decide di esporre questi dati su un grafico a linea spezzata avente sulle ordinate la distanza percettiva e sulle ascisse il tempo, espresso dal numero di frame corrispondente. Un punto sul grafico corrisponderà alla differenza percettiva tra frame a tempo $t$ e la ground truth. Nell'output sarà presente anche l'area sottesa alla curva rappresentata

        $$
            \int_{t=1}^T{f(P_t, \texttt{PTruth}) \, dt} = \sum_{t=1}^T{f(P_t, \texttt{PTruth})}    
        $$

        L'obiettivo è quello di minimizzare il valore sotteso alla curva.

        \section{Ambiente di sviluppo e moduli}
        Lo strumento è stato sviluppato con Python 3.9.13, un ambiente completamente agnostico da quello di sviluppo delle politiche di priorità. Il modulo principale utilizzato è OpenCV, una vasta libreria per la manipolazione delle immagini e la conversione degli spazi colorimetrici. Sono stati inoltre utilizzati Numpy 1.21.2 e Pandas 1.3.3 per la gestione delle strutture dati e Matplotlib 3.4.3 per la produzione dei grafici.
        \\
        L'applicativo non presenta alcuna interfaccia grafica, opera completamente da linea di comando. Prima di invocarlo bisogna preparare l'input. Una directory deve contenere tutti i frame da analizzare, il nome di ognuno di essi deve essere uno scalare che rappresenta l'ordine del frame. Questa nomenclatura, benché rigida, si presta come la più semplice e scalabile. L'utilizzo della data di creazione dell'immagine avrebbe causato conflitti vista la velocità di acquisizione, minando la correttezza dell'ordinamento temporale.
        Da linea di comando accetta due argomenti: 
            \renewcommand\labelitemi{\tiny$\bullet$}
            \begin{itemize}
                \item \verb|-ld|: lista delle directory da fornire in input
                \item \verb|-o|: directory di output dei grafici
            \end{itemize}
        Attualmente lo strumento supporta in input solo file \verb|png| ma è facilmente estendibile ad altri formati. I grafici forniti in output sono in formato \verb|png| con risoluzione 1000x600.
        \\

        Una versione dello strumento è stata resa pubblicamente disponibile all'indirizzo \url{https://github.com/andreaborghesi00/perception-evaluator}.
    % Ma se spostassi tutti i grafici in un appendice apposita?

\chapter{Risultati ed esperimenti}
    % In questo capitolo, mostriamo il risultato dell'applicazione della strategia messa a punto a dataset reali o fittizi.

    % cosa si andrà a vedere nel capitolo
    Con l'utilizzo dello strumento proposto nel precedente capitolo è possibile ottenere delle valutazioni sulle politiche esposte nel Capitolo \ref{cap:strategie}. 

    \section{Testbed utilizzati}
    % testbed utilizzati (uno artificiale e uno realistico)
    % nella artificiale abbiamo i cubozzi che sono esattamente dentro le aabb mentre nella reale sono approssimazioni -> still pretty good per le ombre, funziona comunque molto bene (non abbiamo mai il caso: carica oggetto -> carica ombreggiatore dell'oggetto già presente perché le aabb sono un'approssimazione per eccesso e non per difetto)
    Le valutazioni avverranno su due scene, una fittizia che simula un caso ideale e una reale.
    
    La scena fittizia (Figura \ref{fig:scenafittizia}) presenta un'insieme di cubi con una fonte luminosa con la camera posizionata grossolanamente al centro. Il cubo, nel contesto valutativo considerato, è la mesh ideale poiché, in alcune strategie per la valutazione della priorità (vedi Capitoli \ref{cap:ombreportate}, \ref{cap:sphere-distanza}, \ref{cap:sphere-dimensione}) verranno considerate le AABB delle mesh che, per parallelepipedi rettangoli, coincide perfettamente con la mesh stessa.
    
    La scena contiene circa 2000 istanze, l'inquadratura renderizza circa $2,5 \cdot 10^4$ triangoli e non sono associate tessiture alle mesh renderizzate.
    Il posizionamento delle istanze dei cubi è parzialmente arbitraria; l'obiettivo principale è quello di emulare la presenza di molte istanze e di osservare, con una seconda camera, il corretto funzionamento della strategia in uso.
    È stato intenzionale il posizionamento di gruppi di cubi e della camera, così da avere la certezza che questa inquadrasse delle ombre portate. Infine, in distanza è stato aggiunto un parallelepipedo di grandi dimensioni, per assicurarsi il corretto funzionamento delle valutazioni per dimensione.

    \begin{figure}[htbp!]
        \centering
        \includegraphics[scale=.2]{images/scena_fittizia.png}
        \caption{Scena fittizia, fonte luminosa $30^\circ$}
        \label{fig:scenafittizia}

        \centering
        \includegraphics[scale=.2]{images/sRGB-ground-truth.png}
        \caption{Scena realistica, fonte luminosa $30^\circ$}
        \label{fig:scenarealistica}
    \end{figure}
    
    La scena reale (\ref{fig:scenarealistica}), di un gioco non annunciato - Copyright MixedBag srl -, è di un paese sul mare. Questa scena presenta diversi ostacoli nella quale le prime politiche esposte (Capitoli \ref{cap:closest-first} \ref{cap:closest-first-on-camera}, \ref{cap:sphere-distanza}) saranno più svantaggiate mentre quelle più raffinate incontreranno meno difficoltà. Si osserva che le politiche che fruiscono di raggi soffriranno la presenza di mesh convesse e cave le cui approssimazioni ad AABB saranno molto grezze, nonostante ciò tale approssimazione sarà comunque migliore di altre strategie.
    
    Per eseguire correttamente le valutazioni sulla scena realistica non dovranno essere presenti componenti in movimento come vegetazione soggetta al vento o acqua che scorre.
    La scena presenta al suo interno circa 4000 istanze di asset con una geometria complessa con tessiture associate. L'inquadratura utilizzata renderizza circa $2.4 \cdot 10^6$ triangoli, un notevole incremento di complessità dalla scena fittizia. La posizione della camera è centrale alla cittadina, alle sue spalle sono presenti ulteriori istanze di dimensione variabile.



    % disclaimer sulla simulazione in unity perché limitazioni con rimando al capitolo delle limitazioni e lavoro futuro

    \section{Valutazioni delle politiche}
    % valutazioni: ipotesi su fps, aps, ANGOLO DELLA luce, tempo di esecuzione, asset prioritari presi in considerazione (luce), quantità di luci nella scena, ignoriamo le zone di interesse
    Nelle valutazioni che seguiranno si terrà conto che: 
    \begin{itemize}
        \item si valuterà la differenza per ogni frame dal frame finale (\textit{ground truth})
        \item verranno caricati 16 istanze di asset per frame (16 APF) a 60 FPS
        \item la fonte luminosa è una fonte diretta e angolata a $30^\circ$ quando non specificato
        \item la camera e la fonte luminosa saranno sempre considerati asset prioritari
    \end{itemize}
    Le inquadrature di test usate nelle due scene sono mostrate nelle Figure \ref{fig:scenafittizia} e \ref{fig:scenarealistica}.
        \subsection{Closest-First}
            % closest first evidenziando l'uniformità con l'angolo della luce
            Come è possibile osservare in Figura \ref{fig:eval-cf}, questa politica mostra una tendenza sub-lineare nella scena fittizia mentre un comportamento molto peggiore per la scena realistica. Questo può essere motivato dalla presenza di più istanze vicine non in vista rispetto alla scena fittizia. 
            I salti presenti nella valutazione della scena realistica sono causati dal caricamento delle istanze di grandi dimensioni. Data la considerevole sezione di vista occupata, comportano un grande avvicinamento verso la ground-truth e un'importante amplificazione dell'area sotto la curva fino a quando non vengono caricate.

            Attraverso ulteriori esperimenti si è notato che il comportamento di questa politica non risente di notevoli cambiamenti alterando l'angolo di provenienza della luce.

            \begin{figure}[htbp!]
                \centering
                %% Creator: Matplotlib, PGF backend
%%
%% To include the figure in your LaTeX document, write
%%   \input{<filename>.pgf}
%%
%% Make sure the required packages are loaded in your preamble
%%   \usepackage{pgf}
%%
%% Figures using additional raster images can only be included by \input if
%% they are in the same directory as the main LaTeX file. For loading figures
%% from other directories you can use the `import` package
%%   \usepackage{import}
%%
%% and then include the figures with
%%   \import{<path to file>}{<filename>.pgf}
%%
%% Matplotlib used the following preamble
%%
\begingroup%
\makeatletter%
\begin{pgfpicture}%
\pgfpathrectangle{\pgfpointorigin}{\pgfqpoint{5.900000in}{3.400000in}}%
\pgfusepath{use as bounding box, clip}%
\begin{pgfscope}%
\pgfsetbuttcap%
\pgfsetmiterjoin%
\definecolor{currentfill}{rgb}{1.000000,1.000000,1.000000}%
\pgfsetfillcolor{currentfill}%
\pgfsetlinewidth{0.000000pt}%
\definecolor{currentstroke}{rgb}{1.000000,1.000000,1.000000}%
\pgfsetstrokecolor{currentstroke}%
\pgfsetdash{}{0pt}%
\pgfpathmoveto{\pgfqpoint{0.000000in}{0.000000in}}%
\pgfpathlineto{\pgfqpoint{5.900000in}{0.000000in}}%
\pgfpathlineto{\pgfqpoint{5.900000in}{3.400000in}}%
\pgfpathlineto{\pgfqpoint{0.000000in}{3.400000in}}%
\pgfpathclose%
\pgfusepath{fill}%
\end{pgfscope}%
\begin{pgfscope}%
\pgfsetbuttcap%
\pgfsetmiterjoin%
\definecolor{currentfill}{rgb}{1.000000,1.000000,1.000000}%
\pgfsetfillcolor{currentfill}%
\pgfsetlinewidth{0.000000pt}%
\definecolor{currentstroke}{rgb}{0.000000,0.000000,0.000000}%
\pgfsetstrokecolor{currentstroke}%
\pgfsetstrokeopacity{0.000000}%
\pgfsetdash{}{0pt}%
\pgfpathmoveto{\pgfqpoint{0.634568in}{0.565123in}}%
\pgfpathlineto{\pgfqpoint{5.750000in}{0.565123in}}%
\pgfpathlineto{\pgfqpoint{5.750000in}{3.250000in}}%
\pgfpathlineto{\pgfqpoint{0.634568in}{3.250000in}}%
\pgfpathclose%
\pgfusepath{fill}%
\end{pgfscope}%
\begin{pgfscope}%
\pgfsetbuttcap%
\pgfsetroundjoin%
\definecolor{currentfill}{rgb}{0.000000,0.000000,0.000000}%
\pgfsetfillcolor{currentfill}%
\pgfsetlinewidth{0.803000pt}%
\definecolor{currentstroke}{rgb}{0.000000,0.000000,0.000000}%
\pgfsetstrokecolor{currentstroke}%
\pgfsetdash{}{0pt}%
\pgfsys@defobject{currentmarker}{\pgfqpoint{0.000000in}{-0.048611in}}{\pgfqpoint{0.000000in}{0.000000in}}{%
\pgfpathmoveto{\pgfqpoint{0.000000in}{0.000000in}}%
\pgfpathlineto{\pgfqpoint{0.000000in}{-0.048611in}}%
\pgfusepath{stroke,fill}%
}%
\begin{pgfscope}%
\pgfsys@transformshift{0.867088in}{0.565123in}%
\pgfsys@useobject{currentmarker}{}%
\end{pgfscope}%
\end{pgfscope}%
\begin{pgfscope}%
\definecolor{textcolor}{rgb}{0.000000,0.000000,0.000000}%
\pgfsetstrokecolor{textcolor}%
\pgfsetfillcolor{textcolor}%
\pgftext[x=0.867088in,y=0.467901in,,top]{\color{textcolor}\rmfamily\fontsize{10.000000}{12.000000}\selectfont \(\displaystyle {0}\)}%
\end{pgfscope}%
\begin{pgfscope}%
\pgfsetbuttcap%
\pgfsetroundjoin%
\definecolor{currentfill}{rgb}{0.000000,0.000000,0.000000}%
\pgfsetfillcolor{currentfill}%
\pgfsetlinewidth{0.803000pt}%
\definecolor{currentstroke}{rgb}{0.000000,0.000000,0.000000}%
\pgfsetstrokecolor{currentstroke}%
\pgfsetdash{}{0pt}%
\pgfsys@defobject{currentmarker}{\pgfqpoint{0.000000in}{-0.048611in}}{\pgfqpoint{0.000000in}{0.000000in}}{%
\pgfpathmoveto{\pgfqpoint{0.000000in}{0.000000in}}%
\pgfpathlineto{\pgfqpoint{0.000000in}{-0.048611in}}%
\pgfusepath{stroke,fill}%
}%
\begin{pgfscope}%
\pgfsys@transformshift{1.852341in}{0.565123in}%
\pgfsys@useobject{currentmarker}{}%
\end{pgfscope}%
\end{pgfscope}%
\begin{pgfscope}%
\definecolor{textcolor}{rgb}{0.000000,0.000000,0.000000}%
\pgfsetstrokecolor{textcolor}%
\pgfsetfillcolor{textcolor}%
\pgftext[x=1.852341in,y=0.467901in,,top]{\color{textcolor}\rmfamily\fontsize{10.000000}{12.000000}\selectfont \(\displaystyle {50}\)}%
\end{pgfscope}%
\begin{pgfscope}%
\pgfsetbuttcap%
\pgfsetroundjoin%
\definecolor{currentfill}{rgb}{0.000000,0.000000,0.000000}%
\pgfsetfillcolor{currentfill}%
\pgfsetlinewidth{0.803000pt}%
\definecolor{currentstroke}{rgb}{0.000000,0.000000,0.000000}%
\pgfsetstrokecolor{currentstroke}%
\pgfsetdash{}{0pt}%
\pgfsys@defobject{currentmarker}{\pgfqpoint{0.000000in}{-0.048611in}}{\pgfqpoint{0.000000in}{0.000000in}}{%
\pgfpathmoveto{\pgfqpoint{0.000000in}{0.000000in}}%
\pgfpathlineto{\pgfqpoint{0.000000in}{-0.048611in}}%
\pgfusepath{stroke,fill}%
}%
\begin{pgfscope}%
\pgfsys@transformshift{2.837593in}{0.565123in}%
\pgfsys@useobject{currentmarker}{}%
\end{pgfscope}%
\end{pgfscope}%
\begin{pgfscope}%
\definecolor{textcolor}{rgb}{0.000000,0.000000,0.000000}%
\pgfsetstrokecolor{textcolor}%
\pgfsetfillcolor{textcolor}%
\pgftext[x=2.837593in,y=0.467901in,,top]{\color{textcolor}\rmfamily\fontsize{10.000000}{12.000000}\selectfont \(\displaystyle {100}\)}%
\end{pgfscope}%
\begin{pgfscope}%
\pgfsetbuttcap%
\pgfsetroundjoin%
\definecolor{currentfill}{rgb}{0.000000,0.000000,0.000000}%
\pgfsetfillcolor{currentfill}%
\pgfsetlinewidth{0.803000pt}%
\definecolor{currentstroke}{rgb}{0.000000,0.000000,0.000000}%
\pgfsetstrokecolor{currentstroke}%
\pgfsetdash{}{0pt}%
\pgfsys@defobject{currentmarker}{\pgfqpoint{0.000000in}{-0.048611in}}{\pgfqpoint{0.000000in}{0.000000in}}{%
\pgfpathmoveto{\pgfqpoint{0.000000in}{0.000000in}}%
\pgfpathlineto{\pgfqpoint{0.000000in}{-0.048611in}}%
\pgfusepath{stroke,fill}%
}%
\begin{pgfscope}%
\pgfsys@transformshift{3.822846in}{0.565123in}%
\pgfsys@useobject{currentmarker}{}%
\end{pgfscope}%
\end{pgfscope}%
\begin{pgfscope}%
\definecolor{textcolor}{rgb}{0.000000,0.000000,0.000000}%
\pgfsetstrokecolor{textcolor}%
\pgfsetfillcolor{textcolor}%
\pgftext[x=3.822846in,y=0.467901in,,top]{\color{textcolor}\rmfamily\fontsize{10.000000}{12.000000}\selectfont \(\displaystyle {150}\)}%
\end{pgfscope}%
\begin{pgfscope}%
\pgfsetbuttcap%
\pgfsetroundjoin%
\definecolor{currentfill}{rgb}{0.000000,0.000000,0.000000}%
\pgfsetfillcolor{currentfill}%
\pgfsetlinewidth{0.803000pt}%
\definecolor{currentstroke}{rgb}{0.000000,0.000000,0.000000}%
\pgfsetstrokecolor{currentstroke}%
\pgfsetdash{}{0pt}%
\pgfsys@defobject{currentmarker}{\pgfqpoint{0.000000in}{-0.048611in}}{\pgfqpoint{0.000000in}{0.000000in}}{%
\pgfpathmoveto{\pgfqpoint{0.000000in}{0.000000in}}%
\pgfpathlineto{\pgfqpoint{0.000000in}{-0.048611in}}%
\pgfusepath{stroke,fill}%
}%
\begin{pgfscope}%
\pgfsys@transformshift{4.808099in}{0.565123in}%
\pgfsys@useobject{currentmarker}{}%
\end{pgfscope}%
\end{pgfscope}%
\begin{pgfscope}%
\definecolor{textcolor}{rgb}{0.000000,0.000000,0.000000}%
\pgfsetstrokecolor{textcolor}%
\pgfsetfillcolor{textcolor}%
\pgftext[x=4.808099in,y=0.467901in,,top]{\color{textcolor}\rmfamily\fontsize{10.000000}{12.000000}\selectfont \(\displaystyle {200}\)}%
\end{pgfscope}%
\begin{pgfscope}%
\definecolor{textcolor}{rgb}{0.000000,0.000000,0.000000}%
\pgfsetstrokecolor{textcolor}%
\pgfsetfillcolor{textcolor}%
\pgftext[x=3.192284in,y=0.288889in,,top]{\color{textcolor}\rmfamily\fontsize{10.000000}{12.000000}\selectfont Tempo (Frame)}%
\end{pgfscope}%
\begin{pgfscope}%
\pgfsetbuttcap%
\pgfsetroundjoin%
\definecolor{currentfill}{rgb}{0.000000,0.000000,0.000000}%
\pgfsetfillcolor{currentfill}%
\pgfsetlinewidth{0.803000pt}%
\definecolor{currentstroke}{rgb}{0.000000,0.000000,0.000000}%
\pgfsetstrokecolor{currentstroke}%
\pgfsetdash{}{0pt}%
\pgfsys@defobject{currentmarker}{\pgfqpoint{-0.048611in}{0.000000in}}{\pgfqpoint{-0.000000in}{0.000000in}}{%
\pgfpathmoveto{\pgfqpoint{-0.000000in}{0.000000in}}%
\pgfpathlineto{\pgfqpoint{-0.048611in}{0.000000in}}%
\pgfusepath{stroke,fill}%
}%
\begin{pgfscope}%
\pgfsys@transformshift{0.634568in}{0.687163in}%
\pgfsys@useobject{currentmarker}{}%
\end{pgfscope}%
\end{pgfscope}%
\begin{pgfscope}%
\definecolor{textcolor}{rgb}{0.000000,0.000000,0.000000}%
\pgfsetstrokecolor{textcolor}%
\pgfsetfillcolor{textcolor}%
\pgftext[x=0.467902in, y=0.638938in, left, base]{\color{textcolor}\rmfamily\fontsize{10.000000}{12.000000}\selectfont \(\displaystyle {0}\)}%
\end{pgfscope}%
\begin{pgfscope}%
\pgfsetbuttcap%
\pgfsetroundjoin%
\definecolor{currentfill}{rgb}{0.000000,0.000000,0.000000}%
\pgfsetfillcolor{currentfill}%
\pgfsetlinewidth{0.803000pt}%
\definecolor{currentstroke}{rgb}{0.000000,0.000000,0.000000}%
\pgfsetstrokecolor{currentstroke}%
\pgfsetdash{}{0pt}%
\pgfsys@defobject{currentmarker}{\pgfqpoint{-0.048611in}{0.000000in}}{\pgfqpoint{-0.000000in}{0.000000in}}{%
\pgfpathmoveto{\pgfqpoint{-0.000000in}{0.000000in}}%
\pgfpathlineto{\pgfqpoint{-0.048611in}{0.000000in}}%
\pgfusepath{stroke,fill}%
}%
\begin{pgfscope}%
\pgfsys@transformshift{0.634568in}{1.158585in}%
\pgfsys@useobject{currentmarker}{}%
\end{pgfscope}%
\end{pgfscope}%
\begin{pgfscope}%
\definecolor{textcolor}{rgb}{0.000000,0.000000,0.000000}%
\pgfsetstrokecolor{textcolor}%
\pgfsetfillcolor{textcolor}%
\pgftext[x=0.398457in, y=1.110359in, left, base]{\color{textcolor}\rmfamily\fontsize{10.000000}{12.000000}\selectfont \(\displaystyle {20}\)}%
\end{pgfscope}%
\begin{pgfscope}%
\pgfsetbuttcap%
\pgfsetroundjoin%
\definecolor{currentfill}{rgb}{0.000000,0.000000,0.000000}%
\pgfsetfillcolor{currentfill}%
\pgfsetlinewidth{0.803000pt}%
\definecolor{currentstroke}{rgb}{0.000000,0.000000,0.000000}%
\pgfsetstrokecolor{currentstroke}%
\pgfsetdash{}{0pt}%
\pgfsys@defobject{currentmarker}{\pgfqpoint{-0.048611in}{0.000000in}}{\pgfqpoint{-0.000000in}{0.000000in}}{%
\pgfpathmoveto{\pgfqpoint{-0.000000in}{0.000000in}}%
\pgfpathlineto{\pgfqpoint{-0.048611in}{0.000000in}}%
\pgfusepath{stroke,fill}%
}%
\begin{pgfscope}%
\pgfsys@transformshift{0.634568in}{1.630006in}%
\pgfsys@useobject{currentmarker}{}%
\end{pgfscope}%
\end{pgfscope}%
\begin{pgfscope}%
\definecolor{textcolor}{rgb}{0.000000,0.000000,0.000000}%
\pgfsetstrokecolor{textcolor}%
\pgfsetfillcolor{textcolor}%
\pgftext[x=0.398457in, y=1.581781in, left, base]{\color{textcolor}\rmfamily\fontsize{10.000000}{12.000000}\selectfont \(\displaystyle {40}\)}%
\end{pgfscope}%
\begin{pgfscope}%
\pgfsetbuttcap%
\pgfsetroundjoin%
\definecolor{currentfill}{rgb}{0.000000,0.000000,0.000000}%
\pgfsetfillcolor{currentfill}%
\pgfsetlinewidth{0.803000pt}%
\definecolor{currentstroke}{rgb}{0.000000,0.000000,0.000000}%
\pgfsetstrokecolor{currentstroke}%
\pgfsetdash{}{0pt}%
\pgfsys@defobject{currentmarker}{\pgfqpoint{-0.048611in}{0.000000in}}{\pgfqpoint{-0.000000in}{0.000000in}}{%
\pgfpathmoveto{\pgfqpoint{-0.000000in}{0.000000in}}%
\pgfpathlineto{\pgfqpoint{-0.048611in}{0.000000in}}%
\pgfusepath{stroke,fill}%
}%
\begin{pgfscope}%
\pgfsys@transformshift{0.634568in}{2.101427in}%
\pgfsys@useobject{currentmarker}{}%
\end{pgfscope}%
\end{pgfscope}%
\begin{pgfscope}%
\definecolor{textcolor}{rgb}{0.000000,0.000000,0.000000}%
\pgfsetstrokecolor{textcolor}%
\pgfsetfillcolor{textcolor}%
\pgftext[x=0.398457in, y=2.053202in, left, base]{\color{textcolor}\rmfamily\fontsize{10.000000}{12.000000}\selectfont \(\displaystyle {60}\)}%
\end{pgfscope}%
\begin{pgfscope}%
\pgfsetbuttcap%
\pgfsetroundjoin%
\definecolor{currentfill}{rgb}{0.000000,0.000000,0.000000}%
\pgfsetfillcolor{currentfill}%
\pgfsetlinewidth{0.803000pt}%
\definecolor{currentstroke}{rgb}{0.000000,0.000000,0.000000}%
\pgfsetstrokecolor{currentstroke}%
\pgfsetdash{}{0pt}%
\pgfsys@defobject{currentmarker}{\pgfqpoint{-0.048611in}{0.000000in}}{\pgfqpoint{-0.000000in}{0.000000in}}{%
\pgfpathmoveto{\pgfqpoint{-0.000000in}{0.000000in}}%
\pgfpathlineto{\pgfqpoint{-0.048611in}{0.000000in}}%
\pgfusepath{stroke,fill}%
}%
\begin{pgfscope}%
\pgfsys@transformshift{0.634568in}{2.572849in}%
\pgfsys@useobject{currentmarker}{}%
\end{pgfscope}%
\end{pgfscope}%
\begin{pgfscope}%
\definecolor{textcolor}{rgb}{0.000000,0.000000,0.000000}%
\pgfsetstrokecolor{textcolor}%
\pgfsetfillcolor{textcolor}%
\pgftext[x=0.398457in, y=2.524624in, left, base]{\color{textcolor}\rmfamily\fontsize{10.000000}{12.000000}\selectfont \(\displaystyle {80}\)}%
\end{pgfscope}%
\begin{pgfscope}%
\pgfsetbuttcap%
\pgfsetroundjoin%
\definecolor{currentfill}{rgb}{0.000000,0.000000,0.000000}%
\pgfsetfillcolor{currentfill}%
\pgfsetlinewidth{0.803000pt}%
\definecolor{currentstroke}{rgb}{0.000000,0.000000,0.000000}%
\pgfsetstrokecolor{currentstroke}%
\pgfsetdash{}{0pt}%
\pgfsys@defobject{currentmarker}{\pgfqpoint{-0.048611in}{0.000000in}}{\pgfqpoint{-0.000000in}{0.000000in}}{%
\pgfpathmoveto{\pgfqpoint{-0.000000in}{0.000000in}}%
\pgfpathlineto{\pgfqpoint{-0.048611in}{0.000000in}}%
\pgfusepath{stroke,fill}%
}%
\begin{pgfscope}%
\pgfsys@transformshift{0.634568in}{3.044270in}%
\pgfsys@useobject{currentmarker}{}%
\end{pgfscope}%
\end{pgfscope}%
\begin{pgfscope}%
\definecolor{textcolor}{rgb}{0.000000,0.000000,0.000000}%
\pgfsetstrokecolor{textcolor}%
\pgfsetfillcolor{textcolor}%
\pgftext[x=0.329012in, y=2.996045in, left, base]{\color{textcolor}\rmfamily\fontsize{10.000000}{12.000000}\selectfont \(\displaystyle {100}\)}%
\end{pgfscope}%
\begin{pgfscope}%
\definecolor{textcolor}{rgb}{0.000000,0.000000,0.000000}%
\pgfsetstrokecolor{textcolor}%
\pgfsetfillcolor{textcolor}%
\pgftext[x=0.273457in,y=1.907562in,,bottom,rotate=90.000000]{\color{textcolor}\rmfamily\fontsize{10.000000}{12.000000}\selectfont Differenza percettiva}%
\end{pgfscope}%
\begin{pgfscope}%
\pgfpathrectangle{\pgfqpoint{0.634568in}{0.565123in}}{\pgfqpoint{5.115432in}{2.684877in}}%
\pgfusepath{clip}%
\pgfsetrectcap%
\pgfsetroundjoin%
\pgfsetlinewidth{2.007500pt}%
\definecolor{currentstroke}{rgb}{0.121569,0.466667,0.705882}%
\pgfsetstrokecolor{currentstroke}%
\pgfsetdash{}{0pt}%
\pgfpathmoveto{\pgfqpoint{0.867088in}{1.840957in}}%
\pgfpathlineto{\pgfqpoint{0.886793in}{1.808217in}}%
\pgfpathlineto{\pgfqpoint{0.906498in}{1.806631in}}%
\pgfpathlineto{\pgfqpoint{0.926203in}{1.781585in}}%
\pgfpathlineto{\pgfqpoint{0.945908in}{1.587120in}}%
\pgfpathlineto{\pgfqpoint{0.965613in}{1.551508in}}%
\pgfpathlineto{\pgfqpoint{1.005023in}{1.550429in}}%
\pgfpathlineto{\pgfqpoint{1.024729in}{1.550429in}}%
\pgfpathlineto{\pgfqpoint{1.044434in}{1.546386in}}%
\pgfpathlineto{\pgfqpoint{1.064139in}{1.458658in}}%
\pgfpathlineto{\pgfqpoint{1.083844in}{1.452088in}}%
\pgfpathlineto{\pgfqpoint{1.103549in}{1.434438in}}%
\pgfpathlineto{\pgfqpoint{1.123254in}{1.419754in}}%
\pgfpathlineto{\pgfqpoint{1.142959in}{1.419754in}}%
\pgfpathlineto{\pgfqpoint{1.162664in}{1.409240in}}%
\pgfpathlineto{\pgfqpoint{1.182369in}{1.396779in}}%
\pgfpathlineto{\pgfqpoint{1.202074in}{1.376857in}}%
\pgfpathlineto{\pgfqpoint{1.221779in}{1.365775in}}%
\pgfpathlineto{\pgfqpoint{1.241484in}{1.344893in}}%
\pgfpathlineto{\pgfqpoint{1.261189in}{1.333880in}}%
\pgfpathlineto{\pgfqpoint{1.280894in}{1.324235in}}%
\pgfpathlineto{\pgfqpoint{1.300599in}{1.324235in}}%
\pgfpathlineto{\pgfqpoint{1.320304in}{1.311404in}}%
\pgfpathlineto{\pgfqpoint{1.340009in}{1.309275in}}%
\pgfpathlineto{\pgfqpoint{1.359714in}{1.277013in}}%
\pgfpathlineto{\pgfqpoint{1.399124in}{1.277013in}}%
\pgfpathlineto{\pgfqpoint{1.418830in}{1.260530in}}%
\pgfpathlineto{\pgfqpoint{1.438535in}{1.241869in}}%
\pgfpathlineto{\pgfqpoint{1.458240in}{1.233909in}}%
\pgfpathlineto{\pgfqpoint{1.477945in}{1.207102in}}%
\pgfpathlineto{\pgfqpoint{1.497650in}{1.204991in}}%
\pgfpathlineto{\pgfqpoint{1.517355in}{1.175507in}}%
\pgfpathlineto{\pgfqpoint{1.537060in}{1.175004in}}%
\pgfpathlineto{\pgfqpoint{1.556765in}{1.172427in}}%
\pgfpathlineto{\pgfqpoint{1.576470in}{1.161420in}}%
\pgfpathlineto{\pgfqpoint{1.596175in}{1.156388in}}%
\pgfpathlineto{\pgfqpoint{1.615880in}{1.147536in}}%
\pgfpathlineto{\pgfqpoint{1.635585in}{1.147512in}}%
\pgfpathlineto{\pgfqpoint{1.655290in}{1.143213in}}%
\pgfpathlineto{\pgfqpoint{1.674995in}{1.140664in}}%
\pgfpathlineto{\pgfqpoint{1.714405in}{1.128011in}}%
\pgfpathlineto{\pgfqpoint{1.734110in}{1.112103in}}%
\pgfpathlineto{\pgfqpoint{1.753815in}{1.112072in}}%
\pgfpathlineto{\pgfqpoint{1.773520in}{1.087924in}}%
\pgfpathlineto{\pgfqpoint{1.812931in}{1.087102in}}%
\pgfpathlineto{\pgfqpoint{1.832636in}{1.082891in}}%
\pgfpathlineto{\pgfqpoint{1.872046in}{1.079372in}}%
\pgfpathlineto{\pgfqpoint{1.891751in}{1.076993in}}%
\pgfpathlineto{\pgfqpoint{1.931161in}{1.069018in}}%
\pgfpathlineto{\pgfqpoint{1.950866in}{1.060866in}}%
\pgfpathlineto{\pgfqpoint{1.970571in}{1.035696in}}%
\pgfpathlineto{\pgfqpoint{1.990276in}{1.035696in}}%
\pgfpathlineto{\pgfqpoint{2.009981in}{1.031814in}}%
\pgfpathlineto{\pgfqpoint{2.029686in}{1.024885in}}%
\pgfpathlineto{\pgfqpoint{2.049391in}{1.024885in}}%
\pgfpathlineto{\pgfqpoint{2.069096in}{1.020206in}}%
\pgfpathlineto{\pgfqpoint{2.088801in}{1.013045in}}%
\pgfpathlineto{\pgfqpoint{2.108506in}{1.009771in}}%
\pgfpathlineto{\pgfqpoint{2.128211in}{0.998412in}}%
\pgfpathlineto{\pgfqpoint{2.167622in}{0.995507in}}%
\pgfpathlineto{\pgfqpoint{2.187327in}{0.986860in}}%
\pgfpathlineto{\pgfqpoint{2.207032in}{0.982384in}}%
\pgfpathlineto{\pgfqpoint{2.226737in}{0.975968in}}%
\pgfpathlineto{\pgfqpoint{2.246442in}{0.974044in}}%
\pgfpathlineto{\pgfqpoint{2.266147in}{0.970542in}}%
\pgfpathlineto{\pgfqpoint{2.325262in}{0.966622in}}%
\pgfpathlineto{\pgfqpoint{2.404082in}{0.965822in}}%
\pgfpathlineto{\pgfqpoint{2.423787in}{0.959922in}}%
\pgfpathlineto{\pgfqpoint{2.443492in}{0.942795in}}%
\pgfpathlineto{\pgfqpoint{2.482902in}{0.939935in}}%
\pgfpathlineto{\pgfqpoint{2.502607in}{0.929506in}}%
\pgfpathlineto{\pgfqpoint{2.542018in}{0.926942in}}%
\pgfpathlineto{\pgfqpoint{2.561723in}{0.924696in}}%
\pgfpathlineto{\pgfqpoint{2.581428in}{0.919876in}}%
\pgfpathlineto{\pgfqpoint{2.620838in}{0.919876in}}%
\pgfpathlineto{\pgfqpoint{2.640543in}{0.915534in}}%
\pgfpathlineto{\pgfqpoint{2.660248in}{0.909175in}}%
\pgfpathlineto{\pgfqpoint{2.679953in}{0.900213in}}%
\pgfpathlineto{\pgfqpoint{2.699658in}{0.898505in}}%
\pgfpathlineto{\pgfqpoint{2.739068in}{0.892467in}}%
\pgfpathlineto{\pgfqpoint{2.758773in}{0.886184in}}%
\pgfpathlineto{\pgfqpoint{2.778478in}{0.886184in}}%
\pgfpathlineto{\pgfqpoint{2.798183in}{0.876869in}}%
\pgfpathlineto{\pgfqpoint{2.817888in}{0.873720in}}%
\pgfpathlineto{\pgfqpoint{2.837593in}{0.871917in}}%
\pgfpathlineto{\pgfqpoint{2.857298in}{0.867345in}}%
\pgfpathlineto{\pgfqpoint{2.877003in}{0.864141in}}%
\pgfpathlineto{\pgfqpoint{2.896708in}{0.863304in}}%
\pgfpathlineto{\pgfqpoint{2.936119in}{0.858989in}}%
\pgfpathlineto{\pgfqpoint{2.975529in}{0.852320in}}%
\pgfpathlineto{\pgfqpoint{2.995234in}{0.848946in}}%
\pgfpathlineto{\pgfqpoint{3.014939in}{0.847533in}}%
\pgfpathlineto{\pgfqpoint{3.074054in}{0.846499in}}%
\pgfpathlineto{\pgfqpoint{3.093759in}{0.842771in}}%
\pgfpathlineto{\pgfqpoint{3.172579in}{0.837620in}}%
\pgfpathlineto{\pgfqpoint{3.211989in}{0.835296in}}%
\pgfpathlineto{\pgfqpoint{3.231694in}{0.835296in}}%
\pgfpathlineto{\pgfqpoint{3.251399in}{0.829614in}}%
\pgfpathlineto{\pgfqpoint{3.310515in}{0.825788in}}%
\pgfpathlineto{\pgfqpoint{3.330220in}{0.818564in}}%
\pgfpathlineto{\pgfqpoint{3.349925in}{0.814457in}}%
\pgfpathlineto{\pgfqpoint{3.389335in}{0.810903in}}%
\pgfpathlineto{\pgfqpoint{3.428745in}{0.807368in}}%
\pgfpathlineto{\pgfqpoint{3.448450in}{0.803240in}}%
\pgfpathlineto{\pgfqpoint{3.468155in}{0.728775in}}%
\pgfpathlineto{\pgfqpoint{3.507565in}{0.727968in}}%
\pgfpathlineto{\pgfqpoint{3.546975in}{0.722345in}}%
\pgfpathlineto{\pgfqpoint{3.586385in}{0.721061in}}%
\pgfpathlineto{\pgfqpoint{3.606090in}{0.718756in}}%
\pgfpathlineto{\pgfqpoint{3.645500in}{0.717681in}}%
\pgfpathlineto{\pgfqpoint{3.684911in}{0.714131in}}%
\pgfpathlineto{\pgfqpoint{3.724321in}{0.714130in}}%
\pgfpathlineto{\pgfqpoint{3.744026in}{0.710342in}}%
\pgfpathlineto{\pgfqpoint{3.803141in}{0.706783in}}%
\pgfpathlineto{\pgfqpoint{3.822846in}{0.702267in}}%
\pgfpathlineto{\pgfqpoint{3.862256in}{0.696946in}}%
\pgfpathlineto{\pgfqpoint{3.901666in}{0.696793in}}%
\pgfpathlineto{\pgfqpoint{3.921371in}{0.694780in}}%
\pgfpathlineto{\pgfqpoint{3.941076in}{0.691297in}}%
\pgfpathlineto{\pgfqpoint{3.960781in}{0.690326in}}%
\pgfpathlineto{\pgfqpoint{3.980486in}{0.687163in}}%
\pgfpathlineto{\pgfqpoint{5.497775in}{0.687163in}}%
\pgfpathlineto{\pgfqpoint{5.497775in}{0.687163in}}%
\pgfusepath{stroke}%
\end{pgfscope}%
\begin{pgfscope}%
\pgfpathrectangle{\pgfqpoint{0.634568in}{0.565123in}}{\pgfqpoint{5.115432in}{2.684877in}}%
\pgfusepath{clip}%
\pgfsetrectcap%
\pgfsetroundjoin%
\pgfsetlinewidth{2.007500pt}%
\definecolor{currentstroke}{rgb}{1.000000,0.498039,0.054902}%
\pgfsetstrokecolor{currentstroke}%
\pgfsetdash{}{0pt}%
\pgfpathmoveto{\pgfqpoint{0.886793in}{3.127953in}}%
\pgfpathlineto{\pgfqpoint{1.340009in}{3.127960in}}%
\pgfpathlineto{\pgfqpoint{1.359714in}{3.121231in}}%
\pgfpathlineto{\pgfqpoint{1.379419in}{3.120996in}}%
\pgfpathlineto{\pgfqpoint{1.399124in}{3.096972in}}%
\pgfpathlineto{\pgfqpoint{2.758773in}{3.096960in}}%
\pgfpathlineto{\pgfqpoint{2.778478in}{3.093125in}}%
\pgfpathlineto{\pgfqpoint{2.798183in}{3.093125in}}%
\pgfpathlineto{\pgfqpoint{2.817888in}{3.089724in}}%
\pgfpathlineto{\pgfqpoint{2.837593in}{2.768224in}}%
\pgfpathlineto{\pgfqpoint{2.857298in}{2.747825in}}%
\pgfpathlineto{\pgfqpoint{2.877003in}{2.747824in}}%
\pgfpathlineto{\pgfqpoint{2.896708in}{2.731279in}}%
\pgfpathlineto{\pgfqpoint{3.014939in}{2.731270in}}%
\pgfpathlineto{\pgfqpoint{3.034644in}{2.712415in}}%
\pgfpathlineto{\pgfqpoint{3.074054in}{2.712411in}}%
\pgfpathlineto{\pgfqpoint{3.093759in}{2.631195in}}%
\pgfpathlineto{\pgfqpoint{3.172579in}{2.631175in}}%
\pgfpathlineto{\pgfqpoint{3.192284in}{2.620720in}}%
\pgfpathlineto{\pgfqpoint{3.211989in}{2.446776in}}%
\pgfpathlineto{\pgfqpoint{3.231694in}{2.349242in}}%
\pgfpathlineto{\pgfqpoint{3.251399in}{2.335249in}}%
\pgfpathlineto{\pgfqpoint{3.271104in}{2.225421in}}%
\pgfpathlineto{\pgfqpoint{3.290809in}{1.749546in}}%
\pgfpathlineto{\pgfqpoint{3.310515in}{1.731856in}}%
\pgfpathlineto{\pgfqpoint{3.369630in}{1.731825in}}%
\pgfpathlineto{\pgfqpoint{3.389335in}{1.720050in}}%
\pgfpathlineto{\pgfqpoint{3.409040in}{1.720043in}}%
\pgfpathlineto{\pgfqpoint{3.428745in}{1.716709in}}%
\pgfpathlineto{\pgfqpoint{3.448450in}{1.715627in}}%
\pgfpathlineto{\pgfqpoint{3.468155in}{1.697494in}}%
\pgfpathlineto{\pgfqpoint{3.487860in}{1.564507in}}%
\pgfpathlineto{\pgfqpoint{3.527270in}{1.556797in}}%
\pgfpathlineto{\pgfqpoint{3.546975in}{1.546135in}}%
\pgfpathlineto{\pgfqpoint{3.566680in}{1.546136in}}%
\pgfpathlineto{\pgfqpoint{3.586385in}{1.540574in}}%
\pgfpathlineto{\pgfqpoint{3.606090in}{1.537851in}}%
\pgfpathlineto{\pgfqpoint{3.625795in}{1.537850in}}%
\pgfpathlineto{\pgfqpoint{3.645500in}{1.533960in}}%
\pgfpathlineto{\pgfqpoint{3.665205in}{1.533067in}}%
\pgfpathlineto{\pgfqpoint{3.684911in}{1.527673in}}%
\pgfpathlineto{\pgfqpoint{3.704616in}{1.525005in}}%
\pgfpathlineto{\pgfqpoint{3.724321in}{1.524538in}}%
\pgfpathlineto{\pgfqpoint{3.744026in}{1.031117in}}%
\pgfpathlineto{\pgfqpoint{3.763731in}{1.013163in}}%
\pgfpathlineto{\pgfqpoint{3.783436in}{1.008663in}}%
\pgfpathlineto{\pgfqpoint{3.803141in}{1.005773in}}%
\pgfpathlineto{\pgfqpoint{3.842551in}{0.990557in}}%
\pgfpathlineto{\pgfqpoint{3.921371in}{0.990555in}}%
\pgfpathlineto{\pgfqpoint{3.941076in}{0.759670in}}%
\pgfpathlineto{\pgfqpoint{3.960781in}{0.745976in}}%
\pgfpathlineto{\pgfqpoint{3.980486in}{0.709008in}}%
\pgfpathlineto{\pgfqpoint{4.354882in}{0.708335in}}%
\pgfpathlineto{\pgfqpoint{4.374587in}{0.687179in}}%
\pgfpathlineto{\pgfqpoint{5.517480in}{0.687163in}}%
\pgfpathlineto{\pgfqpoint{5.517480in}{0.687163in}}%
\pgfusepath{stroke}%
\end{pgfscope}%
\begin{pgfscope}%
\pgfsetrectcap%
\pgfsetmiterjoin%
\pgfsetlinewidth{0.803000pt}%
\definecolor{currentstroke}{rgb}{0.000000,0.000000,0.000000}%
\pgfsetstrokecolor{currentstroke}%
\pgfsetdash{}{0pt}%
\pgfpathmoveto{\pgfqpoint{0.634568in}{0.565123in}}%
\pgfpathlineto{\pgfqpoint{0.634568in}{3.250000in}}%
\pgfusepath{stroke}%
\end{pgfscope}%
\begin{pgfscope}%
\pgfsetrectcap%
\pgfsetmiterjoin%
\pgfsetlinewidth{0.803000pt}%
\definecolor{currentstroke}{rgb}{0.000000,0.000000,0.000000}%
\pgfsetstrokecolor{currentstroke}%
\pgfsetdash{}{0pt}%
\pgfpathmoveto{\pgfqpoint{5.750000in}{0.565123in}}%
\pgfpathlineto{\pgfqpoint{5.750000in}{3.250000in}}%
\pgfusepath{stroke}%
\end{pgfscope}%
\begin{pgfscope}%
\pgfsetrectcap%
\pgfsetmiterjoin%
\pgfsetlinewidth{0.803000pt}%
\definecolor{currentstroke}{rgb}{0.000000,0.000000,0.000000}%
\pgfsetstrokecolor{currentstroke}%
\pgfsetdash{}{0pt}%
\pgfpathmoveto{\pgfqpoint{0.634568in}{0.565123in}}%
\pgfpathlineto{\pgfqpoint{5.750000in}{0.565123in}}%
\pgfusepath{stroke}%
\end{pgfscope}%
\begin{pgfscope}%
\pgfsetrectcap%
\pgfsetmiterjoin%
\pgfsetlinewidth{0.803000pt}%
\definecolor{currentstroke}{rgb}{0.000000,0.000000,0.000000}%
\pgfsetstrokecolor{currentstroke}%
\pgfsetdash{}{0pt}%
\pgfpathmoveto{\pgfqpoint{0.634568in}{3.250000in}}%
\pgfpathlineto{\pgfqpoint{5.750000in}{3.250000in}}%
\pgfusepath{stroke}%
\end{pgfscope}%
\begin{pgfscope}%
\pgfsetbuttcap%
\pgfsetmiterjoin%
\definecolor{currentfill}{rgb}{1.000000,1.000000,1.000000}%
\pgfsetfillcolor{currentfill}%
\pgfsetfillopacity{0.800000}%
\pgfsetlinewidth{1.003750pt}%
\definecolor{currentstroke}{rgb}{0.800000,0.800000,0.800000}%
\pgfsetstrokecolor{currentstroke}%
\pgfsetstrokeopacity{0.800000}%
\pgfsetdash{}{0pt}%
\pgfpathmoveto{\pgfqpoint{4.273532in}{2.751543in}}%
\pgfpathlineto{\pgfqpoint{5.652778in}{2.751543in}}%
\pgfpathquadraticcurveto{\pgfqpoint{5.680556in}{2.751543in}}{\pgfqpoint{5.680556in}{2.779321in}}%
\pgfpathlineto{\pgfqpoint{5.680556in}{3.152778in}}%
\pgfpathquadraticcurveto{\pgfqpoint{5.680556in}{3.180556in}}{\pgfqpoint{5.652778in}{3.180556in}}%
\pgfpathlineto{\pgfqpoint{4.273532in}{3.180556in}}%
\pgfpathquadraticcurveto{\pgfqpoint{4.245755in}{3.180556in}}{\pgfqpoint{4.245755in}{3.152778in}}%
\pgfpathlineto{\pgfqpoint{4.245755in}{2.779321in}}%
\pgfpathquadraticcurveto{\pgfqpoint{4.245755in}{2.751543in}}{\pgfqpoint{4.273532in}{2.751543in}}%
\pgfpathclose%
\pgfusepath{stroke,fill}%
\end{pgfscope}%
\begin{pgfscope}%
\pgfsetrectcap%
\pgfsetroundjoin%
\pgfsetlinewidth{2.007500pt}%
\definecolor{currentstroke}{rgb}{0.121569,0.466667,0.705882}%
\pgfsetstrokecolor{currentstroke}%
\pgfsetdash{}{0pt}%
\pgfpathmoveto{\pgfqpoint{4.301310in}{3.076389in}}%
\pgfpathlineto{\pgfqpoint{4.579088in}{3.076389in}}%
\pgfusepath{stroke}%
\end{pgfscope}%
\begin{pgfscope}%
\definecolor{textcolor}{rgb}{0.000000,0.000000,0.000000}%
\pgfsetstrokecolor{textcolor}%
\pgfsetfillcolor{textcolor}%
\pgftext[x=4.690199in,y=3.027778in,left,base]{\color{textcolor}\rmfamily\fontsize{10.000000}{12.000000}\selectfont Scena fittizia}%
\end{pgfscope}%
\begin{pgfscope}%
\pgfsetrectcap%
\pgfsetroundjoin%
\pgfsetlinewidth{2.007500pt}%
\definecolor{currentstroke}{rgb}{1.000000,0.498039,0.054902}%
\pgfsetstrokecolor{currentstroke}%
\pgfsetdash{}{0pt}%
\pgfpathmoveto{\pgfqpoint{4.301310in}{2.882716in}}%
\pgfpathlineto{\pgfqpoint{4.579088in}{2.882716in}}%
\pgfusepath{stroke}%
\end{pgfscope}%
\begin{pgfscope}%
\definecolor{textcolor}{rgb}{0.000000,0.000000,0.000000}%
\pgfsetstrokecolor{textcolor}%
\pgfsetfillcolor{textcolor}%
\pgftext[x=4.690199in,y=2.834105in,left,base]{\color{textcolor}\rmfamily\fontsize{10.000000}{12.000000}\selectfont Scena realistica}%
\end{pgfscope}%
\end{pgfpicture}%
\makeatother%
\endgroup%
%
                \caption{Valutazione Closest-First sulla scena fittizia}
                \label{fig:eval-cf}

                %% Creator: Matplotlib, PGF backend
%%
%% To include the figure in your LaTeX document, write
%%   \input{<filename>.pgf}
%%
%% Make sure the required packages are loaded in your preamble
%%   \usepackage{pgf}
%%
%% Figures using additional raster images can only be included by \input if
%% they are in the same directory as the main LaTeX file. For loading figures
%% from other directories you can use the `import` package
%%   \usepackage{import}
%%
%% and then include the figures with
%%   \import{<path to file>}{<filename>.pgf}
%%
%% Matplotlib used the following preamble
%%
\begingroup%
\makeatletter%
\begin{pgfpicture}%
\pgfpathrectangle{\pgfpointorigin}{\pgfqpoint{5.900000in}{3.400000in}}%
\pgfusepath{use as bounding box, clip}%
\begin{pgfscope}%
\pgfsetbuttcap%
\pgfsetmiterjoin%
\definecolor{currentfill}{rgb}{1.000000,1.000000,1.000000}%
\pgfsetfillcolor{currentfill}%
\pgfsetlinewidth{0.000000pt}%
\definecolor{currentstroke}{rgb}{1.000000,1.000000,1.000000}%
\pgfsetstrokecolor{currentstroke}%
\pgfsetdash{}{0pt}%
\pgfpathmoveto{\pgfqpoint{0.000000in}{0.000000in}}%
\pgfpathlineto{\pgfqpoint{5.900000in}{0.000000in}}%
\pgfpathlineto{\pgfqpoint{5.900000in}{3.400000in}}%
\pgfpathlineto{\pgfqpoint{0.000000in}{3.400000in}}%
\pgfpathclose%
\pgfusepath{fill}%
\end{pgfscope}%
\begin{pgfscope}%
\pgfsetbuttcap%
\pgfsetmiterjoin%
\definecolor{currentfill}{rgb}{1.000000,1.000000,1.000000}%
\pgfsetfillcolor{currentfill}%
\pgfsetlinewidth{0.000000pt}%
\definecolor{currentstroke}{rgb}{0.000000,0.000000,0.000000}%
\pgfsetstrokecolor{currentstroke}%
\pgfsetstrokeopacity{0.000000}%
\pgfsetdash{}{0pt}%
\pgfpathmoveto{\pgfqpoint{0.565124in}{0.565123in}}%
\pgfpathlineto{\pgfqpoint{5.750000in}{0.565123in}}%
\pgfpathlineto{\pgfqpoint{5.750000in}{3.250000in}}%
\pgfpathlineto{\pgfqpoint{0.565124in}{3.250000in}}%
\pgfpathclose%
\pgfusepath{fill}%
\end{pgfscope}%
\begin{pgfscope}%
\pgfsetbuttcap%
\pgfsetroundjoin%
\definecolor{currentfill}{rgb}{0.000000,0.000000,0.000000}%
\pgfsetfillcolor{currentfill}%
\pgfsetlinewidth{0.803000pt}%
\definecolor{currentstroke}{rgb}{0.000000,0.000000,0.000000}%
\pgfsetstrokecolor{currentstroke}%
\pgfsetdash{}{0pt}%
\pgfsys@defobject{currentmarker}{\pgfqpoint{0.000000in}{-0.048611in}}{\pgfqpoint{0.000000in}{0.000000in}}{%
\pgfpathmoveto{\pgfqpoint{0.000000in}{0.000000in}}%
\pgfpathlineto{\pgfqpoint{0.000000in}{-0.048611in}}%
\pgfusepath{stroke,fill}%
}%
\begin{pgfscope}%
\pgfsys@transformshift{0.800800in}{0.565123in}%
\pgfsys@useobject{currentmarker}{}%
\end{pgfscope}%
\end{pgfscope}%
\begin{pgfscope}%
\definecolor{textcolor}{rgb}{0.000000,0.000000,0.000000}%
\pgfsetstrokecolor{textcolor}%
\pgfsetfillcolor{textcolor}%
\pgftext[x=0.800800in,y=0.467901in,,top]{\color{textcolor}\rmfamily\fontsize{10.000000}{12.000000}\selectfont \(\displaystyle {0}\)}%
\end{pgfscope}%
\begin{pgfscope}%
\pgfsetbuttcap%
\pgfsetroundjoin%
\definecolor{currentfill}{rgb}{0.000000,0.000000,0.000000}%
\pgfsetfillcolor{currentfill}%
\pgfsetlinewidth{0.803000pt}%
\definecolor{currentstroke}{rgb}{0.000000,0.000000,0.000000}%
\pgfsetstrokecolor{currentstroke}%
\pgfsetdash{}{0pt}%
\pgfsys@defobject{currentmarker}{\pgfqpoint{0.000000in}{-0.048611in}}{\pgfqpoint{0.000000in}{0.000000in}}{%
\pgfpathmoveto{\pgfqpoint{0.000000in}{0.000000in}}%
\pgfpathlineto{\pgfqpoint{0.000000in}{-0.048611in}}%
\pgfusepath{stroke,fill}%
}%
\begin{pgfscope}%
\pgfsys@transformshift{1.799428in}{0.565123in}%
\pgfsys@useobject{currentmarker}{}%
\end{pgfscope}%
\end{pgfscope}%
\begin{pgfscope}%
\definecolor{textcolor}{rgb}{0.000000,0.000000,0.000000}%
\pgfsetstrokecolor{textcolor}%
\pgfsetfillcolor{textcolor}%
\pgftext[x=1.799428in,y=0.467901in,,top]{\color{textcolor}\rmfamily\fontsize{10.000000}{12.000000}\selectfont \(\displaystyle {50}\)}%
\end{pgfscope}%
\begin{pgfscope}%
\pgfsetbuttcap%
\pgfsetroundjoin%
\definecolor{currentfill}{rgb}{0.000000,0.000000,0.000000}%
\pgfsetfillcolor{currentfill}%
\pgfsetlinewidth{0.803000pt}%
\definecolor{currentstroke}{rgb}{0.000000,0.000000,0.000000}%
\pgfsetstrokecolor{currentstroke}%
\pgfsetdash{}{0pt}%
\pgfsys@defobject{currentmarker}{\pgfqpoint{0.000000in}{-0.048611in}}{\pgfqpoint{0.000000in}{0.000000in}}{%
\pgfpathmoveto{\pgfqpoint{0.000000in}{0.000000in}}%
\pgfpathlineto{\pgfqpoint{0.000000in}{-0.048611in}}%
\pgfusepath{stroke,fill}%
}%
\begin{pgfscope}%
\pgfsys@transformshift{2.798056in}{0.565123in}%
\pgfsys@useobject{currentmarker}{}%
\end{pgfscope}%
\end{pgfscope}%
\begin{pgfscope}%
\definecolor{textcolor}{rgb}{0.000000,0.000000,0.000000}%
\pgfsetstrokecolor{textcolor}%
\pgfsetfillcolor{textcolor}%
\pgftext[x=2.798056in,y=0.467901in,,top]{\color{textcolor}\rmfamily\fontsize{10.000000}{12.000000}\selectfont \(\displaystyle {100}\)}%
\end{pgfscope}%
\begin{pgfscope}%
\pgfsetbuttcap%
\pgfsetroundjoin%
\definecolor{currentfill}{rgb}{0.000000,0.000000,0.000000}%
\pgfsetfillcolor{currentfill}%
\pgfsetlinewidth{0.803000pt}%
\definecolor{currentstroke}{rgb}{0.000000,0.000000,0.000000}%
\pgfsetstrokecolor{currentstroke}%
\pgfsetdash{}{0pt}%
\pgfsys@defobject{currentmarker}{\pgfqpoint{0.000000in}{-0.048611in}}{\pgfqpoint{0.000000in}{0.000000in}}{%
\pgfpathmoveto{\pgfqpoint{0.000000in}{0.000000in}}%
\pgfpathlineto{\pgfqpoint{0.000000in}{-0.048611in}}%
\pgfusepath{stroke,fill}%
}%
\begin{pgfscope}%
\pgfsys@transformshift{3.796684in}{0.565123in}%
\pgfsys@useobject{currentmarker}{}%
\end{pgfscope}%
\end{pgfscope}%
\begin{pgfscope}%
\definecolor{textcolor}{rgb}{0.000000,0.000000,0.000000}%
\pgfsetstrokecolor{textcolor}%
\pgfsetfillcolor{textcolor}%
\pgftext[x=3.796684in,y=0.467901in,,top]{\color{textcolor}\rmfamily\fontsize{10.000000}{12.000000}\selectfont \(\displaystyle {150}\)}%
\end{pgfscope}%
\begin{pgfscope}%
\pgfsetbuttcap%
\pgfsetroundjoin%
\definecolor{currentfill}{rgb}{0.000000,0.000000,0.000000}%
\pgfsetfillcolor{currentfill}%
\pgfsetlinewidth{0.803000pt}%
\definecolor{currentstroke}{rgb}{0.000000,0.000000,0.000000}%
\pgfsetstrokecolor{currentstroke}%
\pgfsetdash{}{0pt}%
\pgfsys@defobject{currentmarker}{\pgfqpoint{0.000000in}{-0.048611in}}{\pgfqpoint{0.000000in}{0.000000in}}{%
\pgfpathmoveto{\pgfqpoint{0.000000in}{0.000000in}}%
\pgfpathlineto{\pgfqpoint{0.000000in}{-0.048611in}}%
\pgfusepath{stroke,fill}%
}%
\begin{pgfscope}%
\pgfsys@transformshift{4.795312in}{0.565123in}%
\pgfsys@useobject{currentmarker}{}%
\end{pgfscope}%
\end{pgfscope}%
\begin{pgfscope}%
\definecolor{textcolor}{rgb}{0.000000,0.000000,0.000000}%
\pgfsetstrokecolor{textcolor}%
\pgfsetfillcolor{textcolor}%
\pgftext[x=4.795312in,y=0.467901in,,top]{\color{textcolor}\rmfamily\fontsize{10.000000}{12.000000}\selectfont \(\displaystyle {200}\)}%
\end{pgfscope}%
\begin{pgfscope}%
\definecolor{textcolor}{rgb}{0.000000,0.000000,0.000000}%
\pgfsetstrokecolor{textcolor}%
\pgfsetfillcolor{textcolor}%
\pgftext[x=3.157562in,y=0.288889in,,top]{\color{textcolor}\rmfamily\fontsize{10.000000}{12.000000}\selectfont Tempo (Frame)}%
\end{pgfscope}%
\begin{pgfscope}%
\pgfsetbuttcap%
\pgfsetroundjoin%
\definecolor{currentfill}{rgb}{0.000000,0.000000,0.000000}%
\pgfsetfillcolor{currentfill}%
\pgfsetlinewidth{0.803000pt}%
\definecolor{currentstroke}{rgb}{0.000000,0.000000,0.000000}%
\pgfsetstrokecolor{currentstroke}%
\pgfsetdash{}{0pt}%
\pgfsys@defobject{currentmarker}{\pgfqpoint{-0.048611in}{0.000000in}}{\pgfqpoint{-0.000000in}{0.000000in}}{%
\pgfpathmoveto{\pgfqpoint{-0.000000in}{0.000000in}}%
\pgfpathlineto{\pgfqpoint{-0.048611in}{0.000000in}}%
\pgfusepath{stroke,fill}%
}%
\begin{pgfscope}%
\pgfsys@transformshift{0.565124in}{0.687163in}%
\pgfsys@useobject{currentmarker}{}%
\end{pgfscope}%
\end{pgfscope}%
\begin{pgfscope}%
\definecolor{textcolor}{rgb}{0.000000,0.000000,0.000000}%
\pgfsetstrokecolor{textcolor}%
\pgfsetfillcolor{textcolor}%
\pgftext[x=0.398457in, y=0.638938in, left, base]{\color{textcolor}\rmfamily\fontsize{10.000000}{12.000000}\selectfont \(\displaystyle {0}\)}%
\end{pgfscope}%
\begin{pgfscope}%
\pgfsetbuttcap%
\pgfsetroundjoin%
\definecolor{currentfill}{rgb}{0.000000,0.000000,0.000000}%
\pgfsetfillcolor{currentfill}%
\pgfsetlinewidth{0.803000pt}%
\definecolor{currentstroke}{rgb}{0.000000,0.000000,0.000000}%
\pgfsetstrokecolor{currentstroke}%
\pgfsetdash{}{0pt}%
\pgfsys@defobject{currentmarker}{\pgfqpoint{-0.048611in}{0.000000in}}{\pgfqpoint{-0.000000in}{0.000000in}}{%
\pgfpathmoveto{\pgfqpoint{-0.000000in}{0.000000in}}%
\pgfpathlineto{\pgfqpoint{-0.048611in}{0.000000in}}%
\pgfusepath{stroke,fill}%
}%
\begin{pgfscope}%
\pgfsys@transformshift{0.565124in}{1.221913in}%
\pgfsys@useobject{currentmarker}{}%
\end{pgfscope}%
\end{pgfscope}%
\begin{pgfscope}%
\definecolor{textcolor}{rgb}{0.000000,0.000000,0.000000}%
\pgfsetstrokecolor{textcolor}%
\pgfsetfillcolor{textcolor}%
\pgftext[x=0.329012in, y=1.173687in, left, base]{\color{textcolor}\rmfamily\fontsize{10.000000}{12.000000}\selectfont \(\displaystyle {20}\)}%
\end{pgfscope}%
\begin{pgfscope}%
\pgfsetbuttcap%
\pgfsetroundjoin%
\definecolor{currentfill}{rgb}{0.000000,0.000000,0.000000}%
\pgfsetfillcolor{currentfill}%
\pgfsetlinewidth{0.803000pt}%
\definecolor{currentstroke}{rgb}{0.000000,0.000000,0.000000}%
\pgfsetstrokecolor{currentstroke}%
\pgfsetdash{}{0pt}%
\pgfsys@defobject{currentmarker}{\pgfqpoint{-0.048611in}{0.000000in}}{\pgfqpoint{-0.000000in}{0.000000in}}{%
\pgfpathmoveto{\pgfqpoint{-0.000000in}{0.000000in}}%
\pgfpathlineto{\pgfqpoint{-0.048611in}{0.000000in}}%
\pgfusepath{stroke,fill}%
}%
\begin{pgfscope}%
\pgfsys@transformshift{0.565124in}{1.756662in}%
\pgfsys@useobject{currentmarker}{}%
\end{pgfscope}%
\end{pgfscope}%
\begin{pgfscope}%
\definecolor{textcolor}{rgb}{0.000000,0.000000,0.000000}%
\pgfsetstrokecolor{textcolor}%
\pgfsetfillcolor{textcolor}%
\pgftext[x=0.329012in, y=1.708437in, left, base]{\color{textcolor}\rmfamily\fontsize{10.000000}{12.000000}\selectfont \(\displaystyle {40}\)}%
\end{pgfscope}%
\begin{pgfscope}%
\pgfsetbuttcap%
\pgfsetroundjoin%
\definecolor{currentfill}{rgb}{0.000000,0.000000,0.000000}%
\pgfsetfillcolor{currentfill}%
\pgfsetlinewidth{0.803000pt}%
\definecolor{currentstroke}{rgb}{0.000000,0.000000,0.000000}%
\pgfsetstrokecolor{currentstroke}%
\pgfsetdash{}{0pt}%
\pgfsys@defobject{currentmarker}{\pgfqpoint{-0.048611in}{0.000000in}}{\pgfqpoint{-0.000000in}{0.000000in}}{%
\pgfpathmoveto{\pgfqpoint{-0.000000in}{0.000000in}}%
\pgfpathlineto{\pgfqpoint{-0.048611in}{0.000000in}}%
\pgfusepath{stroke,fill}%
}%
\begin{pgfscope}%
\pgfsys@transformshift{0.565124in}{2.291412in}%
\pgfsys@useobject{currentmarker}{}%
\end{pgfscope}%
\end{pgfscope}%
\begin{pgfscope}%
\definecolor{textcolor}{rgb}{0.000000,0.000000,0.000000}%
\pgfsetstrokecolor{textcolor}%
\pgfsetfillcolor{textcolor}%
\pgftext[x=0.329012in, y=2.243186in, left, base]{\color{textcolor}\rmfamily\fontsize{10.000000}{12.000000}\selectfont \(\displaystyle {60}\)}%
\end{pgfscope}%
\begin{pgfscope}%
\pgfsetbuttcap%
\pgfsetroundjoin%
\definecolor{currentfill}{rgb}{0.000000,0.000000,0.000000}%
\pgfsetfillcolor{currentfill}%
\pgfsetlinewidth{0.803000pt}%
\definecolor{currentstroke}{rgb}{0.000000,0.000000,0.000000}%
\pgfsetstrokecolor{currentstroke}%
\pgfsetdash{}{0pt}%
\pgfsys@defobject{currentmarker}{\pgfqpoint{-0.048611in}{0.000000in}}{\pgfqpoint{-0.000000in}{0.000000in}}{%
\pgfpathmoveto{\pgfqpoint{-0.000000in}{0.000000in}}%
\pgfpathlineto{\pgfqpoint{-0.048611in}{0.000000in}}%
\pgfusepath{stroke,fill}%
}%
\begin{pgfscope}%
\pgfsys@transformshift{0.565124in}{2.826161in}%
\pgfsys@useobject{currentmarker}{}%
\end{pgfscope}%
\end{pgfscope}%
\begin{pgfscope}%
\definecolor{textcolor}{rgb}{0.000000,0.000000,0.000000}%
\pgfsetstrokecolor{textcolor}%
\pgfsetfillcolor{textcolor}%
\pgftext[x=0.329012in, y=2.777936in, left, base]{\color{textcolor}\rmfamily\fontsize{10.000000}{12.000000}\selectfont \(\displaystyle {80}\)}%
\end{pgfscope}%
\begin{pgfscope}%
\definecolor{textcolor}{rgb}{0.000000,0.000000,0.000000}%
\pgfsetstrokecolor{textcolor}%
\pgfsetfillcolor{textcolor}%
\pgftext[x=0.273457in,y=1.907562in,,bottom,rotate=90.000000]{\color{textcolor}\rmfamily\fontsize{10.000000}{12.000000}\selectfont Differenza percettiva}%
\end{pgfscope}%
\begin{pgfscope}%
\pgfpathrectangle{\pgfqpoint{0.565124in}{0.565123in}}{\pgfqpoint{5.184876in}{2.684877in}}%
\pgfusepath{clip}%
\pgfsetrectcap%
\pgfsetroundjoin%
\pgfsetlinewidth{2.007500pt}%
\definecolor{currentstroke}{rgb}{0.121569,0.466667,0.705882}%
\pgfsetstrokecolor{currentstroke}%
\pgfsetdash{}{0pt}%
\pgfpathmoveto{\pgfqpoint{0.800800in}{1.561468in}}%
\pgfpathlineto{\pgfqpoint{0.820773in}{1.510069in}}%
\pgfpathlineto{\pgfqpoint{0.840745in}{1.467996in}}%
\pgfpathlineto{\pgfqpoint{0.880690in}{1.410692in}}%
\pgfpathlineto{\pgfqpoint{0.900663in}{1.388364in}}%
\pgfpathlineto{\pgfqpoint{0.920635in}{1.378362in}}%
\pgfpathlineto{\pgfqpoint{0.940608in}{1.360700in}}%
\pgfpathlineto{\pgfqpoint{0.960580in}{1.344659in}}%
\pgfpathlineto{\pgfqpoint{0.980553in}{1.319806in}}%
\pgfpathlineto{\pgfqpoint{1.000526in}{1.308910in}}%
\pgfpathlineto{\pgfqpoint{1.020498in}{1.293976in}}%
\pgfpathlineto{\pgfqpoint{1.040471in}{1.282749in}}%
\pgfpathlineto{\pgfqpoint{1.060443in}{1.275091in}}%
\pgfpathlineto{\pgfqpoint{1.080416in}{1.264528in}}%
\pgfpathlineto{\pgfqpoint{1.100388in}{1.246918in}}%
\pgfpathlineto{\pgfqpoint{1.120361in}{1.236678in}}%
\pgfpathlineto{\pgfqpoint{1.140333in}{1.146299in}}%
\pgfpathlineto{\pgfqpoint{1.160306in}{1.143040in}}%
\pgfpathlineto{\pgfqpoint{1.220224in}{1.117246in}}%
\pgfpathlineto{\pgfqpoint{1.240196in}{1.113690in}}%
\pgfpathlineto{\pgfqpoint{1.260169in}{1.105601in}}%
\pgfpathlineto{\pgfqpoint{1.280141in}{1.032741in}}%
\pgfpathlineto{\pgfqpoint{1.300114in}{1.000788in}}%
\pgfpathlineto{\pgfqpoint{1.320086in}{0.994582in}}%
\pgfpathlineto{\pgfqpoint{1.340059in}{0.993358in}}%
\pgfpathlineto{\pgfqpoint{1.360032in}{0.978879in}}%
\pgfpathlineto{\pgfqpoint{1.380004in}{0.975466in}}%
\pgfpathlineto{\pgfqpoint{1.399977in}{0.975447in}}%
\pgfpathlineto{\pgfqpoint{1.419949in}{0.973715in}}%
\pgfpathlineto{\pgfqpoint{1.439922in}{0.973715in}}%
\pgfpathlineto{\pgfqpoint{1.459894in}{0.970967in}}%
\pgfpathlineto{\pgfqpoint{1.479867in}{0.958129in}}%
\pgfpathlineto{\pgfqpoint{1.499839in}{0.958129in}}%
\pgfpathlineto{\pgfqpoint{1.519812in}{0.949113in}}%
\pgfpathlineto{\pgfqpoint{1.539785in}{0.947546in}}%
\pgfpathlineto{\pgfqpoint{1.559757in}{0.929692in}}%
\pgfpathlineto{\pgfqpoint{1.579730in}{0.929692in}}%
\pgfpathlineto{\pgfqpoint{1.599702in}{0.927882in}}%
\pgfpathlineto{\pgfqpoint{1.619675in}{0.912444in}}%
\pgfpathlineto{\pgfqpoint{1.739510in}{0.904262in}}%
\pgfpathlineto{\pgfqpoint{1.759483in}{0.903897in}}%
\pgfpathlineto{\pgfqpoint{1.779455in}{0.868436in}}%
\pgfpathlineto{\pgfqpoint{1.799428in}{0.865097in}}%
\pgfpathlineto{\pgfqpoint{1.819400in}{0.856896in}}%
\pgfpathlineto{\pgfqpoint{1.839373in}{0.851986in}}%
\pgfpathlineto{\pgfqpoint{1.859346in}{0.851985in}}%
\pgfpathlineto{\pgfqpoint{1.879318in}{0.819902in}}%
\pgfpathlineto{\pgfqpoint{1.919263in}{0.818160in}}%
\pgfpathlineto{\pgfqpoint{1.939236in}{0.784476in}}%
\pgfpathlineto{\pgfqpoint{1.959208in}{0.780387in}}%
\pgfpathlineto{\pgfqpoint{1.979181in}{0.780381in}}%
\pgfpathlineto{\pgfqpoint{1.999153in}{0.778382in}}%
\pgfpathlineto{\pgfqpoint{2.039099in}{0.771766in}}%
\pgfpathlineto{\pgfqpoint{2.059071in}{0.770499in}}%
\pgfpathlineto{\pgfqpoint{2.079044in}{0.762402in}}%
\pgfpathlineto{\pgfqpoint{2.099016in}{0.760096in}}%
\pgfpathlineto{\pgfqpoint{2.178906in}{0.756174in}}%
\pgfpathlineto{\pgfqpoint{2.218852in}{0.755998in}}%
\pgfpathlineto{\pgfqpoint{2.238824in}{0.754073in}}%
\pgfpathlineto{\pgfqpoint{2.258797in}{0.753831in}}%
\pgfpathlineto{\pgfqpoint{2.298742in}{0.744237in}}%
\pgfpathlineto{\pgfqpoint{2.318714in}{0.740989in}}%
\pgfpathlineto{\pgfqpoint{2.338687in}{0.715816in}}%
\pgfpathlineto{\pgfqpoint{2.398605in}{0.714878in}}%
\pgfpathlineto{\pgfqpoint{2.418577in}{0.713059in}}%
\pgfpathlineto{\pgfqpoint{2.458522in}{0.712405in}}%
\pgfpathlineto{\pgfqpoint{2.478495in}{0.710825in}}%
\pgfpathlineto{\pgfqpoint{2.538413in}{0.710357in}}%
\pgfpathlineto{\pgfqpoint{2.658248in}{0.710145in}}%
\pgfpathlineto{\pgfqpoint{2.718166in}{0.709840in}}%
\pgfpathlineto{\pgfqpoint{2.738138in}{0.698363in}}%
\pgfpathlineto{\pgfqpoint{2.798056in}{0.697489in}}%
\pgfpathlineto{\pgfqpoint{2.857973in}{0.697082in}}%
\pgfpathlineto{\pgfqpoint{3.017754in}{0.696709in}}%
\pgfpathlineto{\pgfqpoint{3.037727in}{0.689823in}}%
\pgfpathlineto{\pgfqpoint{3.077672in}{0.689771in}}%
\pgfpathlineto{\pgfqpoint{3.097644in}{0.688638in}}%
\pgfpathlineto{\pgfqpoint{3.177534in}{0.687884in}}%
\pgfpathlineto{\pgfqpoint{3.217480in}{0.687165in}}%
\pgfpathlineto{\pgfqpoint{5.494351in}{0.687163in}}%
\pgfpathlineto{\pgfqpoint{5.494351in}{0.687163in}}%
\pgfusepath{stroke}%
\end{pgfscope}%
\begin{pgfscope}%
\pgfpathrectangle{\pgfqpoint{0.565124in}{0.565123in}}{\pgfqpoint{5.184876in}{2.684877in}}%
\pgfusepath{clip}%
\pgfsetrectcap%
\pgfsetroundjoin%
\pgfsetlinewidth{2.007500pt}%
\definecolor{currentstroke}{rgb}{1.000000,0.498039,0.054902}%
\pgfsetstrokecolor{currentstroke}%
\pgfsetdash{}{0pt}%
\pgfpathmoveto{\pgfqpoint{0.820773in}{3.127957in}}%
\pgfpathlineto{\pgfqpoint{0.880690in}{3.127960in}}%
\pgfpathlineto{\pgfqpoint{0.900663in}{3.059568in}}%
\pgfpathlineto{\pgfqpoint{1.160306in}{3.059572in}}%
\pgfpathlineto{\pgfqpoint{1.180279in}{3.036434in}}%
\pgfpathlineto{\pgfqpoint{1.360032in}{3.036438in}}%
\pgfpathlineto{\pgfqpoint{1.380004in}{3.022542in}}%
\pgfpathlineto{\pgfqpoint{1.459894in}{3.022543in}}%
\pgfpathlineto{\pgfqpoint{1.479867in}{3.018213in}}%
\pgfpathlineto{\pgfqpoint{1.499839in}{3.009950in}}%
\pgfpathlineto{\pgfqpoint{1.519812in}{3.009771in}}%
\pgfpathlineto{\pgfqpoint{1.559757in}{2.968472in}}%
\pgfpathlineto{\pgfqpoint{1.639647in}{2.968468in}}%
\pgfpathlineto{\pgfqpoint{1.659620in}{2.956617in}}%
\pgfpathlineto{\pgfqpoint{1.919263in}{2.956600in}}%
\pgfpathlineto{\pgfqpoint{1.939236in}{2.952700in}}%
\pgfpathlineto{\pgfqpoint{2.019126in}{2.952686in}}%
\pgfpathlineto{\pgfqpoint{2.039099in}{2.652358in}}%
\pgfpathlineto{\pgfqpoint{2.138961in}{2.652349in}}%
\pgfpathlineto{\pgfqpoint{2.158934in}{2.596027in}}%
\pgfpathlineto{\pgfqpoint{2.258797in}{2.595358in}}%
\pgfpathlineto{\pgfqpoint{2.278769in}{2.593842in}}%
\pgfpathlineto{\pgfqpoint{2.298742in}{2.098955in}}%
\pgfpathlineto{\pgfqpoint{2.358660in}{2.098329in}}%
\pgfpathlineto{\pgfqpoint{2.378632in}{2.085753in}}%
\pgfpathlineto{\pgfqpoint{2.398605in}{2.081974in}}%
\pgfpathlineto{\pgfqpoint{2.418577in}{2.080183in}}%
\pgfpathlineto{\pgfqpoint{2.438550in}{2.052172in}}%
\pgfpathlineto{\pgfqpoint{2.458522in}{1.901880in}}%
\pgfpathlineto{\pgfqpoint{2.478495in}{1.885648in}}%
\pgfpathlineto{\pgfqpoint{2.498467in}{1.885647in}}%
\pgfpathlineto{\pgfqpoint{2.518440in}{1.884285in}}%
\pgfpathlineto{\pgfqpoint{2.538413in}{1.881313in}}%
\pgfpathlineto{\pgfqpoint{2.558385in}{1.876640in}}%
\pgfpathlineto{\pgfqpoint{2.578358in}{1.875545in}}%
\pgfpathlineto{\pgfqpoint{2.598330in}{1.869515in}}%
\pgfpathlineto{\pgfqpoint{2.618303in}{1.866576in}}%
\pgfpathlineto{\pgfqpoint{2.638275in}{1.356324in}}%
\pgfpathlineto{\pgfqpoint{2.658248in}{1.340066in}}%
\pgfpathlineto{\pgfqpoint{2.678220in}{1.327481in}}%
\pgfpathlineto{\pgfqpoint{2.718166in}{1.327482in}}%
\pgfpathlineto{\pgfqpoint{2.738138in}{1.020726in}}%
\pgfpathlineto{\pgfqpoint{2.778083in}{1.019960in}}%
\pgfpathlineto{\pgfqpoint{2.857973in}{1.019956in}}%
\pgfpathlineto{\pgfqpoint{2.877946in}{0.921188in}}%
\pgfpathlineto{\pgfqpoint{2.937864in}{0.921185in}}%
\pgfpathlineto{\pgfqpoint{2.957836in}{0.917182in}}%
\pgfpathlineto{\pgfqpoint{3.257425in}{0.917180in}}%
\pgfpathlineto{\pgfqpoint{3.277397in}{0.915183in}}%
\pgfpathlineto{\pgfqpoint{3.477123in}{0.915183in}}%
\pgfpathlineto{\pgfqpoint{3.497095in}{0.913301in}}%
\pgfpathlineto{\pgfqpoint{3.517068in}{0.913300in}}%
\pgfpathlineto{\pgfqpoint{3.537040in}{0.904603in}}%
\pgfpathlineto{\pgfqpoint{3.576986in}{0.903555in}}%
\pgfpathlineto{\pgfqpoint{3.596958in}{0.839626in}}%
\pgfpathlineto{\pgfqpoint{3.616931in}{0.839625in}}%
\pgfpathlineto{\pgfqpoint{3.636903in}{0.837011in}}%
\pgfpathlineto{\pgfqpoint{3.656876in}{0.810542in}}%
\pgfpathlineto{\pgfqpoint{3.676848in}{0.802567in}}%
\pgfpathlineto{\pgfqpoint{3.696821in}{0.765725in}}%
\pgfpathlineto{\pgfqpoint{3.796684in}{0.765570in}}%
\pgfpathlineto{\pgfqpoint{3.816656in}{0.754226in}}%
\pgfpathlineto{\pgfqpoint{3.836629in}{0.753059in}}%
\pgfpathlineto{\pgfqpoint{3.856601in}{0.746093in}}%
\pgfpathlineto{\pgfqpoint{3.876574in}{0.726755in}}%
\pgfpathlineto{\pgfqpoint{3.896547in}{0.726219in}}%
\pgfpathlineto{\pgfqpoint{3.916519in}{0.719411in}}%
\pgfpathlineto{\pgfqpoint{3.936492in}{0.715859in}}%
\pgfpathlineto{\pgfqpoint{3.956464in}{0.714092in}}%
\pgfpathlineto{\pgfqpoint{3.976437in}{0.711171in}}%
\pgfpathlineto{\pgfqpoint{4.375888in}{0.711165in}}%
\pgfpathlineto{\pgfqpoint{4.395861in}{0.687174in}}%
\pgfpathlineto{\pgfqpoint{5.514324in}{0.687163in}}%
\pgfpathlineto{\pgfqpoint{5.514324in}{0.687163in}}%
\pgfusepath{stroke}%
\end{pgfscope}%
\begin{pgfscope}%
\pgfsetrectcap%
\pgfsetmiterjoin%
\pgfsetlinewidth{0.803000pt}%
\definecolor{currentstroke}{rgb}{0.000000,0.000000,0.000000}%
\pgfsetstrokecolor{currentstroke}%
\pgfsetdash{}{0pt}%
\pgfpathmoveto{\pgfqpoint{0.565124in}{0.565123in}}%
\pgfpathlineto{\pgfqpoint{0.565124in}{3.250000in}}%
\pgfusepath{stroke}%
\end{pgfscope}%
\begin{pgfscope}%
\pgfsetrectcap%
\pgfsetmiterjoin%
\pgfsetlinewidth{0.803000pt}%
\definecolor{currentstroke}{rgb}{0.000000,0.000000,0.000000}%
\pgfsetstrokecolor{currentstroke}%
\pgfsetdash{}{0pt}%
\pgfpathmoveto{\pgfqpoint{5.750000in}{0.565123in}}%
\pgfpathlineto{\pgfqpoint{5.750000in}{3.250000in}}%
\pgfusepath{stroke}%
\end{pgfscope}%
\begin{pgfscope}%
\pgfsetrectcap%
\pgfsetmiterjoin%
\pgfsetlinewidth{0.803000pt}%
\definecolor{currentstroke}{rgb}{0.000000,0.000000,0.000000}%
\pgfsetstrokecolor{currentstroke}%
\pgfsetdash{}{0pt}%
\pgfpathmoveto{\pgfqpoint{0.565124in}{0.565123in}}%
\pgfpathlineto{\pgfqpoint{5.750000in}{0.565123in}}%
\pgfusepath{stroke}%
\end{pgfscope}%
\begin{pgfscope}%
\pgfsetrectcap%
\pgfsetmiterjoin%
\pgfsetlinewidth{0.803000pt}%
\definecolor{currentstroke}{rgb}{0.000000,0.000000,0.000000}%
\pgfsetstrokecolor{currentstroke}%
\pgfsetdash{}{0pt}%
\pgfpathmoveto{\pgfqpoint{0.565124in}{3.250000in}}%
\pgfpathlineto{\pgfqpoint{5.750000in}{3.250000in}}%
\pgfusepath{stroke}%
\end{pgfscope}%
\begin{pgfscope}%
\pgfsetbuttcap%
\pgfsetmiterjoin%
\definecolor{currentfill}{rgb}{1.000000,1.000000,1.000000}%
\pgfsetfillcolor{currentfill}%
\pgfsetfillopacity{0.800000}%
\pgfsetlinewidth{1.003750pt}%
\definecolor{currentstroke}{rgb}{0.800000,0.800000,0.800000}%
\pgfsetstrokecolor{currentstroke}%
\pgfsetstrokeopacity{0.800000}%
\pgfsetdash{}{0pt}%
\pgfpathmoveto{\pgfqpoint{4.273532in}{2.751543in}}%
\pgfpathlineto{\pgfqpoint{5.652778in}{2.751543in}}%
\pgfpathquadraticcurveto{\pgfqpoint{5.680556in}{2.751543in}}{\pgfqpoint{5.680556in}{2.779321in}}%
\pgfpathlineto{\pgfqpoint{5.680556in}{3.152778in}}%
\pgfpathquadraticcurveto{\pgfqpoint{5.680556in}{3.180556in}}{\pgfqpoint{5.652778in}{3.180556in}}%
\pgfpathlineto{\pgfqpoint{4.273532in}{3.180556in}}%
\pgfpathquadraticcurveto{\pgfqpoint{4.245755in}{3.180556in}}{\pgfqpoint{4.245755in}{3.152778in}}%
\pgfpathlineto{\pgfqpoint{4.245755in}{2.779321in}}%
\pgfpathquadraticcurveto{\pgfqpoint{4.245755in}{2.751543in}}{\pgfqpoint{4.273532in}{2.751543in}}%
\pgfpathclose%
\pgfusepath{stroke,fill}%
\end{pgfscope}%
\begin{pgfscope}%
\pgfsetrectcap%
\pgfsetroundjoin%
\pgfsetlinewidth{2.007500pt}%
\definecolor{currentstroke}{rgb}{0.121569,0.466667,0.705882}%
\pgfsetstrokecolor{currentstroke}%
\pgfsetdash{}{0pt}%
\pgfpathmoveto{\pgfqpoint{4.301310in}{3.076389in}}%
\pgfpathlineto{\pgfqpoint{4.579088in}{3.076389in}}%
\pgfusepath{stroke}%
\end{pgfscope}%
\begin{pgfscope}%
\definecolor{textcolor}{rgb}{0.000000,0.000000,0.000000}%
\pgfsetstrokecolor{textcolor}%
\pgfsetfillcolor{textcolor}%
\pgftext[x=4.690199in,y=3.027778in,left,base]{\color{textcolor}\rmfamily\fontsize{10.000000}{12.000000}\selectfont Scena fittizia}%
\end{pgfscope}%
\begin{pgfscope}%
\pgfsetrectcap%
\pgfsetroundjoin%
\pgfsetlinewidth{2.007500pt}%
\definecolor{currentstroke}{rgb}{1.000000,0.498039,0.054902}%
\pgfsetstrokecolor{currentstroke}%
\pgfsetdash{}{0pt}%
\pgfpathmoveto{\pgfqpoint{4.301310in}{2.882716in}}%
\pgfpathlineto{\pgfqpoint{4.579088in}{2.882716in}}%
\pgfusepath{stroke}%
\end{pgfscope}%
\begin{pgfscope}%
\definecolor{textcolor}{rgb}{0.000000,0.000000,0.000000}%
\pgfsetstrokecolor{textcolor}%
\pgfsetfillcolor{textcolor}%
\pgftext[x=4.690199in,y=2.834105in,left,base]{\color{textcolor}\rmfamily\fontsize{10.000000}{12.000000}\selectfont Scena realistica}%
\end{pgfscope}%
\end{pgfpicture}%
\makeatother%
\endgroup%

                \caption{Valutazione Closest-First on Camera}
                \label{fig:eval-cfv}
            \end{figure}
        
        \subsection{Closest-First on Camera}
            La seguente politica, fornendo priorità alle istanze di asset interne al view-frustum, elimina il problema che si presentava nella strategia analizzata al paragrafo precedente: la presenza di un grande numero di istanze di asset non in vista. Questa raffinatura presenta un miglioramento del 58.3\% per la scena fittizia e del 41.6\% per la scena realistica.

        \newpage            
        \subsection{Sphere Tracing}
            % sphere tracing: % evidenzia le differenze con l'angolo della luce
            \subsubsection{Valutazione della distanza}
            Considerare prioritari le istanze che compongono la vista, per quanto ragionevole, non porta apparenti miglioramenti nella scena realistica (figura \ref{fig:eval-stdist}). Questo comportamento può essere motivato dal fatto che l'inquadratura considerata è quasi priva di istanze occluse più vicine di altre istanze non occluse, eliminando il beneficio principale di questa politica. Nella scena fittizia invece, grazie alla presenza di istanze vicine e completamente occluse, si nota un miglioramento, benché leggero, del 6.1\%.

            È opportuno notare che i risultati di questa strategia si alterano al variare dell'angolo della luce. Variando l'angolo da $30^\circ$ a $5^\circ$ si nota una variazione del 5.6\% per la scena fittizia e del 7.1\% per la scena realistica rispetto alle valutazioni effettuate con la fonte luminosa posta a $30^\circ$. Questa politica fornisce priorità agli oggetti in vista, non considerando l'impatto percettivo delle ombre portate causate da istanze di asset non in vista. Questo implica che aumentare o diminuire la quantità di ombre portate generate da asset non in vista, causato da una modifica dell'angolo di provenienza della luce, implica una diversa efficacia della strategia.
            Si osserva che il caso ideale è quello in cui la fonte luminosa splenda verso la camera, cosicché gli asset generatori di ombre portate siano principalmente quelli in vista. 

            \begin{figure}[htb!]
                \centering
                %% Creator: Matplotlib, PGF backend
%%
%% To include the figure in your LaTeX document, write
%%   \input{<filename>.pgf}
%%
%% Make sure the required packages are loaded in your preamble
%%   \usepackage{pgf}
%%
%% Figures using additional raster images can only be included by \input if
%% they are in the same directory as the main LaTeX file. For loading figures
%% from other directories you can use the `import` package
%%   \usepackage{import}
%%
%% and then include the figures with
%%   \import{<path to file>}{<filename>.pgf}
%%
%% Matplotlib used the following preamble
%%
\begingroup%
\makeatletter%
\begin{pgfpicture}%
\pgfpathrectangle{\pgfpointorigin}{\pgfqpoint{5.900000in}{3.400000in}}%
\pgfusepath{use as bounding box, clip}%
\begin{pgfscope}%
\pgfsetbuttcap%
\pgfsetmiterjoin%
\definecolor{currentfill}{rgb}{1.000000,1.000000,1.000000}%
\pgfsetfillcolor{currentfill}%
\pgfsetlinewidth{0.000000pt}%
\definecolor{currentstroke}{rgb}{1.000000,1.000000,1.000000}%
\pgfsetstrokecolor{currentstroke}%
\pgfsetdash{}{0pt}%
\pgfpathmoveto{\pgfqpoint{0.000000in}{0.000000in}}%
\pgfpathlineto{\pgfqpoint{5.900000in}{0.000000in}}%
\pgfpathlineto{\pgfqpoint{5.900000in}{3.400000in}}%
\pgfpathlineto{\pgfqpoint{0.000000in}{3.400000in}}%
\pgfpathclose%
\pgfusepath{fill}%
\end{pgfscope}%
\begin{pgfscope}%
\pgfsetbuttcap%
\pgfsetmiterjoin%
\definecolor{currentfill}{rgb}{1.000000,1.000000,1.000000}%
\pgfsetfillcolor{currentfill}%
\pgfsetlinewidth{0.000000pt}%
\definecolor{currentstroke}{rgb}{0.000000,0.000000,0.000000}%
\pgfsetstrokecolor{currentstroke}%
\pgfsetstrokeopacity{0.000000}%
\pgfsetdash{}{0pt}%
\pgfpathmoveto{\pgfqpoint{0.565124in}{0.565123in}}%
\pgfpathlineto{\pgfqpoint{5.750000in}{0.565123in}}%
\pgfpathlineto{\pgfqpoint{5.750000in}{3.250000in}}%
\pgfpathlineto{\pgfqpoint{0.565124in}{3.250000in}}%
\pgfpathclose%
\pgfusepath{fill}%
\end{pgfscope}%
\begin{pgfscope}%
\pgfsetbuttcap%
\pgfsetroundjoin%
\definecolor{currentfill}{rgb}{0.000000,0.000000,0.000000}%
\pgfsetfillcolor{currentfill}%
\pgfsetlinewidth{0.803000pt}%
\definecolor{currentstroke}{rgb}{0.000000,0.000000,0.000000}%
\pgfsetstrokecolor{currentstroke}%
\pgfsetdash{}{0pt}%
\pgfsys@defobject{currentmarker}{\pgfqpoint{0.000000in}{-0.048611in}}{\pgfqpoint{0.000000in}{0.000000in}}{%
\pgfpathmoveto{\pgfqpoint{0.000000in}{0.000000in}}%
\pgfpathlineto{\pgfqpoint{0.000000in}{-0.048611in}}%
\pgfusepath{stroke,fill}%
}%
\begin{pgfscope}%
\pgfsys@transformshift{0.800800in}{0.565123in}%
\pgfsys@useobject{currentmarker}{}%
\end{pgfscope}%
\end{pgfscope}%
\begin{pgfscope}%
\definecolor{textcolor}{rgb}{0.000000,0.000000,0.000000}%
\pgfsetstrokecolor{textcolor}%
\pgfsetfillcolor{textcolor}%
\pgftext[x=0.800800in,y=0.467901in,,top]{\color{textcolor}\rmfamily\fontsize{10.000000}{12.000000}\selectfont \(\displaystyle {0}\)}%
\end{pgfscope}%
\begin{pgfscope}%
\pgfsetbuttcap%
\pgfsetroundjoin%
\definecolor{currentfill}{rgb}{0.000000,0.000000,0.000000}%
\pgfsetfillcolor{currentfill}%
\pgfsetlinewidth{0.803000pt}%
\definecolor{currentstroke}{rgb}{0.000000,0.000000,0.000000}%
\pgfsetstrokecolor{currentstroke}%
\pgfsetdash{}{0pt}%
\pgfsys@defobject{currentmarker}{\pgfqpoint{0.000000in}{-0.048611in}}{\pgfqpoint{0.000000in}{0.000000in}}{%
\pgfpathmoveto{\pgfqpoint{0.000000in}{0.000000in}}%
\pgfpathlineto{\pgfqpoint{0.000000in}{-0.048611in}}%
\pgfusepath{stroke,fill}%
}%
\begin{pgfscope}%
\pgfsys@transformshift{1.799428in}{0.565123in}%
\pgfsys@useobject{currentmarker}{}%
\end{pgfscope}%
\end{pgfscope}%
\begin{pgfscope}%
\definecolor{textcolor}{rgb}{0.000000,0.000000,0.000000}%
\pgfsetstrokecolor{textcolor}%
\pgfsetfillcolor{textcolor}%
\pgftext[x=1.799428in,y=0.467901in,,top]{\color{textcolor}\rmfamily\fontsize{10.000000}{12.000000}\selectfont \(\displaystyle {50}\)}%
\end{pgfscope}%
\begin{pgfscope}%
\pgfsetbuttcap%
\pgfsetroundjoin%
\definecolor{currentfill}{rgb}{0.000000,0.000000,0.000000}%
\pgfsetfillcolor{currentfill}%
\pgfsetlinewidth{0.803000pt}%
\definecolor{currentstroke}{rgb}{0.000000,0.000000,0.000000}%
\pgfsetstrokecolor{currentstroke}%
\pgfsetdash{}{0pt}%
\pgfsys@defobject{currentmarker}{\pgfqpoint{0.000000in}{-0.048611in}}{\pgfqpoint{0.000000in}{0.000000in}}{%
\pgfpathmoveto{\pgfqpoint{0.000000in}{0.000000in}}%
\pgfpathlineto{\pgfqpoint{0.000000in}{-0.048611in}}%
\pgfusepath{stroke,fill}%
}%
\begin{pgfscope}%
\pgfsys@transformshift{2.798056in}{0.565123in}%
\pgfsys@useobject{currentmarker}{}%
\end{pgfscope}%
\end{pgfscope}%
\begin{pgfscope}%
\definecolor{textcolor}{rgb}{0.000000,0.000000,0.000000}%
\pgfsetstrokecolor{textcolor}%
\pgfsetfillcolor{textcolor}%
\pgftext[x=2.798056in,y=0.467901in,,top]{\color{textcolor}\rmfamily\fontsize{10.000000}{12.000000}\selectfont \(\displaystyle {100}\)}%
\end{pgfscope}%
\begin{pgfscope}%
\pgfsetbuttcap%
\pgfsetroundjoin%
\definecolor{currentfill}{rgb}{0.000000,0.000000,0.000000}%
\pgfsetfillcolor{currentfill}%
\pgfsetlinewidth{0.803000pt}%
\definecolor{currentstroke}{rgb}{0.000000,0.000000,0.000000}%
\pgfsetstrokecolor{currentstroke}%
\pgfsetdash{}{0pt}%
\pgfsys@defobject{currentmarker}{\pgfqpoint{0.000000in}{-0.048611in}}{\pgfqpoint{0.000000in}{0.000000in}}{%
\pgfpathmoveto{\pgfqpoint{0.000000in}{0.000000in}}%
\pgfpathlineto{\pgfqpoint{0.000000in}{-0.048611in}}%
\pgfusepath{stroke,fill}%
}%
\begin{pgfscope}%
\pgfsys@transformshift{3.796684in}{0.565123in}%
\pgfsys@useobject{currentmarker}{}%
\end{pgfscope}%
\end{pgfscope}%
\begin{pgfscope}%
\definecolor{textcolor}{rgb}{0.000000,0.000000,0.000000}%
\pgfsetstrokecolor{textcolor}%
\pgfsetfillcolor{textcolor}%
\pgftext[x=3.796684in,y=0.467901in,,top]{\color{textcolor}\rmfamily\fontsize{10.000000}{12.000000}\selectfont \(\displaystyle {150}\)}%
\end{pgfscope}%
\begin{pgfscope}%
\pgfsetbuttcap%
\pgfsetroundjoin%
\definecolor{currentfill}{rgb}{0.000000,0.000000,0.000000}%
\pgfsetfillcolor{currentfill}%
\pgfsetlinewidth{0.803000pt}%
\definecolor{currentstroke}{rgb}{0.000000,0.000000,0.000000}%
\pgfsetstrokecolor{currentstroke}%
\pgfsetdash{}{0pt}%
\pgfsys@defobject{currentmarker}{\pgfqpoint{0.000000in}{-0.048611in}}{\pgfqpoint{0.000000in}{0.000000in}}{%
\pgfpathmoveto{\pgfqpoint{0.000000in}{0.000000in}}%
\pgfpathlineto{\pgfqpoint{0.000000in}{-0.048611in}}%
\pgfusepath{stroke,fill}%
}%
\begin{pgfscope}%
\pgfsys@transformshift{4.795312in}{0.565123in}%
\pgfsys@useobject{currentmarker}{}%
\end{pgfscope}%
\end{pgfscope}%
\begin{pgfscope}%
\definecolor{textcolor}{rgb}{0.000000,0.000000,0.000000}%
\pgfsetstrokecolor{textcolor}%
\pgfsetfillcolor{textcolor}%
\pgftext[x=4.795312in,y=0.467901in,,top]{\color{textcolor}\rmfamily\fontsize{10.000000}{12.000000}\selectfont \(\displaystyle {200}\)}%
\end{pgfscope}%
\begin{pgfscope}%
\definecolor{textcolor}{rgb}{0.000000,0.000000,0.000000}%
\pgfsetstrokecolor{textcolor}%
\pgfsetfillcolor{textcolor}%
\pgftext[x=3.157562in,y=0.288889in,,top]{\color{textcolor}\rmfamily\fontsize{10.000000}{12.000000}\selectfont Tempo (Frame)}%
\end{pgfscope}%
\begin{pgfscope}%
\pgfsetbuttcap%
\pgfsetroundjoin%
\definecolor{currentfill}{rgb}{0.000000,0.000000,0.000000}%
\pgfsetfillcolor{currentfill}%
\pgfsetlinewidth{0.803000pt}%
\definecolor{currentstroke}{rgb}{0.000000,0.000000,0.000000}%
\pgfsetstrokecolor{currentstroke}%
\pgfsetdash{}{0pt}%
\pgfsys@defobject{currentmarker}{\pgfqpoint{-0.048611in}{0.000000in}}{\pgfqpoint{-0.000000in}{0.000000in}}{%
\pgfpathmoveto{\pgfqpoint{-0.000000in}{0.000000in}}%
\pgfpathlineto{\pgfqpoint{-0.048611in}{0.000000in}}%
\pgfusepath{stroke,fill}%
}%
\begin{pgfscope}%
\pgfsys@transformshift{0.565124in}{0.687163in}%
\pgfsys@useobject{currentmarker}{}%
\end{pgfscope}%
\end{pgfscope}%
\begin{pgfscope}%
\definecolor{textcolor}{rgb}{0.000000,0.000000,0.000000}%
\pgfsetstrokecolor{textcolor}%
\pgfsetfillcolor{textcolor}%
\pgftext[x=0.398457in, y=0.638938in, left, base]{\color{textcolor}\rmfamily\fontsize{10.000000}{12.000000}\selectfont \(\displaystyle {0}\)}%
\end{pgfscope}%
\begin{pgfscope}%
\pgfsetbuttcap%
\pgfsetroundjoin%
\definecolor{currentfill}{rgb}{0.000000,0.000000,0.000000}%
\pgfsetfillcolor{currentfill}%
\pgfsetlinewidth{0.803000pt}%
\definecolor{currentstroke}{rgb}{0.000000,0.000000,0.000000}%
\pgfsetstrokecolor{currentstroke}%
\pgfsetdash{}{0pt}%
\pgfsys@defobject{currentmarker}{\pgfqpoint{-0.048611in}{0.000000in}}{\pgfqpoint{-0.000000in}{0.000000in}}{%
\pgfpathmoveto{\pgfqpoint{-0.000000in}{0.000000in}}%
\pgfpathlineto{\pgfqpoint{-0.048611in}{0.000000in}}%
\pgfusepath{stroke,fill}%
}%
\begin{pgfscope}%
\pgfsys@transformshift{0.565124in}{1.221913in}%
\pgfsys@useobject{currentmarker}{}%
\end{pgfscope}%
\end{pgfscope}%
\begin{pgfscope}%
\definecolor{textcolor}{rgb}{0.000000,0.000000,0.000000}%
\pgfsetstrokecolor{textcolor}%
\pgfsetfillcolor{textcolor}%
\pgftext[x=0.329012in, y=1.173687in, left, base]{\color{textcolor}\rmfamily\fontsize{10.000000}{12.000000}\selectfont \(\displaystyle {20}\)}%
\end{pgfscope}%
\begin{pgfscope}%
\pgfsetbuttcap%
\pgfsetroundjoin%
\definecolor{currentfill}{rgb}{0.000000,0.000000,0.000000}%
\pgfsetfillcolor{currentfill}%
\pgfsetlinewidth{0.803000pt}%
\definecolor{currentstroke}{rgb}{0.000000,0.000000,0.000000}%
\pgfsetstrokecolor{currentstroke}%
\pgfsetdash{}{0pt}%
\pgfsys@defobject{currentmarker}{\pgfqpoint{-0.048611in}{0.000000in}}{\pgfqpoint{-0.000000in}{0.000000in}}{%
\pgfpathmoveto{\pgfqpoint{-0.000000in}{0.000000in}}%
\pgfpathlineto{\pgfqpoint{-0.048611in}{0.000000in}}%
\pgfusepath{stroke,fill}%
}%
\begin{pgfscope}%
\pgfsys@transformshift{0.565124in}{1.756662in}%
\pgfsys@useobject{currentmarker}{}%
\end{pgfscope}%
\end{pgfscope}%
\begin{pgfscope}%
\definecolor{textcolor}{rgb}{0.000000,0.000000,0.000000}%
\pgfsetstrokecolor{textcolor}%
\pgfsetfillcolor{textcolor}%
\pgftext[x=0.329012in, y=1.708437in, left, base]{\color{textcolor}\rmfamily\fontsize{10.000000}{12.000000}\selectfont \(\displaystyle {40}\)}%
\end{pgfscope}%
\begin{pgfscope}%
\pgfsetbuttcap%
\pgfsetroundjoin%
\definecolor{currentfill}{rgb}{0.000000,0.000000,0.000000}%
\pgfsetfillcolor{currentfill}%
\pgfsetlinewidth{0.803000pt}%
\definecolor{currentstroke}{rgb}{0.000000,0.000000,0.000000}%
\pgfsetstrokecolor{currentstroke}%
\pgfsetdash{}{0pt}%
\pgfsys@defobject{currentmarker}{\pgfqpoint{-0.048611in}{0.000000in}}{\pgfqpoint{-0.000000in}{0.000000in}}{%
\pgfpathmoveto{\pgfqpoint{-0.000000in}{0.000000in}}%
\pgfpathlineto{\pgfqpoint{-0.048611in}{0.000000in}}%
\pgfusepath{stroke,fill}%
}%
\begin{pgfscope}%
\pgfsys@transformshift{0.565124in}{2.291411in}%
\pgfsys@useobject{currentmarker}{}%
\end{pgfscope}%
\end{pgfscope}%
\begin{pgfscope}%
\definecolor{textcolor}{rgb}{0.000000,0.000000,0.000000}%
\pgfsetstrokecolor{textcolor}%
\pgfsetfillcolor{textcolor}%
\pgftext[x=0.329012in, y=2.243186in, left, base]{\color{textcolor}\rmfamily\fontsize{10.000000}{12.000000}\selectfont \(\displaystyle {60}\)}%
\end{pgfscope}%
\begin{pgfscope}%
\pgfsetbuttcap%
\pgfsetroundjoin%
\definecolor{currentfill}{rgb}{0.000000,0.000000,0.000000}%
\pgfsetfillcolor{currentfill}%
\pgfsetlinewidth{0.803000pt}%
\definecolor{currentstroke}{rgb}{0.000000,0.000000,0.000000}%
\pgfsetstrokecolor{currentstroke}%
\pgfsetdash{}{0pt}%
\pgfsys@defobject{currentmarker}{\pgfqpoint{-0.048611in}{0.000000in}}{\pgfqpoint{-0.000000in}{0.000000in}}{%
\pgfpathmoveto{\pgfqpoint{-0.000000in}{0.000000in}}%
\pgfpathlineto{\pgfqpoint{-0.048611in}{0.000000in}}%
\pgfusepath{stroke,fill}%
}%
\begin{pgfscope}%
\pgfsys@transformshift{0.565124in}{2.826161in}%
\pgfsys@useobject{currentmarker}{}%
\end{pgfscope}%
\end{pgfscope}%
\begin{pgfscope}%
\definecolor{textcolor}{rgb}{0.000000,0.000000,0.000000}%
\pgfsetstrokecolor{textcolor}%
\pgfsetfillcolor{textcolor}%
\pgftext[x=0.329012in, y=2.777936in, left, base]{\color{textcolor}\rmfamily\fontsize{10.000000}{12.000000}\selectfont \(\displaystyle {80}\)}%
\end{pgfscope}%
\begin{pgfscope}%
\definecolor{textcolor}{rgb}{0.000000,0.000000,0.000000}%
\pgfsetstrokecolor{textcolor}%
\pgfsetfillcolor{textcolor}%
\pgftext[x=0.273457in,y=1.907562in,,bottom,rotate=90.000000]{\color{textcolor}\rmfamily\fontsize{10.000000}{12.000000}\selectfont Differenza percettiva}%
\end{pgfscope}%
\begin{pgfscope}%
\pgfpathrectangle{\pgfqpoint{0.565124in}{0.565123in}}{\pgfqpoint{5.184876in}{2.684877in}}%
\pgfusepath{clip}%
\pgfsetrectcap%
\pgfsetroundjoin%
\pgfsetlinewidth{2.007500pt}%
\definecolor{currentstroke}{rgb}{0.121569,0.466667,0.705882}%
\pgfsetstrokecolor{currentstroke}%
\pgfsetdash{}{0pt}%
\pgfpathmoveto{\pgfqpoint{0.800800in}{1.536914in}}%
\pgfpathlineto{\pgfqpoint{0.820773in}{1.498452in}}%
\pgfpathlineto{\pgfqpoint{0.840745in}{1.445509in}}%
\pgfpathlineto{\pgfqpoint{0.860718in}{1.416978in}}%
\pgfpathlineto{\pgfqpoint{0.900663in}{1.362714in}}%
\pgfpathlineto{\pgfqpoint{0.920635in}{1.335696in}}%
\pgfpathlineto{\pgfqpoint{0.940608in}{1.313683in}}%
\pgfpathlineto{\pgfqpoint{0.980553in}{1.286399in}}%
\pgfpathlineto{\pgfqpoint{1.000526in}{1.272161in}}%
\pgfpathlineto{\pgfqpoint{1.020498in}{1.250083in}}%
\pgfpathlineto{\pgfqpoint{1.040471in}{1.155334in}}%
\pgfpathlineto{\pgfqpoint{1.060443in}{1.143916in}}%
\pgfpathlineto{\pgfqpoint{1.080416in}{1.131025in}}%
\pgfpathlineto{\pgfqpoint{1.100388in}{1.123360in}}%
\pgfpathlineto{\pgfqpoint{1.120361in}{1.110800in}}%
\pgfpathlineto{\pgfqpoint{1.140333in}{1.109269in}}%
\pgfpathlineto{\pgfqpoint{1.160306in}{1.106322in}}%
\pgfpathlineto{\pgfqpoint{1.260169in}{1.106011in}}%
\pgfpathlineto{\pgfqpoint{1.280141in}{1.039467in}}%
\pgfpathlineto{\pgfqpoint{1.300114in}{0.987513in}}%
\pgfpathlineto{\pgfqpoint{1.320086in}{0.982331in}}%
\pgfpathlineto{\pgfqpoint{1.340059in}{0.981201in}}%
\pgfpathlineto{\pgfqpoint{1.360032in}{0.966721in}}%
\pgfpathlineto{\pgfqpoint{1.419949in}{0.963916in}}%
\pgfpathlineto{\pgfqpoint{1.439922in}{0.963916in}}%
\pgfpathlineto{\pgfqpoint{1.459894in}{0.955330in}}%
\pgfpathlineto{\pgfqpoint{1.479867in}{0.948330in}}%
\pgfpathlineto{\pgfqpoint{1.499839in}{0.948330in}}%
\pgfpathlineto{\pgfqpoint{1.519812in}{0.938640in}}%
\pgfpathlineto{\pgfqpoint{1.539785in}{0.937747in}}%
\pgfpathlineto{\pgfqpoint{1.559757in}{0.919893in}}%
\pgfpathlineto{\pgfqpoint{1.579730in}{0.919893in}}%
\pgfpathlineto{\pgfqpoint{1.599702in}{0.918083in}}%
\pgfpathlineto{\pgfqpoint{1.619675in}{0.902027in}}%
\pgfpathlineto{\pgfqpoint{1.659620in}{0.899402in}}%
\pgfpathlineto{\pgfqpoint{1.719538in}{0.895249in}}%
\pgfpathlineto{\pgfqpoint{1.759483in}{0.894098in}}%
\pgfpathlineto{\pgfqpoint{1.779455in}{0.867288in}}%
\pgfpathlineto{\pgfqpoint{1.799428in}{0.863949in}}%
\pgfpathlineto{\pgfqpoint{1.819400in}{0.855748in}}%
\pgfpathlineto{\pgfqpoint{1.839373in}{0.850837in}}%
\pgfpathlineto{\pgfqpoint{1.859346in}{0.850836in}}%
\pgfpathlineto{\pgfqpoint{1.879318in}{0.818754in}}%
\pgfpathlineto{\pgfqpoint{1.899291in}{0.818394in}}%
\pgfpathlineto{\pgfqpoint{1.919263in}{0.783566in}}%
\pgfpathlineto{\pgfqpoint{1.939236in}{0.783328in}}%
\pgfpathlineto{\pgfqpoint{1.959208in}{0.779239in}}%
\pgfpathlineto{\pgfqpoint{1.979181in}{0.779233in}}%
\pgfpathlineto{\pgfqpoint{1.999153in}{0.777234in}}%
\pgfpathlineto{\pgfqpoint{2.039099in}{0.770618in}}%
\pgfpathlineto{\pgfqpoint{2.059071in}{0.769351in}}%
\pgfpathlineto{\pgfqpoint{2.079044in}{0.761254in}}%
\pgfpathlineto{\pgfqpoint{2.099016in}{0.758948in}}%
\pgfpathlineto{\pgfqpoint{2.178906in}{0.755026in}}%
\pgfpathlineto{\pgfqpoint{2.218852in}{0.754850in}}%
\pgfpathlineto{\pgfqpoint{2.238824in}{0.752925in}}%
\pgfpathlineto{\pgfqpoint{2.258797in}{0.752683in}}%
\pgfpathlineto{\pgfqpoint{2.298742in}{0.743089in}}%
\pgfpathlineto{\pgfqpoint{2.318714in}{0.739841in}}%
\pgfpathlineto{\pgfqpoint{2.338687in}{0.714668in}}%
\pgfpathlineto{\pgfqpoint{2.398605in}{0.713730in}}%
\pgfpathlineto{\pgfqpoint{2.418577in}{0.711910in}}%
\pgfpathlineto{\pgfqpoint{2.458522in}{0.711256in}}%
\pgfpathlineto{\pgfqpoint{2.478495in}{0.709677in}}%
\pgfpathlineto{\pgfqpoint{2.538413in}{0.709209in}}%
\pgfpathlineto{\pgfqpoint{2.658248in}{0.708894in}}%
\pgfpathlineto{\pgfqpoint{2.718166in}{0.708692in}}%
\pgfpathlineto{\pgfqpoint{2.738138in}{0.698363in}}%
\pgfpathlineto{\pgfqpoint{2.798056in}{0.697489in}}%
\pgfpathlineto{\pgfqpoint{2.857973in}{0.697082in}}%
\pgfpathlineto{\pgfqpoint{3.017754in}{0.696709in}}%
\pgfpathlineto{\pgfqpoint{3.037727in}{0.689823in}}%
\pgfpathlineto{\pgfqpoint{3.077672in}{0.689771in}}%
\pgfpathlineto{\pgfqpoint{3.097644in}{0.688638in}}%
\pgfpathlineto{\pgfqpoint{3.177534in}{0.687884in}}%
\pgfpathlineto{\pgfqpoint{3.217480in}{0.687165in}}%
\pgfpathlineto{\pgfqpoint{5.494351in}{0.687163in}}%
\pgfpathlineto{\pgfqpoint{5.494351in}{0.687163in}}%
\pgfusepath{stroke}%
\end{pgfscope}%
\begin{pgfscope}%
\pgfpathrectangle{\pgfqpoint{0.565124in}{0.565123in}}{\pgfqpoint{5.184876in}{2.684877in}}%
\pgfusepath{clip}%
\pgfsetrectcap%
\pgfsetroundjoin%
\pgfsetlinewidth{2.007500pt}%
\definecolor{currentstroke}{rgb}{1.000000,0.498039,0.054902}%
\pgfsetstrokecolor{currentstroke}%
\pgfsetdash{}{0pt}%
\pgfpathmoveto{\pgfqpoint{0.820773in}{3.127957in}}%
\pgfpathlineto{\pgfqpoint{0.880690in}{3.127960in}}%
\pgfpathlineto{\pgfqpoint{0.900663in}{3.059568in}}%
\pgfpathlineto{\pgfqpoint{1.160306in}{3.059572in}}%
\pgfpathlineto{\pgfqpoint{1.180279in}{3.036434in}}%
\pgfpathlineto{\pgfqpoint{1.360032in}{3.036438in}}%
\pgfpathlineto{\pgfqpoint{1.380004in}{3.022542in}}%
\pgfpathlineto{\pgfqpoint{1.459894in}{3.022543in}}%
\pgfpathlineto{\pgfqpoint{1.479867in}{3.018213in}}%
\pgfpathlineto{\pgfqpoint{1.499839in}{3.009949in}}%
\pgfpathlineto{\pgfqpoint{1.519812in}{3.009771in}}%
\pgfpathlineto{\pgfqpoint{1.559757in}{2.968472in}}%
\pgfpathlineto{\pgfqpoint{1.639647in}{2.968468in}}%
\pgfpathlineto{\pgfqpoint{1.659620in}{2.956617in}}%
\pgfpathlineto{\pgfqpoint{1.919263in}{2.956599in}}%
\pgfpathlineto{\pgfqpoint{1.939236in}{2.952700in}}%
\pgfpathlineto{\pgfqpoint{2.019126in}{2.952685in}}%
\pgfpathlineto{\pgfqpoint{2.039099in}{2.652357in}}%
\pgfpathlineto{\pgfqpoint{2.138961in}{2.652347in}}%
\pgfpathlineto{\pgfqpoint{2.158934in}{2.596026in}}%
\pgfpathlineto{\pgfqpoint{2.258797in}{2.595356in}}%
\pgfpathlineto{\pgfqpoint{2.278769in}{2.593840in}}%
\pgfpathlineto{\pgfqpoint{2.298742in}{2.098954in}}%
\pgfpathlineto{\pgfqpoint{2.358660in}{2.098327in}}%
\pgfpathlineto{\pgfqpoint{2.378632in}{2.085750in}}%
\pgfpathlineto{\pgfqpoint{2.398605in}{2.081972in}}%
\pgfpathlineto{\pgfqpoint{2.418577in}{2.080180in}}%
\pgfpathlineto{\pgfqpoint{2.438550in}{2.052169in}}%
\pgfpathlineto{\pgfqpoint{2.458522in}{1.901880in}}%
\pgfpathlineto{\pgfqpoint{2.478495in}{1.885647in}}%
\pgfpathlineto{\pgfqpoint{2.498467in}{1.885646in}}%
\pgfpathlineto{\pgfqpoint{2.518440in}{1.884284in}}%
\pgfpathlineto{\pgfqpoint{2.538413in}{1.881311in}}%
\pgfpathlineto{\pgfqpoint{2.558385in}{1.876639in}}%
\pgfpathlineto{\pgfqpoint{2.578358in}{1.875546in}}%
\pgfpathlineto{\pgfqpoint{2.598330in}{1.869515in}}%
\pgfpathlineto{\pgfqpoint{2.618303in}{1.866576in}}%
\pgfpathlineto{\pgfqpoint{2.638275in}{1.356323in}}%
\pgfpathlineto{\pgfqpoint{2.658248in}{1.340063in}}%
\pgfpathlineto{\pgfqpoint{2.678220in}{1.327478in}}%
\pgfpathlineto{\pgfqpoint{2.718166in}{1.327480in}}%
\pgfpathlineto{\pgfqpoint{2.738138in}{1.020725in}}%
\pgfpathlineto{\pgfqpoint{2.778083in}{1.019958in}}%
\pgfpathlineto{\pgfqpoint{2.857973in}{1.019952in}}%
\pgfpathlineto{\pgfqpoint{2.877946in}{0.921185in}}%
\pgfpathlineto{\pgfqpoint{2.937864in}{0.921188in}}%
\pgfpathlineto{\pgfqpoint{2.957836in}{0.917184in}}%
\pgfpathlineto{\pgfqpoint{3.257425in}{0.917181in}}%
\pgfpathlineto{\pgfqpoint{3.277397in}{0.915182in}}%
\pgfpathlineto{\pgfqpoint{3.477123in}{0.915184in}}%
\pgfpathlineto{\pgfqpoint{3.497095in}{0.913304in}}%
\pgfpathlineto{\pgfqpoint{3.517068in}{0.913306in}}%
\pgfpathlineto{\pgfqpoint{3.537040in}{0.904606in}}%
\pgfpathlineto{\pgfqpoint{3.576986in}{0.903552in}}%
\pgfpathlineto{\pgfqpoint{3.596958in}{0.839624in}}%
\pgfpathlineto{\pgfqpoint{3.616931in}{0.839623in}}%
\pgfpathlineto{\pgfqpoint{3.636903in}{0.837014in}}%
\pgfpathlineto{\pgfqpoint{3.656876in}{0.810547in}}%
\pgfpathlineto{\pgfqpoint{3.676848in}{0.802568in}}%
\pgfpathlineto{\pgfqpoint{3.696821in}{0.765721in}}%
\pgfpathlineto{\pgfqpoint{3.796684in}{0.765572in}}%
\pgfpathlineto{\pgfqpoint{3.816656in}{0.754227in}}%
\pgfpathlineto{\pgfqpoint{3.836629in}{0.753055in}}%
\pgfpathlineto{\pgfqpoint{3.856601in}{0.746090in}}%
\pgfpathlineto{\pgfqpoint{3.876574in}{0.726753in}}%
\pgfpathlineto{\pgfqpoint{3.896547in}{0.726218in}}%
\pgfpathlineto{\pgfqpoint{3.916519in}{0.719408in}}%
\pgfpathlineto{\pgfqpoint{3.936492in}{0.715859in}}%
\pgfpathlineto{\pgfqpoint{3.956464in}{0.714093in}}%
\pgfpathlineto{\pgfqpoint{3.976437in}{0.711172in}}%
\pgfpathlineto{\pgfqpoint{4.375888in}{0.711163in}}%
\pgfpathlineto{\pgfqpoint{4.395861in}{0.687170in}}%
\pgfpathlineto{\pgfqpoint{5.514324in}{0.687163in}}%
\pgfpathlineto{\pgfqpoint{5.514324in}{0.687163in}}%
\pgfusepath{stroke}%
\end{pgfscope}%
\begin{pgfscope}%
\pgfsetrectcap%
\pgfsetmiterjoin%
\pgfsetlinewidth{0.803000pt}%
\definecolor{currentstroke}{rgb}{0.000000,0.000000,0.000000}%
\pgfsetstrokecolor{currentstroke}%
\pgfsetdash{}{0pt}%
\pgfpathmoveto{\pgfqpoint{0.565124in}{0.565123in}}%
\pgfpathlineto{\pgfqpoint{0.565124in}{3.250000in}}%
\pgfusepath{stroke}%
\end{pgfscope}%
\begin{pgfscope}%
\pgfsetrectcap%
\pgfsetmiterjoin%
\pgfsetlinewidth{0.803000pt}%
\definecolor{currentstroke}{rgb}{0.000000,0.000000,0.000000}%
\pgfsetstrokecolor{currentstroke}%
\pgfsetdash{}{0pt}%
\pgfpathmoveto{\pgfqpoint{5.750000in}{0.565123in}}%
\pgfpathlineto{\pgfqpoint{5.750000in}{3.250000in}}%
\pgfusepath{stroke}%
\end{pgfscope}%
\begin{pgfscope}%
\pgfsetrectcap%
\pgfsetmiterjoin%
\pgfsetlinewidth{0.803000pt}%
\definecolor{currentstroke}{rgb}{0.000000,0.000000,0.000000}%
\pgfsetstrokecolor{currentstroke}%
\pgfsetdash{}{0pt}%
\pgfpathmoveto{\pgfqpoint{0.565124in}{0.565123in}}%
\pgfpathlineto{\pgfqpoint{5.750000in}{0.565123in}}%
\pgfusepath{stroke}%
\end{pgfscope}%
\begin{pgfscope}%
\pgfsetrectcap%
\pgfsetmiterjoin%
\pgfsetlinewidth{0.803000pt}%
\definecolor{currentstroke}{rgb}{0.000000,0.000000,0.000000}%
\pgfsetstrokecolor{currentstroke}%
\pgfsetdash{}{0pt}%
\pgfpathmoveto{\pgfqpoint{0.565124in}{3.250000in}}%
\pgfpathlineto{\pgfqpoint{5.750000in}{3.250000in}}%
\pgfusepath{stroke}%
\end{pgfscope}%
\begin{pgfscope}%
\pgfsetbuttcap%
\pgfsetmiterjoin%
\definecolor{currentfill}{rgb}{1.000000,1.000000,1.000000}%
\pgfsetfillcolor{currentfill}%
\pgfsetfillopacity{0.800000}%
\pgfsetlinewidth{1.003750pt}%
\definecolor{currentstroke}{rgb}{0.800000,0.800000,0.800000}%
\pgfsetstrokecolor{currentstroke}%
\pgfsetstrokeopacity{0.800000}%
\pgfsetdash{}{0pt}%
\pgfpathmoveto{\pgfqpoint{4.273532in}{2.751543in}}%
\pgfpathlineto{\pgfqpoint{5.652778in}{2.751543in}}%
\pgfpathquadraticcurveto{\pgfqpoint{5.680556in}{2.751543in}}{\pgfqpoint{5.680556in}{2.779321in}}%
\pgfpathlineto{\pgfqpoint{5.680556in}{3.152778in}}%
\pgfpathquadraticcurveto{\pgfqpoint{5.680556in}{3.180556in}}{\pgfqpoint{5.652778in}{3.180556in}}%
\pgfpathlineto{\pgfqpoint{4.273532in}{3.180556in}}%
\pgfpathquadraticcurveto{\pgfqpoint{4.245755in}{3.180556in}}{\pgfqpoint{4.245755in}{3.152778in}}%
\pgfpathlineto{\pgfqpoint{4.245755in}{2.779321in}}%
\pgfpathquadraticcurveto{\pgfqpoint{4.245755in}{2.751543in}}{\pgfqpoint{4.273532in}{2.751543in}}%
\pgfpathclose%
\pgfusepath{stroke,fill}%
\end{pgfscope}%
\begin{pgfscope}%
\pgfsetrectcap%
\pgfsetroundjoin%
\pgfsetlinewidth{2.007500pt}%
\definecolor{currentstroke}{rgb}{0.121569,0.466667,0.705882}%
\pgfsetstrokecolor{currentstroke}%
\pgfsetdash{}{0pt}%
\pgfpathmoveto{\pgfqpoint{4.301310in}{3.076389in}}%
\pgfpathlineto{\pgfqpoint{4.579088in}{3.076389in}}%
\pgfusepath{stroke}%
\end{pgfscope}%
\begin{pgfscope}%
\definecolor{textcolor}{rgb}{0.000000,0.000000,0.000000}%
\pgfsetstrokecolor{textcolor}%
\pgfsetfillcolor{textcolor}%
\pgftext[x=4.690199in,y=3.027778in,left,base]{\color{textcolor}\rmfamily\fontsize{10.000000}{12.000000}\selectfont Scena fittizia}%
\end{pgfscope}%
\begin{pgfscope}%
\pgfsetrectcap%
\pgfsetroundjoin%
\pgfsetlinewidth{2.007500pt}%
\definecolor{currentstroke}{rgb}{1.000000,0.498039,0.054902}%
\pgfsetstrokecolor{currentstroke}%
\pgfsetdash{}{0pt}%
\pgfpathmoveto{\pgfqpoint{4.301310in}{2.882716in}}%
\pgfpathlineto{\pgfqpoint{4.579088in}{2.882716in}}%
\pgfusepath{stroke}%
\end{pgfscope}%
\begin{pgfscope}%
\definecolor{textcolor}{rgb}{0.000000,0.000000,0.000000}%
\pgfsetstrokecolor{textcolor}%
\pgfsetfillcolor{textcolor}%
\pgftext[x=4.690199in,y=2.834105in,left,base]{\color{textcolor}\rmfamily\fontsize{10.000000}{12.000000}\selectfont Scena realistica}%
\end{pgfscope}%
\end{pgfpicture}%
\makeatother%
\endgroup%

                \caption{Valutazione Sphere Tracing valutato per la distanza}
                \label{fig:eval-stdist}
            \end{figure}

            \newpage
            \subsubsection{Valutazione della dimensione}
            Valutare le istanze in vista per dimensione porta un notevole guadagno nella scena realistica (Figura \ref{fig:eval-stdim}): del 28.3\% rispetto alla valutazione per distanza (e alla valutazione Closest-First on Camera) e del 58.38\% da Closest-First. Questo miglioramento era prevedibile dato che la scena realistica è composta da elementi di grandi dimensioni che occupano un'importante sezione della vista anche se posti distanti dalla camera. 
            
            È presente un guadagno anche nella scena fittizia ma, dato che le istanze di asset che la popolano presentano una maggiore uniformità in dimensione, fatte poche eccezioni, il guadagno è minore, del 5.7\% rispetto alla valutazione per distanza, del 11.5\% dalla strategia Closest-First on camera e del 63.12\% da Closest-First.
            %discussione riguardo gli angoli della luce
                % \begin{figure}[htb!]
                %     \centering
                %     \includegraphics[scale=.5]{images/valutazioni/fittizia/28-04-23 04-20-50 SphereTracingSizePriority 809.5422230853329.png}
                %     \caption{Valutazione Sphere Tracing valutato per la dimensione sulla scena fittizia}
                %     \label{fig:eval-stdim-fit}
                % \end{figure}
    
                % \begin{figure}[htb!]
                %     \centering
                %     \includegraphics[scale=.5]{images/valutazioni/realistica/28-04-23 05-01-17 SphereTracingSizePriority 5462.929151987784.png}
                %     \caption{Valutazione Sphere Tracing valutato per la dimensione sulla scena realistica}
                %     \label{fig:eval-stdim-re}
                % \end{figure}
            % \begin{figure}[htbp]
            %     \centering
            %     \includegraphics[width=\textwidth]{images/sequences/sequence-stdim-re.png}
            %     \par
            %     \vspace{15pt}
            %     \centering
            %     \includegraphics[width=\textwidth]{images/sequences/sequence-stdim-fit.png}
            %     \caption{Sequenza di carimento di Sphere Tracing valutato per la dimensione}
            %     \label{fig:seq-stdim}
            % \end{figure}
                
            \begin{figure}[htb!]
                \centering
                %% Creator: Matplotlib, PGF backend
%%
%% To include the figure in your LaTeX document, write
%%   \input{<filename>.pgf}
%%
%% Make sure the required packages are loaded in your preamble
%%   \usepackage{pgf}
%%
%% Figures using additional raster images can only be included by \input if
%% they are in the same directory as the main LaTeX file. For loading figures
%% from other directories you can use the `import` package
%%   \usepackage{import}
%%
%% and then include the figures with
%%   \import{<path to file>}{<filename>.pgf}
%%
%% Matplotlib used the following preamble
%%
\begingroup%
\makeatletter%
\begin{pgfpicture}%
\pgfpathrectangle{\pgfpointorigin}{\pgfqpoint{5.900000in}{3.400000in}}%
\pgfusepath{use as bounding box, clip}%
\begin{pgfscope}%
\pgfsetbuttcap%
\pgfsetmiterjoin%
\definecolor{currentfill}{rgb}{1.000000,1.000000,1.000000}%
\pgfsetfillcolor{currentfill}%
\pgfsetlinewidth{0.000000pt}%
\definecolor{currentstroke}{rgb}{1.000000,1.000000,1.000000}%
\pgfsetstrokecolor{currentstroke}%
\pgfsetdash{}{0pt}%
\pgfpathmoveto{\pgfqpoint{0.000000in}{0.000000in}}%
\pgfpathlineto{\pgfqpoint{5.900000in}{0.000000in}}%
\pgfpathlineto{\pgfqpoint{5.900000in}{3.400000in}}%
\pgfpathlineto{\pgfqpoint{0.000000in}{3.400000in}}%
\pgfpathclose%
\pgfusepath{fill}%
\end{pgfscope}%
\begin{pgfscope}%
\pgfsetbuttcap%
\pgfsetmiterjoin%
\definecolor{currentfill}{rgb}{1.000000,1.000000,1.000000}%
\pgfsetfillcolor{currentfill}%
\pgfsetlinewidth{0.000000pt}%
\definecolor{currentstroke}{rgb}{0.000000,0.000000,0.000000}%
\pgfsetstrokecolor{currentstroke}%
\pgfsetstrokeopacity{0.000000}%
\pgfsetdash{}{0pt}%
\pgfpathmoveto{\pgfqpoint{0.565124in}{0.565123in}}%
\pgfpathlineto{\pgfqpoint{5.750000in}{0.565123in}}%
\pgfpathlineto{\pgfqpoint{5.750000in}{3.250000in}}%
\pgfpathlineto{\pgfqpoint{0.565124in}{3.250000in}}%
\pgfpathclose%
\pgfusepath{fill}%
\end{pgfscope}%
\begin{pgfscope}%
\pgfsetbuttcap%
\pgfsetroundjoin%
\definecolor{currentfill}{rgb}{0.000000,0.000000,0.000000}%
\pgfsetfillcolor{currentfill}%
\pgfsetlinewidth{0.803000pt}%
\definecolor{currentstroke}{rgb}{0.000000,0.000000,0.000000}%
\pgfsetstrokecolor{currentstroke}%
\pgfsetdash{}{0pt}%
\pgfsys@defobject{currentmarker}{\pgfqpoint{0.000000in}{-0.048611in}}{\pgfqpoint{0.000000in}{0.000000in}}{%
\pgfpathmoveto{\pgfqpoint{0.000000in}{0.000000in}}%
\pgfpathlineto{\pgfqpoint{0.000000in}{-0.048611in}}%
\pgfusepath{stroke,fill}%
}%
\begin{pgfscope}%
\pgfsys@transformshift{0.800800in}{0.565123in}%
\pgfsys@useobject{currentmarker}{}%
\end{pgfscope}%
\end{pgfscope}%
\begin{pgfscope}%
\definecolor{textcolor}{rgb}{0.000000,0.000000,0.000000}%
\pgfsetstrokecolor{textcolor}%
\pgfsetfillcolor{textcolor}%
\pgftext[x=0.800800in,y=0.467901in,,top]{\color{textcolor}\rmfamily\fontsize{10.000000}{12.000000}\selectfont \(\displaystyle {0}\)}%
\end{pgfscope}%
\begin{pgfscope}%
\pgfsetbuttcap%
\pgfsetroundjoin%
\definecolor{currentfill}{rgb}{0.000000,0.000000,0.000000}%
\pgfsetfillcolor{currentfill}%
\pgfsetlinewidth{0.803000pt}%
\definecolor{currentstroke}{rgb}{0.000000,0.000000,0.000000}%
\pgfsetstrokecolor{currentstroke}%
\pgfsetdash{}{0pt}%
\pgfsys@defobject{currentmarker}{\pgfqpoint{0.000000in}{-0.048611in}}{\pgfqpoint{0.000000in}{0.000000in}}{%
\pgfpathmoveto{\pgfqpoint{0.000000in}{0.000000in}}%
\pgfpathlineto{\pgfqpoint{0.000000in}{-0.048611in}}%
\pgfusepath{stroke,fill}%
}%
\begin{pgfscope}%
\pgfsys@transformshift{1.799428in}{0.565123in}%
\pgfsys@useobject{currentmarker}{}%
\end{pgfscope}%
\end{pgfscope}%
\begin{pgfscope}%
\definecolor{textcolor}{rgb}{0.000000,0.000000,0.000000}%
\pgfsetstrokecolor{textcolor}%
\pgfsetfillcolor{textcolor}%
\pgftext[x=1.799428in,y=0.467901in,,top]{\color{textcolor}\rmfamily\fontsize{10.000000}{12.000000}\selectfont \(\displaystyle {50}\)}%
\end{pgfscope}%
\begin{pgfscope}%
\pgfsetbuttcap%
\pgfsetroundjoin%
\definecolor{currentfill}{rgb}{0.000000,0.000000,0.000000}%
\pgfsetfillcolor{currentfill}%
\pgfsetlinewidth{0.803000pt}%
\definecolor{currentstroke}{rgb}{0.000000,0.000000,0.000000}%
\pgfsetstrokecolor{currentstroke}%
\pgfsetdash{}{0pt}%
\pgfsys@defobject{currentmarker}{\pgfqpoint{0.000000in}{-0.048611in}}{\pgfqpoint{0.000000in}{0.000000in}}{%
\pgfpathmoveto{\pgfqpoint{0.000000in}{0.000000in}}%
\pgfpathlineto{\pgfqpoint{0.000000in}{-0.048611in}}%
\pgfusepath{stroke,fill}%
}%
\begin{pgfscope}%
\pgfsys@transformshift{2.798056in}{0.565123in}%
\pgfsys@useobject{currentmarker}{}%
\end{pgfscope}%
\end{pgfscope}%
\begin{pgfscope}%
\definecolor{textcolor}{rgb}{0.000000,0.000000,0.000000}%
\pgfsetstrokecolor{textcolor}%
\pgfsetfillcolor{textcolor}%
\pgftext[x=2.798056in,y=0.467901in,,top]{\color{textcolor}\rmfamily\fontsize{10.000000}{12.000000}\selectfont \(\displaystyle {100}\)}%
\end{pgfscope}%
\begin{pgfscope}%
\pgfsetbuttcap%
\pgfsetroundjoin%
\definecolor{currentfill}{rgb}{0.000000,0.000000,0.000000}%
\pgfsetfillcolor{currentfill}%
\pgfsetlinewidth{0.803000pt}%
\definecolor{currentstroke}{rgb}{0.000000,0.000000,0.000000}%
\pgfsetstrokecolor{currentstroke}%
\pgfsetdash{}{0pt}%
\pgfsys@defobject{currentmarker}{\pgfqpoint{0.000000in}{-0.048611in}}{\pgfqpoint{0.000000in}{0.000000in}}{%
\pgfpathmoveto{\pgfqpoint{0.000000in}{0.000000in}}%
\pgfpathlineto{\pgfqpoint{0.000000in}{-0.048611in}}%
\pgfusepath{stroke,fill}%
}%
\begin{pgfscope}%
\pgfsys@transformshift{3.796684in}{0.565123in}%
\pgfsys@useobject{currentmarker}{}%
\end{pgfscope}%
\end{pgfscope}%
\begin{pgfscope}%
\definecolor{textcolor}{rgb}{0.000000,0.000000,0.000000}%
\pgfsetstrokecolor{textcolor}%
\pgfsetfillcolor{textcolor}%
\pgftext[x=3.796684in,y=0.467901in,,top]{\color{textcolor}\rmfamily\fontsize{10.000000}{12.000000}\selectfont \(\displaystyle {150}\)}%
\end{pgfscope}%
\begin{pgfscope}%
\pgfsetbuttcap%
\pgfsetroundjoin%
\definecolor{currentfill}{rgb}{0.000000,0.000000,0.000000}%
\pgfsetfillcolor{currentfill}%
\pgfsetlinewidth{0.803000pt}%
\definecolor{currentstroke}{rgb}{0.000000,0.000000,0.000000}%
\pgfsetstrokecolor{currentstroke}%
\pgfsetdash{}{0pt}%
\pgfsys@defobject{currentmarker}{\pgfqpoint{0.000000in}{-0.048611in}}{\pgfqpoint{0.000000in}{0.000000in}}{%
\pgfpathmoveto{\pgfqpoint{0.000000in}{0.000000in}}%
\pgfpathlineto{\pgfqpoint{0.000000in}{-0.048611in}}%
\pgfusepath{stroke,fill}%
}%
\begin{pgfscope}%
\pgfsys@transformshift{4.795312in}{0.565123in}%
\pgfsys@useobject{currentmarker}{}%
\end{pgfscope}%
\end{pgfscope}%
\begin{pgfscope}%
\definecolor{textcolor}{rgb}{0.000000,0.000000,0.000000}%
\pgfsetstrokecolor{textcolor}%
\pgfsetfillcolor{textcolor}%
\pgftext[x=4.795312in,y=0.467901in,,top]{\color{textcolor}\rmfamily\fontsize{10.000000}{12.000000}\selectfont \(\displaystyle {200}\)}%
\end{pgfscope}%
\begin{pgfscope}%
\definecolor{textcolor}{rgb}{0.000000,0.000000,0.000000}%
\pgfsetstrokecolor{textcolor}%
\pgfsetfillcolor{textcolor}%
\pgftext[x=3.157562in,y=0.288889in,,top]{\color{textcolor}\rmfamily\fontsize{10.000000}{12.000000}\selectfont Tempo (Frame)}%
\end{pgfscope}%
\begin{pgfscope}%
\pgfsetbuttcap%
\pgfsetroundjoin%
\definecolor{currentfill}{rgb}{0.000000,0.000000,0.000000}%
\pgfsetfillcolor{currentfill}%
\pgfsetlinewidth{0.803000pt}%
\definecolor{currentstroke}{rgb}{0.000000,0.000000,0.000000}%
\pgfsetstrokecolor{currentstroke}%
\pgfsetdash{}{0pt}%
\pgfsys@defobject{currentmarker}{\pgfqpoint{-0.048611in}{0.000000in}}{\pgfqpoint{-0.000000in}{0.000000in}}{%
\pgfpathmoveto{\pgfqpoint{-0.000000in}{0.000000in}}%
\pgfpathlineto{\pgfqpoint{-0.048611in}{0.000000in}}%
\pgfusepath{stroke,fill}%
}%
\begin{pgfscope}%
\pgfsys@transformshift{0.565124in}{0.687163in}%
\pgfsys@useobject{currentmarker}{}%
\end{pgfscope}%
\end{pgfscope}%
\begin{pgfscope}%
\definecolor{textcolor}{rgb}{0.000000,0.000000,0.000000}%
\pgfsetstrokecolor{textcolor}%
\pgfsetfillcolor{textcolor}%
\pgftext[x=0.398457in, y=0.638938in, left, base]{\color{textcolor}\rmfamily\fontsize{10.000000}{12.000000}\selectfont \(\displaystyle {0}\)}%
\end{pgfscope}%
\begin{pgfscope}%
\pgfsetbuttcap%
\pgfsetroundjoin%
\definecolor{currentfill}{rgb}{0.000000,0.000000,0.000000}%
\pgfsetfillcolor{currentfill}%
\pgfsetlinewidth{0.803000pt}%
\definecolor{currentstroke}{rgb}{0.000000,0.000000,0.000000}%
\pgfsetstrokecolor{currentstroke}%
\pgfsetdash{}{0pt}%
\pgfsys@defobject{currentmarker}{\pgfqpoint{-0.048611in}{0.000000in}}{\pgfqpoint{-0.000000in}{0.000000in}}{%
\pgfpathmoveto{\pgfqpoint{-0.000000in}{0.000000in}}%
\pgfpathlineto{\pgfqpoint{-0.048611in}{0.000000in}}%
\pgfusepath{stroke,fill}%
}%
\begin{pgfscope}%
\pgfsys@transformshift{0.565124in}{1.239248in}%
\pgfsys@useobject{currentmarker}{}%
\end{pgfscope}%
\end{pgfscope}%
\begin{pgfscope}%
\definecolor{textcolor}{rgb}{0.000000,0.000000,0.000000}%
\pgfsetstrokecolor{textcolor}%
\pgfsetfillcolor{textcolor}%
\pgftext[x=0.329012in, y=1.191023in, left, base]{\color{textcolor}\rmfamily\fontsize{10.000000}{12.000000}\selectfont \(\displaystyle {20}\)}%
\end{pgfscope}%
\begin{pgfscope}%
\pgfsetbuttcap%
\pgfsetroundjoin%
\definecolor{currentfill}{rgb}{0.000000,0.000000,0.000000}%
\pgfsetfillcolor{currentfill}%
\pgfsetlinewidth{0.803000pt}%
\definecolor{currentstroke}{rgb}{0.000000,0.000000,0.000000}%
\pgfsetstrokecolor{currentstroke}%
\pgfsetdash{}{0pt}%
\pgfsys@defobject{currentmarker}{\pgfqpoint{-0.048611in}{0.000000in}}{\pgfqpoint{-0.000000in}{0.000000in}}{%
\pgfpathmoveto{\pgfqpoint{-0.000000in}{0.000000in}}%
\pgfpathlineto{\pgfqpoint{-0.048611in}{0.000000in}}%
\pgfusepath{stroke,fill}%
}%
\begin{pgfscope}%
\pgfsys@transformshift{0.565124in}{1.791334in}%
\pgfsys@useobject{currentmarker}{}%
\end{pgfscope}%
\end{pgfscope}%
\begin{pgfscope}%
\definecolor{textcolor}{rgb}{0.000000,0.000000,0.000000}%
\pgfsetstrokecolor{textcolor}%
\pgfsetfillcolor{textcolor}%
\pgftext[x=0.329012in, y=1.743108in, left, base]{\color{textcolor}\rmfamily\fontsize{10.000000}{12.000000}\selectfont \(\displaystyle {40}\)}%
\end{pgfscope}%
\begin{pgfscope}%
\pgfsetbuttcap%
\pgfsetroundjoin%
\definecolor{currentfill}{rgb}{0.000000,0.000000,0.000000}%
\pgfsetfillcolor{currentfill}%
\pgfsetlinewidth{0.803000pt}%
\definecolor{currentstroke}{rgb}{0.000000,0.000000,0.000000}%
\pgfsetstrokecolor{currentstroke}%
\pgfsetdash{}{0pt}%
\pgfsys@defobject{currentmarker}{\pgfqpoint{-0.048611in}{0.000000in}}{\pgfqpoint{-0.000000in}{0.000000in}}{%
\pgfpathmoveto{\pgfqpoint{-0.000000in}{0.000000in}}%
\pgfpathlineto{\pgfqpoint{-0.048611in}{0.000000in}}%
\pgfusepath{stroke,fill}%
}%
\begin{pgfscope}%
\pgfsys@transformshift{0.565124in}{2.343419in}%
\pgfsys@useobject{currentmarker}{}%
\end{pgfscope}%
\end{pgfscope}%
\begin{pgfscope}%
\definecolor{textcolor}{rgb}{0.000000,0.000000,0.000000}%
\pgfsetstrokecolor{textcolor}%
\pgfsetfillcolor{textcolor}%
\pgftext[x=0.329012in, y=2.295194in, left, base]{\color{textcolor}\rmfamily\fontsize{10.000000}{12.000000}\selectfont \(\displaystyle {60}\)}%
\end{pgfscope}%
\begin{pgfscope}%
\pgfsetbuttcap%
\pgfsetroundjoin%
\definecolor{currentfill}{rgb}{0.000000,0.000000,0.000000}%
\pgfsetfillcolor{currentfill}%
\pgfsetlinewidth{0.803000pt}%
\definecolor{currentstroke}{rgb}{0.000000,0.000000,0.000000}%
\pgfsetstrokecolor{currentstroke}%
\pgfsetdash{}{0pt}%
\pgfsys@defobject{currentmarker}{\pgfqpoint{-0.048611in}{0.000000in}}{\pgfqpoint{-0.000000in}{0.000000in}}{%
\pgfpathmoveto{\pgfqpoint{-0.000000in}{0.000000in}}%
\pgfpathlineto{\pgfqpoint{-0.048611in}{0.000000in}}%
\pgfusepath{stroke,fill}%
}%
\begin{pgfscope}%
\pgfsys@transformshift{0.565124in}{2.895504in}%
\pgfsys@useobject{currentmarker}{}%
\end{pgfscope}%
\end{pgfscope}%
\begin{pgfscope}%
\definecolor{textcolor}{rgb}{0.000000,0.000000,0.000000}%
\pgfsetstrokecolor{textcolor}%
\pgfsetfillcolor{textcolor}%
\pgftext[x=0.329012in, y=2.847279in, left, base]{\color{textcolor}\rmfamily\fontsize{10.000000}{12.000000}\selectfont \(\displaystyle {80}\)}%
\end{pgfscope}%
\begin{pgfscope}%
\definecolor{textcolor}{rgb}{0.000000,0.000000,0.000000}%
\pgfsetstrokecolor{textcolor}%
\pgfsetfillcolor{textcolor}%
\pgftext[x=0.273457in,y=1.907562in,,bottom,rotate=90.000000]{\color{textcolor}\rmfamily\fontsize{10.000000}{12.000000}\selectfont Differenza percettiva}%
\end{pgfscope}%
\begin{pgfscope}%
\pgfpathrectangle{\pgfqpoint{0.565124in}{0.565123in}}{\pgfqpoint{5.184876in}{2.684877in}}%
\pgfusepath{clip}%
\pgfsetrectcap%
\pgfsetroundjoin%
\pgfsetlinewidth{2.007500pt}%
\definecolor{currentstroke}{rgb}{0.121569,0.466667,0.705882}%
\pgfsetstrokecolor{currentstroke}%
\pgfsetdash{}{0pt}%
\pgfpathmoveto{\pgfqpoint{0.800800in}{1.566187in}}%
\pgfpathlineto{\pgfqpoint{0.820773in}{1.436116in}}%
\pgfpathlineto{\pgfqpoint{0.840745in}{1.376603in}}%
\pgfpathlineto{\pgfqpoint{0.860718in}{1.336835in}}%
\pgfpathlineto{\pgfqpoint{0.880690in}{1.303095in}}%
\pgfpathlineto{\pgfqpoint{0.900663in}{1.259225in}}%
\pgfpathlineto{\pgfqpoint{0.940608in}{1.212086in}}%
\pgfpathlineto{\pgfqpoint{0.980553in}{1.175465in}}%
\pgfpathlineto{\pgfqpoint{1.020498in}{1.145133in}}%
\pgfpathlineto{\pgfqpoint{1.040471in}{1.134587in}}%
\pgfpathlineto{\pgfqpoint{1.060443in}{1.128409in}}%
\pgfpathlineto{\pgfqpoint{1.080416in}{1.126074in}}%
\pgfpathlineto{\pgfqpoint{1.160306in}{1.124565in}}%
\pgfpathlineto{\pgfqpoint{1.180279in}{1.120349in}}%
\pgfpathlineto{\pgfqpoint{1.240196in}{1.119922in}}%
\pgfpathlineto{\pgfqpoint{1.260169in}{1.119922in}}%
\pgfpathlineto{\pgfqpoint{1.280141in}{1.051221in}}%
\pgfpathlineto{\pgfqpoint{1.300114in}{0.998736in}}%
\pgfpathlineto{\pgfqpoint{1.320086in}{0.992329in}}%
\pgfpathlineto{\pgfqpoint{1.340059in}{0.991066in}}%
\pgfpathlineto{\pgfqpoint{1.360032in}{0.976117in}}%
\pgfpathlineto{\pgfqpoint{1.439922in}{0.971490in}}%
\pgfpathlineto{\pgfqpoint{1.459894in}{0.961494in}}%
\pgfpathlineto{\pgfqpoint{1.479867in}{0.957129in}}%
\pgfpathlineto{\pgfqpoint{1.499839in}{0.947951in}}%
\pgfpathlineto{\pgfqpoint{1.519812in}{0.946301in}}%
\pgfpathlineto{\pgfqpoint{1.539785in}{0.927931in}}%
\pgfpathlineto{\pgfqpoint{1.579730in}{0.927435in}}%
\pgfpathlineto{\pgfqpoint{1.599702in}{0.925567in}}%
\pgfpathlineto{\pgfqpoint{1.619675in}{0.909629in}}%
\pgfpathlineto{\pgfqpoint{1.739510in}{0.901181in}}%
\pgfpathlineto{\pgfqpoint{1.759483in}{0.900805in}}%
\pgfpathlineto{\pgfqpoint{1.779455in}{0.873125in}}%
\pgfpathlineto{\pgfqpoint{1.799428in}{0.869677in}}%
\pgfpathlineto{\pgfqpoint{1.819400in}{0.861213in}}%
\pgfpathlineto{\pgfqpoint{1.839373in}{0.856144in}}%
\pgfpathlineto{\pgfqpoint{1.859346in}{0.856143in}}%
\pgfpathlineto{\pgfqpoint{1.879318in}{0.823020in}}%
\pgfpathlineto{\pgfqpoint{1.919263in}{0.821221in}}%
\pgfpathlineto{\pgfqpoint{1.939236in}{0.786445in}}%
\pgfpathlineto{\pgfqpoint{1.959208in}{0.782224in}}%
\pgfpathlineto{\pgfqpoint{1.979181in}{0.782217in}}%
\pgfpathlineto{\pgfqpoint{1.999153in}{0.780154in}}%
\pgfpathlineto{\pgfqpoint{2.039099in}{0.773324in}}%
\pgfpathlineto{\pgfqpoint{2.059071in}{0.772015in}}%
\pgfpathlineto{\pgfqpoint{2.079044in}{0.763656in}}%
\pgfpathlineto{\pgfqpoint{2.099016in}{0.761275in}}%
\pgfpathlineto{\pgfqpoint{2.178906in}{0.757226in}}%
\pgfpathlineto{\pgfqpoint{2.218852in}{0.757045in}}%
\pgfpathlineto{\pgfqpoint{2.238824in}{0.755057in}}%
\pgfpathlineto{\pgfqpoint{2.258797in}{0.754807in}}%
\pgfpathlineto{\pgfqpoint{2.298742in}{0.744902in}}%
\pgfpathlineto{\pgfqpoint{2.318714in}{0.741549in}}%
\pgfpathlineto{\pgfqpoint{2.338687in}{0.715560in}}%
\pgfpathlineto{\pgfqpoint{2.398605in}{0.714591in}}%
\pgfpathlineto{\pgfqpoint{2.418577in}{0.712713in}}%
\pgfpathlineto{\pgfqpoint{2.458522in}{0.712037in}}%
\pgfpathlineto{\pgfqpoint{2.478495in}{0.710406in}}%
\pgfpathlineto{\pgfqpoint{2.538413in}{0.709924in}}%
\pgfpathlineto{\pgfqpoint{2.658248in}{0.709704in}}%
\pgfpathlineto{\pgfqpoint{2.718166in}{0.709390in}}%
\pgfpathlineto{\pgfqpoint{2.738138in}{0.698726in}}%
\pgfpathlineto{\pgfqpoint{2.798056in}{0.697824in}}%
\pgfpathlineto{\pgfqpoint{2.857973in}{0.697403in}}%
\pgfpathlineto{\pgfqpoint{3.017754in}{0.697019in}}%
\pgfpathlineto{\pgfqpoint{3.037727in}{0.689910in}}%
\pgfpathlineto{\pgfqpoint{3.077672in}{0.689855in}}%
\pgfpathlineto{\pgfqpoint{3.097644in}{0.688686in}}%
\pgfpathlineto{\pgfqpoint{3.177534in}{0.687907in}}%
\pgfpathlineto{\pgfqpoint{3.217480in}{0.687165in}}%
\pgfpathlineto{\pgfqpoint{5.494351in}{0.687163in}}%
\pgfpathlineto{\pgfqpoint{5.494351in}{0.687163in}}%
\pgfusepath{stroke}%
\end{pgfscope}%
\begin{pgfscope}%
\pgfpathrectangle{\pgfqpoint{0.565124in}{0.565123in}}{\pgfqpoint{5.184876in}{2.684877in}}%
\pgfusepath{clip}%
\pgfsetrectcap%
\pgfsetroundjoin%
\pgfsetlinewidth{2.007500pt}%
\definecolor{currentstroke}{rgb}{1.000000,0.498039,0.054902}%
\pgfsetstrokecolor{currentstroke}%
\pgfsetdash{}{0pt}%
\pgfpathmoveto{\pgfqpoint{0.820773in}{3.127958in}}%
\pgfpathlineto{\pgfqpoint{0.880690in}{3.127960in}}%
\pgfpathlineto{\pgfqpoint{0.900663in}{3.104083in}}%
\pgfpathlineto{\pgfqpoint{1.180279in}{3.104088in}}%
\pgfpathlineto{\pgfqpoint{1.200251in}{3.089740in}}%
\pgfpathlineto{\pgfqpoint{1.280141in}{3.089742in}}%
\pgfpathlineto{\pgfqpoint{1.300114in}{3.085256in}}%
\pgfpathlineto{\pgfqpoint{1.479867in}{3.085257in}}%
\pgfpathlineto{\pgfqpoint{1.499839in}{3.076688in}}%
\pgfpathlineto{\pgfqpoint{1.519812in}{3.076502in}}%
\pgfpathlineto{\pgfqpoint{1.539785in}{3.054427in}}%
\pgfpathlineto{\pgfqpoint{1.559757in}{2.707670in}}%
\pgfpathlineto{\pgfqpoint{1.579730in}{2.649023in}}%
\pgfpathlineto{\pgfqpoint{1.599702in}{2.159039in}}%
\pgfpathlineto{\pgfqpoint{1.619675in}{2.148198in}}%
\pgfpathlineto{\pgfqpoint{1.639647in}{2.143333in}}%
\pgfpathlineto{\pgfqpoint{1.659620in}{2.122208in}}%
\pgfpathlineto{\pgfqpoint{1.679593in}{1.976787in}}%
\pgfpathlineto{\pgfqpoint{1.699565in}{1.976787in}}%
\pgfpathlineto{\pgfqpoint{1.719538in}{1.963167in}}%
\pgfpathlineto{\pgfqpoint{1.739510in}{1.935682in}}%
\pgfpathlineto{\pgfqpoint{1.759483in}{1.935682in}}%
\pgfpathlineto{\pgfqpoint{1.779455in}{1.928859in}}%
\pgfpathlineto{\pgfqpoint{1.819400in}{1.928537in}}%
\pgfpathlineto{\pgfqpoint{1.839373in}{1.916278in}}%
\pgfpathlineto{\pgfqpoint{1.859346in}{1.915046in}}%
\pgfpathlineto{\pgfqpoint{1.919263in}{1.906885in}}%
\pgfpathlineto{\pgfqpoint{1.939236in}{1.901395in}}%
\pgfpathlineto{\pgfqpoint{1.959208in}{1.897524in}}%
\pgfpathlineto{\pgfqpoint{1.979181in}{1.897300in}}%
\pgfpathlineto{\pgfqpoint{1.999153in}{1.384465in}}%
\pgfpathlineto{\pgfqpoint{2.019126in}{1.380341in}}%
\pgfpathlineto{\pgfqpoint{2.039099in}{1.380076in}}%
\pgfpathlineto{\pgfqpoint{2.059071in}{1.361916in}}%
\pgfpathlineto{\pgfqpoint{2.079044in}{1.355952in}}%
\pgfpathlineto{\pgfqpoint{2.099016in}{1.342161in}}%
\pgfpathlineto{\pgfqpoint{2.138961in}{1.341356in}}%
\pgfpathlineto{\pgfqpoint{2.378632in}{1.341351in}}%
\pgfpathlineto{\pgfqpoint{2.398605in}{1.332163in}}%
\pgfpathlineto{\pgfqpoint{2.618303in}{1.332103in}}%
\pgfpathlineto{\pgfqpoint{2.638275in}{1.324071in}}%
\pgfpathlineto{\pgfqpoint{2.678220in}{1.324068in}}%
\pgfpathlineto{\pgfqpoint{2.698193in}{0.994400in}}%
\pgfpathlineto{\pgfqpoint{2.738138in}{0.993609in}}%
\pgfpathlineto{\pgfqpoint{2.758111in}{0.993608in}}%
\pgfpathlineto{\pgfqpoint{2.778083in}{0.887861in}}%
\pgfpathlineto{\pgfqpoint{2.798056in}{0.887863in}}%
\pgfpathlineto{\pgfqpoint{2.838001in}{0.882141in}}%
\pgfpathlineto{\pgfqpoint{2.857973in}{0.882138in}}%
\pgfpathlineto{\pgfqpoint{2.877946in}{0.857498in}}%
\pgfpathlineto{\pgfqpoint{2.937864in}{0.857500in}}%
\pgfpathlineto{\pgfqpoint{2.957836in}{0.853485in}}%
\pgfpathlineto{\pgfqpoint{3.077672in}{0.853485in}}%
\pgfpathlineto{\pgfqpoint{3.097644in}{0.849755in}}%
\pgfpathlineto{\pgfqpoint{3.117617in}{0.849752in}}%
\pgfpathlineto{\pgfqpoint{3.137589in}{0.845537in}}%
\pgfpathlineto{\pgfqpoint{3.197507in}{0.845544in}}%
\pgfpathlineto{\pgfqpoint{3.217480in}{0.837663in}}%
\pgfpathlineto{\pgfqpoint{3.237452in}{0.837661in}}%
\pgfpathlineto{\pgfqpoint{3.257425in}{0.833058in}}%
\pgfpathlineto{\pgfqpoint{3.437178in}{0.833058in}}%
\pgfpathlineto{\pgfqpoint{3.457150in}{0.830378in}}%
\pgfpathlineto{\pgfqpoint{3.477123in}{0.829110in}}%
\pgfpathlineto{\pgfqpoint{3.497095in}{0.818908in}}%
\pgfpathlineto{\pgfqpoint{3.537040in}{0.818905in}}%
\pgfpathlineto{\pgfqpoint{3.557013in}{0.815842in}}%
\pgfpathlineto{\pgfqpoint{3.576986in}{0.815845in}}%
\pgfpathlineto{\pgfqpoint{3.596958in}{0.809390in}}%
\pgfpathlineto{\pgfqpoint{3.616931in}{0.809388in}}%
\pgfpathlineto{\pgfqpoint{3.636903in}{0.804747in}}%
\pgfpathlineto{\pgfqpoint{3.656876in}{0.804746in}}%
\pgfpathlineto{\pgfqpoint{3.676848in}{0.768455in}}%
\pgfpathlineto{\pgfqpoint{3.696821in}{0.739299in}}%
\pgfpathlineto{\pgfqpoint{3.796684in}{0.739284in}}%
\pgfpathlineto{\pgfqpoint{3.816656in}{0.727582in}}%
\pgfpathlineto{\pgfqpoint{3.836629in}{0.727583in}}%
\pgfpathlineto{\pgfqpoint{3.856601in}{0.720642in}}%
\pgfpathlineto{\pgfqpoint{3.876574in}{0.703283in}}%
\pgfpathlineto{\pgfqpoint{4.355915in}{0.702912in}}%
\pgfpathlineto{\pgfqpoint{4.375888in}{0.687327in}}%
\pgfpathlineto{\pgfqpoint{5.514324in}{0.687163in}}%
\pgfpathlineto{\pgfqpoint{5.514324in}{0.687163in}}%
\pgfusepath{stroke}%
\end{pgfscope}%
\begin{pgfscope}%
\pgfsetrectcap%
\pgfsetmiterjoin%
\pgfsetlinewidth{0.803000pt}%
\definecolor{currentstroke}{rgb}{0.000000,0.000000,0.000000}%
\pgfsetstrokecolor{currentstroke}%
\pgfsetdash{}{0pt}%
\pgfpathmoveto{\pgfqpoint{0.565124in}{0.565123in}}%
\pgfpathlineto{\pgfqpoint{0.565124in}{3.250000in}}%
\pgfusepath{stroke}%
\end{pgfscope}%
\begin{pgfscope}%
\pgfsetrectcap%
\pgfsetmiterjoin%
\pgfsetlinewidth{0.803000pt}%
\definecolor{currentstroke}{rgb}{0.000000,0.000000,0.000000}%
\pgfsetstrokecolor{currentstroke}%
\pgfsetdash{}{0pt}%
\pgfpathmoveto{\pgfqpoint{5.750000in}{0.565123in}}%
\pgfpathlineto{\pgfqpoint{5.750000in}{3.250000in}}%
\pgfusepath{stroke}%
\end{pgfscope}%
\begin{pgfscope}%
\pgfsetrectcap%
\pgfsetmiterjoin%
\pgfsetlinewidth{0.803000pt}%
\definecolor{currentstroke}{rgb}{0.000000,0.000000,0.000000}%
\pgfsetstrokecolor{currentstroke}%
\pgfsetdash{}{0pt}%
\pgfpathmoveto{\pgfqpoint{0.565124in}{0.565123in}}%
\pgfpathlineto{\pgfqpoint{5.750000in}{0.565123in}}%
\pgfusepath{stroke}%
\end{pgfscope}%
\begin{pgfscope}%
\pgfsetrectcap%
\pgfsetmiterjoin%
\pgfsetlinewidth{0.803000pt}%
\definecolor{currentstroke}{rgb}{0.000000,0.000000,0.000000}%
\pgfsetstrokecolor{currentstroke}%
\pgfsetdash{}{0pt}%
\pgfpathmoveto{\pgfqpoint{0.565124in}{3.250000in}}%
\pgfpathlineto{\pgfqpoint{5.750000in}{3.250000in}}%
\pgfusepath{stroke}%
\end{pgfscope}%
\begin{pgfscope}%
\pgfsetbuttcap%
\pgfsetmiterjoin%
\definecolor{currentfill}{rgb}{1.000000,1.000000,1.000000}%
\pgfsetfillcolor{currentfill}%
\pgfsetfillopacity{0.800000}%
\pgfsetlinewidth{1.003750pt}%
\definecolor{currentstroke}{rgb}{0.800000,0.800000,0.800000}%
\pgfsetstrokecolor{currentstroke}%
\pgfsetstrokeopacity{0.800000}%
\pgfsetdash{}{0pt}%
\pgfpathmoveto{\pgfqpoint{4.273532in}{2.751543in}}%
\pgfpathlineto{\pgfqpoint{5.652778in}{2.751543in}}%
\pgfpathquadraticcurveto{\pgfqpoint{5.680556in}{2.751543in}}{\pgfqpoint{5.680556in}{2.779321in}}%
\pgfpathlineto{\pgfqpoint{5.680556in}{3.152778in}}%
\pgfpathquadraticcurveto{\pgfqpoint{5.680556in}{3.180556in}}{\pgfqpoint{5.652778in}{3.180556in}}%
\pgfpathlineto{\pgfqpoint{4.273532in}{3.180556in}}%
\pgfpathquadraticcurveto{\pgfqpoint{4.245755in}{3.180556in}}{\pgfqpoint{4.245755in}{3.152778in}}%
\pgfpathlineto{\pgfqpoint{4.245755in}{2.779321in}}%
\pgfpathquadraticcurveto{\pgfqpoint{4.245755in}{2.751543in}}{\pgfqpoint{4.273532in}{2.751543in}}%
\pgfpathclose%
\pgfusepath{stroke,fill}%
\end{pgfscope}%
\begin{pgfscope}%
\pgfsetrectcap%
\pgfsetroundjoin%
\pgfsetlinewidth{2.007500pt}%
\definecolor{currentstroke}{rgb}{0.121569,0.466667,0.705882}%
\pgfsetstrokecolor{currentstroke}%
\pgfsetdash{}{0pt}%
\pgfpathmoveto{\pgfqpoint{4.301310in}{3.076389in}}%
\pgfpathlineto{\pgfqpoint{4.579088in}{3.076389in}}%
\pgfusepath{stroke}%
\end{pgfscope}%
\begin{pgfscope}%
\definecolor{textcolor}{rgb}{0.000000,0.000000,0.000000}%
\pgfsetstrokecolor{textcolor}%
\pgfsetfillcolor{textcolor}%
\pgftext[x=4.690199in,y=3.027778in,left,base]{\color{textcolor}\rmfamily\fontsize{10.000000}{12.000000}\selectfont Scena fittizia}%
\end{pgfscope}%
\begin{pgfscope}%
\pgfsetrectcap%
\pgfsetroundjoin%
\pgfsetlinewidth{2.007500pt}%
\definecolor{currentstroke}{rgb}{1.000000,0.498039,0.054902}%
\pgfsetstrokecolor{currentstroke}%
\pgfsetdash{}{0pt}%
\pgfpathmoveto{\pgfqpoint{4.301310in}{2.882716in}}%
\pgfpathlineto{\pgfqpoint{4.579088in}{2.882716in}}%
\pgfusepath{stroke}%
\end{pgfscope}%
\begin{pgfscope}%
\definecolor{textcolor}{rgb}{0.000000,0.000000,0.000000}%
\pgfsetstrokecolor{textcolor}%
\pgfsetfillcolor{textcolor}%
\pgftext[x=4.690199in,y=2.834105in,left,base]{\color{textcolor}\rmfamily\fontsize{10.000000}{12.000000}\selectfont Scena realistica}%
\end{pgfscope}%
\end{pgfpicture}%
\makeatother%
\endgroup%

                \caption{Valutazione Sphere Tracing valutato per la dimensione}
                \label{fig:eval-stdim}
            \end{figure}             

        \newpage
        \subsection{Ray Tracing - Ombre portate}
        Combinare la determinazione delle istanze in vista e la previsione delle ombre portate porta un forte incremento di performance in entrambe le scene. Nella scena realistica si percepisce un miglioramento del 47\% rispetto alla strategia di valutazione per dimensione con lo Sphere Tracing, e del 62.6\% e 78.2\% per le strategie Closest-First on Camera e Closest-First rispettivamente. 
        
        I risultati sono promettenti anche per la scena fittizia data la presenza di un maggior numero di ombre portate provenienti da molte istanze non inquadrate. Si osserva un miglioramento del 75.9\% dalla valutazione per dimensione con lo Sphere Tracing e del 77.3\% per la valutazione per la distanza sempre dello stesso, del 78.6\% per il Closest-First on Camera e del 91.1\% per il Closest-First. È importante sottolineare che queste prestazioni sono mantenute indipendentemente dall'angolo di provenienza della luce a differenza delle politiche valutate nei paragrafi precedenti. Il parametro $\alpha$ è stato impostato ad $1$ per le valutazioni. Attraverso esperimenti empirici è il valore che ha portato i migliori risultati.
            % \begin{figure}[htb!]
            %     \centering
            %     % \includegraphics[scale=.55]{images/valutazioni/fittizia/28-04-23 03-31-24 RayTracing 195.07658568058338.png}
            %     %% Creator: Matplotlib, PGF backend
%%
%% To include the figure in your LaTeX document, write
%%   \input{<filename>.pgf}
%%
%% Make sure the required packages are loaded in your preamble
%%   \usepackage{pgf}
%%
%% Figures using additional raster images can only be included by \input if
%% they are in the same directory as the main LaTeX file. For loading figures
%% from other directories you can use the `import` package
%%   \usepackage{import}
%%
%% and then include the figures with
%%   \import{<path to file>}{<filename>.pgf}
%%
%% Matplotlib used the following preamble
%%
\begingroup%
\makeatletter%
\begin{pgfpicture}%
\pgfpathrectangle{\pgfpointorigin}{\pgfqpoint{5.900000in}{3.400000in}}%
\pgfusepath{use as bounding box, clip}%
\begin{pgfscope}%
\pgfsetbuttcap%
\pgfsetmiterjoin%
\definecolor{currentfill}{rgb}{1.000000,1.000000,1.000000}%
\pgfsetfillcolor{currentfill}%
\pgfsetlinewidth{0.000000pt}%
\definecolor{currentstroke}{rgb}{1.000000,1.000000,1.000000}%
\pgfsetstrokecolor{currentstroke}%
\pgfsetdash{}{0pt}%
\pgfpathmoveto{\pgfqpoint{0.000000in}{0.000000in}}%
\pgfpathlineto{\pgfqpoint{5.900000in}{0.000000in}}%
\pgfpathlineto{\pgfqpoint{5.900000in}{3.400000in}}%
\pgfpathlineto{\pgfqpoint{0.000000in}{3.400000in}}%
\pgfpathclose%
\pgfusepath{fill}%
\end{pgfscope}%
\begin{pgfscope}%
\pgfsetbuttcap%
\pgfsetmiterjoin%
\definecolor{currentfill}{rgb}{1.000000,1.000000,1.000000}%
\pgfsetfillcolor{currentfill}%
\pgfsetlinewidth{0.000000pt}%
\definecolor{currentstroke}{rgb}{0.000000,0.000000,0.000000}%
\pgfsetstrokecolor{currentstroke}%
\pgfsetstrokeopacity{0.000000}%
\pgfsetdash{}{0pt}%
\pgfpathmoveto{\pgfqpoint{0.565124in}{0.549691in}}%
\pgfpathlineto{\pgfqpoint{5.750000in}{0.549691in}}%
\pgfpathlineto{\pgfqpoint{5.750000in}{3.250000in}}%
\pgfpathlineto{\pgfqpoint{0.565124in}{3.250000in}}%
\pgfpathclose%
\pgfusepath{fill}%
\end{pgfscope}%
\begin{pgfscope}%
\pgfsetbuttcap%
\pgfsetroundjoin%
\definecolor{currentfill}{rgb}{0.000000,0.000000,0.000000}%
\pgfsetfillcolor{currentfill}%
\pgfsetlinewidth{0.803000pt}%
\definecolor{currentstroke}{rgb}{0.000000,0.000000,0.000000}%
\pgfsetstrokecolor{currentstroke}%
\pgfsetdash{}{0pt}%
\pgfsys@defobject{currentmarker}{\pgfqpoint{0.000000in}{-0.048611in}}{\pgfqpoint{0.000000in}{0.000000in}}{%
\pgfpathmoveto{\pgfqpoint{0.000000in}{0.000000in}}%
\pgfpathlineto{\pgfqpoint{0.000000in}{-0.048611in}}%
\pgfusepath{stroke,fill}%
}%
\begin{pgfscope}%
\pgfsys@transformshift{0.770968in}{0.549691in}%
\pgfsys@useobject{currentmarker}{}%
\end{pgfscope}%
\end{pgfscope}%
\begin{pgfscope}%
\definecolor{textcolor}{rgb}{0.000000,0.000000,0.000000}%
\pgfsetstrokecolor{textcolor}%
\pgfsetfillcolor{textcolor}%
\pgftext[x=0.770968in,y=0.452469in,,top]{\color{textcolor}\rmfamily\fontsize{10.000000}{12.000000}\selectfont \(\displaystyle {0}\)}%
\end{pgfscope}%
\begin{pgfscope}%
\pgfsetbuttcap%
\pgfsetroundjoin%
\definecolor{currentfill}{rgb}{0.000000,0.000000,0.000000}%
\pgfsetfillcolor{currentfill}%
\pgfsetlinewidth{0.803000pt}%
\definecolor{currentstroke}{rgb}{0.000000,0.000000,0.000000}%
\pgfsetstrokecolor{currentstroke}%
\pgfsetdash{}{0pt}%
\pgfsys@defobject{currentmarker}{\pgfqpoint{0.000000in}{-0.048611in}}{\pgfqpoint{0.000000in}{0.000000in}}{%
\pgfpathmoveto{\pgfqpoint{0.000000in}{0.000000in}}%
\pgfpathlineto{\pgfqpoint{0.000000in}{-0.048611in}}%
\pgfusepath{stroke,fill}%
}%
\begin{pgfscope}%
\pgfsys@transformshift{1.367616in}{0.549691in}%
\pgfsys@useobject{currentmarker}{}%
\end{pgfscope}%
\end{pgfscope}%
\begin{pgfscope}%
\definecolor{textcolor}{rgb}{0.000000,0.000000,0.000000}%
\pgfsetstrokecolor{textcolor}%
\pgfsetfillcolor{textcolor}%
\pgftext[x=1.367616in,y=0.452469in,,top]{\color{textcolor}\rmfamily\fontsize{10.000000}{12.000000}\selectfont \(\displaystyle {20}\)}%
\end{pgfscope}%
\begin{pgfscope}%
\pgfsetbuttcap%
\pgfsetroundjoin%
\definecolor{currentfill}{rgb}{0.000000,0.000000,0.000000}%
\pgfsetfillcolor{currentfill}%
\pgfsetlinewidth{0.803000pt}%
\definecolor{currentstroke}{rgb}{0.000000,0.000000,0.000000}%
\pgfsetstrokecolor{currentstroke}%
\pgfsetdash{}{0pt}%
\pgfsys@defobject{currentmarker}{\pgfqpoint{0.000000in}{-0.048611in}}{\pgfqpoint{0.000000in}{0.000000in}}{%
\pgfpathmoveto{\pgfqpoint{0.000000in}{0.000000in}}%
\pgfpathlineto{\pgfqpoint{0.000000in}{-0.048611in}}%
\pgfusepath{stroke,fill}%
}%
\begin{pgfscope}%
\pgfsys@transformshift{1.964265in}{0.549691in}%
\pgfsys@useobject{currentmarker}{}%
\end{pgfscope}%
\end{pgfscope}%
\begin{pgfscope}%
\definecolor{textcolor}{rgb}{0.000000,0.000000,0.000000}%
\pgfsetstrokecolor{textcolor}%
\pgfsetfillcolor{textcolor}%
\pgftext[x=1.964265in,y=0.452469in,,top]{\color{textcolor}\rmfamily\fontsize{10.000000}{12.000000}\selectfont \(\displaystyle {40}\)}%
\end{pgfscope}%
\begin{pgfscope}%
\pgfsetbuttcap%
\pgfsetroundjoin%
\definecolor{currentfill}{rgb}{0.000000,0.000000,0.000000}%
\pgfsetfillcolor{currentfill}%
\pgfsetlinewidth{0.803000pt}%
\definecolor{currentstroke}{rgb}{0.000000,0.000000,0.000000}%
\pgfsetstrokecolor{currentstroke}%
\pgfsetdash{}{0pt}%
\pgfsys@defobject{currentmarker}{\pgfqpoint{0.000000in}{-0.048611in}}{\pgfqpoint{0.000000in}{0.000000in}}{%
\pgfpathmoveto{\pgfqpoint{0.000000in}{0.000000in}}%
\pgfpathlineto{\pgfqpoint{0.000000in}{-0.048611in}}%
\pgfusepath{stroke,fill}%
}%
\begin{pgfscope}%
\pgfsys@transformshift{2.560913in}{0.549691in}%
\pgfsys@useobject{currentmarker}{}%
\end{pgfscope}%
\end{pgfscope}%
\begin{pgfscope}%
\definecolor{textcolor}{rgb}{0.000000,0.000000,0.000000}%
\pgfsetstrokecolor{textcolor}%
\pgfsetfillcolor{textcolor}%
\pgftext[x=2.560913in,y=0.452469in,,top]{\color{textcolor}\rmfamily\fontsize{10.000000}{12.000000}\selectfont \(\displaystyle {60}\)}%
\end{pgfscope}%
\begin{pgfscope}%
\pgfsetbuttcap%
\pgfsetroundjoin%
\definecolor{currentfill}{rgb}{0.000000,0.000000,0.000000}%
\pgfsetfillcolor{currentfill}%
\pgfsetlinewidth{0.803000pt}%
\definecolor{currentstroke}{rgb}{0.000000,0.000000,0.000000}%
\pgfsetstrokecolor{currentstroke}%
\pgfsetdash{}{0pt}%
\pgfsys@defobject{currentmarker}{\pgfqpoint{0.000000in}{-0.048611in}}{\pgfqpoint{0.000000in}{0.000000in}}{%
\pgfpathmoveto{\pgfqpoint{0.000000in}{0.000000in}}%
\pgfpathlineto{\pgfqpoint{0.000000in}{-0.048611in}}%
\pgfusepath{stroke,fill}%
}%
\begin{pgfscope}%
\pgfsys@transformshift{3.157562in}{0.549691in}%
\pgfsys@useobject{currentmarker}{}%
\end{pgfscope}%
\end{pgfscope}%
\begin{pgfscope}%
\definecolor{textcolor}{rgb}{0.000000,0.000000,0.000000}%
\pgfsetstrokecolor{textcolor}%
\pgfsetfillcolor{textcolor}%
\pgftext[x=3.157562in,y=0.452469in,,top]{\color{textcolor}\rmfamily\fontsize{10.000000}{12.000000}\selectfont \(\displaystyle {80}\)}%
\end{pgfscope}%
\begin{pgfscope}%
\pgfsetbuttcap%
\pgfsetroundjoin%
\definecolor{currentfill}{rgb}{0.000000,0.000000,0.000000}%
\pgfsetfillcolor{currentfill}%
\pgfsetlinewidth{0.803000pt}%
\definecolor{currentstroke}{rgb}{0.000000,0.000000,0.000000}%
\pgfsetstrokecolor{currentstroke}%
\pgfsetdash{}{0pt}%
\pgfsys@defobject{currentmarker}{\pgfqpoint{0.000000in}{-0.048611in}}{\pgfqpoint{0.000000in}{0.000000in}}{%
\pgfpathmoveto{\pgfqpoint{0.000000in}{0.000000in}}%
\pgfpathlineto{\pgfqpoint{0.000000in}{-0.048611in}}%
\pgfusepath{stroke,fill}%
}%
\begin{pgfscope}%
\pgfsys@transformshift{3.754210in}{0.549691in}%
\pgfsys@useobject{currentmarker}{}%
\end{pgfscope}%
\end{pgfscope}%
\begin{pgfscope}%
\definecolor{textcolor}{rgb}{0.000000,0.000000,0.000000}%
\pgfsetstrokecolor{textcolor}%
\pgfsetfillcolor{textcolor}%
\pgftext[x=3.754210in,y=0.452469in,,top]{\color{textcolor}\rmfamily\fontsize{10.000000}{12.000000}\selectfont \(\displaystyle {100}\)}%
\end{pgfscope}%
\begin{pgfscope}%
\pgfsetbuttcap%
\pgfsetroundjoin%
\definecolor{currentfill}{rgb}{0.000000,0.000000,0.000000}%
\pgfsetfillcolor{currentfill}%
\pgfsetlinewidth{0.803000pt}%
\definecolor{currentstroke}{rgb}{0.000000,0.000000,0.000000}%
\pgfsetstrokecolor{currentstroke}%
\pgfsetdash{}{0pt}%
\pgfsys@defobject{currentmarker}{\pgfqpoint{0.000000in}{-0.048611in}}{\pgfqpoint{0.000000in}{0.000000in}}{%
\pgfpathmoveto{\pgfqpoint{0.000000in}{0.000000in}}%
\pgfpathlineto{\pgfqpoint{0.000000in}{-0.048611in}}%
\pgfusepath{stroke,fill}%
}%
\begin{pgfscope}%
\pgfsys@transformshift{4.350859in}{0.549691in}%
\pgfsys@useobject{currentmarker}{}%
\end{pgfscope}%
\end{pgfscope}%
\begin{pgfscope}%
\definecolor{textcolor}{rgb}{0.000000,0.000000,0.000000}%
\pgfsetstrokecolor{textcolor}%
\pgfsetfillcolor{textcolor}%
\pgftext[x=4.350859in,y=0.452469in,,top]{\color{textcolor}\rmfamily\fontsize{10.000000}{12.000000}\selectfont \(\displaystyle {120}\)}%
\end{pgfscope}%
\begin{pgfscope}%
\pgfsetbuttcap%
\pgfsetroundjoin%
\definecolor{currentfill}{rgb}{0.000000,0.000000,0.000000}%
\pgfsetfillcolor{currentfill}%
\pgfsetlinewidth{0.803000pt}%
\definecolor{currentstroke}{rgb}{0.000000,0.000000,0.000000}%
\pgfsetstrokecolor{currentstroke}%
\pgfsetdash{}{0pt}%
\pgfsys@defobject{currentmarker}{\pgfqpoint{0.000000in}{-0.048611in}}{\pgfqpoint{0.000000in}{0.000000in}}{%
\pgfpathmoveto{\pgfqpoint{0.000000in}{0.000000in}}%
\pgfpathlineto{\pgfqpoint{0.000000in}{-0.048611in}}%
\pgfusepath{stroke,fill}%
}%
\begin{pgfscope}%
\pgfsys@transformshift{4.947508in}{0.549691in}%
\pgfsys@useobject{currentmarker}{}%
\end{pgfscope}%
\end{pgfscope}%
\begin{pgfscope}%
\definecolor{textcolor}{rgb}{0.000000,0.000000,0.000000}%
\pgfsetstrokecolor{textcolor}%
\pgfsetfillcolor{textcolor}%
\pgftext[x=4.947508in,y=0.452469in,,top]{\color{textcolor}\rmfamily\fontsize{10.000000}{12.000000}\selectfont \(\displaystyle {140}\)}%
\end{pgfscope}%
\begin{pgfscope}%
\pgfsetbuttcap%
\pgfsetroundjoin%
\definecolor{currentfill}{rgb}{0.000000,0.000000,0.000000}%
\pgfsetfillcolor{currentfill}%
\pgfsetlinewidth{0.803000pt}%
\definecolor{currentstroke}{rgb}{0.000000,0.000000,0.000000}%
\pgfsetstrokecolor{currentstroke}%
\pgfsetdash{}{0pt}%
\pgfsys@defobject{currentmarker}{\pgfqpoint{0.000000in}{-0.048611in}}{\pgfqpoint{0.000000in}{0.000000in}}{%
\pgfpathmoveto{\pgfqpoint{0.000000in}{0.000000in}}%
\pgfpathlineto{\pgfqpoint{0.000000in}{-0.048611in}}%
\pgfusepath{stroke,fill}%
}%
\begin{pgfscope}%
\pgfsys@transformshift{5.544156in}{0.549691in}%
\pgfsys@useobject{currentmarker}{}%
\end{pgfscope}%
\end{pgfscope}%
\begin{pgfscope}%
\definecolor{textcolor}{rgb}{0.000000,0.000000,0.000000}%
\pgfsetstrokecolor{textcolor}%
\pgfsetfillcolor{textcolor}%
\pgftext[x=5.544156in,y=0.452469in,,top]{\color{textcolor}\rmfamily\fontsize{10.000000}{12.000000}\selectfont \(\displaystyle {160}\)}%
\end{pgfscope}%
\begin{pgfscope}%
\definecolor{textcolor}{rgb}{0.000000,0.000000,0.000000}%
\pgfsetstrokecolor{textcolor}%
\pgfsetfillcolor{textcolor}%
\pgftext[x=3.157562in,y=0.273457in,,top]{\color{textcolor}\rmfamily\fontsize{10.000000}{12.000000}\selectfont Tempo}%
\end{pgfscope}%
\begin{pgfscope}%
\pgfsetbuttcap%
\pgfsetroundjoin%
\definecolor{currentfill}{rgb}{0.000000,0.000000,0.000000}%
\pgfsetfillcolor{currentfill}%
\pgfsetlinewidth{0.803000pt}%
\definecolor{currentstroke}{rgb}{0.000000,0.000000,0.000000}%
\pgfsetstrokecolor{currentstroke}%
\pgfsetdash{}{0pt}%
\pgfsys@defobject{currentmarker}{\pgfqpoint{-0.048611in}{0.000000in}}{\pgfqpoint{-0.000000in}{0.000000in}}{%
\pgfpathmoveto{\pgfqpoint{-0.000000in}{0.000000in}}%
\pgfpathlineto{\pgfqpoint{-0.048611in}{0.000000in}}%
\pgfusepath{stroke,fill}%
}%
\begin{pgfscope}%
\pgfsys@transformshift{0.565124in}{0.672432in}%
\pgfsys@useobject{currentmarker}{}%
\end{pgfscope}%
\end{pgfscope}%
\begin{pgfscope}%
\definecolor{textcolor}{rgb}{0.000000,0.000000,0.000000}%
\pgfsetstrokecolor{textcolor}%
\pgfsetfillcolor{textcolor}%
\pgftext[x=0.398457in, y=0.624207in, left, base]{\color{textcolor}\rmfamily\fontsize{10.000000}{12.000000}\selectfont \(\displaystyle {0}\)}%
\end{pgfscope}%
\begin{pgfscope}%
\pgfsetbuttcap%
\pgfsetroundjoin%
\definecolor{currentfill}{rgb}{0.000000,0.000000,0.000000}%
\pgfsetfillcolor{currentfill}%
\pgfsetlinewidth{0.803000pt}%
\definecolor{currentstroke}{rgb}{0.000000,0.000000,0.000000}%
\pgfsetstrokecolor{currentstroke}%
\pgfsetdash{}{0pt}%
\pgfsys@defobject{currentmarker}{\pgfqpoint{-0.048611in}{0.000000in}}{\pgfqpoint{-0.000000in}{0.000000in}}{%
\pgfpathmoveto{\pgfqpoint{-0.000000in}{0.000000in}}%
\pgfpathlineto{\pgfqpoint{-0.048611in}{0.000000in}}%
\pgfusepath{stroke,fill}%
}%
\begin{pgfscope}%
\pgfsys@transformshift{0.565124in}{1.089062in}%
\pgfsys@useobject{currentmarker}{}%
\end{pgfscope}%
\end{pgfscope}%
\begin{pgfscope}%
\definecolor{textcolor}{rgb}{0.000000,0.000000,0.000000}%
\pgfsetstrokecolor{textcolor}%
\pgfsetfillcolor{textcolor}%
\pgftext[x=0.398457in, y=1.040837in, left, base]{\color{textcolor}\rmfamily\fontsize{10.000000}{12.000000}\selectfont \(\displaystyle {5}\)}%
\end{pgfscope}%
\begin{pgfscope}%
\pgfsetbuttcap%
\pgfsetroundjoin%
\definecolor{currentfill}{rgb}{0.000000,0.000000,0.000000}%
\pgfsetfillcolor{currentfill}%
\pgfsetlinewidth{0.803000pt}%
\definecolor{currentstroke}{rgb}{0.000000,0.000000,0.000000}%
\pgfsetstrokecolor{currentstroke}%
\pgfsetdash{}{0pt}%
\pgfsys@defobject{currentmarker}{\pgfqpoint{-0.048611in}{0.000000in}}{\pgfqpoint{-0.000000in}{0.000000in}}{%
\pgfpathmoveto{\pgfqpoint{-0.000000in}{0.000000in}}%
\pgfpathlineto{\pgfqpoint{-0.048611in}{0.000000in}}%
\pgfusepath{stroke,fill}%
}%
\begin{pgfscope}%
\pgfsys@transformshift{0.565124in}{1.505692in}%
\pgfsys@useobject{currentmarker}{}%
\end{pgfscope}%
\end{pgfscope}%
\begin{pgfscope}%
\definecolor{textcolor}{rgb}{0.000000,0.000000,0.000000}%
\pgfsetstrokecolor{textcolor}%
\pgfsetfillcolor{textcolor}%
\pgftext[x=0.329012in, y=1.457467in, left, base]{\color{textcolor}\rmfamily\fontsize{10.000000}{12.000000}\selectfont \(\displaystyle {10}\)}%
\end{pgfscope}%
\begin{pgfscope}%
\pgfsetbuttcap%
\pgfsetroundjoin%
\definecolor{currentfill}{rgb}{0.000000,0.000000,0.000000}%
\pgfsetfillcolor{currentfill}%
\pgfsetlinewidth{0.803000pt}%
\definecolor{currentstroke}{rgb}{0.000000,0.000000,0.000000}%
\pgfsetstrokecolor{currentstroke}%
\pgfsetdash{}{0pt}%
\pgfsys@defobject{currentmarker}{\pgfqpoint{-0.048611in}{0.000000in}}{\pgfqpoint{-0.000000in}{0.000000in}}{%
\pgfpathmoveto{\pgfqpoint{-0.000000in}{0.000000in}}%
\pgfpathlineto{\pgfqpoint{-0.048611in}{0.000000in}}%
\pgfusepath{stroke,fill}%
}%
\begin{pgfscope}%
\pgfsys@transformshift{0.565124in}{1.922322in}%
\pgfsys@useobject{currentmarker}{}%
\end{pgfscope}%
\end{pgfscope}%
\begin{pgfscope}%
\definecolor{textcolor}{rgb}{0.000000,0.000000,0.000000}%
\pgfsetstrokecolor{textcolor}%
\pgfsetfillcolor{textcolor}%
\pgftext[x=0.329012in, y=1.874097in, left, base]{\color{textcolor}\rmfamily\fontsize{10.000000}{12.000000}\selectfont \(\displaystyle {15}\)}%
\end{pgfscope}%
\begin{pgfscope}%
\pgfsetbuttcap%
\pgfsetroundjoin%
\definecolor{currentfill}{rgb}{0.000000,0.000000,0.000000}%
\pgfsetfillcolor{currentfill}%
\pgfsetlinewidth{0.803000pt}%
\definecolor{currentstroke}{rgb}{0.000000,0.000000,0.000000}%
\pgfsetstrokecolor{currentstroke}%
\pgfsetdash{}{0pt}%
\pgfsys@defobject{currentmarker}{\pgfqpoint{-0.048611in}{0.000000in}}{\pgfqpoint{-0.000000in}{0.000000in}}{%
\pgfpathmoveto{\pgfqpoint{-0.000000in}{0.000000in}}%
\pgfpathlineto{\pgfqpoint{-0.048611in}{0.000000in}}%
\pgfusepath{stroke,fill}%
}%
\begin{pgfscope}%
\pgfsys@transformshift{0.565124in}{2.338952in}%
\pgfsys@useobject{currentmarker}{}%
\end{pgfscope}%
\end{pgfscope}%
\begin{pgfscope}%
\definecolor{textcolor}{rgb}{0.000000,0.000000,0.000000}%
\pgfsetstrokecolor{textcolor}%
\pgfsetfillcolor{textcolor}%
\pgftext[x=0.329012in, y=2.290727in, left, base]{\color{textcolor}\rmfamily\fontsize{10.000000}{12.000000}\selectfont \(\displaystyle {20}\)}%
\end{pgfscope}%
\begin{pgfscope}%
\pgfsetbuttcap%
\pgfsetroundjoin%
\definecolor{currentfill}{rgb}{0.000000,0.000000,0.000000}%
\pgfsetfillcolor{currentfill}%
\pgfsetlinewidth{0.803000pt}%
\definecolor{currentstroke}{rgb}{0.000000,0.000000,0.000000}%
\pgfsetstrokecolor{currentstroke}%
\pgfsetdash{}{0pt}%
\pgfsys@defobject{currentmarker}{\pgfqpoint{-0.048611in}{0.000000in}}{\pgfqpoint{-0.000000in}{0.000000in}}{%
\pgfpathmoveto{\pgfqpoint{-0.000000in}{0.000000in}}%
\pgfpathlineto{\pgfqpoint{-0.048611in}{0.000000in}}%
\pgfusepath{stroke,fill}%
}%
\begin{pgfscope}%
\pgfsys@transformshift{0.565124in}{2.755582in}%
\pgfsys@useobject{currentmarker}{}%
\end{pgfscope}%
\end{pgfscope}%
\begin{pgfscope}%
\definecolor{textcolor}{rgb}{0.000000,0.000000,0.000000}%
\pgfsetstrokecolor{textcolor}%
\pgfsetfillcolor{textcolor}%
\pgftext[x=0.329012in, y=2.707357in, left, base]{\color{textcolor}\rmfamily\fontsize{10.000000}{12.000000}\selectfont \(\displaystyle {25}\)}%
\end{pgfscope}%
\begin{pgfscope}%
\pgfsetbuttcap%
\pgfsetroundjoin%
\definecolor{currentfill}{rgb}{0.000000,0.000000,0.000000}%
\pgfsetfillcolor{currentfill}%
\pgfsetlinewidth{0.803000pt}%
\definecolor{currentstroke}{rgb}{0.000000,0.000000,0.000000}%
\pgfsetstrokecolor{currentstroke}%
\pgfsetdash{}{0pt}%
\pgfsys@defobject{currentmarker}{\pgfqpoint{-0.048611in}{0.000000in}}{\pgfqpoint{-0.000000in}{0.000000in}}{%
\pgfpathmoveto{\pgfqpoint{-0.000000in}{0.000000in}}%
\pgfpathlineto{\pgfqpoint{-0.048611in}{0.000000in}}%
\pgfusepath{stroke,fill}%
}%
\begin{pgfscope}%
\pgfsys@transformshift{0.565124in}{3.172212in}%
\pgfsys@useobject{currentmarker}{}%
\end{pgfscope}%
\end{pgfscope}%
\begin{pgfscope}%
\definecolor{textcolor}{rgb}{0.000000,0.000000,0.000000}%
\pgfsetstrokecolor{textcolor}%
\pgfsetfillcolor{textcolor}%
\pgftext[x=0.329012in, y=3.123986in, left, base]{\color{textcolor}\rmfamily\fontsize{10.000000}{12.000000}\selectfont \(\displaystyle {30}\)}%
\end{pgfscope}%
\begin{pgfscope}%
\definecolor{textcolor}{rgb}{0.000000,0.000000,0.000000}%
\pgfsetstrokecolor{textcolor}%
\pgfsetfillcolor{textcolor}%
\pgftext[x=0.273457in,y=1.899846in,,bottom,rotate=90.000000]{\color{textcolor}\rmfamily\fontsize{10.000000}{12.000000}\selectfont Differenza percettiva}%
\end{pgfscope}%
\begin{pgfscope}%
\pgfpathrectangle{\pgfqpoint{0.565124in}{0.549691in}}{\pgfqpoint{5.184876in}{2.700309in}}%
\pgfusepath{clip}%
\pgfsetrectcap%
\pgfsetroundjoin%
\pgfsetlinewidth{2.007500pt}%
\definecolor{currentstroke}{rgb}{0.121569,0.466667,0.705882}%
\pgfsetstrokecolor{currentstroke}%
\pgfsetdash{}{0pt}%
\pgfpathmoveto{\pgfqpoint{0.800800in}{3.127259in}}%
\pgfpathlineto{\pgfqpoint{0.830632in}{2.547361in}}%
\pgfpathlineto{\pgfqpoint{0.860465in}{2.301762in}}%
\pgfpathlineto{\pgfqpoint{0.890297in}{2.070902in}}%
\pgfpathlineto{\pgfqpoint{0.920130in}{1.779736in}}%
\pgfpathlineto{\pgfqpoint{0.949962in}{1.678325in}}%
\pgfpathlineto{\pgfqpoint{0.979795in}{1.514068in}}%
\pgfpathlineto{\pgfqpoint{1.009627in}{1.455498in}}%
\pgfpathlineto{\pgfqpoint{1.039459in}{1.376207in}}%
\pgfpathlineto{\pgfqpoint{1.069292in}{1.292514in}}%
\pgfpathlineto{\pgfqpoint{1.099124in}{1.251603in}}%
\pgfpathlineto{\pgfqpoint{1.128957in}{1.125702in}}%
\pgfpathlineto{\pgfqpoint{1.158789in}{1.065341in}}%
\pgfpathlineto{\pgfqpoint{1.188622in}{1.029981in}}%
\pgfpathlineto{\pgfqpoint{1.218454in}{0.989446in}}%
\pgfpathlineto{\pgfqpoint{1.248286in}{0.938726in}}%
\pgfpathlineto{\pgfqpoint{1.278119in}{0.907757in}}%
\pgfpathlineto{\pgfqpoint{1.307951in}{0.868069in}}%
\pgfpathlineto{\pgfqpoint{1.337784in}{0.833323in}}%
\pgfpathlineto{\pgfqpoint{1.367616in}{0.810448in}}%
\pgfpathlineto{\pgfqpoint{1.397449in}{0.762493in}}%
\pgfpathlineto{\pgfqpoint{1.427281in}{0.749261in}}%
\pgfpathlineto{\pgfqpoint{1.457113in}{0.732959in}}%
\pgfpathlineto{\pgfqpoint{1.546611in}{0.702302in}}%
\pgfpathlineto{\pgfqpoint{1.576443in}{0.698231in}}%
\pgfpathlineto{\pgfqpoint{1.606276in}{0.691983in}}%
\pgfpathlineto{\pgfqpoint{1.636108in}{0.687326in}}%
\pgfpathlineto{\pgfqpoint{1.725605in}{0.682748in}}%
\pgfpathlineto{\pgfqpoint{1.785270in}{0.681754in}}%
\pgfpathlineto{\pgfqpoint{1.815103in}{0.680166in}}%
\pgfpathlineto{\pgfqpoint{2.859238in}{0.680101in}}%
\pgfpathlineto{\pgfqpoint{2.889070in}{0.672444in}}%
\pgfpathlineto{\pgfqpoint{5.514324in}{0.672432in}}%
\pgfpathlineto{\pgfqpoint{5.514324in}{0.672432in}}%
\pgfusepath{stroke}%
\end{pgfscope}%
\begin{pgfscope}%
\pgfsetrectcap%
\pgfsetmiterjoin%
\pgfsetlinewidth{0.803000pt}%
\definecolor{currentstroke}{rgb}{0.000000,0.000000,0.000000}%
\pgfsetstrokecolor{currentstroke}%
\pgfsetdash{}{0pt}%
\pgfpathmoveto{\pgfqpoint{0.565124in}{0.549691in}}%
\pgfpathlineto{\pgfqpoint{0.565124in}{3.250000in}}%
\pgfusepath{stroke}%
\end{pgfscope}%
\begin{pgfscope}%
\pgfsetrectcap%
\pgfsetmiterjoin%
\pgfsetlinewidth{0.803000pt}%
\definecolor{currentstroke}{rgb}{0.000000,0.000000,0.000000}%
\pgfsetstrokecolor{currentstroke}%
\pgfsetdash{}{0pt}%
\pgfpathmoveto{\pgfqpoint{5.750000in}{0.549691in}}%
\pgfpathlineto{\pgfqpoint{5.750000in}{3.250000in}}%
\pgfusepath{stroke}%
\end{pgfscope}%
\begin{pgfscope}%
\pgfsetrectcap%
\pgfsetmiterjoin%
\pgfsetlinewidth{0.803000pt}%
\definecolor{currentstroke}{rgb}{0.000000,0.000000,0.000000}%
\pgfsetstrokecolor{currentstroke}%
\pgfsetdash{}{0pt}%
\pgfpathmoveto{\pgfqpoint{0.565124in}{0.549691in}}%
\pgfpathlineto{\pgfqpoint{5.750000in}{0.549691in}}%
\pgfusepath{stroke}%
\end{pgfscope}%
\begin{pgfscope}%
\pgfsetrectcap%
\pgfsetmiterjoin%
\pgfsetlinewidth{0.803000pt}%
\definecolor{currentstroke}{rgb}{0.000000,0.000000,0.000000}%
\pgfsetstrokecolor{currentstroke}%
\pgfsetdash{}{0pt}%
\pgfpathmoveto{\pgfqpoint{0.565124in}{3.250000in}}%
\pgfpathlineto{\pgfqpoint{5.750000in}{3.250000in}}%
\pgfusepath{stroke}%
\end{pgfscope}%
\end{pgfpicture}%
\makeatother%
\endgroup%

            %     \caption{Valutazione Ray Tracing - ombre portate sulla scena fittizia}
            %     \label{fig:eval-raytracing-fit}
            % \end{figure}

            % \begin{figure}[htb!]
            %     \centering
            %     \includegraphics[scale=.55]{images/valutazioni/realistica/28-04-23 03-32-23 RayTracing 2860.4245073910374.png}
            %     \caption{Valutazione Ray tracing - ombre portate sulla scena realistica}
            %     \label{fig:eval-raytracing-re}
            % \end{figure}

            % \begin{figure}[htbp]
            %     \centering
            %     \includegraphics[width=\textwidth]{images/sequences/sequence-rt-re.png}
            %     \par
            %     \vspace{15pt}
            %     \centering
            %     \includegraphics[width=\textwidth]{images/sequences/sequence-rt-fit.png}
            %     \caption{Sequenza di caricamento di Ray Tracing}
            %     \label{fig:seq-rt}
            % \end{figure}

            \begin{figure}[htb!]
                \centering
                %% Creator: Matplotlib, PGF backend
%%
%% To include the figure in your LaTeX document, write
%%   \input{<filename>.pgf}
%%
%% Make sure the required packages are loaded in your preamble
%%   \usepackage{pgf}
%%
%% Figures using additional raster images can only be included by \input if
%% they are in the same directory as the main LaTeX file. For loading figures
%% from other directories you can use the `import` package
%%   \usepackage{import}
%%
%% and then include the figures with
%%   \import{<path to file>}{<filename>.pgf}
%%
%% Matplotlib used the following preamble
%%
\begingroup%
\makeatletter%
\begin{pgfpicture}%
\pgfpathrectangle{\pgfpointorigin}{\pgfqpoint{5.900000in}{3.400000in}}%
\pgfusepath{use as bounding box, clip}%
\begin{pgfscope}%
\pgfsetbuttcap%
\pgfsetmiterjoin%
\definecolor{currentfill}{rgb}{1.000000,1.000000,1.000000}%
\pgfsetfillcolor{currentfill}%
\pgfsetlinewidth{0.000000pt}%
\definecolor{currentstroke}{rgb}{1.000000,1.000000,1.000000}%
\pgfsetstrokecolor{currentstroke}%
\pgfsetdash{}{0pt}%
\pgfpathmoveto{\pgfqpoint{0.000000in}{0.000000in}}%
\pgfpathlineto{\pgfqpoint{5.900000in}{0.000000in}}%
\pgfpathlineto{\pgfqpoint{5.900000in}{3.400000in}}%
\pgfpathlineto{\pgfqpoint{0.000000in}{3.400000in}}%
\pgfpathclose%
\pgfusepath{fill}%
\end{pgfscope}%
\begin{pgfscope}%
\pgfsetbuttcap%
\pgfsetmiterjoin%
\definecolor{currentfill}{rgb}{1.000000,1.000000,1.000000}%
\pgfsetfillcolor{currentfill}%
\pgfsetlinewidth{0.000000pt}%
\definecolor{currentstroke}{rgb}{0.000000,0.000000,0.000000}%
\pgfsetstrokecolor{currentstroke}%
\pgfsetstrokeopacity{0.000000}%
\pgfsetdash{}{0pt}%
\pgfpathmoveto{\pgfqpoint{0.565124in}{0.565123in}}%
\pgfpathlineto{\pgfqpoint{5.750000in}{0.565123in}}%
\pgfpathlineto{\pgfqpoint{5.750000in}{3.250000in}}%
\pgfpathlineto{\pgfqpoint{0.565124in}{3.250000in}}%
\pgfpathclose%
\pgfusepath{fill}%
\end{pgfscope}%
\begin{pgfscope}%
\pgfsetbuttcap%
\pgfsetroundjoin%
\definecolor{currentfill}{rgb}{0.000000,0.000000,0.000000}%
\pgfsetfillcolor{currentfill}%
\pgfsetlinewidth{0.803000pt}%
\definecolor{currentstroke}{rgb}{0.000000,0.000000,0.000000}%
\pgfsetstrokecolor{currentstroke}%
\pgfsetdash{}{0pt}%
\pgfsys@defobject{currentmarker}{\pgfqpoint{0.000000in}{-0.048611in}}{\pgfqpoint{0.000000in}{0.000000in}}{%
\pgfpathmoveto{\pgfqpoint{0.000000in}{0.000000in}}%
\pgfpathlineto{\pgfqpoint{0.000000in}{-0.048611in}}%
\pgfusepath{stroke,fill}%
}%
\begin{pgfscope}%
\pgfsys@transformshift{0.800800in}{0.565123in}%
\pgfsys@useobject{currentmarker}{}%
\end{pgfscope}%
\end{pgfscope}%
\begin{pgfscope}%
\definecolor{textcolor}{rgb}{0.000000,0.000000,0.000000}%
\pgfsetstrokecolor{textcolor}%
\pgfsetfillcolor{textcolor}%
\pgftext[x=0.800800in,y=0.467901in,,top]{\color{textcolor}\rmfamily\fontsize{10.000000}{12.000000}\selectfont \(\displaystyle {0}\)}%
\end{pgfscope}%
\begin{pgfscope}%
\pgfsetbuttcap%
\pgfsetroundjoin%
\definecolor{currentfill}{rgb}{0.000000,0.000000,0.000000}%
\pgfsetfillcolor{currentfill}%
\pgfsetlinewidth{0.803000pt}%
\definecolor{currentstroke}{rgb}{0.000000,0.000000,0.000000}%
\pgfsetstrokecolor{currentstroke}%
\pgfsetdash{}{0pt}%
\pgfsys@defobject{currentmarker}{\pgfqpoint{0.000000in}{-0.048611in}}{\pgfqpoint{0.000000in}{0.000000in}}{%
\pgfpathmoveto{\pgfqpoint{0.000000in}{0.000000in}}%
\pgfpathlineto{\pgfqpoint{0.000000in}{-0.048611in}}%
\pgfusepath{stroke,fill}%
}%
\begin{pgfscope}%
\pgfsys@transformshift{1.799428in}{0.565123in}%
\pgfsys@useobject{currentmarker}{}%
\end{pgfscope}%
\end{pgfscope}%
\begin{pgfscope}%
\definecolor{textcolor}{rgb}{0.000000,0.000000,0.000000}%
\pgfsetstrokecolor{textcolor}%
\pgfsetfillcolor{textcolor}%
\pgftext[x=1.799428in,y=0.467901in,,top]{\color{textcolor}\rmfamily\fontsize{10.000000}{12.000000}\selectfont \(\displaystyle {50}\)}%
\end{pgfscope}%
\begin{pgfscope}%
\pgfsetbuttcap%
\pgfsetroundjoin%
\definecolor{currentfill}{rgb}{0.000000,0.000000,0.000000}%
\pgfsetfillcolor{currentfill}%
\pgfsetlinewidth{0.803000pt}%
\definecolor{currentstroke}{rgb}{0.000000,0.000000,0.000000}%
\pgfsetstrokecolor{currentstroke}%
\pgfsetdash{}{0pt}%
\pgfsys@defobject{currentmarker}{\pgfqpoint{0.000000in}{-0.048611in}}{\pgfqpoint{0.000000in}{0.000000in}}{%
\pgfpathmoveto{\pgfqpoint{0.000000in}{0.000000in}}%
\pgfpathlineto{\pgfqpoint{0.000000in}{-0.048611in}}%
\pgfusepath{stroke,fill}%
}%
\begin{pgfscope}%
\pgfsys@transformshift{2.798056in}{0.565123in}%
\pgfsys@useobject{currentmarker}{}%
\end{pgfscope}%
\end{pgfscope}%
\begin{pgfscope}%
\definecolor{textcolor}{rgb}{0.000000,0.000000,0.000000}%
\pgfsetstrokecolor{textcolor}%
\pgfsetfillcolor{textcolor}%
\pgftext[x=2.798056in,y=0.467901in,,top]{\color{textcolor}\rmfamily\fontsize{10.000000}{12.000000}\selectfont \(\displaystyle {100}\)}%
\end{pgfscope}%
\begin{pgfscope}%
\pgfsetbuttcap%
\pgfsetroundjoin%
\definecolor{currentfill}{rgb}{0.000000,0.000000,0.000000}%
\pgfsetfillcolor{currentfill}%
\pgfsetlinewidth{0.803000pt}%
\definecolor{currentstroke}{rgb}{0.000000,0.000000,0.000000}%
\pgfsetstrokecolor{currentstroke}%
\pgfsetdash{}{0pt}%
\pgfsys@defobject{currentmarker}{\pgfqpoint{0.000000in}{-0.048611in}}{\pgfqpoint{0.000000in}{0.000000in}}{%
\pgfpathmoveto{\pgfqpoint{0.000000in}{0.000000in}}%
\pgfpathlineto{\pgfqpoint{0.000000in}{-0.048611in}}%
\pgfusepath{stroke,fill}%
}%
\begin{pgfscope}%
\pgfsys@transformshift{3.796684in}{0.565123in}%
\pgfsys@useobject{currentmarker}{}%
\end{pgfscope}%
\end{pgfscope}%
\begin{pgfscope}%
\definecolor{textcolor}{rgb}{0.000000,0.000000,0.000000}%
\pgfsetstrokecolor{textcolor}%
\pgfsetfillcolor{textcolor}%
\pgftext[x=3.796684in,y=0.467901in,,top]{\color{textcolor}\rmfamily\fontsize{10.000000}{12.000000}\selectfont \(\displaystyle {150}\)}%
\end{pgfscope}%
\begin{pgfscope}%
\pgfsetbuttcap%
\pgfsetroundjoin%
\definecolor{currentfill}{rgb}{0.000000,0.000000,0.000000}%
\pgfsetfillcolor{currentfill}%
\pgfsetlinewidth{0.803000pt}%
\definecolor{currentstroke}{rgb}{0.000000,0.000000,0.000000}%
\pgfsetstrokecolor{currentstroke}%
\pgfsetdash{}{0pt}%
\pgfsys@defobject{currentmarker}{\pgfqpoint{0.000000in}{-0.048611in}}{\pgfqpoint{0.000000in}{0.000000in}}{%
\pgfpathmoveto{\pgfqpoint{0.000000in}{0.000000in}}%
\pgfpathlineto{\pgfqpoint{0.000000in}{-0.048611in}}%
\pgfusepath{stroke,fill}%
}%
\begin{pgfscope}%
\pgfsys@transformshift{4.795312in}{0.565123in}%
\pgfsys@useobject{currentmarker}{}%
\end{pgfscope}%
\end{pgfscope}%
\begin{pgfscope}%
\definecolor{textcolor}{rgb}{0.000000,0.000000,0.000000}%
\pgfsetstrokecolor{textcolor}%
\pgfsetfillcolor{textcolor}%
\pgftext[x=4.795312in,y=0.467901in,,top]{\color{textcolor}\rmfamily\fontsize{10.000000}{12.000000}\selectfont \(\displaystyle {200}\)}%
\end{pgfscope}%
\begin{pgfscope}%
\definecolor{textcolor}{rgb}{0.000000,0.000000,0.000000}%
\pgfsetstrokecolor{textcolor}%
\pgfsetfillcolor{textcolor}%
\pgftext[x=3.157562in,y=0.288889in,,top]{\color{textcolor}\rmfamily\fontsize{10.000000}{12.000000}\selectfont Tempo (Frame)}%
\end{pgfscope}%
\begin{pgfscope}%
\pgfsetbuttcap%
\pgfsetroundjoin%
\definecolor{currentfill}{rgb}{0.000000,0.000000,0.000000}%
\pgfsetfillcolor{currentfill}%
\pgfsetlinewidth{0.803000pt}%
\definecolor{currentstroke}{rgb}{0.000000,0.000000,0.000000}%
\pgfsetstrokecolor{currentstroke}%
\pgfsetdash{}{0pt}%
\pgfsys@defobject{currentmarker}{\pgfqpoint{-0.048611in}{0.000000in}}{\pgfqpoint{-0.000000in}{0.000000in}}{%
\pgfpathmoveto{\pgfqpoint{-0.000000in}{0.000000in}}%
\pgfpathlineto{\pgfqpoint{-0.048611in}{0.000000in}}%
\pgfusepath{stroke,fill}%
}%
\begin{pgfscope}%
\pgfsys@transformshift{0.565124in}{0.687163in}%
\pgfsys@useobject{currentmarker}{}%
\end{pgfscope}%
\end{pgfscope}%
\begin{pgfscope}%
\definecolor{textcolor}{rgb}{0.000000,0.000000,0.000000}%
\pgfsetstrokecolor{textcolor}%
\pgfsetfillcolor{textcolor}%
\pgftext[x=0.398457in, y=0.638938in, left, base]{\color{textcolor}\rmfamily\fontsize{10.000000}{12.000000}\selectfont \(\displaystyle {0}\)}%
\end{pgfscope}%
\begin{pgfscope}%
\pgfsetbuttcap%
\pgfsetroundjoin%
\definecolor{currentfill}{rgb}{0.000000,0.000000,0.000000}%
\pgfsetfillcolor{currentfill}%
\pgfsetlinewidth{0.803000pt}%
\definecolor{currentstroke}{rgb}{0.000000,0.000000,0.000000}%
\pgfsetstrokecolor{currentstroke}%
\pgfsetdash{}{0pt}%
\pgfsys@defobject{currentmarker}{\pgfqpoint{-0.048611in}{0.000000in}}{\pgfqpoint{-0.000000in}{0.000000in}}{%
\pgfpathmoveto{\pgfqpoint{-0.000000in}{0.000000in}}%
\pgfpathlineto{\pgfqpoint{-0.048611in}{0.000000in}}%
\pgfusepath{stroke,fill}%
}%
\begin{pgfscope}%
\pgfsys@transformshift{0.565124in}{0.999067in}%
\pgfsys@useobject{currentmarker}{}%
\end{pgfscope}%
\end{pgfscope}%
\begin{pgfscope}%
\definecolor{textcolor}{rgb}{0.000000,0.000000,0.000000}%
\pgfsetstrokecolor{textcolor}%
\pgfsetfillcolor{textcolor}%
\pgftext[x=0.398457in, y=0.950842in, left, base]{\color{textcolor}\rmfamily\fontsize{10.000000}{12.000000}\selectfont \(\displaystyle {5}\)}%
\end{pgfscope}%
\begin{pgfscope}%
\pgfsetbuttcap%
\pgfsetroundjoin%
\definecolor{currentfill}{rgb}{0.000000,0.000000,0.000000}%
\pgfsetfillcolor{currentfill}%
\pgfsetlinewidth{0.803000pt}%
\definecolor{currentstroke}{rgb}{0.000000,0.000000,0.000000}%
\pgfsetstrokecolor{currentstroke}%
\pgfsetdash{}{0pt}%
\pgfsys@defobject{currentmarker}{\pgfqpoint{-0.048611in}{0.000000in}}{\pgfqpoint{-0.000000in}{0.000000in}}{%
\pgfpathmoveto{\pgfqpoint{-0.000000in}{0.000000in}}%
\pgfpathlineto{\pgfqpoint{-0.048611in}{0.000000in}}%
\pgfusepath{stroke,fill}%
}%
\begin{pgfscope}%
\pgfsys@transformshift{0.565124in}{1.310971in}%
\pgfsys@useobject{currentmarker}{}%
\end{pgfscope}%
\end{pgfscope}%
\begin{pgfscope}%
\definecolor{textcolor}{rgb}{0.000000,0.000000,0.000000}%
\pgfsetstrokecolor{textcolor}%
\pgfsetfillcolor{textcolor}%
\pgftext[x=0.329012in, y=1.262746in, left, base]{\color{textcolor}\rmfamily\fontsize{10.000000}{12.000000}\selectfont \(\displaystyle {10}\)}%
\end{pgfscope}%
\begin{pgfscope}%
\pgfsetbuttcap%
\pgfsetroundjoin%
\definecolor{currentfill}{rgb}{0.000000,0.000000,0.000000}%
\pgfsetfillcolor{currentfill}%
\pgfsetlinewidth{0.803000pt}%
\definecolor{currentstroke}{rgb}{0.000000,0.000000,0.000000}%
\pgfsetstrokecolor{currentstroke}%
\pgfsetdash{}{0pt}%
\pgfsys@defobject{currentmarker}{\pgfqpoint{-0.048611in}{0.000000in}}{\pgfqpoint{-0.000000in}{0.000000in}}{%
\pgfpathmoveto{\pgfqpoint{-0.000000in}{0.000000in}}%
\pgfpathlineto{\pgfqpoint{-0.048611in}{0.000000in}}%
\pgfusepath{stroke,fill}%
}%
\begin{pgfscope}%
\pgfsys@transformshift{0.565124in}{1.622875in}%
\pgfsys@useobject{currentmarker}{}%
\end{pgfscope}%
\end{pgfscope}%
\begin{pgfscope}%
\definecolor{textcolor}{rgb}{0.000000,0.000000,0.000000}%
\pgfsetstrokecolor{textcolor}%
\pgfsetfillcolor{textcolor}%
\pgftext[x=0.329012in, y=1.574650in, left, base]{\color{textcolor}\rmfamily\fontsize{10.000000}{12.000000}\selectfont \(\displaystyle {15}\)}%
\end{pgfscope}%
\begin{pgfscope}%
\pgfsetbuttcap%
\pgfsetroundjoin%
\definecolor{currentfill}{rgb}{0.000000,0.000000,0.000000}%
\pgfsetfillcolor{currentfill}%
\pgfsetlinewidth{0.803000pt}%
\definecolor{currentstroke}{rgb}{0.000000,0.000000,0.000000}%
\pgfsetstrokecolor{currentstroke}%
\pgfsetdash{}{0pt}%
\pgfsys@defobject{currentmarker}{\pgfqpoint{-0.048611in}{0.000000in}}{\pgfqpoint{-0.000000in}{0.000000in}}{%
\pgfpathmoveto{\pgfqpoint{-0.000000in}{0.000000in}}%
\pgfpathlineto{\pgfqpoint{-0.048611in}{0.000000in}}%
\pgfusepath{stroke,fill}%
}%
\begin{pgfscope}%
\pgfsys@transformshift{0.565124in}{1.934779in}%
\pgfsys@useobject{currentmarker}{}%
\end{pgfscope}%
\end{pgfscope}%
\begin{pgfscope}%
\definecolor{textcolor}{rgb}{0.000000,0.000000,0.000000}%
\pgfsetstrokecolor{textcolor}%
\pgfsetfillcolor{textcolor}%
\pgftext[x=0.329012in, y=1.886554in, left, base]{\color{textcolor}\rmfamily\fontsize{10.000000}{12.000000}\selectfont \(\displaystyle {20}\)}%
\end{pgfscope}%
\begin{pgfscope}%
\pgfsetbuttcap%
\pgfsetroundjoin%
\definecolor{currentfill}{rgb}{0.000000,0.000000,0.000000}%
\pgfsetfillcolor{currentfill}%
\pgfsetlinewidth{0.803000pt}%
\definecolor{currentstroke}{rgb}{0.000000,0.000000,0.000000}%
\pgfsetstrokecolor{currentstroke}%
\pgfsetdash{}{0pt}%
\pgfsys@defobject{currentmarker}{\pgfqpoint{-0.048611in}{0.000000in}}{\pgfqpoint{-0.000000in}{0.000000in}}{%
\pgfpathmoveto{\pgfqpoint{-0.000000in}{0.000000in}}%
\pgfpathlineto{\pgfqpoint{-0.048611in}{0.000000in}}%
\pgfusepath{stroke,fill}%
}%
\begin{pgfscope}%
\pgfsys@transformshift{0.565124in}{2.246684in}%
\pgfsys@useobject{currentmarker}{}%
\end{pgfscope}%
\end{pgfscope}%
\begin{pgfscope}%
\definecolor{textcolor}{rgb}{0.000000,0.000000,0.000000}%
\pgfsetstrokecolor{textcolor}%
\pgfsetfillcolor{textcolor}%
\pgftext[x=0.329012in, y=2.198458in, left, base]{\color{textcolor}\rmfamily\fontsize{10.000000}{12.000000}\selectfont \(\displaystyle {25}\)}%
\end{pgfscope}%
\begin{pgfscope}%
\pgfsetbuttcap%
\pgfsetroundjoin%
\definecolor{currentfill}{rgb}{0.000000,0.000000,0.000000}%
\pgfsetfillcolor{currentfill}%
\pgfsetlinewidth{0.803000pt}%
\definecolor{currentstroke}{rgb}{0.000000,0.000000,0.000000}%
\pgfsetstrokecolor{currentstroke}%
\pgfsetdash{}{0pt}%
\pgfsys@defobject{currentmarker}{\pgfqpoint{-0.048611in}{0.000000in}}{\pgfqpoint{-0.000000in}{0.000000in}}{%
\pgfpathmoveto{\pgfqpoint{-0.000000in}{0.000000in}}%
\pgfpathlineto{\pgfqpoint{-0.048611in}{0.000000in}}%
\pgfusepath{stroke,fill}%
}%
\begin{pgfscope}%
\pgfsys@transformshift{0.565124in}{2.558588in}%
\pgfsys@useobject{currentmarker}{}%
\end{pgfscope}%
\end{pgfscope}%
\begin{pgfscope}%
\definecolor{textcolor}{rgb}{0.000000,0.000000,0.000000}%
\pgfsetstrokecolor{textcolor}%
\pgfsetfillcolor{textcolor}%
\pgftext[x=0.329012in, y=2.510362in, left, base]{\color{textcolor}\rmfamily\fontsize{10.000000}{12.000000}\selectfont \(\displaystyle {30}\)}%
\end{pgfscope}%
\begin{pgfscope}%
\pgfsetbuttcap%
\pgfsetroundjoin%
\definecolor{currentfill}{rgb}{0.000000,0.000000,0.000000}%
\pgfsetfillcolor{currentfill}%
\pgfsetlinewidth{0.803000pt}%
\definecolor{currentstroke}{rgb}{0.000000,0.000000,0.000000}%
\pgfsetstrokecolor{currentstroke}%
\pgfsetdash{}{0pt}%
\pgfsys@defobject{currentmarker}{\pgfqpoint{-0.048611in}{0.000000in}}{\pgfqpoint{-0.000000in}{0.000000in}}{%
\pgfpathmoveto{\pgfqpoint{-0.000000in}{0.000000in}}%
\pgfpathlineto{\pgfqpoint{-0.048611in}{0.000000in}}%
\pgfusepath{stroke,fill}%
}%
\begin{pgfscope}%
\pgfsys@transformshift{0.565124in}{2.870492in}%
\pgfsys@useobject{currentmarker}{}%
\end{pgfscope}%
\end{pgfscope}%
\begin{pgfscope}%
\definecolor{textcolor}{rgb}{0.000000,0.000000,0.000000}%
\pgfsetstrokecolor{textcolor}%
\pgfsetfillcolor{textcolor}%
\pgftext[x=0.329012in, y=2.822266in, left, base]{\color{textcolor}\rmfamily\fontsize{10.000000}{12.000000}\selectfont \(\displaystyle {35}\)}%
\end{pgfscope}%
\begin{pgfscope}%
\pgfsetbuttcap%
\pgfsetroundjoin%
\definecolor{currentfill}{rgb}{0.000000,0.000000,0.000000}%
\pgfsetfillcolor{currentfill}%
\pgfsetlinewidth{0.803000pt}%
\definecolor{currentstroke}{rgb}{0.000000,0.000000,0.000000}%
\pgfsetstrokecolor{currentstroke}%
\pgfsetdash{}{0pt}%
\pgfsys@defobject{currentmarker}{\pgfqpoint{-0.048611in}{0.000000in}}{\pgfqpoint{-0.000000in}{0.000000in}}{%
\pgfpathmoveto{\pgfqpoint{-0.000000in}{0.000000in}}%
\pgfpathlineto{\pgfqpoint{-0.048611in}{0.000000in}}%
\pgfusepath{stroke,fill}%
}%
\begin{pgfscope}%
\pgfsys@transformshift{0.565124in}{3.182396in}%
\pgfsys@useobject{currentmarker}{}%
\end{pgfscope}%
\end{pgfscope}%
\begin{pgfscope}%
\definecolor{textcolor}{rgb}{0.000000,0.000000,0.000000}%
\pgfsetstrokecolor{textcolor}%
\pgfsetfillcolor{textcolor}%
\pgftext[x=0.329012in, y=3.134170in, left, base]{\color{textcolor}\rmfamily\fontsize{10.000000}{12.000000}\selectfont \(\displaystyle {40}\)}%
\end{pgfscope}%
\begin{pgfscope}%
\definecolor{textcolor}{rgb}{0.000000,0.000000,0.000000}%
\pgfsetstrokecolor{textcolor}%
\pgfsetfillcolor{textcolor}%
\pgftext[x=0.273457in,y=1.907562in,,bottom,rotate=90.000000]{\color{textcolor}\rmfamily\fontsize{10.000000}{12.000000}\selectfont Differenza percettiva}%
\end{pgfscope}%
\begin{pgfscope}%
\pgfpathrectangle{\pgfqpoint{0.565124in}{0.565123in}}{\pgfqpoint{5.184876in}{2.684877in}}%
\pgfusepath{clip}%
\pgfsetrectcap%
\pgfsetroundjoin%
\pgfsetlinewidth{2.007500pt}%
\definecolor{currentstroke}{rgb}{0.121569,0.466667,0.705882}%
\pgfsetstrokecolor{currentstroke}%
\pgfsetdash{}{0pt}%
\pgfpathmoveto{\pgfqpoint{0.800800in}{2.524934in}}%
\pgfpathlineto{\pgfqpoint{0.820773in}{2.090802in}}%
\pgfpathlineto{\pgfqpoint{0.840745in}{1.906938in}}%
\pgfpathlineto{\pgfqpoint{0.860718in}{1.734108in}}%
\pgfpathlineto{\pgfqpoint{0.880690in}{1.516130in}}%
\pgfpathlineto{\pgfqpoint{0.900663in}{1.440210in}}%
\pgfpathlineto{\pgfqpoint{0.920635in}{1.317242in}}%
\pgfpathlineto{\pgfqpoint{0.940608in}{1.273394in}}%
\pgfpathlineto{\pgfqpoint{0.980553in}{1.151378in}}%
\pgfpathlineto{\pgfqpoint{1.000526in}{1.120751in}}%
\pgfpathlineto{\pgfqpoint{1.020498in}{1.026497in}}%
\pgfpathlineto{\pgfqpoint{1.040471in}{0.981309in}}%
\pgfpathlineto{\pgfqpoint{1.060443in}{0.954837in}}%
\pgfpathlineto{\pgfqpoint{1.080416in}{0.924491in}}%
\pgfpathlineto{\pgfqpoint{1.100388in}{0.886520in}}%
\pgfpathlineto{\pgfqpoint{1.120361in}{0.863336in}}%
\pgfpathlineto{\pgfqpoint{1.140333in}{0.833624in}}%
\pgfpathlineto{\pgfqpoint{1.160306in}{0.807611in}}%
\pgfpathlineto{\pgfqpoint{1.180279in}{0.790487in}}%
\pgfpathlineto{\pgfqpoint{1.200251in}{0.754586in}}%
\pgfpathlineto{\pgfqpoint{1.220224in}{0.744680in}}%
\pgfpathlineto{\pgfqpoint{1.240196in}{0.732476in}}%
\pgfpathlineto{\pgfqpoint{1.300114in}{0.709525in}}%
\pgfpathlineto{\pgfqpoint{1.320086in}{0.706477in}}%
\pgfpathlineto{\pgfqpoint{1.340059in}{0.701799in}}%
\pgfpathlineto{\pgfqpoint{1.360032in}{0.698313in}}%
\pgfpathlineto{\pgfqpoint{1.419949in}{0.694886in}}%
\pgfpathlineto{\pgfqpoint{1.599702in}{0.692950in}}%
\pgfpathlineto{\pgfqpoint{2.178906in}{0.692904in}}%
\pgfpathlineto{\pgfqpoint{2.198879in}{0.687172in}}%
\pgfpathlineto{\pgfqpoint{5.494351in}{0.687163in}}%
\pgfpathlineto{\pgfqpoint{5.494351in}{0.687163in}}%
\pgfusepath{stroke}%
\end{pgfscope}%
\begin{pgfscope}%
\pgfpathrectangle{\pgfqpoint{0.565124in}{0.565123in}}{\pgfqpoint{5.184876in}{2.684877in}}%
\pgfusepath{clip}%
\pgfsetrectcap%
\pgfsetroundjoin%
\pgfsetlinewidth{2.007500pt}%
\definecolor{currentstroke}{rgb}{1.000000,0.498039,0.054902}%
\pgfsetstrokecolor{currentstroke}%
\pgfsetdash{}{0pt}%
\pgfpathmoveto{\pgfqpoint{0.820773in}{3.127960in}}%
\pgfpathlineto{\pgfqpoint{0.880690in}{3.127960in}}%
\pgfpathlineto{\pgfqpoint{0.900663in}{3.063141in}}%
\pgfpathlineto{\pgfqpoint{0.920635in}{2.574733in}}%
\pgfpathlineto{\pgfqpoint{0.940608in}{2.507245in}}%
\pgfpathlineto{\pgfqpoint{1.020498in}{2.507245in}}%
\pgfpathlineto{\pgfqpoint{1.040471in}{2.495573in}}%
\pgfpathlineto{\pgfqpoint{1.120361in}{2.495573in}}%
\pgfpathlineto{\pgfqpoint{1.140333in}{2.344408in}}%
\pgfpathlineto{\pgfqpoint{1.160306in}{2.344408in}}%
\pgfpathlineto{\pgfqpoint{1.180279in}{2.319286in}}%
\pgfpathlineto{\pgfqpoint{1.200251in}{2.298776in}}%
\pgfpathlineto{\pgfqpoint{1.280141in}{2.298776in}}%
\pgfpathlineto{\pgfqpoint{1.300114in}{2.292041in}}%
\pgfpathlineto{\pgfqpoint{1.320086in}{2.252226in}}%
\pgfpathlineto{\pgfqpoint{1.419949in}{2.252222in}}%
\pgfpathlineto{\pgfqpoint{1.439922in}{2.247107in}}%
\pgfpathlineto{\pgfqpoint{1.619675in}{2.247101in}}%
\pgfpathlineto{\pgfqpoint{1.639647in}{2.229765in}}%
\pgfpathlineto{\pgfqpoint{1.719538in}{2.229769in}}%
\pgfpathlineto{\pgfqpoint{1.739510in}{2.225207in}}%
\pgfpathlineto{\pgfqpoint{1.999153in}{2.225212in}}%
\pgfpathlineto{\pgfqpoint{2.019126in}{2.223867in}}%
\pgfpathlineto{\pgfqpoint{2.039099in}{2.218884in}}%
\pgfpathlineto{\pgfqpoint{2.099016in}{2.218879in}}%
\pgfpathlineto{\pgfqpoint{2.118989in}{2.196587in}}%
\pgfpathlineto{\pgfqpoint{2.318714in}{2.196594in}}%
\pgfpathlineto{\pgfqpoint{2.338687in}{2.193603in}}%
\pgfpathlineto{\pgfqpoint{2.358660in}{1.739626in}}%
\pgfpathlineto{\pgfqpoint{2.378632in}{1.736838in}}%
\pgfpathlineto{\pgfqpoint{2.398605in}{1.722055in}}%
\pgfpathlineto{\pgfqpoint{2.418577in}{1.721865in}}%
\pgfpathlineto{\pgfqpoint{2.438550in}{1.718159in}}%
\pgfpathlineto{\pgfqpoint{2.518440in}{1.718161in}}%
\pgfpathlineto{\pgfqpoint{2.538413in}{1.707136in}}%
\pgfpathlineto{\pgfqpoint{2.618303in}{1.707140in}}%
\pgfpathlineto{\pgfqpoint{2.638275in}{1.617504in}}%
\pgfpathlineto{\pgfqpoint{2.718166in}{1.617498in}}%
\pgfpathlineto{\pgfqpoint{2.758111in}{1.593837in}}%
\pgfpathlineto{\pgfqpoint{2.778083in}{1.560893in}}%
\pgfpathlineto{\pgfqpoint{2.798056in}{1.555603in}}%
\pgfpathlineto{\pgfqpoint{2.838001in}{1.548142in}}%
\pgfpathlineto{\pgfqpoint{2.857973in}{1.542353in}}%
\pgfpathlineto{\pgfqpoint{2.937864in}{1.542187in}}%
\pgfpathlineto{\pgfqpoint{2.957836in}{1.530936in}}%
\pgfpathlineto{\pgfqpoint{2.997781in}{1.510922in}}%
\pgfpathlineto{\pgfqpoint{3.017754in}{1.510355in}}%
\pgfpathlineto{\pgfqpoint{3.037727in}{1.491130in}}%
\pgfpathlineto{\pgfqpoint{3.057699in}{1.491129in}}%
\pgfpathlineto{\pgfqpoint{3.077672in}{1.475418in}}%
\pgfpathlineto{\pgfqpoint{3.097644in}{1.467319in}}%
\pgfpathlineto{\pgfqpoint{3.117617in}{1.463231in}}%
\pgfpathlineto{\pgfqpoint{3.137589in}{1.426648in}}%
\pgfpathlineto{\pgfqpoint{3.217480in}{1.426637in}}%
\pgfpathlineto{\pgfqpoint{3.237452in}{1.422627in}}%
\pgfpathlineto{\pgfqpoint{3.277397in}{1.422634in}}%
\pgfpathlineto{\pgfqpoint{3.297370in}{1.388010in}}%
\pgfpathlineto{\pgfqpoint{3.317342in}{1.379338in}}%
\pgfpathlineto{\pgfqpoint{3.377260in}{1.379320in}}%
\pgfpathlineto{\pgfqpoint{3.397233in}{1.130760in}}%
\pgfpathlineto{\pgfqpoint{3.417205in}{1.110742in}}%
\pgfpathlineto{\pgfqpoint{3.437178in}{1.110668in}}%
\pgfpathlineto{\pgfqpoint{3.457150in}{0.978853in}}%
\pgfpathlineto{\pgfqpoint{3.477123in}{0.963034in}}%
\pgfpathlineto{\pgfqpoint{3.616931in}{0.963028in}}%
\pgfpathlineto{\pgfqpoint{3.636903in}{0.952544in}}%
\pgfpathlineto{\pgfqpoint{3.656876in}{0.952543in}}%
\pgfpathlineto{\pgfqpoint{3.676848in}{0.870533in}}%
\pgfpathlineto{\pgfqpoint{3.696821in}{0.804642in}}%
\pgfpathlineto{\pgfqpoint{3.796684in}{0.804608in}}%
\pgfpathlineto{\pgfqpoint{3.816656in}{0.778153in}}%
\pgfpathlineto{\pgfqpoint{3.836629in}{0.778158in}}%
\pgfpathlineto{\pgfqpoint{3.856601in}{0.762484in}}%
\pgfpathlineto{\pgfqpoint{3.876574in}{0.723260in}}%
\pgfpathlineto{\pgfqpoint{4.355915in}{0.722413in}}%
\pgfpathlineto{\pgfqpoint{4.375888in}{0.687189in}}%
\pgfpathlineto{\pgfqpoint{5.514324in}{0.687163in}}%
\pgfpathlineto{\pgfqpoint{5.514324in}{0.687163in}}%
\pgfusepath{stroke}%
\end{pgfscope}%
\begin{pgfscope}%
\pgfsetrectcap%
\pgfsetmiterjoin%
\pgfsetlinewidth{0.803000pt}%
\definecolor{currentstroke}{rgb}{0.000000,0.000000,0.000000}%
\pgfsetstrokecolor{currentstroke}%
\pgfsetdash{}{0pt}%
\pgfpathmoveto{\pgfqpoint{0.565124in}{0.565123in}}%
\pgfpathlineto{\pgfqpoint{0.565124in}{3.250000in}}%
\pgfusepath{stroke}%
\end{pgfscope}%
\begin{pgfscope}%
\pgfsetrectcap%
\pgfsetmiterjoin%
\pgfsetlinewidth{0.803000pt}%
\definecolor{currentstroke}{rgb}{0.000000,0.000000,0.000000}%
\pgfsetstrokecolor{currentstroke}%
\pgfsetdash{}{0pt}%
\pgfpathmoveto{\pgfqpoint{5.750000in}{0.565123in}}%
\pgfpathlineto{\pgfqpoint{5.750000in}{3.250000in}}%
\pgfusepath{stroke}%
\end{pgfscope}%
\begin{pgfscope}%
\pgfsetrectcap%
\pgfsetmiterjoin%
\pgfsetlinewidth{0.803000pt}%
\definecolor{currentstroke}{rgb}{0.000000,0.000000,0.000000}%
\pgfsetstrokecolor{currentstroke}%
\pgfsetdash{}{0pt}%
\pgfpathmoveto{\pgfqpoint{0.565124in}{0.565123in}}%
\pgfpathlineto{\pgfqpoint{5.750000in}{0.565123in}}%
\pgfusepath{stroke}%
\end{pgfscope}%
\begin{pgfscope}%
\pgfsetrectcap%
\pgfsetmiterjoin%
\pgfsetlinewidth{0.803000pt}%
\definecolor{currentstroke}{rgb}{0.000000,0.000000,0.000000}%
\pgfsetstrokecolor{currentstroke}%
\pgfsetdash{}{0pt}%
\pgfpathmoveto{\pgfqpoint{0.565124in}{3.250000in}}%
\pgfpathlineto{\pgfqpoint{5.750000in}{3.250000in}}%
\pgfusepath{stroke}%
\end{pgfscope}%
\begin{pgfscope}%
\pgfsetbuttcap%
\pgfsetmiterjoin%
\definecolor{currentfill}{rgb}{1.000000,1.000000,1.000000}%
\pgfsetfillcolor{currentfill}%
\pgfsetfillopacity{0.800000}%
\pgfsetlinewidth{1.003750pt}%
\definecolor{currentstroke}{rgb}{0.800000,0.800000,0.800000}%
\pgfsetstrokecolor{currentstroke}%
\pgfsetstrokeopacity{0.800000}%
\pgfsetdash{}{0pt}%
\pgfpathmoveto{\pgfqpoint{4.273532in}{2.751543in}}%
\pgfpathlineto{\pgfqpoint{5.652778in}{2.751543in}}%
\pgfpathquadraticcurveto{\pgfqpoint{5.680556in}{2.751543in}}{\pgfqpoint{5.680556in}{2.779321in}}%
\pgfpathlineto{\pgfqpoint{5.680556in}{3.152778in}}%
\pgfpathquadraticcurveto{\pgfqpoint{5.680556in}{3.180556in}}{\pgfqpoint{5.652778in}{3.180556in}}%
\pgfpathlineto{\pgfqpoint{4.273532in}{3.180556in}}%
\pgfpathquadraticcurveto{\pgfqpoint{4.245755in}{3.180556in}}{\pgfqpoint{4.245755in}{3.152778in}}%
\pgfpathlineto{\pgfqpoint{4.245755in}{2.779321in}}%
\pgfpathquadraticcurveto{\pgfqpoint{4.245755in}{2.751543in}}{\pgfqpoint{4.273532in}{2.751543in}}%
\pgfpathclose%
\pgfusepath{stroke,fill}%
\end{pgfscope}%
\begin{pgfscope}%
\pgfsetrectcap%
\pgfsetroundjoin%
\pgfsetlinewidth{2.007500pt}%
\definecolor{currentstroke}{rgb}{0.121569,0.466667,0.705882}%
\pgfsetstrokecolor{currentstroke}%
\pgfsetdash{}{0pt}%
\pgfpathmoveto{\pgfqpoint{4.301310in}{3.076389in}}%
\pgfpathlineto{\pgfqpoint{4.579088in}{3.076389in}}%
\pgfusepath{stroke}%
\end{pgfscope}%
\begin{pgfscope}%
\definecolor{textcolor}{rgb}{0.000000,0.000000,0.000000}%
\pgfsetstrokecolor{textcolor}%
\pgfsetfillcolor{textcolor}%
\pgftext[x=4.690199in,y=3.027778in,left,base]{\color{textcolor}\rmfamily\fontsize{10.000000}{12.000000}\selectfont Scena fittizia}%
\end{pgfscope}%
\begin{pgfscope}%
\pgfsetrectcap%
\pgfsetroundjoin%
\pgfsetlinewidth{2.007500pt}%
\definecolor{currentstroke}{rgb}{1.000000,0.498039,0.054902}%
\pgfsetstrokecolor{currentstroke}%
\pgfsetdash{}{0pt}%
\pgfpathmoveto{\pgfqpoint{4.301310in}{2.882716in}}%
\pgfpathlineto{\pgfqpoint{4.579088in}{2.882716in}}%
\pgfusepath{stroke}%
\end{pgfscope}%
\begin{pgfscope}%
\definecolor{textcolor}{rgb}{0.000000,0.000000,0.000000}%
\pgfsetstrokecolor{textcolor}%
\pgfsetfillcolor{textcolor}%
\pgftext[x=4.690199in,y=2.834105in,left,base]{\color{textcolor}\rmfamily\fontsize{10.000000}{12.000000}\selectfont Scena realistica}%
\end{pgfscope}%
\end{pgfpicture}%
\makeatother%
\endgroup%

                \caption{Valutazione Ray Tracing - Ombre portate}
                \label{fig:eval-rt}
            \end{figure}         


    \begin{sidewaysfigure}
        \centering
        \includegraphics[width=\textheight,height=.70793\textwidth]{images/valutazioni/eval-fittizia-cluster.png}
        \caption{Sequenze di caricamento della scena fittizia utilizzando le strategie proposte}
        \label{fig:fittizia-cluster}
    \end{sidewaysfigure}

    \begin{sidewaysfigure}
        \centering
        \includegraphics[width=\textheight,height=.71123834149044233077846523224674\textwidth]{images/valutazioni/eval-realistica-cluster.png}
        \caption{Sequenze di caricamento della scena realistica utilizzando le strategie proposte}
        \label{fig:realistica-cluster}
    \end{sidewaysfigure}
    % le valutazioni messe a confronto: ray tracing wins
        \chapter{Conclusioni}    
        % Breve. Alcune cose che si possono dire sono:
        
        % Come è valutata, in totale, la stategia dallo strumento di assessment? 
        
        % E' utilizzata la strategia di prioritizzazione dal gioco per il qualeera stata pensata? perché

        Delle strategie proposte \textit{Ray Tracing - Ombre portate} si pone come la migliore valutata dallo strumento preposto e la più consistente sotto l'aspetto dell'angolo di provenienza della luce ma, la versione implementata, segue una singola fonte luminosa. Nonostante ciò, questa metodologia può essere estesa a molteplici fonti pagando un costo di tempo di calcolo che può essere trascurato data l'assenza di necessità di essere computato in tempi brevi.
        
        Lo strumento valutativo propone delle valutazioni ragionevoli e ben supportate, rendendolo utilizzabile anche in diversi contesti essendo indipendente dalle strategie di prioritizzazione.

        Delle strategie proposte si propone un confronto nelle figure \ref{fig:conclusioni-fittizia} e \ref{fig:conclusioni-realistica} le quali mostrano la totale differenza percettiva di ogni strategia per la scena fittizia e quella realistica, mettendo alla luce i miglioramenti portati dalle intuizioni del capitolo \ref{cap:strategie}.

        % Le strategie proposte possono essere implementate in un vero contesto di produzione per 
        \begin{figure}
                \centering
                %% Creator: Matplotlib, PGF backend
%%
%% To include the figure in your LaTeX document, write
%%   \input{<filename>.pgf}
%%
%% Make sure the required packages are loaded in your preamble
%%   \usepackage{pgf}
%%
%% Figures using additional raster images can only be included by \input if
%% they are in the same directory as the main LaTeX file. For loading figures
%% from other directories you can use the `import` package
%%   \usepackage{import}
%%
%% and then include the figures with
%%   \import{<path to file>}{<filename>.pgf}
%%
%% Matplotlib used the following preamble
%%
\begingroup%
\makeatletter%
\begin{pgfpicture}%
\pgfpathrectangle{\pgfpointorigin}{\pgfqpoint{5.900000in}{3.350000in}}%
\pgfusepath{use as bounding box, clip}%
\begin{pgfscope}%
\pgfsetbuttcap%
\pgfsetmiterjoin%
\definecolor{currentfill}{rgb}{1.000000,1.000000,1.000000}%
\pgfsetfillcolor{currentfill}%
\pgfsetlinewidth{0.000000pt}%
\definecolor{currentstroke}{rgb}{1.000000,1.000000,1.000000}%
\pgfsetstrokecolor{currentstroke}%
\pgfsetdash{}{0pt}%
\pgfpathmoveto{\pgfqpoint{0.000000in}{0.000000in}}%
\pgfpathlineto{\pgfqpoint{5.900000in}{0.000000in}}%
\pgfpathlineto{\pgfqpoint{5.900000in}{3.350000in}}%
\pgfpathlineto{\pgfqpoint{0.000000in}{3.350000in}}%
\pgfpathclose%
\pgfusepath{fill}%
\end{pgfscope}%
\begin{pgfscope}%
\pgfsetbuttcap%
\pgfsetmiterjoin%
\definecolor{currentfill}{rgb}{1.000000,1.000000,1.000000}%
\pgfsetfillcolor{currentfill}%
\pgfsetlinewidth{0.000000pt}%
\definecolor{currentstroke}{rgb}{0.000000,0.000000,0.000000}%
\pgfsetstrokecolor{currentstroke}%
\pgfsetstrokeopacity{0.000000}%
\pgfsetdash{}{0pt}%
\pgfpathmoveto{\pgfqpoint{0.525001in}{0.513426in}}%
\pgfpathlineto{\pgfqpoint{5.750000in}{0.513426in}}%
\pgfpathlineto{\pgfqpoint{5.750000in}{3.000926in}}%
\pgfpathlineto{\pgfqpoint{0.525001in}{3.000926in}}%
\pgfpathclose%
\pgfusepath{fill}%
\end{pgfscope}%
\begin{pgfscope}%
\pgfpathrectangle{\pgfqpoint{0.525001in}{0.513426in}}{\pgfqpoint{5.224999in}{2.487501in}}%
\pgfusepath{clip}%
\pgfsetbuttcap%
\pgfsetmiterjoin%
\definecolor{currentfill}{rgb}{0.023529,0.482353,0.760784}%
\pgfsetfillcolor{currentfill}%
\pgfsetlinewidth{0.000000pt}%
\definecolor{currentstroke}{rgb}{0.000000,0.000000,0.000000}%
\pgfsetstrokecolor{currentstroke}%
\pgfsetstrokeopacity{0.000000}%
\pgfsetdash{}{0pt}%
\pgfpathmoveto{\pgfqpoint{0.762501in}{0.513426in}}%
\pgfpathlineto{\pgfqpoint{1.290278in}{0.513426in}}%
\pgfpathlineto{\pgfqpoint{1.290278in}{2.882474in}}%
\pgfpathlineto{\pgfqpoint{0.762501in}{2.882474in}}%
\pgfpathclose%
\pgfusepath{fill}%
\end{pgfscope}%
\begin{pgfscope}%
\pgfpathrectangle{\pgfqpoint{0.525001in}{0.513426in}}{\pgfqpoint{5.224999in}{2.487501in}}%
\pgfusepath{clip}%
\pgfsetbuttcap%
\pgfsetmiterjoin%
\definecolor{currentfill}{rgb}{0.517647,0.737255,0.854902}%
\pgfsetfillcolor{currentfill}%
\pgfsetlinewidth{0.000000pt}%
\definecolor{currentstroke}{rgb}{0.000000,0.000000,0.000000}%
\pgfsetstrokecolor{currentstroke}%
\pgfsetstrokeopacity{0.000000}%
\pgfsetdash{}{0pt}%
\pgfpathmoveto{\pgfqpoint{1.818056in}{0.513426in}}%
\pgfpathlineto{\pgfqpoint{2.345834in}{0.513426in}}%
\pgfpathlineto{\pgfqpoint{2.345834in}{1.500799in}}%
\pgfpathlineto{\pgfqpoint{1.818056in}{1.500799in}}%
\pgfpathclose%
\pgfusepath{fill}%
\end{pgfscope}%
\begin{pgfscope}%
\pgfpathrectangle{\pgfqpoint{0.525001in}{0.513426in}}{\pgfqpoint{5.224999in}{2.487501in}}%
\pgfusepath{clip}%
\pgfsetbuttcap%
\pgfsetmiterjoin%
\definecolor{currentfill}{rgb}{0.925490,0.764706,0.043137}%
\pgfsetfillcolor{currentfill}%
\pgfsetlinewidth{0.000000pt}%
\definecolor{currentstroke}{rgb}{0.000000,0.000000,0.000000}%
\pgfsetstrokecolor{currentstroke}%
\pgfsetstrokeopacity{0.000000}%
\pgfsetdash{}{0pt}%
\pgfpathmoveto{\pgfqpoint{2.873612in}{0.513426in}}%
\pgfpathlineto{\pgfqpoint{3.401389in}{0.513426in}}%
\pgfpathlineto{\pgfqpoint{3.401389in}{1.440304in}}%
\pgfpathlineto{\pgfqpoint{2.873612in}{1.440304in}}%
\pgfpathclose%
\pgfusepath{fill}%
\end{pgfscope}%
\begin{pgfscope}%
\pgfpathrectangle{\pgfqpoint{0.525001in}{0.513426in}}{\pgfqpoint{5.224999in}{2.487501in}}%
\pgfusepath{clip}%
\pgfsetbuttcap%
\pgfsetmiterjoin%
\definecolor{currentfill}{rgb}{0.952941,0.466667,0.282353}%
\pgfsetfillcolor{currentfill}%
\pgfsetlinewidth{0.000000pt}%
\definecolor{currentstroke}{rgb}{0.000000,0.000000,0.000000}%
\pgfsetstrokecolor{currentstroke}%
\pgfsetstrokeopacity{0.000000}%
\pgfsetdash{}{0pt}%
\pgfpathmoveto{\pgfqpoint{3.929167in}{0.513426in}}%
\pgfpathlineto{\pgfqpoint{4.456945in}{0.513426in}}%
\pgfpathlineto{\pgfqpoint{4.456945in}{1.387370in}}%
\pgfpathlineto{\pgfqpoint{3.929167in}{1.387370in}}%
\pgfpathclose%
\pgfusepath{fill}%
\end{pgfscope}%
\begin{pgfscope}%
\pgfpathrectangle{\pgfqpoint{0.525001in}{0.513426in}}{\pgfqpoint{5.224999in}{2.487501in}}%
\pgfusepath{clip}%
\pgfsetbuttcap%
\pgfsetmiterjoin%
\definecolor{currentfill}{rgb}{0.835294,0.376471,0.384314}%
\pgfsetfillcolor{currentfill}%
\pgfsetlinewidth{0.000000pt}%
\definecolor{currentstroke}{rgb}{0.000000,0.000000,0.000000}%
\pgfsetstrokecolor{currentstroke}%
\pgfsetstrokeopacity{0.000000}%
\pgfsetdash{}{0pt}%
\pgfpathmoveto{\pgfqpoint{4.984722in}{0.513426in}}%
\pgfpathlineto{\pgfqpoint{5.512500in}{0.513426in}}%
\pgfpathlineto{\pgfqpoint{5.512500in}{0.724080in}}%
\pgfpathlineto{\pgfqpoint{4.984722in}{0.724080in}}%
\pgfpathclose%
\pgfusepath{fill}%
\end{pgfscope}%
\begin{pgfscope}%
\pgfsetbuttcap%
\pgfsetroundjoin%
\definecolor{currentfill}{rgb}{0.000000,0.000000,0.000000}%
\pgfsetfillcolor{currentfill}%
\pgfsetlinewidth{0.803000pt}%
\definecolor{currentstroke}{rgb}{0.000000,0.000000,0.000000}%
\pgfsetstrokecolor{currentstroke}%
\pgfsetdash{}{0pt}%
\pgfsys@defobject{currentmarker}{\pgfqpoint{0.000000in}{-0.048611in}}{\pgfqpoint{0.000000in}{0.000000in}}{%
\pgfpathmoveto{\pgfqpoint{0.000000in}{0.000000in}}%
\pgfpathlineto{\pgfqpoint{0.000000in}{-0.048611in}}%
\pgfusepath{stroke,fill}%
}%
\begin{pgfscope}%
\pgfsys@transformshift{1.026390in}{0.513426in}%
\pgfsys@useobject{currentmarker}{}%
\end{pgfscope}%
\end{pgfscope}%
\begin{pgfscope}%
\definecolor{textcolor}{rgb}{0.000000,0.000000,0.000000}%
\pgfsetstrokecolor{textcolor}%
\pgfsetfillcolor{textcolor}%
\pgftext[x=1.026390in,y=0.416203in,,top]{\color{textcolor}\rmfamily\fontsize{10.000000}{12.000000}\selectfont Closest-first}%
\end{pgfscope}%
\begin{pgfscope}%
\pgfsetbuttcap%
\pgfsetroundjoin%
\definecolor{currentfill}{rgb}{0.000000,0.000000,0.000000}%
\pgfsetfillcolor{currentfill}%
\pgfsetlinewidth{0.803000pt}%
\definecolor{currentstroke}{rgb}{0.000000,0.000000,0.000000}%
\pgfsetstrokecolor{currentstroke}%
\pgfsetdash{}{0pt}%
\pgfsys@defobject{currentmarker}{\pgfqpoint{0.000000in}{-0.048611in}}{\pgfqpoint{0.000000in}{0.000000in}}{%
\pgfpathmoveto{\pgfqpoint{0.000000in}{0.000000in}}%
\pgfpathlineto{\pgfqpoint{0.000000in}{-0.048611in}}%
\pgfusepath{stroke,fill}%
}%
\begin{pgfscope}%
\pgfsys@transformshift{2.081945in}{0.513426in}%
\pgfsys@useobject{currentmarker}{}%
\end{pgfscope}%
\end{pgfscope}%
\begin{pgfscope}%
\definecolor{textcolor}{rgb}{0.000000,0.000000,0.000000}%
\pgfsetstrokecolor{textcolor}%
\pgfsetfillcolor{textcolor}%
\pgftext[x=1.721798in, y=0.319753in, left, base]{\color{textcolor}\rmfamily\fontsize{10.000000}{12.000000}\selectfont Closest-first}%
\end{pgfscope}%
\begin{pgfscope}%
\definecolor{textcolor}{rgb}{0.000000,0.000000,0.000000}%
\pgfsetstrokecolor{textcolor}%
\pgfsetfillcolor{textcolor}%
\pgftext[x=1.848534in, y=0.177006in, left, base]{\color{textcolor}\rmfamily\fontsize{10.000000}{12.000000}\selectfont in View}%
\end{pgfscope}%
\begin{pgfscope}%
\pgfsetbuttcap%
\pgfsetroundjoin%
\definecolor{currentfill}{rgb}{0.000000,0.000000,0.000000}%
\pgfsetfillcolor{currentfill}%
\pgfsetlinewidth{0.803000pt}%
\definecolor{currentstroke}{rgb}{0.000000,0.000000,0.000000}%
\pgfsetstrokecolor{currentstroke}%
\pgfsetdash{}{0pt}%
\pgfsys@defobject{currentmarker}{\pgfqpoint{0.000000in}{-0.048611in}}{\pgfqpoint{0.000000in}{0.000000in}}{%
\pgfpathmoveto{\pgfqpoint{0.000000in}{0.000000in}}%
\pgfpathlineto{\pgfqpoint{0.000000in}{-0.048611in}}%
\pgfusepath{stroke,fill}%
}%
\begin{pgfscope}%
\pgfsys@transformshift{3.137500in}{0.513426in}%
\pgfsys@useobject{currentmarker}{}%
\end{pgfscope}%
\end{pgfscope}%
\begin{pgfscope}%
\definecolor{textcolor}{rgb}{0.000000,0.000000,0.000000}%
\pgfsetstrokecolor{textcolor}%
\pgfsetfillcolor{textcolor}%
\pgftext[x=2.703086in, y=0.319753in, left, base]{\color{textcolor}\rmfamily\fontsize{10.000000}{12.000000}\selectfont SphereTracing}%
\end{pgfscope}%
\begin{pgfscope}%
\definecolor{textcolor}{rgb}{0.000000,0.000000,0.000000}%
\pgfsetstrokecolor{textcolor}%
\pgfsetfillcolor{textcolor}%
\pgftext[x=2.890201in, y=0.177006in, left, base]{\color{textcolor}\rmfamily\fontsize{10.000000}{12.000000}\selectfont distance}%
\end{pgfscope}%
\begin{pgfscope}%
\pgfsetbuttcap%
\pgfsetroundjoin%
\definecolor{currentfill}{rgb}{0.000000,0.000000,0.000000}%
\pgfsetfillcolor{currentfill}%
\pgfsetlinewidth{0.803000pt}%
\definecolor{currentstroke}{rgb}{0.000000,0.000000,0.000000}%
\pgfsetstrokecolor{currentstroke}%
\pgfsetdash{}{0pt}%
\pgfsys@defobject{currentmarker}{\pgfqpoint{0.000000in}{-0.048611in}}{\pgfqpoint{0.000000in}{0.000000in}}{%
\pgfpathmoveto{\pgfqpoint{0.000000in}{0.000000in}}%
\pgfpathlineto{\pgfqpoint{0.000000in}{-0.048611in}}%
\pgfusepath{stroke,fill}%
}%
\begin{pgfscope}%
\pgfsys@transformshift{4.193056in}{0.513426in}%
\pgfsys@useobject{currentmarker}{}%
\end{pgfscope}%
\end{pgfscope}%
\begin{pgfscope}%
\definecolor{textcolor}{rgb}{0.000000,0.000000,0.000000}%
\pgfsetstrokecolor{textcolor}%
\pgfsetfillcolor{textcolor}%
\pgftext[x=3.758641in, y=0.319753in, left, base]{\color{textcolor}\rmfamily\fontsize{10.000000}{12.000000}\selectfont SphereTracing}%
\end{pgfscope}%
\begin{pgfscope}%
\definecolor{textcolor}{rgb}{0.000000,0.000000,0.000000}%
\pgfsetstrokecolor{textcolor}%
\pgfsetfillcolor{textcolor}%
\pgftext[x=3.887885in, y=0.177006in, left, base]{\color{textcolor}\rmfamily\fontsize{10.000000}{12.000000}\selectfont dimension}%
\end{pgfscope}%
\begin{pgfscope}%
\pgfsetbuttcap%
\pgfsetroundjoin%
\definecolor{currentfill}{rgb}{0.000000,0.000000,0.000000}%
\pgfsetfillcolor{currentfill}%
\pgfsetlinewidth{0.803000pt}%
\definecolor{currentstroke}{rgb}{0.000000,0.000000,0.000000}%
\pgfsetstrokecolor{currentstroke}%
\pgfsetdash{}{0pt}%
\pgfsys@defobject{currentmarker}{\pgfqpoint{0.000000in}{-0.048611in}}{\pgfqpoint{0.000000in}{0.000000in}}{%
\pgfpathmoveto{\pgfqpoint{0.000000in}{0.000000in}}%
\pgfpathlineto{\pgfqpoint{0.000000in}{-0.048611in}}%
\pgfusepath{stroke,fill}%
}%
\begin{pgfscope}%
\pgfsys@transformshift{5.248611in}{0.513426in}%
\pgfsys@useobject{currentmarker}{}%
\end{pgfscope}%
\end{pgfscope}%
\begin{pgfscope}%
\definecolor{textcolor}{rgb}{0.000000,0.000000,0.000000}%
\pgfsetstrokecolor{textcolor}%
\pgfsetfillcolor{textcolor}%
\pgftext[x=4.898302in, y=0.319753in, left, base]{\color{textcolor}\rmfamily\fontsize{10.000000}{12.000000}\selectfont RayTracing}%
\end{pgfscope}%
\begin{pgfscope}%
\definecolor{textcolor}{rgb}{0.000000,0.000000,0.000000}%
\pgfsetstrokecolor{textcolor}%
\pgfsetfillcolor{textcolor}%
\pgftext[x=4.836574in, y=0.177006in, left, base]{\color{textcolor}\rmfamily\fontsize{10.000000}{12.000000}\selectfont Cast shadows}%
\end{pgfscope}%
\begin{pgfscope}%
\pgfsetbuttcap%
\pgfsetroundjoin%
\definecolor{currentfill}{rgb}{0.000000,0.000000,0.000000}%
\pgfsetfillcolor{currentfill}%
\pgfsetlinewidth{0.803000pt}%
\definecolor{currentstroke}{rgb}{0.000000,0.000000,0.000000}%
\pgfsetstrokecolor{currentstroke}%
\pgfsetdash{}{0pt}%
\pgfsys@defobject{currentmarker}{\pgfqpoint{-0.048611in}{0.000000in}}{\pgfqpoint{-0.000000in}{0.000000in}}{%
\pgfpathmoveto{\pgfqpoint{-0.000000in}{0.000000in}}%
\pgfpathlineto{\pgfqpoint{-0.048611in}{0.000000in}}%
\pgfusepath{stroke,fill}%
}%
\begin{pgfscope}%
\pgfsys@transformshift{0.525001in}{0.513426in}%
\pgfsys@useobject{currentmarker}{}%
\end{pgfscope}%
\end{pgfscope}%
\begin{pgfscope}%
\definecolor{textcolor}{rgb}{0.000000,0.000000,0.000000}%
\pgfsetstrokecolor{textcolor}%
\pgfsetfillcolor{textcolor}%
\pgftext[x=0.358334in, y=0.465200in, left, base]{\color{textcolor}\rmfamily\fontsize{10.000000}{12.000000}\selectfont \(\displaystyle {0}\)}%
\end{pgfscope}%
\begin{pgfscope}%
\pgfsetbuttcap%
\pgfsetroundjoin%
\definecolor{currentfill}{rgb}{0.000000,0.000000,0.000000}%
\pgfsetfillcolor{currentfill}%
\pgfsetlinewidth{0.803000pt}%
\definecolor{currentstroke}{rgb}{0.000000,0.000000,0.000000}%
\pgfsetstrokecolor{currentstroke}%
\pgfsetdash{}{0pt}%
\pgfsys@defobject{currentmarker}{\pgfqpoint{-0.048611in}{0.000000in}}{\pgfqpoint{-0.000000in}{0.000000in}}{%
\pgfpathmoveto{\pgfqpoint{-0.000000in}{0.000000in}}%
\pgfpathlineto{\pgfqpoint{-0.048611in}{0.000000in}}%
\pgfusepath{stroke,fill}%
}%
\begin{pgfscope}%
\pgfsys@transformshift{0.525001in}{1.053564in}%
\pgfsys@useobject{currentmarker}{}%
\end{pgfscope}%
\end{pgfscope}%
\begin{pgfscope}%
\definecolor{textcolor}{rgb}{0.000000,0.000000,0.000000}%
\pgfsetstrokecolor{textcolor}%
\pgfsetfillcolor{textcolor}%
\pgftext[x=0.219445in, y=1.005339in, left, base]{\color{textcolor}\rmfamily\fontsize{10.000000}{12.000000}\selectfont \(\displaystyle {500}\)}%
\end{pgfscope}%
\begin{pgfscope}%
\pgfsetbuttcap%
\pgfsetroundjoin%
\definecolor{currentfill}{rgb}{0.000000,0.000000,0.000000}%
\pgfsetfillcolor{currentfill}%
\pgfsetlinewidth{0.803000pt}%
\definecolor{currentstroke}{rgb}{0.000000,0.000000,0.000000}%
\pgfsetstrokecolor{currentstroke}%
\pgfsetdash{}{0pt}%
\pgfsys@defobject{currentmarker}{\pgfqpoint{-0.048611in}{0.000000in}}{\pgfqpoint{-0.000000in}{0.000000in}}{%
\pgfpathmoveto{\pgfqpoint{-0.000000in}{0.000000in}}%
\pgfpathlineto{\pgfqpoint{-0.048611in}{0.000000in}}%
\pgfusepath{stroke,fill}%
}%
\begin{pgfscope}%
\pgfsys@transformshift{0.525001in}{1.593703in}%
\pgfsys@useobject{currentmarker}{}%
\end{pgfscope}%
\end{pgfscope}%
\begin{pgfscope}%
\definecolor{textcolor}{rgb}{0.000000,0.000000,0.000000}%
\pgfsetstrokecolor{textcolor}%
\pgfsetfillcolor{textcolor}%
\pgftext[x=0.150000in, y=1.545478in, left, base]{\color{textcolor}\rmfamily\fontsize{10.000000}{12.000000}\selectfont \(\displaystyle {1000}\)}%
\end{pgfscope}%
\begin{pgfscope}%
\pgfsetbuttcap%
\pgfsetroundjoin%
\definecolor{currentfill}{rgb}{0.000000,0.000000,0.000000}%
\pgfsetfillcolor{currentfill}%
\pgfsetlinewidth{0.803000pt}%
\definecolor{currentstroke}{rgb}{0.000000,0.000000,0.000000}%
\pgfsetstrokecolor{currentstroke}%
\pgfsetdash{}{0pt}%
\pgfsys@defobject{currentmarker}{\pgfqpoint{-0.048611in}{0.000000in}}{\pgfqpoint{-0.000000in}{0.000000in}}{%
\pgfpathmoveto{\pgfqpoint{-0.000000in}{0.000000in}}%
\pgfpathlineto{\pgfqpoint{-0.048611in}{0.000000in}}%
\pgfusepath{stroke,fill}%
}%
\begin{pgfscope}%
\pgfsys@transformshift{0.525001in}{2.133842in}%
\pgfsys@useobject{currentmarker}{}%
\end{pgfscope}%
\end{pgfscope}%
\begin{pgfscope}%
\definecolor{textcolor}{rgb}{0.000000,0.000000,0.000000}%
\pgfsetstrokecolor{textcolor}%
\pgfsetfillcolor{textcolor}%
\pgftext[x=0.150000in, y=2.085616in, left, base]{\color{textcolor}\rmfamily\fontsize{10.000000}{12.000000}\selectfont \(\displaystyle {1500}\)}%
\end{pgfscope}%
\begin{pgfscope}%
\pgfsetbuttcap%
\pgfsetroundjoin%
\definecolor{currentfill}{rgb}{0.000000,0.000000,0.000000}%
\pgfsetfillcolor{currentfill}%
\pgfsetlinewidth{0.803000pt}%
\definecolor{currentstroke}{rgb}{0.000000,0.000000,0.000000}%
\pgfsetstrokecolor{currentstroke}%
\pgfsetdash{}{0pt}%
\pgfsys@defobject{currentmarker}{\pgfqpoint{-0.048611in}{0.000000in}}{\pgfqpoint{-0.000000in}{0.000000in}}{%
\pgfpathmoveto{\pgfqpoint{-0.000000in}{0.000000in}}%
\pgfpathlineto{\pgfqpoint{-0.048611in}{0.000000in}}%
\pgfusepath{stroke,fill}%
}%
\begin{pgfscope}%
\pgfsys@transformshift{0.525001in}{2.673980in}%
\pgfsys@useobject{currentmarker}{}%
\end{pgfscope}%
\end{pgfscope}%
\begin{pgfscope}%
\definecolor{textcolor}{rgb}{0.000000,0.000000,0.000000}%
\pgfsetstrokecolor{textcolor}%
\pgfsetfillcolor{textcolor}%
\pgftext[x=0.150000in, y=2.625755in, left, base]{\color{textcolor}\rmfamily\fontsize{10.000000}{12.000000}\selectfont \(\displaystyle {2000}\)}%
\end{pgfscope}%
\begin{pgfscope}%
\pgfsetrectcap%
\pgfsetmiterjoin%
\pgfsetlinewidth{0.803000pt}%
\definecolor{currentstroke}{rgb}{0.000000,0.000000,0.000000}%
\pgfsetstrokecolor{currentstroke}%
\pgfsetdash{}{0pt}%
\pgfpathmoveto{\pgfqpoint{0.525001in}{0.513426in}}%
\pgfpathlineto{\pgfqpoint{0.525001in}{3.000926in}}%
\pgfusepath{stroke}%
\end{pgfscope}%
\begin{pgfscope}%
\pgfsetrectcap%
\pgfsetmiterjoin%
\pgfsetlinewidth{0.803000pt}%
\definecolor{currentstroke}{rgb}{0.000000,0.000000,0.000000}%
\pgfsetstrokecolor{currentstroke}%
\pgfsetdash{}{0pt}%
\pgfpathmoveto{\pgfqpoint{5.750000in}{0.513426in}}%
\pgfpathlineto{\pgfqpoint{5.750000in}{3.000926in}}%
\pgfusepath{stroke}%
\end{pgfscope}%
\begin{pgfscope}%
\pgfsetrectcap%
\pgfsetmiterjoin%
\pgfsetlinewidth{0.803000pt}%
\definecolor{currentstroke}{rgb}{0.000000,0.000000,0.000000}%
\pgfsetstrokecolor{currentstroke}%
\pgfsetdash{}{0pt}%
\pgfpathmoveto{\pgfqpoint{0.525001in}{0.513426in}}%
\pgfpathlineto{\pgfqpoint{5.750000in}{0.513426in}}%
\pgfusepath{stroke}%
\end{pgfscope}%
\begin{pgfscope}%
\pgfsetrectcap%
\pgfsetmiterjoin%
\pgfsetlinewidth{0.803000pt}%
\definecolor{currentstroke}{rgb}{0.000000,0.000000,0.000000}%
\pgfsetstrokecolor{currentstroke}%
\pgfsetdash{}{0pt}%
\pgfpathmoveto{\pgfqpoint{0.525001in}{3.000926in}}%
\pgfpathlineto{\pgfqpoint{5.750000in}{3.000926in}}%
\pgfusepath{stroke}%
\end{pgfscope}%
\begin{pgfscope}%
\definecolor{textcolor}{rgb}{0.000000,0.000000,0.000000}%
\pgfsetstrokecolor{textcolor}%
\pgfsetfillcolor{textcolor}%
\pgftext[x=3.137500in,y=3.084260in,,base]{\color{textcolor}\rmfamily\fontsize{12.000000}{14.400000}\selectfont Differenza percettiva - Scena fittizia}%
\end{pgfscope}%
\end{pgfpicture}%
\makeatother%
\endgroup%

                \caption{Valutazioni messe a confronto: scena fittizia}
                \label{fig:conclusioni-fittizia}

                %% Creator: Matplotlib, PGF backend
%%
%% To include the figure in your LaTeX document, write
%%   \input{<filename>.pgf}
%%
%% Make sure the required packages are loaded in your preamble
%%   \usepackage{pgf}
%%
%% Figures using additional raster images can only be included by \input if
%% they are in the same directory as the main LaTeX file. For loading figures
%% from other directories you can use the `import` package
%%   \usepackage{import}
%%
%% and then include the figures with
%%   \import{<path to file>}{<filename>.pgf}
%%
%% Matplotlib used the following preamble
%%
\begingroup%
\makeatletter%
\begin{pgfpicture}%
\pgfpathrectangle{\pgfpointorigin}{\pgfqpoint{5.900000in}{3.350000in}}%
\pgfusepath{use as bounding box, clip}%
\begin{pgfscope}%
\pgfsetbuttcap%
\pgfsetmiterjoin%
\definecolor{currentfill}{rgb}{1.000000,1.000000,1.000000}%
\pgfsetfillcolor{currentfill}%
\pgfsetlinewidth{0.000000pt}%
\definecolor{currentstroke}{rgb}{1.000000,1.000000,1.000000}%
\pgfsetstrokecolor{currentstroke}%
\pgfsetdash{}{0pt}%
\pgfpathmoveto{\pgfqpoint{0.000000in}{0.000000in}}%
\pgfpathlineto{\pgfqpoint{5.900000in}{0.000000in}}%
\pgfpathlineto{\pgfqpoint{5.900000in}{3.350000in}}%
\pgfpathlineto{\pgfqpoint{0.000000in}{3.350000in}}%
\pgfpathclose%
\pgfusepath{fill}%
\end{pgfscope}%
\begin{pgfscope}%
\pgfsetbuttcap%
\pgfsetmiterjoin%
\definecolor{currentfill}{rgb}{1.000000,1.000000,1.000000}%
\pgfsetfillcolor{currentfill}%
\pgfsetlinewidth{0.000000pt}%
\definecolor{currentstroke}{rgb}{0.000000,0.000000,0.000000}%
\pgfsetstrokecolor{currentstroke}%
\pgfsetstrokeopacity{0.000000}%
\pgfsetdash{}{0pt}%
\pgfpathmoveto{\pgfqpoint{0.594446in}{0.513426in}}%
\pgfpathlineto{\pgfqpoint{5.750000in}{0.513426in}}%
\pgfpathlineto{\pgfqpoint{5.750000in}{3.000926in}}%
\pgfpathlineto{\pgfqpoint{0.594446in}{3.000926in}}%
\pgfpathclose%
\pgfusepath{fill}%
\end{pgfscope}%
\begin{pgfscope}%
\pgfpathrectangle{\pgfqpoint{0.594446in}{0.513426in}}{\pgfqpoint{5.155554in}{2.487501in}}%
\pgfusepath{clip}%
\pgfsetbuttcap%
\pgfsetmiterjoin%
\definecolor{currentfill}{rgb}{0.023529,0.482353,0.760784}%
\pgfsetfillcolor{currentfill}%
\pgfsetlinewidth{0.000000pt}%
\definecolor{currentstroke}{rgb}{0.000000,0.000000,0.000000}%
\pgfsetstrokecolor{currentstroke}%
\pgfsetstrokeopacity{0.000000}%
\pgfsetdash{}{0pt}%
\pgfpathmoveto{\pgfqpoint{0.828789in}{0.513426in}}%
\pgfpathlineto{\pgfqpoint{1.349552in}{0.513426in}}%
\pgfpathlineto{\pgfqpoint{1.349552in}{2.882474in}}%
\pgfpathlineto{\pgfqpoint{0.828789in}{2.882474in}}%
\pgfpathclose%
\pgfusepath{fill}%
\end{pgfscope}%
\begin{pgfscope}%
\pgfpathrectangle{\pgfqpoint{0.594446in}{0.513426in}}{\pgfqpoint{5.155554in}{2.487501in}}%
\pgfusepath{clip}%
\pgfsetbuttcap%
\pgfsetmiterjoin%
\definecolor{currentfill}{rgb}{0.517647,0.737255,0.854902}%
\pgfsetfillcolor{currentfill}%
\pgfsetlinewidth{0.000000pt}%
\definecolor{currentstroke}{rgb}{0.000000,0.000000,0.000000}%
\pgfsetstrokecolor{currentstroke}%
\pgfsetstrokeopacity{0.000000}%
\pgfsetdash{}{0pt}%
\pgfpathmoveto{\pgfqpoint{1.870315in}{0.513426in}}%
\pgfpathlineto{\pgfqpoint{2.391078in}{0.513426in}}%
\pgfpathlineto{\pgfqpoint{2.391078in}{1.896333in}}%
\pgfpathlineto{\pgfqpoint{1.870315in}{1.896333in}}%
\pgfpathclose%
\pgfusepath{fill}%
\end{pgfscope}%
\begin{pgfscope}%
\pgfpathrectangle{\pgfqpoint{0.594446in}{0.513426in}}{\pgfqpoint{5.155554in}{2.487501in}}%
\pgfusepath{clip}%
\pgfsetbuttcap%
\pgfsetmiterjoin%
\definecolor{currentfill}{rgb}{0.925490,0.764706,0.043137}%
\pgfsetfillcolor{currentfill}%
\pgfsetlinewidth{0.000000pt}%
\definecolor{currentstroke}{rgb}{0.000000,0.000000,0.000000}%
\pgfsetstrokecolor{currentstroke}%
\pgfsetstrokeopacity{0.000000}%
\pgfsetdash{}{0pt}%
\pgfpathmoveto{\pgfqpoint{2.911841in}{0.513426in}}%
\pgfpathlineto{\pgfqpoint{3.432604in}{0.513426in}}%
\pgfpathlineto{\pgfqpoint{3.432604in}{1.896333in}}%
\pgfpathlineto{\pgfqpoint{2.911841in}{1.896333in}}%
\pgfpathclose%
\pgfusepath{fill}%
\end{pgfscope}%
\begin{pgfscope}%
\pgfpathrectangle{\pgfqpoint{0.594446in}{0.513426in}}{\pgfqpoint{5.155554in}{2.487501in}}%
\pgfusepath{clip}%
\pgfsetbuttcap%
\pgfsetmiterjoin%
\definecolor{currentfill}{rgb}{0.952941,0.466667,0.282353}%
\pgfsetfillcolor{currentfill}%
\pgfsetlinewidth{0.000000pt}%
\definecolor{currentstroke}{rgb}{0.000000,0.000000,0.000000}%
\pgfsetstrokecolor{currentstroke}%
\pgfsetstrokeopacity{0.000000}%
\pgfsetdash{}{0pt}%
\pgfpathmoveto{\pgfqpoint{3.953367in}{0.513426in}}%
\pgfpathlineto{\pgfqpoint{4.474130in}{0.513426in}}%
\pgfpathlineto{\pgfqpoint{4.474130in}{1.499386in}}%
\pgfpathlineto{\pgfqpoint{3.953367in}{1.499386in}}%
\pgfpathclose%
\pgfusepath{fill}%
\end{pgfscope}%
\begin{pgfscope}%
\pgfpathrectangle{\pgfqpoint{0.594446in}{0.513426in}}{\pgfqpoint{5.155554in}{2.487501in}}%
\pgfusepath{clip}%
\pgfsetbuttcap%
\pgfsetmiterjoin%
\definecolor{currentfill}{rgb}{0.835294,0.376471,0.384314}%
\pgfsetfillcolor{currentfill}%
\pgfsetlinewidth{0.000000pt}%
\definecolor{currentstroke}{rgb}{0.000000,0.000000,0.000000}%
\pgfsetstrokecolor{currentstroke}%
\pgfsetstrokeopacity{0.000000}%
\pgfsetdash{}{0pt}%
\pgfpathmoveto{\pgfqpoint{4.994894in}{0.513426in}}%
\pgfpathlineto{\pgfqpoint{5.515657in}{0.513426in}}%
\pgfpathlineto{\pgfqpoint{5.515657in}{1.029692in}}%
\pgfpathlineto{\pgfqpoint{4.994894in}{1.029692in}}%
\pgfpathclose%
\pgfusepath{fill}%
\end{pgfscope}%
\begin{pgfscope}%
\pgfsetbuttcap%
\pgfsetroundjoin%
\definecolor{currentfill}{rgb}{0.000000,0.000000,0.000000}%
\pgfsetfillcolor{currentfill}%
\pgfsetlinewidth{0.803000pt}%
\definecolor{currentstroke}{rgb}{0.000000,0.000000,0.000000}%
\pgfsetstrokecolor{currentstroke}%
\pgfsetdash{}{0pt}%
\pgfsys@defobject{currentmarker}{\pgfqpoint{0.000000in}{-0.048611in}}{\pgfqpoint{0.000000in}{0.000000in}}{%
\pgfpathmoveto{\pgfqpoint{0.000000in}{0.000000in}}%
\pgfpathlineto{\pgfqpoint{0.000000in}{-0.048611in}}%
\pgfusepath{stroke,fill}%
}%
\begin{pgfscope}%
\pgfsys@transformshift{1.089170in}{0.513426in}%
\pgfsys@useobject{currentmarker}{}%
\end{pgfscope}%
\end{pgfscope}%
\begin{pgfscope}%
\definecolor{textcolor}{rgb}{0.000000,0.000000,0.000000}%
\pgfsetstrokecolor{textcolor}%
\pgfsetfillcolor{textcolor}%
\pgftext[x=1.089170in,y=0.416203in,,top]{\color{textcolor}\rmfamily\fontsize{10.000000}{12.000000}\selectfont Closest-first}%
\end{pgfscope}%
\begin{pgfscope}%
\pgfsetbuttcap%
\pgfsetroundjoin%
\definecolor{currentfill}{rgb}{0.000000,0.000000,0.000000}%
\pgfsetfillcolor{currentfill}%
\pgfsetlinewidth{0.803000pt}%
\definecolor{currentstroke}{rgb}{0.000000,0.000000,0.000000}%
\pgfsetstrokecolor{currentstroke}%
\pgfsetdash{}{0pt}%
\pgfsys@defobject{currentmarker}{\pgfqpoint{0.000000in}{-0.048611in}}{\pgfqpoint{0.000000in}{0.000000in}}{%
\pgfpathmoveto{\pgfqpoint{0.000000in}{0.000000in}}%
\pgfpathlineto{\pgfqpoint{0.000000in}{-0.048611in}}%
\pgfusepath{stroke,fill}%
}%
\begin{pgfscope}%
\pgfsys@transformshift{2.130697in}{0.513426in}%
\pgfsys@useobject{currentmarker}{}%
\end{pgfscope}%
\end{pgfscope}%
\begin{pgfscope}%
\definecolor{textcolor}{rgb}{0.000000,0.000000,0.000000}%
\pgfsetstrokecolor{textcolor}%
\pgfsetfillcolor{textcolor}%
\pgftext[x=1.770550in, y=0.319753in, left, base]{\color{textcolor}\rmfamily\fontsize{10.000000}{12.000000}\selectfont Closest-first}%
\end{pgfscope}%
\begin{pgfscope}%
\definecolor{textcolor}{rgb}{0.000000,0.000000,0.000000}%
\pgfsetstrokecolor{textcolor}%
\pgfsetfillcolor{textcolor}%
\pgftext[x=1.897286in, y=0.177006in, left, base]{\color{textcolor}\rmfamily\fontsize{10.000000}{12.000000}\selectfont in View}%
\end{pgfscope}%
\begin{pgfscope}%
\pgfsetbuttcap%
\pgfsetroundjoin%
\definecolor{currentfill}{rgb}{0.000000,0.000000,0.000000}%
\pgfsetfillcolor{currentfill}%
\pgfsetlinewidth{0.803000pt}%
\definecolor{currentstroke}{rgb}{0.000000,0.000000,0.000000}%
\pgfsetstrokecolor{currentstroke}%
\pgfsetdash{}{0pt}%
\pgfsys@defobject{currentmarker}{\pgfqpoint{0.000000in}{-0.048611in}}{\pgfqpoint{0.000000in}{0.000000in}}{%
\pgfpathmoveto{\pgfqpoint{0.000000in}{0.000000in}}%
\pgfpathlineto{\pgfqpoint{0.000000in}{-0.048611in}}%
\pgfusepath{stroke,fill}%
}%
\begin{pgfscope}%
\pgfsys@transformshift{3.172223in}{0.513426in}%
\pgfsys@useobject{currentmarker}{}%
\end{pgfscope}%
\end{pgfscope}%
\begin{pgfscope}%
\definecolor{textcolor}{rgb}{0.000000,0.000000,0.000000}%
\pgfsetstrokecolor{textcolor}%
\pgfsetfillcolor{textcolor}%
\pgftext[x=2.737808in, y=0.319753in, left, base]{\color{textcolor}\rmfamily\fontsize{10.000000}{12.000000}\selectfont SphereTracing}%
\end{pgfscope}%
\begin{pgfscope}%
\definecolor{textcolor}{rgb}{0.000000,0.000000,0.000000}%
\pgfsetstrokecolor{textcolor}%
\pgfsetfillcolor{textcolor}%
\pgftext[x=2.924923in, y=0.177006in, left, base]{\color{textcolor}\rmfamily\fontsize{10.000000}{12.000000}\selectfont distance}%
\end{pgfscope}%
\begin{pgfscope}%
\pgfsetbuttcap%
\pgfsetroundjoin%
\definecolor{currentfill}{rgb}{0.000000,0.000000,0.000000}%
\pgfsetfillcolor{currentfill}%
\pgfsetlinewidth{0.803000pt}%
\definecolor{currentstroke}{rgb}{0.000000,0.000000,0.000000}%
\pgfsetstrokecolor{currentstroke}%
\pgfsetdash{}{0pt}%
\pgfsys@defobject{currentmarker}{\pgfqpoint{0.000000in}{-0.048611in}}{\pgfqpoint{0.000000in}{0.000000in}}{%
\pgfpathmoveto{\pgfqpoint{0.000000in}{0.000000in}}%
\pgfpathlineto{\pgfqpoint{0.000000in}{-0.048611in}}%
\pgfusepath{stroke,fill}%
}%
\begin{pgfscope}%
\pgfsys@transformshift{4.213749in}{0.513426in}%
\pgfsys@useobject{currentmarker}{}%
\end{pgfscope}%
\end{pgfscope}%
\begin{pgfscope}%
\definecolor{textcolor}{rgb}{0.000000,0.000000,0.000000}%
\pgfsetstrokecolor{textcolor}%
\pgfsetfillcolor{textcolor}%
\pgftext[x=3.779335in, y=0.319753in, left, base]{\color{textcolor}\rmfamily\fontsize{10.000000}{12.000000}\selectfont SphereTracing}%
\end{pgfscope}%
\begin{pgfscope}%
\definecolor{textcolor}{rgb}{0.000000,0.000000,0.000000}%
\pgfsetstrokecolor{textcolor}%
\pgfsetfillcolor{textcolor}%
\pgftext[x=3.908578in, y=0.177006in, left, base]{\color{textcolor}\rmfamily\fontsize{10.000000}{12.000000}\selectfont dimension}%
\end{pgfscope}%
\begin{pgfscope}%
\pgfsetbuttcap%
\pgfsetroundjoin%
\definecolor{currentfill}{rgb}{0.000000,0.000000,0.000000}%
\pgfsetfillcolor{currentfill}%
\pgfsetlinewidth{0.803000pt}%
\definecolor{currentstroke}{rgb}{0.000000,0.000000,0.000000}%
\pgfsetstrokecolor{currentstroke}%
\pgfsetdash{}{0pt}%
\pgfsys@defobject{currentmarker}{\pgfqpoint{0.000000in}{-0.048611in}}{\pgfqpoint{0.000000in}{0.000000in}}{%
\pgfpathmoveto{\pgfqpoint{0.000000in}{0.000000in}}%
\pgfpathlineto{\pgfqpoint{0.000000in}{-0.048611in}}%
\pgfusepath{stroke,fill}%
}%
\begin{pgfscope}%
\pgfsys@transformshift{5.255275in}{0.513426in}%
\pgfsys@useobject{currentmarker}{}%
\end{pgfscope}%
\end{pgfscope}%
\begin{pgfscope}%
\definecolor{textcolor}{rgb}{0.000000,0.000000,0.000000}%
\pgfsetstrokecolor{textcolor}%
\pgfsetfillcolor{textcolor}%
\pgftext[x=4.904966in, y=0.319753in, left, base]{\color{textcolor}\rmfamily\fontsize{10.000000}{12.000000}\selectfont RayTracing}%
\end{pgfscope}%
\begin{pgfscope}%
\definecolor{textcolor}{rgb}{0.000000,0.000000,0.000000}%
\pgfsetstrokecolor{textcolor}%
\pgfsetfillcolor{textcolor}%
\pgftext[x=4.843237in, y=0.177006in, left, base]{\color{textcolor}\rmfamily\fontsize{10.000000}{12.000000}\selectfont Cast shadows}%
\end{pgfscope}%
\begin{pgfscope}%
\pgfsetbuttcap%
\pgfsetroundjoin%
\definecolor{currentfill}{rgb}{0.000000,0.000000,0.000000}%
\pgfsetfillcolor{currentfill}%
\pgfsetlinewidth{0.803000pt}%
\definecolor{currentstroke}{rgb}{0.000000,0.000000,0.000000}%
\pgfsetstrokecolor{currentstroke}%
\pgfsetdash{}{0pt}%
\pgfsys@defobject{currentmarker}{\pgfqpoint{-0.048611in}{0.000000in}}{\pgfqpoint{-0.000000in}{0.000000in}}{%
\pgfpathmoveto{\pgfqpoint{-0.000000in}{0.000000in}}%
\pgfpathlineto{\pgfqpoint{-0.048611in}{0.000000in}}%
\pgfusepath{stroke,fill}%
}%
\begin{pgfscope}%
\pgfsys@transformshift{0.594446in}{0.513426in}%
\pgfsys@useobject{currentmarker}{}%
\end{pgfscope}%
\end{pgfscope}%
\begin{pgfscope}%
\definecolor{textcolor}{rgb}{0.000000,0.000000,0.000000}%
\pgfsetstrokecolor{textcolor}%
\pgfsetfillcolor{textcolor}%
\pgftext[x=0.427779in, y=0.465200in, left, base]{\color{textcolor}\rmfamily\fontsize{10.000000}{12.000000}\selectfont \(\displaystyle {0}\)}%
\end{pgfscope}%
\begin{pgfscope}%
\pgfsetbuttcap%
\pgfsetroundjoin%
\definecolor{currentfill}{rgb}{0.000000,0.000000,0.000000}%
\pgfsetfillcolor{currentfill}%
\pgfsetlinewidth{0.803000pt}%
\definecolor{currentstroke}{rgb}{0.000000,0.000000,0.000000}%
\pgfsetstrokecolor{currentstroke}%
\pgfsetdash{}{0pt}%
\pgfsys@defobject{currentmarker}{\pgfqpoint{-0.048611in}{0.000000in}}{\pgfqpoint{-0.000000in}{0.000000in}}{%
\pgfpathmoveto{\pgfqpoint{-0.000000in}{0.000000in}}%
\pgfpathlineto{\pgfqpoint{-0.048611in}{0.000000in}}%
\pgfusepath{stroke,fill}%
}%
\begin{pgfscope}%
\pgfsys@transformshift{0.594446in}{0.874451in}%
\pgfsys@useobject{currentmarker}{}%
\end{pgfscope}%
\end{pgfscope}%
\begin{pgfscope}%
\definecolor{textcolor}{rgb}{0.000000,0.000000,0.000000}%
\pgfsetstrokecolor{textcolor}%
\pgfsetfillcolor{textcolor}%
\pgftext[x=0.219445in, y=0.826226in, left, base]{\color{textcolor}\rmfamily\fontsize{10.000000}{12.000000}\selectfont \(\displaystyle {2000}\)}%
\end{pgfscope}%
\begin{pgfscope}%
\pgfsetbuttcap%
\pgfsetroundjoin%
\definecolor{currentfill}{rgb}{0.000000,0.000000,0.000000}%
\pgfsetfillcolor{currentfill}%
\pgfsetlinewidth{0.803000pt}%
\definecolor{currentstroke}{rgb}{0.000000,0.000000,0.000000}%
\pgfsetstrokecolor{currentstroke}%
\pgfsetdash{}{0pt}%
\pgfsys@defobject{currentmarker}{\pgfqpoint{-0.048611in}{0.000000in}}{\pgfqpoint{-0.000000in}{0.000000in}}{%
\pgfpathmoveto{\pgfqpoint{-0.000000in}{0.000000in}}%
\pgfpathlineto{\pgfqpoint{-0.048611in}{0.000000in}}%
\pgfusepath{stroke,fill}%
}%
\begin{pgfscope}%
\pgfsys@transformshift{0.594446in}{1.235476in}%
\pgfsys@useobject{currentmarker}{}%
\end{pgfscope}%
\end{pgfscope}%
\begin{pgfscope}%
\definecolor{textcolor}{rgb}{0.000000,0.000000,0.000000}%
\pgfsetstrokecolor{textcolor}%
\pgfsetfillcolor{textcolor}%
\pgftext[x=0.219445in, y=1.187251in, left, base]{\color{textcolor}\rmfamily\fontsize{10.000000}{12.000000}\selectfont \(\displaystyle {4000}\)}%
\end{pgfscope}%
\begin{pgfscope}%
\pgfsetbuttcap%
\pgfsetroundjoin%
\definecolor{currentfill}{rgb}{0.000000,0.000000,0.000000}%
\pgfsetfillcolor{currentfill}%
\pgfsetlinewidth{0.803000pt}%
\definecolor{currentstroke}{rgb}{0.000000,0.000000,0.000000}%
\pgfsetstrokecolor{currentstroke}%
\pgfsetdash{}{0pt}%
\pgfsys@defobject{currentmarker}{\pgfqpoint{-0.048611in}{0.000000in}}{\pgfqpoint{-0.000000in}{0.000000in}}{%
\pgfpathmoveto{\pgfqpoint{-0.000000in}{0.000000in}}%
\pgfpathlineto{\pgfqpoint{-0.048611in}{0.000000in}}%
\pgfusepath{stroke,fill}%
}%
\begin{pgfscope}%
\pgfsys@transformshift{0.594446in}{1.596502in}%
\pgfsys@useobject{currentmarker}{}%
\end{pgfscope}%
\end{pgfscope}%
\begin{pgfscope}%
\definecolor{textcolor}{rgb}{0.000000,0.000000,0.000000}%
\pgfsetstrokecolor{textcolor}%
\pgfsetfillcolor{textcolor}%
\pgftext[x=0.219445in, y=1.548276in, left, base]{\color{textcolor}\rmfamily\fontsize{10.000000}{12.000000}\selectfont \(\displaystyle {6000}\)}%
\end{pgfscope}%
\begin{pgfscope}%
\pgfsetbuttcap%
\pgfsetroundjoin%
\definecolor{currentfill}{rgb}{0.000000,0.000000,0.000000}%
\pgfsetfillcolor{currentfill}%
\pgfsetlinewidth{0.803000pt}%
\definecolor{currentstroke}{rgb}{0.000000,0.000000,0.000000}%
\pgfsetstrokecolor{currentstroke}%
\pgfsetdash{}{0pt}%
\pgfsys@defobject{currentmarker}{\pgfqpoint{-0.048611in}{0.000000in}}{\pgfqpoint{-0.000000in}{0.000000in}}{%
\pgfpathmoveto{\pgfqpoint{-0.000000in}{0.000000in}}%
\pgfpathlineto{\pgfqpoint{-0.048611in}{0.000000in}}%
\pgfusepath{stroke,fill}%
}%
\begin{pgfscope}%
\pgfsys@transformshift{0.594446in}{1.957527in}%
\pgfsys@useobject{currentmarker}{}%
\end{pgfscope}%
\end{pgfscope}%
\begin{pgfscope}%
\definecolor{textcolor}{rgb}{0.000000,0.000000,0.000000}%
\pgfsetstrokecolor{textcolor}%
\pgfsetfillcolor{textcolor}%
\pgftext[x=0.219445in, y=1.909302in, left, base]{\color{textcolor}\rmfamily\fontsize{10.000000}{12.000000}\selectfont \(\displaystyle {8000}\)}%
\end{pgfscope}%
\begin{pgfscope}%
\pgfsetbuttcap%
\pgfsetroundjoin%
\definecolor{currentfill}{rgb}{0.000000,0.000000,0.000000}%
\pgfsetfillcolor{currentfill}%
\pgfsetlinewidth{0.803000pt}%
\definecolor{currentstroke}{rgb}{0.000000,0.000000,0.000000}%
\pgfsetstrokecolor{currentstroke}%
\pgfsetdash{}{0pt}%
\pgfsys@defobject{currentmarker}{\pgfqpoint{-0.048611in}{0.000000in}}{\pgfqpoint{-0.000000in}{0.000000in}}{%
\pgfpathmoveto{\pgfqpoint{-0.000000in}{0.000000in}}%
\pgfpathlineto{\pgfqpoint{-0.048611in}{0.000000in}}%
\pgfusepath{stroke,fill}%
}%
\begin{pgfscope}%
\pgfsys@transformshift{0.594446in}{2.318552in}%
\pgfsys@useobject{currentmarker}{}%
\end{pgfscope}%
\end{pgfscope}%
\begin{pgfscope}%
\definecolor{textcolor}{rgb}{0.000000,0.000000,0.000000}%
\pgfsetstrokecolor{textcolor}%
\pgfsetfillcolor{textcolor}%
\pgftext[x=0.150000in, y=2.270327in, left, base]{\color{textcolor}\rmfamily\fontsize{10.000000}{12.000000}\selectfont \(\displaystyle {10000}\)}%
\end{pgfscope}%
\begin{pgfscope}%
\pgfsetbuttcap%
\pgfsetroundjoin%
\definecolor{currentfill}{rgb}{0.000000,0.000000,0.000000}%
\pgfsetfillcolor{currentfill}%
\pgfsetlinewidth{0.803000pt}%
\definecolor{currentstroke}{rgb}{0.000000,0.000000,0.000000}%
\pgfsetstrokecolor{currentstroke}%
\pgfsetdash{}{0pt}%
\pgfsys@defobject{currentmarker}{\pgfqpoint{-0.048611in}{0.000000in}}{\pgfqpoint{-0.000000in}{0.000000in}}{%
\pgfpathmoveto{\pgfqpoint{-0.000000in}{0.000000in}}%
\pgfpathlineto{\pgfqpoint{-0.048611in}{0.000000in}}%
\pgfusepath{stroke,fill}%
}%
\begin{pgfscope}%
\pgfsys@transformshift{0.594446in}{2.679578in}%
\pgfsys@useobject{currentmarker}{}%
\end{pgfscope}%
\end{pgfscope}%
\begin{pgfscope}%
\definecolor{textcolor}{rgb}{0.000000,0.000000,0.000000}%
\pgfsetstrokecolor{textcolor}%
\pgfsetfillcolor{textcolor}%
\pgftext[x=0.150000in, y=2.631352in, left, base]{\color{textcolor}\rmfamily\fontsize{10.000000}{12.000000}\selectfont \(\displaystyle {12000}\)}%
\end{pgfscope}%
\begin{pgfscope}%
\pgfsetrectcap%
\pgfsetmiterjoin%
\pgfsetlinewidth{0.803000pt}%
\definecolor{currentstroke}{rgb}{0.000000,0.000000,0.000000}%
\pgfsetstrokecolor{currentstroke}%
\pgfsetdash{}{0pt}%
\pgfpathmoveto{\pgfqpoint{0.594446in}{0.513426in}}%
\pgfpathlineto{\pgfqpoint{0.594446in}{3.000926in}}%
\pgfusepath{stroke}%
\end{pgfscope}%
\begin{pgfscope}%
\pgfsetrectcap%
\pgfsetmiterjoin%
\pgfsetlinewidth{0.803000pt}%
\definecolor{currentstroke}{rgb}{0.000000,0.000000,0.000000}%
\pgfsetstrokecolor{currentstroke}%
\pgfsetdash{}{0pt}%
\pgfpathmoveto{\pgfqpoint{5.750000in}{0.513426in}}%
\pgfpathlineto{\pgfqpoint{5.750000in}{3.000926in}}%
\pgfusepath{stroke}%
\end{pgfscope}%
\begin{pgfscope}%
\pgfsetrectcap%
\pgfsetmiterjoin%
\pgfsetlinewidth{0.803000pt}%
\definecolor{currentstroke}{rgb}{0.000000,0.000000,0.000000}%
\pgfsetstrokecolor{currentstroke}%
\pgfsetdash{}{0pt}%
\pgfpathmoveto{\pgfqpoint{0.594446in}{0.513426in}}%
\pgfpathlineto{\pgfqpoint{5.750000in}{0.513426in}}%
\pgfusepath{stroke}%
\end{pgfscope}%
\begin{pgfscope}%
\pgfsetrectcap%
\pgfsetmiterjoin%
\pgfsetlinewidth{0.803000pt}%
\definecolor{currentstroke}{rgb}{0.000000,0.000000,0.000000}%
\pgfsetstrokecolor{currentstroke}%
\pgfsetdash{}{0pt}%
\pgfpathmoveto{\pgfqpoint{0.594446in}{3.000926in}}%
\pgfpathlineto{\pgfqpoint{5.750000in}{3.000926in}}%
\pgfusepath{stroke}%
\end{pgfscope}%
\begin{pgfscope}%
\definecolor{textcolor}{rgb}{0.000000,0.000000,0.000000}%
\pgfsetstrokecolor{textcolor}%
\pgfsetfillcolor{textcolor}%
\pgftext[x=3.172223in,y=3.084260in,,base]{\color{textcolor}\rmfamily\fontsize{12.000000}{14.400000}\selectfont Differenza percettiva - Scena realistica}%
\end{pgfscope}%
\end{pgfpicture}%
\makeatother%
\endgroup%

                \caption{Valutazioni messe a confronto: scena realistica}
                \label{fig:conclusioni-realistica}
        \end{figure}

        Sono già esistenti soluzioni hardware per velocizzare i tempi di trasferimento dei dati da memoria di massa \cite{thompson_newburn_2019}. Queste evidenziano la necessità di accelerare i tempi di caricamento, in questo contesto, delle scene. Il lavoro presentato in questa tesi mostra quindi come la scelta della prioritizzazione abbia un importante impatto nel conseguimento di questi risultati a livello software, che per natura, hanno un costo implementativo molto inferiore.

\newpage
\section{Limitazioni}
\label{cap:limitazioni&lavorofuturo}

\paragraph*{Testing sul campo}
    L'implementazione non è ancora stata utilizzata dall'azienda ospitante a causa dei limiti posti dal motore grafico in utilizzo: Unity \cite{unity}, il quale non permette di gestire la memoria liberamente se non ad un livello macroscopico, garantendo una gestione di pacchetti di asset \cite{addressables, asset-bundles}. Di conseguenza per eseguire comunque gli esperimenti è stato simulato il comportamento di caricamento sempre sullo stesso motore grafico a soli scopi valutativi.
    Nonostante queste limitazioni è opportuno sottolineare che Unity fornisce strumenti per evitare sprechi di memoria \cite{farclipplaneunitydoc, occlusionunitydoc}.


\section{Lavoro futuro}

Il lavoro di questa tesi può essere il punto di partenza per un numero di sviluppi futuri.

\paragraph*{Parallelismo.}
    Nell'attuale implementazione, tutte le strategie sono eseguite in modo sequenziale su CPU single-threaded, quando invece molte di queste, specialmente quelle che fanno uso delle tecniche di Ray Tracing e Ray Marching, potrebbero avvantaggiarsi di una implementazione parallelizzata su GPU. Questa limitazione non ha previene l'esecuzione degli esperimenti ma, in contesto lavorativo, si vorrebbe che questi ordinamenti, benché calcolati offline, avvengano nel minor tempo possibile ma senza rinunciare alla precisione dell'ordinamento.

\paragraph*{Testing ulteriore.}
    Sarebbe inoltre interessante provare a valutare ulteriori scene, prese da contesti diversi, per verificare o contraddire l'efficacia delle strategie nei diversi contesti. Ad esempio, nel contesto videoludico sarebbe opportuno effettuare valutazioni di scene provenienti da generi differenti. 

\paragraph*{User evaluation.} La parte valutativa di questo progetto si basa su un semplice confronto pixel-a-pixel di immagini (seppur valutata in uno spazio colorimetrico basato su considerazioni percettive). Questo rappresenta una cruda approssimazione del dato che si vuole misurare. Uno \textit{user study} con utenti reali si rende necessario per determinare l'accuratezza di questa valutazione.


\paragraph*{Estensione ad altri tipi di asset.}
    Il concetto di caricamento di un'istanza, utilizzato in questa tesi, ha sempre supposto che venissero caricati anche altri asset utilizzati da questa. Quando è necessario caricare l'istanza di un edificio vengono caricati: mesh 3D, tessiture, materiali, animazioni, generici script ecc. Si potrebbe ideare invece una strategia che separa questo elemento atomico in tutti gli asset che lo compongono. Aggiungendo la necessità di introdurre ulteriori politiche per decidere come ordinare questi specifici tipi di asset (per esempio in ordine di dimensione su disco). Ad esempio, una strategia potrebbe anteporre il caricamento della sola istanza di una mesh 3D al caricamento delle tessiture di colore o mappa delle normali associate, risultando in un diverso compromesso fra qualità della scena parziale e tempo di caricamento. Similmente, una mesh animata che sia parte della scena potrebbe essere caricata in \textit{rest pose}, e mostrata senza la corrispondente animazione. Lo strumento di valutazione (introdotto nel Capitolo \ref{cap:toolvalutazione}) è già in grado di gestire queste estensioni, perché si basa su una misura dei rendering risultanti. 

\paragraph*{Estensione ad asset multirisoluzione.}
   Infine, questo lavoro può essere esteso nella direzione di valutare strategie che prescrivono di caricare livelli di dettaglio di risoluzione inferiore a quella che verrebbe normalmente utilizzata, per fornire scene parziali con un vantaggioso rapporto fra tempo di caricamento e qualità visuale. Esempi di questo tipo di asset includono: \begin{itemize}
       \item piramidi di livello di dettaglio di mesh; 
       \item livelli di MIP-map per tessiture;
       \item micro-mesh, una struttura di recente introduzione per la rappresentazione compatta di mesh ad alta risoluzione\cite{micromeshConstruction}, per le quali è possibile limitare il caricamento alla mesh-base tralasciando inizialmente le strutture di displacement scalare associate.
   \end{itemize} 
    


    \include{sections/LimitazioniELavoroFuturo}
    \chapter*{Glossario}

\begin{itemize}[label={}]
  \item \textbf{Differenza (o distanza) percettiva}: la distanza euclidea tra due colori espressi da una tripla di valori, o la media tra la distanza euclidea tra due insiemi ordinati di triple di valori, in base al contesto.
  \item \textbf{Asset}: In termini generali, una rappresentazione digitale (tipicamente contenuta in uno o più file) che modella un elemento costitutivo dell'ambiente virtuale, quale una mesh, un materiale, una tessitura, un file audio, etc, pronto per essere utilizzato durante l'esecuzione dell'applicativo.
  \item \textbf{Istanza di un asset}: anche detta solo istanza, è l'istanziazione di un asset in una scena. Una modifica ad un'istanza non comporta alla modifica dell'asset né alla modifica di altre istanze dello stesso.
  \item \textbf{Frame}: immagine che compone un video in movimento. Come immagine può essere espressa come una matrice di pixel (o frammenti).
  \item \textbf{Vista (o View)}: Risultato della rendering pipeline
  \item \textbf{View Frustum (o frustum della camera)}: La regione di spazio che verrà renderizzata a schermo
  \item \textbf{Scena}: Spazio contenente istanze di asset  
\end{itemize}

    \chapter*{Ringraziamenti}
        Ringrazio vivamente i miei genitori per avermi permesso di vivere questa esperienza, lasciandomi libero di imparare a camminare con le mie stesse gambe.
        
        Un sentito ringraziamento anche alla mia seconda famiglia: Alessandro Barba, Lucrezia, Letizia e al resto della famiglia Zanetti. Grazie per avermi supportato e sopportato per tutti questi anni. La pizza con voi è sempre più buona. 
        
        % Ho imparato tanto da voi, ad ascoltare, a mantenere la calma e la razionalità anche nelle discussioni più accese e persino qualche parola in bresciano.
        
        Grazie ad Alessandro Clerici e a Edoardo Della Rossa, degli amici e compagni di corso brillanti. Con la nostra comicità e complicità abbiamo vissuto una versione dell'esperienza universitaria unica e indimenticabile, nonostante la distanza e il buio delle nostre stanze. Mi avete insegnato l'importanza dell'impegno, della critica e della puntigliosità. Nonostante facessi fatica a stare al vostro passo, non mi avete mai lasciato indietro nemmeno per un momento.

        Grazie anche al prof. Masetti e alla prof.ssa Moretti per avermi fatto innamorare di questa disciplina. Probabilmente se non avessi avuto voi come professori, questa tesi sarebbe rimasta nella penna per sempre. 

        Un ultimo Grazie a Lucia che mi ha accompagnato durante questo ultimo impegnativo periodo, grande squadra.

    \printbibliography[heading=bibintoc]


\end{document}
