% Ma se spostassi tutti i grafici in un appendice apposita?

\chapter{Risultati ed esperimenti}
    % In questo capitolo, mostriamo il risultato dell'applicazione della strategia messa a punto a dataset reali o fittizi.

    % cosa si andrà a vedere nel capitolo
    Con l'utilizzo dello strumento proposto nel precedente capitolo è possibile ottenere delle valutazioni sulle politiche esposte nel Capitolo \ref{cap:strategie}. 

    \section{Testbed utilizzati}
    % testbed utilizzati (uno artificiale e uno realistico)
    % nella artificiale abbiamo i cubozzi che sono esattamente dentro le aabb mentre nella reale sono approssimazioni -> still pretty good per le ombre, funziona comunque molto bene (non abbiamo mai il caso: carica oggetto -> carica ombreggiatore dell'oggetto già presente perché le aabb sono un'approssimazione per eccesso e non per difetto)
    Le valutazioni avverranno su due scene, una fittizia che simula un caso ideale e una reale.
    
    La scena fittizia (Figura \ref{fig:scenafittizia}) presenta un'insieme di cubi con una fonte luminosa con la camera posizionata grossolanamente al centro. Il cubo, nel contesto valutativo considerato, è la mesh ideale poiché, in alcune strategie per la valutazione della priorità (vedi Capitoli \ref{cap:ombreportate}, \ref{cap:sphere-distanza}, \ref{cap:sphere-dimensione}) verranno considerate le AABB delle mesh che, per parallelepipedi rettangoli, coincide perfettamente con la mesh stessa.
    
    La scena contiene circa 2000 istanze, l'inquadratura renderizza circa $2,5 \cdot 10^4$ triangoli e non sono associate tessiture alle mesh renderizzate.
    Il posizionamento delle istanze dei cubi è parzialmente arbitraria; l'obiettivo principale è quello di emulare la presenza di molte istanze e di osservare, con una seconda camera, il corretto funzionamento della strategia in uso.
    È stato intenzionale il posizionamento di gruppi di cubi e della camera, così da avere la certezza che questa inquadrasse delle ombre portate. Infine, in distanza è stato aggiunto un parallelepipedo di grandi dimensioni, per assicurarsi il corretto funzionamento delle valutazioni per dimensione.

    \begin{figure}[htbp!]
        \centering
        \includegraphics[scale=.2]{images/scena_fittizia.png}
        \caption{Scena fittizia, fonte luminosa $30^\circ$}
        \label{fig:scenafittizia}

        \centering
        \includegraphics[scale=.2]{images/sRGB-ground-truth.png}
        \caption{Scena realistica, fonte luminosa $30^\circ$}
        \label{fig:scenarealistica}
    \end{figure}
    
    La scena reale (\ref{fig:scenarealistica}), di un gioco non annunciato - Copyright MixedBag srl -, è di un paese sul mare. Questa scena presenta diversi ostacoli nella quale le prime politiche esposte (Capitoli \ref{cap:closest-first} \ref{cap:closest-first-on-camera}, \ref{cap:sphere-distanza}) saranno più svantaggiate mentre quelle più raffinate incontreranno meno difficoltà. Si osserva che le politiche che fruiscono di raggi soffriranno la presenza di mesh convesse e cave le cui approssimazioni ad AABB saranno molto grezze, nonostante ciò tale approssimazione sarà comunque migliore di altre strategie.
    
    Per eseguire correttamente le valutazioni sulla scena realistica non dovranno essere presenti componenti in movimento come vegetazione soggetta al vento o acqua che scorre.
    La scena presenta al suo interno circa 4000 istanze di asset con una geometria complessa con tessiture associate. L'inquadratura utilizzata renderizza circa $2.4 \cdot 10^6$ triangoli, un notevole incremento di complessità dalla scena fittizia. La posizione della camera è centrale alla cittadina, alle sue spalle sono presenti ulteriori istanze di dimensione variabile.



    % disclaimer sulla simulazione in unity perché limitazioni con rimando al capitolo delle limitazioni e lavoro futuro

    \section{Valutazioni delle politiche}
    % valutazioni: ipotesi su fps, aps, ANGOLO DELLA luce, tempo di esecuzione, asset prioritari presi in considerazione (luce), quantità di luci nella scena, ignoriamo le zone di interesse
    Nelle valutazioni che seguiranno si terrà conto che: 
    \begin{itemize}
        \item si valuterà la differenza per ogni frame dal frame finale (\textit{ground truth})
        \item verranno caricati 16 istanze di asset per frame (16 APF) a 60 FPS
        \item la fonte luminosa è una fonte diretta e angolata a $30^\circ$ quando non specificato
        \item la camera e la fonte luminosa saranno sempre considerati asset prioritari
    \end{itemize}
    Le inquadrature di test usate nelle due scene sono mostrate nelle Figure \ref{fig:scenafittizia} e \ref{fig:scenarealistica}.
        \subsection{Closest-First}
            % closest first evidenziando l'uniformità con l'angolo della luce
            Come è possibile osservare in Figura \ref{fig:eval-cf}, questa politica mostra una tendenza sub-lineare nella scena fittizia mentre un comportamento molto peggiore per la scena realistica. Questo può essere motivato dalla presenza di più istanze vicine non in vista rispetto alla scena fittizia. 
            I salti presenti nella valutazione della scena realistica sono causati dal caricamento delle istanze di grandi dimensioni. Data la considerevole sezione di vista occupata, comportano un grande avvicinamento verso la ground-truth e un'importante amplificazione dell'area sotto la curva fino a quando non vengono caricate.

            Attraverso ulteriori esperimenti si è notato che il comportamento di questa politica non risente di notevoli cambiamenti alterando l'angolo di provenienza della luce.

            \begin{figure}[htbp!]
                \centering
                %% Creator: Matplotlib, PGF backend
%%
%% To include the figure in your LaTeX document, write
%%   \input{<filename>.pgf}
%%
%% Make sure the required packages are loaded in your preamble
%%   \usepackage{pgf}
%%
%% Figures using additional raster images can only be included by \input if
%% they are in the same directory as the main LaTeX file. For loading figures
%% from other directories you can use the `import` package
%%   \usepackage{import}
%%
%% and then include the figures with
%%   \import{<path to file>}{<filename>.pgf}
%%
%% Matplotlib used the following preamble
%%
\begingroup%
\makeatletter%
\begin{pgfpicture}%
\pgfpathrectangle{\pgfpointorigin}{\pgfqpoint{5.900000in}{3.400000in}}%
\pgfusepath{use as bounding box, clip}%
\begin{pgfscope}%
\pgfsetbuttcap%
\pgfsetmiterjoin%
\definecolor{currentfill}{rgb}{1.000000,1.000000,1.000000}%
\pgfsetfillcolor{currentfill}%
\pgfsetlinewidth{0.000000pt}%
\definecolor{currentstroke}{rgb}{1.000000,1.000000,1.000000}%
\pgfsetstrokecolor{currentstroke}%
\pgfsetdash{}{0pt}%
\pgfpathmoveto{\pgfqpoint{0.000000in}{0.000000in}}%
\pgfpathlineto{\pgfqpoint{5.900000in}{0.000000in}}%
\pgfpathlineto{\pgfqpoint{5.900000in}{3.400000in}}%
\pgfpathlineto{\pgfqpoint{0.000000in}{3.400000in}}%
\pgfpathclose%
\pgfusepath{fill}%
\end{pgfscope}%
\begin{pgfscope}%
\pgfsetbuttcap%
\pgfsetmiterjoin%
\definecolor{currentfill}{rgb}{1.000000,1.000000,1.000000}%
\pgfsetfillcolor{currentfill}%
\pgfsetlinewidth{0.000000pt}%
\definecolor{currentstroke}{rgb}{0.000000,0.000000,0.000000}%
\pgfsetstrokecolor{currentstroke}%
\pgfsetstrokeopacity{0.000000}%
\pgfsetdash{}{0pt}%
\pgfpathmoveto{\pgfqpoint{0.634568in}{0.565123in}}%
\pgfpathlineto{\pgfqpoint{5.750000in}{0.565123in}}%
\pgfpathlineto{\pgfqpoint{5.750000in}{3.250000in}}%
\pgfpathlineto{\pgfqpoint{0.634568in}{3.250000in}}%
\pgfpathclose%
\pgfusepath{fill}%
\end{pgfscope}%
\begin{pgfscope}%
\pgfsetbuttcap%
\pgfsetroundjoin%
\definecolor{currentfill}{rgb}{0.000000,0.000000,0.000000}%
\pgfsetfillcolor{currentfill}%
\pgfsetlinewidth{0.803000pt}%
\definecolor{currentstroke}{rgb}{0.000000,0.000000,0.000000}%
\pgfsetstrokecolor{currentstroke}%
\pgfsetdash{}{0pt}%
\pgfsys@defobject{currentmarker}{\pgfqpoint{0.000000in}{-0.048611in}}{\pgfqpoint{0.000000in}{0.000000in}}{%
\pgfpathmoveto{\pgfqpoint{0.000000in}{0.000000in}}%
\pgfpathlineto{\pgfqpoint{0.000000in}{-0.048611in}}%
\pgfusepath{stroke,fill}%
}%
\begin{pgfscope}%
\pgfsys@transformshift{0.867088in}{0.565123in}%
\pgfsys@useobject{currentmarker}{}%
\end{pgfscope}%
\end{pgfscope}%
\begin{pgfscope}%
\definecolor{textcolor}{rgb}{0.000000,0.000000,0.000000}%
\pgfsetstrokecolor{textcolor}%
\pgfsetfillcolor{textcolor}%
\pgftext[x=0.867088in,y=0.467901in,,top]{\color{textcolor}\rmfamily\fontsize{10.000000}{12.000000}\selectfont \(\displaystyle {0}\)}%
\end{pgfscope}%
\begin{pgfscope}%
\pgfsetbuttcap%
\pgfsetroundjoin%
\definecolor{currentfill}{rgb}{0.000000,0.000000,0.000000}%
\pgfsetfillcolor{currentfill}%
\pgfsetlinewidth{0.803000pt}%
\definecolor{currentstroke}{rgb}{0.000000,0.000000,0.000000}%
\pgfsetstrokecolor{currentstroke}%
\pgfsetdash{}{0pt}%
\pgfsys@defobject{currentmarker}{\pgfqpoint{0.000000in}{-0.048611in}}{\pgfqpoint{0.000000in}{0.000000in}}{%
\pgfpathmoveto{\pgfqpoint{0.000000in}{0.000000in}}%
\pgfpathlineto{\pgfqpoint{0.000000in}{-0.048611in}}%
\pgfusepath{stroke,fill}%
}%
\begin{pgfscope}%
\pgfsys@transformshift{1.852341in}{0.565123in}%
\pgfsys@useobject{currentmarker}{}%
\end{pgfscope}%
\end{pgfscope}%
\begin{pgfscope}%
\definecolor{textcolor}{rgb}{0.000000,0.000000,0.000000}%
\pgfsetstrokecolor{textcolor}%
\pgfsetfillcolor{textcolor}%
\pgftext[x=1.852341in,y=0.467901in,,top]{\color{textcolor}\rmfamily\fontsize{10.000000}{12.000000}\selectfont \(\displaystyle {50}\)}%
\end{pgfscope}%
\begin{pgfscope}%
\pgfsetbuttcap%
\pgfsetroundjoin%
\definecolor{currentfill}{rgb}{0.000000,0.000000,0.000000}%
\pgfsetfillcolor{currentfill}%
\pgfsetlinewidth{0.803000pt}%
\definecolor{currentstroke}{rgb}{0.000000,0.000000,0.000000}%
\pgfsetstrokecolor{currentstroke}%
\pgfsetdash{}{0pt}%
\pgfsys@defobject{currentmarker}{\pgfqpoint{0.000000in}{-0.048611in}}{\pgfqpoint{0.000000in}{0.000000in}}{%
\pgfpathmoveto{\pgfqpoint{0.000000in}{0.000000in}}%
\pgfpathlineto{\pgfqpoint{0.000000in}{-0.048611in}}%
\pgfusepath{stroke,fill}%
}%
\begin{pgfscope}%
\pgfsys@transformshift{2.837593in}{0.565123in}%
\pgfsys@useobject{currentmarker}{}%
\end{pgfscope}%
\end{pgfscope}%
\begin{pgfscope}%
\definecolor{textcolor}{rgb}{0.000000,0.000000,0.000000}%
\pgfsetstrokecolor{textcolor}%
\pgfsetfillcolor{textcolor}%
\pgftext[x=2.837593in,y=0.467901in,,top]{\color{textcolor}\rmfamily\fontsize{10.000000}{12.000000}\selectfont \(\displaystyle {100}\)}%
\end{pgfscope}%
\begin{pgfscope}%
\pgfsetbuttcap%
\pgfsetroundjoin%
\definecolor{currentfill}{rgb}{0.000000,0.000000,0.000000}%
\pgfsetfillcolor{currentfill}%
\pgfsetlinewidth{0.803000pt}%
\definecolor{currentstroke}{rgb}{0.000000,0.000000,0.000000}%
\pgfsetstrokecolor{currentstroke}%
\pgfsetdash{}{0pt}%
\pgfsys@defobject{currentmarker}{\pgfqpoint{0.000000in}{-0.048611in}}{\pgfqpoint{0.000000in}{0.000000in}}{%
\pgfpathmoveto{\pgfqpoint{0.000000in}{0.000000in}}%
\pgfpathlineto{\pgfqpoint{0.000000in}{-0.048611in}}%
\pgfusepath{stroke,fill}%
}%
\begin{pgfscope}%
\pgfsys@transformshift{3.822846in}{0.565123in}%
\pgfsys@useobject{currentmarker}{}%
\end{pgfscope}%
\end{pgfscope}%
\begin{pgfscope}%
\definecolor{textcolor}{rgb}{0.000000,0.000000,0.000000}%
\pgfsetstrokecolor{textcolor}%
\pgfsetfillcolor{textcolor}%
\pgftext[x=3.822846in,y=0.467901in,,top]{\color{textcolor}\rmfamily\fontsize{10.000000}{12.000000}\selectfont \(\displaystyle {150}\)}%
\end{pgfscope}%
\begin{pgfscope}%
\pgfsetbuttcap%
\pgfsetroundjoin%
\definecolor{currentfill}{rgb}{0.000000,0.000000,0.000000}%
\pgfsetfillcolor{currentfill}%
\pgfsetlinewidth{0.803000pt}%
\definecolor{currentstroke}{rgb}{0.000000,0.000000,0.000000}%
\pgfsetstrokecolor{currentstroke}%
\pgfsetdash{}{0pt}%
\pgfsys@defobject{currentmarker}{\pgfqpoint{0.000000in}{-0.048611in}}{\pgfqpoint{0.000000in}{0.000000in}}{%
\pgfpathmoveto{\pgfqpoint{0.000000in}{0.000000in}}%
\pgfpathlineto{\pgfqpoint{0.000000in}{-0.048611in}}%
\pgfusepath{stroke,fill}%
}%
\begin{pgfscope}%
\pgfsys@transformshift{4.808099in}{0.565123in}%
\pgfsys@useobject{currentmarker}{}%
\end{pgfscope}%
\end{pgfscope}%
\begin{pgfscope}%
\definecolor{textcolor}{rgb}{0.000000,0.000000,0.000000}%
\pgfsetstrokecolor{textcolor}%
\pgfsetfillcolor{textcolor}%
\pgftext[x=4.808099in,y=0.467901in,,top]{\color{textcolor}\rmfamily\fontsize{10.000000}{12.000000}\selectfont \(\displaystyle {200}\)}%
\end{pgfscope}%
\begin{pgfscope}%
\definecolor{textcolor}{rgb}{0.000000,0.000000,0.000000}%
\pgfsetstrokecolor{textcolor}%
\pgfsetfillcolor{textcolor}%
\pgftext[x=3.192284in,y=0.288889in,,top]{\color{textcolor}\rmfamily\fontsize{10.000000}{12.000000}\selectfont Tempo (Frame)}%
\end{pgfscope}%
\begin{pgfscope}%
\pgfsetbuttcap%
\pgfsetroundjoin%
\definecolor{currentfill}{rgb}{0.000000,0.000000,0.000000}%
\pgfsetfillcolor{currentfill}%
\pgfsetlinewidth{0.803000pt}%
\definecolor{currentstroke}{rgb}{0.000000,0.000000,0.000000}%
\pgfsetstrokecolor{currentstroke}%
\pgfsetdash{}{0pt}%
\pgfsys@defobject{currentmarker}{\pgfqpoint{-0.048611in}{0.000000in}}{\pgfqpoint{-0.000000in}{0.000000in}}{%
\pgfpathmoveto{\pgfqpoint{-0.000000in}{0.000000in}}%
\pgfpathlineto{\pgfqpoint{-0.048611in}{0.000000in}}%
\pgfusepath{stroke,fill}%
}%
\begin{pgfscope}%
\pgfsys@transformshift{0.634568in}{0.687163in}%
\pgfsys@useobject{currentmarker}{}%
\end{pgfscope}%
\end{pgfscope}%
\begin{pgfscope}%
\definecolor{textcolor}{rgb}{0.000000,0.000000,0.000000}%
\pgfsetstrokecolor{textcolor}%
\pgfsetfillcolor{textcolor}%
\pgftext[x=0.467902in, y=0.638938in, left, base]{\color{textcolor}\rmfamily\fontsize{10.000000}{12.000000}\selectfont \(\displaystyle {0}\)}%
\end{pgfscope}%
\begin{pgfscope}%
\pgfsetbuttcap%
\pgfsetroundjoin%
\definecolor{currentfill}{rgb}{0.000000,0.000000,0.000000}%
\pgfsetfillcolor{currentfill}%
\pgfsetlinewidth{0.803000pt}%
\definecolor{currentstroke}{rgb}{0.000000,0.000000,0.000000}%
\pgfsetstrokecolor{currentstroke}%
\pgfsetdash{}{0pt}%
\pgfsys@defobject{currentmarker}{\pgfqpoint{-0.048611in}{0.000000in}}{\pgfqpoint{-0.000000in}{0.000000in}}{%
\pgfpathmoveto{\pgfqpoint{-0.000000in}{0.000000in}}%
\pgfpathlineto{\pgfqpoint{-0.048611in}{0.000000in}}%
\pgfusepath{stroke,fill}%
}%
\begin{pgfscope}%
\pgfsys@transformshift{0.634568in}{1.158585in}%
\pgfsys@useobject{currentmarker}{}%
\end{pgfscope}%
\end{pgfscope}%
\begin{pgfscope}%
\definecolor{textcolor}{rgb}{0.000000,0.000000,0.000000}%
\pgfsetstrokecolor{textcolor}%
\pgfsetfillcolor{textcolor}%
\pgftext[x=0.398457in, y=1.110359in, left, base]{\color{textcolor}\rmfamily\fontsize{10.000000}{12.000000}\selectfont \(\displaystyle {20}\)}%
\end{pgfscope}%
\begin{pgfscope}%
\pgfsetbuttcap%
\pgfsetroundjoin%
\definecolor{currentfill}{rgb}{0.000000,0.000000,0.000000}%
\pgfsetfillcolor{currentfill}%
\pgfsetlinewidth{0.803000pt}%
\definecolor{currentstroke}{rgb}{0.000000,0.000000,0.000000}%
\pgfsetstrokecolor{currentstroke}%
\pgfsetdash{}{0pt}%
\pgfsys@defobject{currentmarker}{\pgfqpoint{-0.048611in}{0.000000in}}{\pgfqpoint{-0.000000in}{0.000000in}}{%
\pgfpathmoveto{\pgfqpoint{-0.000000in}{0.000000in}}%
\pgfpathlineto{\pgfqpoint{-0.048611in}{0.000000in}}%
\pgfusepath{stroke,fill}%
}%
\begin{pgfscope}%
\pgfsys@transformshift{0.634568in}{1.630006in}%
\pgfsys@useobject{currentmarker}{}%
\end{pgfscope}%
\end{pgfscope}%
\begin{pgfscope}%
\definecolor{textcolor}{rgb}{0.000000,0.000000,0.000000}%
\pgfsetstrokecolor{textcolor}%
\pgfsetfillcolor{textcolor}%
\pgftext[x=0.398457in, y=1.581781in, left, base]{\color{textcolor}\rmfamily\fontsize{10.000000}{12.000000}\selectfont \(\displaystyle {40}\)}%
\end{pgfscope}%
\begin{pgfscope}%
\pgfsetbuttcap%
\pgfsetroundjoin%
\definecolor{currentfill}{rgb}{0.000000,0.000000,0.000000}%
\pgfsetfillcolor{currentfill}%
\pgfsetlinewidth{0.803000pt}%
\definecolor{currentstroke}{rgb}{0.000000,0.000000,0.000000}%
\pgfsetstrokecolor{currentstroke}%
\pgfsetdash{}{0pt}%
\pgfsys@defobject{currentmarker}{\pgfqpoint{-0.048611in}{0.000000in}}{\pgfqpoint{-0.000000in}{0.000000in}}{%
\pgfpathmoveto{\pgfqpoint{-0.000000in}{0.000000in}}%
\pgfpathlineto{\pgfqpoint{-0.048611in}{0.000000in}}%
\pgfusepath{stroke,fill}%
}%
\begin{pgfscope}%
\pgfsys@transformshift{0.634568in}{2.101427in}%
\pgfsys@useobject{currentmarker}{}%
\end{pgfscope}%
\end{pgfscope}%
\begin{pgfscope}%
\definecolor{textcolor}{rgb}{0.000000,0.000000,0.000000}%
\pgfsetstrokecolor{textcolor}%
\pgfsetfillcolor{textcolor}%
\pgftext[x=0.398457in, y=2.053202in, left, base]{\color{textcolor}\rmfamily\fontsize{10.000000}{12.000000}\selectfont \(\displaystyle {60}\)}%
\end{pgfscope}%
\begin{pgfscope}%
\pgfsetbuttcap%
\pgfsetroundjoin%
\definecolor{currentfill}{rgb}{0.000000,0.000000,0.000000}%
\pgfsetfillcolor{currentfill}%
\pgfsetlinewidth{0.803000pt}%
\definecolor{currentstroke}{rgb}{0.000000,0.000000,0.000000}%
\pgfsetstrokecolor{currentstroke}%
\pgfsetdash{}{0pt}%
\pgfsys@defobject{currentmarker}{\pgfqpoint{-0.048611in}{0.000000in}}{\pgfqpoint{-0.000000in}{0.000000in}}{%
\pgfpathmoveto{\pgfqpoint{-0.000000in}{0.000000in}}%
\pgfpathlineto{\pgfqpoint{-0.048611in}{0.000000in}}%
\pgfusepath{stroke,fill}%
}%
\begin{pgfscope}%
\pgfsys@transformshift{0.634568in}{2.572849in}%
\pgfsys@useobject{currentmarker}{}%
\end{pgfscope}%
\end{pgfscope}%
\begin{pgfscope}%
\definecolor{textcolor}{rgb}{0.000000,0.000000,0.000000}%
\pgfsetstrokecolor{textcolor}%
\pgfsetfillcolor{textcolor}%
\pgftext[x=0.398457in, y=2.524624in, left, base]{\color{textcolor}\rmfamily\fontsize{10.000000}{12.000000}\selectfont \(\displaystyle {80}\)}%
\end{pgfscope}%
\begin{pgfscope}%
\pgfsetbuttcap%
\pgfsetroundjoin%
\definecolor{currentfill}{rgb}{0.000000,0.000000,0.000000}%
\pgfsetfillcolor{currentfill}%
\pgfsetlinewidth{0.803000pt}%
\definecolor{currentstroke}{rgb}{0.000000,0.000000,0.000000}%
\pgfsetstrokecolor{currentstroke}%
\pgfsetdash{}{0pt}%
\pgfsys@defobject{currentmarker}{\pgfqpoint{-0.048611in}{0.000000in}}{\pgfqpoint{-0.000000in}{0.000000in}}{%
\pgfpathmoveto{\pgfqpoint{-0.000000in}{0.000000in}}%
\pgfpathlineto{\pgfqpoint{-0.048611in}{0.000000in}}%
\pgfusepath{stroke,fill}%
}%
\begin{pgfscope}%
\pgfsys@transformshift{0.634568in}{3.044270in}%
\pgfsys@useobject{currentmarker}{}%
\end{pgfscope}%
\end{pgfscope}%
\begin{pgfscope}%
\definecolor{textcolor}{rgb}{0.000000,0.000000,0.000000}%
\pgfsetstrokecolor{textcolor}%
\pgfsetfillcolor{textcolor}%
\pgftext[x=0.329012in, y=2.996045in, left, base]{\color{textcolor}\rmfamily\fontsize{10.000000}{12.000000}\selectfont \(\displaystyle {100}\)}%
\end{pgfscope}%
\begin{pgfscope}%
\definecolor{textcolor}{rgb}{0.000000,0.000000,0.000000}%
\pgfsetstrokecolor{textcolor}%
\pgfsetfillcolor{textcolor}%
\pgftext[x=0.273457in,y=1.907562in,,bottom,rotate=90.000000]{\color{textcolor}\rmfamily\fontsize{10.000000}{12.000000}\selectfont Differenza percettiva}%
\end{pgfscope}%
\begin{pgfscope}%
\pgfpathrectangle{\pgfqpoint{0.634568in}{0.565123in}}{\pgfqpoint{5.115432in}{2.684877in}}%
\pgfusepath{clip}%
\pgfsetrectcap%
\pgfsetroundjoin%
\pgfsetlinewidth{2.007500pt}%
\definecolor{currentstroke}{rgb}{0.121569,0.466667,0.705882}%
\pgfsetstrokecolor{currentstroke}%
\pgfsetdash{}{0pt}%
\pgfpathmoveto{\pgfqpoint{0.867088in}{1.840957in}}%
\pgfpathlineto{\pgfqpoint{0.886793in}{1.808217in}}%
\pgfpathlineto{\pgfqpoint{0.906498in}{1.806631in}}%
\pgfpathlineto{\pgfqpoint{0.926203in}{1.781585in}}%
\pgfpathlineto{\pgfqpoint{0.945908in}{1.587120in}}%
\pgfpathlineto{\pgfqpoint{0.965613in}{1.551508in}}%
\pgfpathlineto{\pgfqpoint{1.005023in}{1.550429in}}%
\pgfpathlineto{\pgfqpoint{1.024729in}{1.550429in}}%
\pgfpathlineto{\pgfqpoint{1.044434in}{1.546386in}}%
\pgfpathlineto{\pgfqpoint{1.064139in}{1.458658in}}%
\pgfpathlineto{\pgfqpoint{1.083844in}{1.452088in}}%
\pgfpathlineto{\pgfqpoint{1.103549in}{1.434438in}}%
\pgfpathlineto{\pgfqpoint{1.123254in}{1.419754in}}%
\pgfpathlineto{\pgfqpoint{1.142959in}{1.419754in}}%
\pgfpathlineto{\pgfqpoint{1.162664in}{1.409240in}}%
\pgfpathlineto{\pgfqpoint{1.182369in}{1.396779in}}%
\pgfpathlineto{\pgfqpoint{1.202074in}{1.376857in}}%
\pgfpathlineto{\pgfqpoint{1.221779in}{1.365775in}}%
\pgfpathlineto{\pgfqpoint{1.241484in}{1.344893in}}%
\pgfpathlineto{\pgfqpoint{1.261189in}{1.333880in}}%
\pgfpathlineto{\pgfqpoint{1.280894in}{1.324235in}}%
\pgfpathlineto{\pgfqpoint{1.300599in}{1.324235in}}%
\pgfpathlineto{\pgfqpoint{1.320304in}{1.311404in}}%
\pgfpathlineto{\pgfqpoint{1.340009in}{1.309275in}}%
\pgfpathlineto{\pgfqpoint{1.359714in}{1.277013in}}%
\pgfpathlineto{\pgfqpoint{1.399124in}{1.277013in}}%
\pgfpathlineto{\pgfqpoint{1.418830in}{1.260530in}}%
\pgfpathlineto{\pgfqpoint{1.438535in}{1.241869in}}%
\pgfpathlineto{\pgfqpoint{1.458240in}{1.233909in}}%
\pgfpathlineto{\pgfqpoint{1.477945in}{1.207102in}}%
\pgfpathlineto{\pgfqpoint{1.497650in}{1.204991in}}%
\pgfpathlineto{\pgfqpoint{1.517355in}{1.175507in}}%
\pgfpathlineto{\pgfqpoint{1.537060in}{1.175004in}}%
\pgfpathlineto{\pgfqpoint{1.556765in}{1.172427in}}%
\pgfpathlineto{\pgfqpoint{1.576470in}{1.161420in}}%
\pgfpathlineto{\pgfqpoint{1.596175in}{1.156388in}}%
\pgfpathlineto{\pgfqpoint{1.615880in}{1.147536in}}%
\pgfpathlineto{\pgfqpoint{1.635585in}{1.147512in}}%
\pgfpathlineto{\pgfqpoint{1.655290in}{1.143213in}}%
\pgfpathlineto{\pgfqpoint{1.674995in}{1.140664in}}%
\pgfpathlineto{\pgfqpoint{1.714405in}{1.128011in}}%
\pgfpathlineto{\pgfqpoint{1.734110in}{1.112103in}}%
\pgfpathlineto{\pgfqpoint{1.753815in}{1.112072in}}%
\pgfpathlineto{\pgfqpoint{1.773520in}{1.087924in}}%
\pgfpathlineto{\pgfqpoint{1.812931in}{1.087102in}}%
\pgfpathlineto{\pgfqpoint{1.832636in}{1.082891in}}%
\pgfpathlineto{\pgfqpoint{1.872046in}{1.079372in}}%
\pgfpathlineto{\pgfqpoint{1.891751in}{1.076993in}}%
\pgfpathlineto{\pgfqpoint{1.931161in}{1.069018in}}%
\pgfpathlineto{\pgfqpoint{1.950866in}{1.060866in}}%
\pgfpathlineto{\pgfqpoint{1.970571in}{1.035696in}}%
\pgfpathlineto{\pgfqpoint{1.990276in}{1.035696in}}%
\pgfpathlineto{\pgfqpoint{2.009981in}{1.031814in}}%
\pgfpathlineto{\pgfqpoint{2.029686in}{1.024885in}}%
\pgfpathlineto{\pgfqpoint{2.049391in}{1.024885in}}%
\pgfpathlineto{\pgfqpoint{2.069096in}{1.020206in}}%
\pgfpathlineto{\pgfqpoint{2.088801in}{1.013045in}}%
\pgfpathlineto{\pgfqpoint{2.108506in}{1.009771in}}%
\pgfpathlineto{\pgfqpoint{2.128211in}{0.998412in}}%
\pgfpathlineto{\pgfqpoint{2.167622in}{0.995507in}}%
\pgfpathlineto{\pgfqpoint{2.187327in}{0.986860in}}%
\pgfpathlineto{\pgfqpoint{2.207032in}{0.982384in}}%
\pgfpathlineto{\pgfqpoint{2.226737in}{0.975968in}}%
\pgfpathlineto{\pgfqpoint{2.246442in}{0.974044in}}%
\pgfpathlineto{\pgfqpoint{2.266147in}{0.970542in}}%
\pgfpathlineto{\pgfqpoint{2.325262in}{0.966622in}}%
\pgfpathlineto{\pgfqpoint{2.404082in}{0.965822in}}%
\pgfpathlineto{\pgfqpoint{2.423787in}{0.959922in}}%
\pgfpathlineto{\pgfqpoint{2.443492in}{0.942795in}}%
\pgfpathlineto{\pgfqpoint{2.482902in}{0.939935in}}%
\pgfpathlineto{\pgfqpoint{2.502607in}{0.929506in}}%
\pgfpathlineto{\pgfqpoint{2.542018in}{0.926942in}}%
\pgfpathlineto{\pgfqpoint{2.561723in}{0.924696in}}%
\pgfpathlineto{\pgfqpoint{2.581428in}{0.919876in}}%
\pgfpathlineto{\pgfqpoint{2.620838in}{0.919876in}}%
\pgfpathlineto{\pgfqpoint{2.640543in}{0.915534in}}%
\pgfpathlineto{\pgfqpoint{2.660248in}{0.909175in}}%
\pgfpathlineto{\pgfqpoint{2.679953in}{0.900213in}}%
\pgfpathlineto{\pgfqpoint{2.699658in}{0.898505in}}%
\pgfpathlineto{\pgfqpoint{2.739068in}{0.892467in}}%
\pgfpathlineto{\pgfqpoint{2.758773in}{0.886184in}}%
\pgfpathlineto{\pgfqpoint{2.778478in}{0.886184in}}%
\pgfpathlineto{\pgfqpoint{2.798183in}{0.876869in}}%
\pgfpathlineto{\pgfqpoint{2.817888in}{0.873720in}}%
\pgfpathlineto{\pgfqpoint{2.837593in}{0.871917in}}%
\pgfpathlineto{\pgfqpoint{2.857298in}{0.867345in}}%
\pgfpathlineto{\pgfqpoint{2.877003in}{0.864141in}}%
\pgfpathlineto{\pgfqpoint{2.896708in}{0.863304in}}%
\pgfpathlineto{\pgfqpoint{2.936119in}{0.858989in}}%
\pgfpathlineto{\pgfqpoint{2.975529in}{0.852320in}}%
\pgfpathlineto{\pgfqpoint{2.995234in}{0.848946in}}%
\pgfpathlineto{\pgfqpoint{3.014939in}{0.847533in}}%
\pgfpathlineto{\pgfqpoint{3.074054in}{0.846499in}}%
\pgfpathlineto{\pgfqpoint{3.093759in}{0.842771in}}%
\pgfpathlineto{\pgfqpoint{3.172579in}{0.837620in}}%
\pgfpathlineto{\pgfqpoint{3.211989in}{0.835296in}}%
\pgfpathlineto{\pgfqpoint{3.231694in}{0.835296in}}%
\pgfpathlineto{\pgfqpoint{3.251399in}{0.829614in}}%
\pgfpathlineto{\pgfqpoint{3.310515in}{0.825788in}}%
\pgfpathlineto{\pgfqpoint{3.330220in}{0.818564in}}%
\pgfpathlineto{\pgfqpoint{3.349925in}{0.814457in}}%
\pgfpathlineto{\pgfqpoint{3.389335in}{0.810903in}}%
\pgfpathlineto{\pgfqpoint{3.428745in}{0.807368in}}%
\pgfpathlineto{\pgfqpoint{3.448450in}{0.803240in}}%
\pgfpathlineto{\pgfqpoint{3.468155in}{0.728775in}}%
\pgfpathlineto{\pgfqpoint{3.507565in}{0.727968in}}%
\pgfpathlineto{\pgfqpoint{3.546975in}{0.722345in}}%
\pgfpathlineto{\pgfqpoint{3.586385in}{0.721061in}}%
\pgfpathlineto{\pgfqpoint{3.606090in}{0.718756in}}%
\pgfpathlineto{\pgfqpoint{3.645500in}{0.717681in}}%
\pgfpathlineto{\pgfqpoint{3.684911in}{0.714131in}}%
\pgfpathlineto{\pgfqpoint{3.724321in}{0.714130in}}%
\pgfpathlineto{\pgfqpoint{3.744026in}{0.710342in}}%
\pgfpathlineto{\pgfqpoint{3.803141in}{0.706783in}}%
\pgfpathlineto{\pgfqpoint{3.822846in}{0.702267in}}%
\pgfpathlineto{\pgfqpoint{3.862256in}{0.696946in}}%
\pgfpathlineto{\pgfqpoint{3.901666in}{0.696793in}}%
\pgfpathlineto{\pgfqpoint{3.921371in}{0.694780in}}%
\pgfpathlineto{\pgfqpoint{3.941076in}{0.691297in}}%
\pgfpathlineto{\pgfqpoint{3.960781in}{0.690326in}}%
\pgfpathlineto{\pgfqpoint{3.980486in}{0.687163in}}%
\pgfpathlineto{\pgfqpoint{5.497775in}{0.687163in}}%
\pgfpathlineto{\pgfqpoint{5.497775in}{0.687163in}}%
\pgfusepath{stroke}%
\end{pgfscope}%
\begin{pgfscope}%
\pgfpathrectangle{\pgfqpoint{0.634568in}{0.565123in}}{\pgfqpoint{5.115432in}{2.684877in}}%
\pgfusepath{clip}%
\pgfsetrectcap%
\pgfsetroundjoin%
\pgfsetlinewidth{2.007500pt}%
\definecolor{currentstroke}{rgb}{1.000000,0.498039,0.054902}%
\pgfsetstrokecolor{currentstroke}%
\pgfsetdash{}{0pt}%
\pgfpathmoveto{\pgfqpoint{0.886793in}{3.127953in}}%
\pgfpathlineto{\pgfqpoint{1.340009in}{3.127960in}}%
\pgfpathlineto{\pgfqpoint{1.359714in}{3.121231in}}%
\pgfpathlineto{\pgfqpoint{1.379419in}{3.120996in}}%
\pgfpathlineto{\pgfqpoint{1.399124in}{3.096972in}}%
\pgfpathlineto{\pgfqpoint{2.758773in}{3.096960in}}%
\pgfpathlineto{\pgfqpoint{2.778478in}{3.093125in}}%
\pgfpathlineto{\pgfqpoint{2.798183in}{3.093125in}}%
\pgfpathlineto{\pgfqpoint{2.817888in}{3.089724in}}%
\pgfpathlineto{\pgfqpoint{2.837593in}{2.768224in}}%
\pgfpathlineto{\pgfqpoint{2.857298in}{2.747825in}}%
\pgfpathlineto{\pgfqpoint{2.877003in}{2.747824in}}%
\pgfpathlineto{\pgfqpoint{2.896708in}{2.731279in}}%
\pgfpathlineto{\pgfqpoint{3.014939in}{2.731270in}}%
\pgfpathlineto{\pgfqpoint{3.034644in}{2.712415in}}%
\pgfpathlineto{\pgfqpoint{3.074054in}{2.712411in}}%
\pgfpathlineto{\pgfqpoint{3.093759in}{2.631195in}}%
\pgfpathlineto{\pgfqpoint{3.172579in}{2.631175in}}%
\pgfpathlineto{\pgfqpoint{3.192284in}{2.620720in}}%
\pgfpathlineto{\pgfqpoint{3.211989in}{2.446776in}}%
\pgfpathlineto{\pgfqpoint{3.231694in}{2.349242in}}%
\pgfpathlineto{\pgfqpoint{3.251399in}{2.335249in}}%
\pgfpathlineto{\pgfqpoint{3.271104in}{2.225421in}}%
\pgfpathlineto{\pgfqpoint{3.290809in}{1.749546in}}%
\pgfpathlineto{\pgfqpoint{3.310515in}{1.731856in}}%
\pgfpathlineto{\pgfqpoint{3.369630in}{1.731825in}}%
\pgfpathlineto{\pgfqpoint{3.389335in}{1.720050in}}%
\pgfpathlineto{\pgfqpoint{3.409040in}{1.720043in}}%
\pgfpathlineto{\pgfqpoint{3.428745in}{1.716709in}}%
\pgfpathlineto{\pgfqpoint{3.448450in}{1.715627in}}%
\pgfpathlineto{\pgfqpoint{3.468155in}{1.697494in}}%
\pgfpathlineto{\pgfqpoint{3.487860in}{1.564507in}}%
\pgfpathlineto{\pgfqpoint{3.527270in}{1.556797in}}%
\pgfpathlineto{\pgfqpoint{3.546975in}{1.546135in}}%
\pgfpathlineto{\pgfqpoint{3.566680in}{1.546136in}}%
\pgfpathlineto{\pgfqpoint{3.586385in}{1.540574in}}%
\pgfpathlineto{\pgfqpoint{3.606090in}{1.537851in}}%
\pgfpathlineto{\pgfqpoint{3.625795in}{1.537850in}}%
\pgfpathlineto{\pgfqpoint{3.645500in}{1.533960in}}%
\pgfpathlineto{\pgfqpoint{3.665205in}{1.533067in}}%
\pgfpathlineto{\pgfqpoint{3.684911in}{1.527673in}}%
\pgfpathlineto{\pgfqpoint{3.704616in}{1.525005in}}%
\pgfpathlineto{\pgfqpoint{3.724321in}{1.524538in}}%
\pgfpathlineto{\pgfqpoint{3.744026in}{1.031117in}}%
\pgfpathlineto{\pgfqpoint{3.763731in}{1.013163in}}%
\pgfpathlineto{\pgfqpoint{3.783436in}{1.008663in}}%
\pgfpathlineto{\pgfqpoint{3.803141in}{1.005773in}}%
\pgfpathlineto{\pgfqpoint{3.842551in}{0.990557in}}%
\pgfpathlineto{\pgfqpoint{3.921371in}{0.990555in}}%
\pgfpathlineto{\pgfqpoint{3.941076in}{0.759670in}}%
\pgfpathlineto{\pgfqpoint{3.960781in}{0.745976in}}%
\pgfpathlineto{\pgfqpoint{3.980486in}{0.709008in}}%
\pgfpathlineto{\pgfqpoint{4.354882in}{0.708335in}}%
\pgfpathlineto{\pgfqpoint{4.374587in}{0.687179in}}%
\pgfpathlineto{\pgfqpoint{5.517480in}{0.687163in}}%
\pgfpathlineto{\pgfqpoint{5.517480in}{0.687163in}}%
\pgfusepath{stroke}%
\end{pgfscope}%
\begin{pgfscope}%
\pgfsetrectcap%
\pgfsetmiterjoin%
\pgfsetlinewidth{0.803000pt}%
\definecolor{currentstroke}{rgb}{0.000000,0.000000,0.000000}%
\pgfsetstrokecolor{currentstroke}%
\pgfsetdash{}{0pt}%
\pgfpathmoveto{\pgfqpoint{0.634568in}{0.565123in}}%
\pgfpathlineto{\pgfqpoint{0.634568in}{3.250000in}}%
\pgfusepath{stroke}%
\end{pgfscope}%
\begin{pgfscope}%
\pgfsetrectcap%
\pgfsetmiterjoin%
\pgfsetlinewidth{0.803000pt}%
\definecolor{currentstroke}{rgb}{0.000000,0.000000,0.000000}%
\pgfsetstrokecolor{currentstroke}%
\pgfsetdash{}{0pt}%
\pgfpathmoveto{\pgfqpoint{5.750000in}{0.565123in}}%
\pgfpathlineto{\pgfqpoint{5.750000in}{3.250000in}}%
\pgfusepath{stroke}%
\end{pgfscope}%
\begin{pgfscope}%
\pgfsetrectcap%
\pgfsetmiterjoin%
\pgfsetlinewidth{0.803000pt}%
\definecolor{currentstroke}{rgb}{0.000000,0.000000,0.000000}%
\pgfsetstrokecolor{currentstroke}%
\pgfsetdash{}{0pt}%
\pgfpathmoveto{\pgfqpoint{0.634568in}{0.565123in}}%
\pgfpathlineto{\pgfqpoint{5.750000in}{0.565123in}}%
\pgfusepath{stroke}%
\end{pgfscope}%
\begin{pgfscope}%
\pgfsetrectcap%
\pgfsetmiterjoin%
\pgfsetlinewidth{0.803000pt}%
\definecolor{currentstroke}{rgb}{0.000000,0.000000,0.000000}%
\pgfsetstrokecolor{currentstroke}%
\pgfsetdash{}{0pt}%
\pgfpathmoveto{\pgfqpoint{0.634568in}{3.250000in}}%
\pgfpathlineto{\pgfqpoint{5.750000in}{3.250000in}}%
\pgfusepath{stroke}%
\end{pgfscope}%
\begin{pgfscope}%
\pgfsetbuttcap%
\pgfsetmiterjoin%
\definecolor{currentfill}{rgb}{1.000000,1.000000,1.000000}%
\pgfsetfillcolor{currentfill}%
\pgfsetfillopacity{0.800000}%
\pgfsetlinewidth{1.003750pt}%
\definecolor{currentstroke}{rgb}{0.800000,0.800000,0.800000}%
\pgfsetstrokecolor{currentstroke}%
\pgfsetstrokeopacity{0.800000}%
\pgfsetdash{}{0pt}%
\pgfpathmoveto{\pgfqpoint{4.273532in}{2.751543in}}%
\pgfpathlineto{\pgfqpoint{5.652778in}{2.751543in}}%
\pgfpathquadraticcurveto{\pgfqpoint{5.680556in}{2.751543in}}{\pgfqpoint{5.680556in}{2.779321in}}%
\pgfpathlineto{\pgfqpoint{5.680556in}{3.152778in}}%
\pgfpathquadraticcurveto{\pgfqpoint{5.680556in}{3.180556in}}{\pgfqpoint{5.652778in}{3.180556in}}%
\pgfpathlineto{\pgfqpoint{4.273532in}{3.180556in}}%
\pgfpathquadraticcurveto{\pgfqpoint{4.245755in}{3.180556in}}{\pgfqpoint{4.245755in}{3.152778in}}%
\pgfpathlineto{\pgfqpoint{4.245755in}{2.779321in}}%
\pgfpathquadraticcurveto{\pgfqpoint{4.245755in}{2.751543in}}{\pgfqpoint{4.273532in}{2.751543in}}%
\pgfpathclose%
\pgfusepath{stroke,fill}%
\end{pgfscope}%
\begin{pgfscope}%
\pgfsetrectcap%
\pgfsetroundjoin%
\pgfsetlinewidth{2.007500pt}%
\definecolor{currentstroke}{rgb}{0.121569,0.466667,0.705882}%
\pgfsetstrokecolor{currentstroke}%
\pgfsetdash{}{0pt}%
\pgfpathmoveto{\pgfqpoint{4.301310in}{3.076389in}}%
\pgfpathlineto{\pgfqpoint{4.579088in}{3.076389in}}%
\pgfusepath{stroke}%
\end{pgfscope}%
\begin{pgfscope}%
\definecolor{textcolor}{rgb}{0.000000,0.000000,0.000000}%
\pgfsetstrokecolor{textcolor}%
\pgfsetfillcolor{textcolor}%
\pgftext[x=4.690199in,y=3.027778in,left,base]{\color{textcolor}\rmfamily\fontsize{10.000000}{12.000000}\selectfont Scena fittizia}%
\end{pgfscope}%
\begin{pgfscope}%
\pgfsetrectcap%
\pgfsetroundjoin%
\pgfsetlinewidth{2.007500pt}%
\definecolor{currentstroke}{rgb}{1.000000,0.498039,0.054902}%
\pgfsetstrokecolor{currentstroke}%
\pgfsetdash{}{0pt}%
\pgfpathmoveto{\pgfqpoint{4.301310in}{2.882716in}}%
\pgfpathlineto{\pgfqpoint{4.579088in}{2.882716in}}%
\pgfusepath{stroke}%
\end{pgfscope}%
\begin{pgfscope}%
\definecolor{textcolor}{rgb}{0.000000,0.000000,0.000000}%
\pgfsetstrokecolor{textcolor}%
\pgfsetfillcolor{textcolor}%
\pgftext[x=4.690199in,y=2.834105in,left,base]{\color{textcolor}\rmfamily\fontsize{10.000000}{12.000000}\selectfont Scena realistica}%
\end{pgfscope}%
\end{pgfpicture}%
\makeatother%
\endgroup%
%
                \caption{Valutazione Closest-First sulla scena fittizia}
                \label{fig:eval-cf}

                %% Creator: Matplotlib, PGF backend
%%
%% To include the figure in your LaTeX document, write
%%   \input{<filename>.pgf}
%%
%% Make sure the required packages are loaded in your preamble
%%   \usepackage{pgf}
%%
%% Figures using additional raster images can only be included by \input if
%% they are in the same directory as the main LaTeX file. For loading figures
%% from other directories you can use the `import` package
%%   \usepackage{import}
%%
%% and then include the figures with
%%   \import{<path to file>}{<filename>.pgf}
%%
%% Matplotlib used the following preamble
%%
\begingroup%
\makeatletter%
\begin{pgfpicture}%
\pgfpathrectangle{\pgfpointorigin}{\pgfqpoint{5.900000in}{3.400000in}}%
\pgfusepath{use as bounding box, clip}%
\begin{pgfscope}%
\pgfsetbuttcap%
\pgfsetmiterjoin%
\definecolor{currentfill}{rgb}{1.000000,1.000000,1.000000}%
\pgfsetfillcolor{currentfill}%
\pgfsetlinewidth{0.000000pt}%
\definecolor{currentstroke}{rgb}{1.000000,1.000000,1.000000}%
\pgfsetstrokecolor{currentstroke}%
\pgfsetdash{}{0pt}%
\pgfpathmoveto{\pgfqpoint{0.000000in}{0.000000in}}%
\pgfpathlineto{\pgfqpoint{5.900000in}{0.000000in}}%
\pgfpathlineto{\pgfqpoint{5.900000in}{3.400000in}}%
\pgfpathlineto{\pgfqpoint{0.000000in}{3.400000in}}%
\pgfpathclose%
\pgfusepath{fill}%
\end{pgfscope}%
\begin{pgfscope}%
\pgfsetbuttcap%
\pgfsetmiterjoin%
\definecolor{currentfill}{rgb}{1.000000,1.000000,1.000000}%
\pgfsetfillcolor{currentfill}%
\pgfsetlinewidth{0.000000pt}%
\definecolor{currentstroke}{rgb}{0.000000,0.000000,0.000000}%
\pgfsetstrokecolor{currentstroke}%
\pgfsetstrokeopacity{0.000000}%
\pgfsetdash{}{0pt}%
\pgfpathmoveto{\pgfqpoint{0.565124in}{0.565123in}}%
\pgfpathlineto{\pgfqpoint{5.750000in}{0.565123in}}%
\pgfpathlineto{\pgfqpoint{5.750000in}{3.250000in}}%
\pgfpathlineto{\pgfqpoint{0.565124in}{3.250000in}}%
\pgfpathclose%
\pgfusepath{fill}%
\end{pgfscope}%
\begin{pgfscope}%
\pgfsetbuttcap%
\pgfsetroundjoin%
\definecolor{currentfill}{rgb}{0.000000,0.000000,0.000000}%
\pgfsetfillcolor{currentfill}%
\pgfsetlinewidth{0.803000pt}%
\definecolor{currentstroke}{rgb}{0.000000,0.000000,0.000000}%
\pgfsetstrokecolor{currentstroke}%
\pgfsetdash{}{0pt}%
\pgfsys@defobject{currentmarker}{\pgfqpoint{0.000000in}{-0.048611in}}{\pgfqpoint{0.000000in}{0.000000in}}{%
\pgfpathmoveto{\pgfqpoint{0.000000in}{0.000000in}}%
\pgfpathlineto{\pgfqpoint{0.000000in}{-0.048611in}}%
\pgfusepath{stroke,fill}%
}%
\begin{pgfscope}%
\pgfsys@transformshift{0.800800in}{0.565123in}%
\pgfsys@useobject{currentmarker}{}%
\end{pgfscope}%
\end{pgfscope}%
\begin{pgfscope}%
\definecolor{textcolor}{rgb}{0.000000,0.000000,0.000000}%
\pgfsetstrokecolor{textcolor}%
\pgfsetfillcolor{textcolor}%
\pgftext[x=0.800800in,y=0.467901in,,top]{\color{textcolor}\rmfamily\fontsize{10.000000}{12.000000}\selectfont \(\displaystyle {0}\)}%
\end{pgfscope}%
\begin{pgfscope}%
\pgfsetbuttcap%
\pgfsetroundjoin%
\definecolor{currentfill}{rgb}{0.000000,0.000000,0.000000}%
\pgfsetfillcolor{currentfill}%
\pgfsetlinewidth{0.803000pt}%
\definecolor{currentstroke}{rgb}{0.000000,0.000000,0.000000}%
\pgfsetstrokecolor{currentstroke}%
\pgfsetdash{}{0pt}%
\pgfsys@defobject{currentmarker}{\pgfqpoint{0.000000in}{-0.048611in}}{\pgfqpoint{0.000000in}{0.000000in}}{%
\pgfpathmoveto{\pgfqpoint{0.000000in}{0.000000in}}%
\pgfpathlineto{\pgfqpoint{0.000000in}{-0.048611in}}%
\pgfusepath{stroke,fill}%
}%
\begin{pgfscope}%
\pgfsys@transformshift{1.799428in}{0.565123in}%
\pgfsys@useobject{currentmarker}{}%
\end{pgfscope}%
\end{pgfscope}%
\begin{pgfscope}%
\definecolor{textcolor}{rgb}{0.000000,0.000000,0.000000}%
\pgfsetstrokecolor{textcolor}%
\pgfsetfillcolor{textcolor}%
\pgftext[x=1.799428in,y=0.467901in,,top]{\color{textcolor}\rmfamily\fontsize{10.000000}{12.000000}\selectfont \(\displaystyle {50}\)}%
\end{pgfscope}%
\begin{pgfscope}%
\pgfsetbuttcap%
\pgfsetroundjoin%
\definecolor{currentfill}{rgb}{0.000000,0.000000,0.000000}%
\pgfsetfillcolor{currentfill}%
\pgfsetlinewidth{0.803000pt}%
\definecolor{currentstroke}{rgb}{0.000000,0.000000,0.000000}%
\pgfsetstrokecolor{currentstroke}%
\pgfsetdash{}{0pt}%
\pgfsys@defobject{currentmarker}{\pgfqpoint{0.000000in}{-0.048611in}}{\pgfqpoint{0.000000in}{0.000000in}}{%
\pgfpathmoveto{\pgfqpoint{0.000000in}{0.000000in}}%
\pgfpathlineto{\pgfqpoint{0.000000in}{-0.048611in}}%
\pgfusepath{stroke,fill}%
}%
\begin{pgfscope}%
\pgfsys@transformshift{2.798056in}{0.565123in}%
\pgfsys@useobject{currentmarker}{}%
\end{pgfscope}%
\end{pgfscope}%
\begin{pgfscope}%
\definecolor{textcolor}{rgb}{0.000000,0.000000,0.000000}%
\pgfsetstrokecolor{textcolor}%
\pgfsetfillcolor{textcolor}%
\pgftext[x=2.798056in,y=0.467901in,,top]{\color{textcolor}\rmfamily\fontsize{10.000000}{12.000000}\selectfont \(\displaystyle {100}\)}%
\end{pgfscope}%
\begin{pgfscope}%
\pgfsetbuttcap%
\pgfsetroundjoin%
\definecolor{currentfill}{rgb}{0.000000,0.000000,0.000000}%
\pgfsetfillcolor{currentfill}%
\pgfsetlinewidth{0.803000pt}%
\definecolor{currentstroke}{rgb}{0.000000,0.000000,0.000000}%
\pgfsetstrokecolor{currentstroke}%
\pgfsetdash{}{0pt}%
\pgfsys@defobject{currentmarker}{\pgfqpoint{0.000000in}{-0.048611in}}{\pgfqpoint{0.000000in}{0.000000in}}{%
\pgfpathmoveto{\pgfqpoint{0.000000in}{0.000000in}}%
\pgfpathlineto{\pgfqpoint{0.000000in}{-0.048611in}}%
\pgfusepath{stroke,fill}%
}%
\begin{pgfscope}%
\pgfsys@transformshift{3.796684in}{0.565123in}%
\pgfsys@useobject{currentmarker}{}%
\end{pgfscope}%
\end{pgfscope}%
\begin{pgfscope}%
\definecolor{textcolor}{rgb}{0.000000,0.000000,0.000000}%
\pgfsetstrokecolor{textcolor}%
\pgfsetfillcolor{textcolor}%
\pgftext[x=3.796684in,y=0.467901in,,top]{\color{textcolor}\rmfamily\fontsize{10.000000}{12.000000}\selectfont \(\displaystyle {150}\)}%
\end{pgfscope}%
\begin{pgfscope}%
\pgfsetbuttcap%
\pgfsetroundjoin%
\definecolor{currentfill}{rgb}{0.000000,0.000000,0.000000}%
\pgfsetfillcolor{currentfill}%
\pgfsetlinewidth{0.803000pt}%
\definecolor{currentstroke}{rgb}{0.000000,0.000000,0.000000}%
\pgfsetstrokecolor{currentstroke}%
\pgfsetdash{}{0pt}%
\pgfsys@defobject{currentmarker}{\pgfqpoint{0.000000in}{-0.048611in}}{\pgfqpoint{0.000000in}{0.000000in}}{%
\pgfpathmoveto{\pgfqpoint{0.000000in}{0.000000in}}%
\pgfpathlineto{\pgfqpoint{0.000000in}{-0.048611in}}%
\pgfusepath{stroke,fill}%
}%
\begin{pgfscope}%
\pgfsys@transformshift{4.795312in}{0.565123in}%
\pgfsys@useobject{currentmarker}{}%
\end{pgfscope}%
\end{pgfscope}%
\begin{pgfscope}%
\definecolor{textcolor}{rgb}{0.000000,0.000000,0.000000}%
\pgfsetstrokecolor{textcolor}%
\pgfsetfillcolor{textcolor}%
\pgftext[x=4.795312in,y=0.467901in,,top]{\color{textcolor}\rmfamily\fontsize{10.000000}{12.000000}\selectfont \(\displaystyle {200}\)}%
\end{pgfscope}%
\begin{pgfscope}%
\definecolor{textcolor}{rgb}{0.000000,0.000000,0.000000}%
\pgfsetstrokecolor{textcolor}%
\pgfsetfillcolor{textcolor}%
\pgftext[x=3.157562in,y=0.288889in,,top]{\color{textcolor}\rmfamily\fontsize{10.000000}{12.000000}\selectfont Tempo (Frame)}%
\end{pgfscope}%
\begin{pgfscope}%
\pgfsetbuttcap%
\pgfsetroundjoin%
\definecolor{currentfill}{rgb}{0.000000,0.000000,0.000000}%
\pgfsetfillcolor{currentfill}%
\pgfsetlinewidth{0.803000pt}%
\definecolor{currentstroke}{rgb}{0.000000,0.000000,0.000000}%
\pgfsetstrokecolor{currentstroke}%
\pgfsetdash{}{0pt}%
\pgfsys@defobject{currentmarker}{\pgfqpoint{-0.048611in}{0.000000in}}{\pgfqpoint{-0.000000in}{0.000000in}}{%
\pgfpathmoveto{\pgfqpoint{-0.000000in}{0.000000in}}%
\pgfpathlineto{\pgfqpoint{-0.048611in}{0.000000in}}%
\pgfusepath{stroke,fill}%
}%
\begin{pgfscope}%
\pgfsys@transformshift{0.565124in}{0.687163in}%
\pgfsys@useobject{currentmarker}{}%
\end{pgfscope}%
\end{pgfscope}%
\begin{pgfscope}%
\definecolor{textcolor}{rgb}{0.000000,0.000000,0.000000}%
\pgfsetstrokecolor{textcolor}%
\pgfsetfillcolor{textcolor}%
\pgftext[x=0.398457in, y=0.638938in, left, base]{\color{textcolor}\rmfamily\fontsize{10.000000}{12.000000}\selectfont \(\displaystyle {0}\)}%
\end{pgfscope}%
\begin{pgfscope}%
\pgfsetbuttcap%
\pgfsetroundjoin%
\definecolor{currentfill}{rgb}{0.000000,0.000000,0.000000}%
\pgfsetfillcolor{currentfill}%
\pgfsetlinewidth{0.803000pt}%
\definecolor{currentstroke}{rgb}{0.000000,0.000000,0.000000}%
\pgfsetstrokecolor{currentstroke}%
\pgfsetdash{}{0pt}%
\pgfsys@defobject{currentmarker}{\pgfqpoint{-0.048611in}{0.000000in}}{\pgfqpoint{-0.000000in}{0.000000in}}{%
\pgfpathmoveto{\pgfqpoint{-0.000000in}{0.000000in}}%
\pgfpathlineto{\pgfqpoint{-0.048611in}{0.000000in}}%
\pgfusepath{stroke,fill}%
}%
\begin{pgfscope}%
\pgfsys@transformshift{0.565124in}{1.221913in}%
\pgfsys@useobject{currentmarker}{}%
\end{pgfscope}%
\end{pgfscope}%
\begin{pgfscope}%
\definecolor{textcolor}{rgb}{0.000000,0.000000,0.000000}%
\pgfsetstrokecolor{textcolor}%
\pgfsetfillcolor{textcolor}%
\pgftext[x=0.329012in, y=1.173687in, left, base]{\color{textcolor}\rmfamily\fontsize{10.000000}{12.000000}\selectfont \(\displaystyle {20}\)}%
\end{pgfscope}%
\begin{pgfscope}%
\pgfsetbuttcap%
\pgfsetroundjoin%
\definecolor{currentfill}{rgb}{0.000000,0.000000,0.000000}%
\pgfsetfillcolor{currentfill}%
\pgfsetlinewidth{0.803000pt}%
\definecolor{currentstroke}{rgb}{0.000000,0.000000,0.000000}%
\pgfsetstrokecolor{currentstroke}%
\pgfsetdash{}{0pt}%
\pgfsys@defobject{currentmarker}{\pgfqpoint{-0.048611in}{0.000000in}}{\pgfqpoint{-0.000000in}{0.000000in}}{%
\pgfpathmoveto{\pgfqpoint{-0.000000in}{0.000000in}}%
\pgfpathlineto{\pgfqpoint{-0.048611in}{0.000000in}}%
\pgfusepath{stroke,fill}%
}%
\begin{pgfscope}%
\pgfsys@transformshift{0.565124in}{1.756662in}%
\pgfsys@useobject{currentmarker}{}%
\end{pgfscope}%
\end{pgfscope}%
\begin{pgfscope}%
\definecolor{textcolor}{rgb}{0.000000,0.000000,0.000000}%
\pgfsetstrokecolor{textcolor}%
\pgfsetfillcolor{textcolor}%
\pgftext[x=0.329012in, y=1.708437in, left, base]{\color{textcolor}\rmfamily\fontsize{10.000000}{12.000000}\selectfont \(\displaystyle {40}\)}%
\end{pgfscope}%
\begin{pgfscope}%
\pgfsetbuttcap%
\pgfsetroundjoin%
\definecolor{currentfill}{rgb}{0.000000,0.000000,0.000000}%
\pgfsetfillcolor{currentfill}%
\pgfsetlinewidth{0.803000pt}%
\definecolor{currentstroke}{rgb}{0.000000,0.000000,0.000000}%
\pgfsetstrokecolor{currentstroke}%
\pgfsetdash{}{0pt}%
\pgfsys@defobject{currentmarker}{\pgfqpoint{-0.048611in}{0.000000in}}{\pgfqpoint{-0.000000in}{0.000000in}}{%
\pgfpathmoveto{\pgfqpoint{-0.000000in}{0.000000in}}%
\pgfpathlineto{\pgfqpoint{-0.048611in}{0.000000in}}%
\pgfusepath{stroke,fill}%
}%
\begin{pgfscope}%
\pgfsys@transformshift{0.565124in}{2.291412in}%
\pgfsys@useobject{currentmarker}{}%
\end{pgfscope}%
\end{pgfscope}%
\begin{pgfscope}%
\definecolor{textcolor}{rgb}{0.000000,0.000000,0.000000}%
\pgfsetstrokecolor{textcolor}%
\pgfsetfillcolor{textcolor}%
\pgftext[x=0.329012in, y=2.243186in, left, base]{\color{textcolor}\rmfamily\fontsize{10.000000}{12.000000}\selectfont \(\displaystyle {60}\)}%
\end{pgfscope}%
\begin{pgfscope}%
\pgfsetbuttcap%
\pgfsetroundjoin%
\definecolor{currentfill}{rgb}{0.000000,0.000000,0.000000}%
\pgfsetfillcolor{currentfill}%
\pgfsetlinewidth{0.803000pt}%
\definecolor{currentstroke}{rgb}{0.000000,0.000000,0.000000}%
\pgfsetstrokecolor{currentstroke}%
\pgfsetdash{}{0pt}%
\pgfsys@defobject{currentmarker}{\pgfqpoint{-0.048611in}{0.000000in}}{\pgfqpoint{-0.000000in}{0.000000in}}{%
\pgfpathmoveto{\pgfqpoint{-0.000000in}{0.000000in}}%
\pgfpathlineto{\pgfqpoint{-0.048611in}{0.000000in}}%
\pgfusepath{stroke,fill}%
}%
\begin{pgfscope}%
\pgfsys@transformshift{0.565124in}{2.826161in}%
\pgfsys@useobject{currentmarker}{}%
\end{pgfscope}%
\end{pgfscope}%
\begin{pgfscope}%
\definecolor{textcolor}{rgb}{0.000000,0.000000,0.000000}%
\pgfsetstrokecolor{textcolor}%
\pgfsetfillcolor{textcolor}%
\pgftext[x=0.329012in, y=2.777936in, left, base]{\color{textcolor}\rmfamily\fontsize{10.000000}{12.000000}\selectfont \(\displaystyle {80}\)}%
\end{pgfscope}%
\begin{pgfscope}%
\definecolor{textcolor}{rgb}{0.000000,0.000000,0.000000}%
\pgfsetstrokecolor{textcolor}%
\pgfsetfillcolor{textcolor}%
\pgftext[x=0.273457in,y=1.907562in,,bottom,rotate=90.000000]{\color{textcolor}\rmfamily\fontsize{10.000000}{12.000000}\selectfont Differenza percettiva}%
\end{pgfscope}%
\begin{pgfscope}%
\pgfpathrectangle{\pgfqpoint{0.565124in}{0.565123in}}{\pgfqpoint{5.184876in}{2.684877in}}%
\pgfusepath{clip}%
\pgfsetrectcap%
\pgfsetroundjoin%
\pgfsetlinewidth{2.007500pt}%
\definecolor{currentstroke}{rgb}{0.121569,0.466667,0.705882}%
\pgfsetstrokecolor{currentstroke}%
\pgfsetdash{}{0pt}%
\pgfpathmoveto{\pgfqpoint{0.800800in}{1.561468in}}%
\pgfpathlineto{\pgfqpoint{0.820773in}{1.510069in}}%
\pgfpathlineto{\pgfqpoint{0.840745in}{1.467996in}}%
\pgfpathlineto{\pgfqpoint{0.880690in}{1.410692in}}%
\pgfpathlineto{\pgfqpoint{0.900663in}{1.388364in}}%
\pgfpathlineto{\pgfqpoint{0.920635in}{1.378362in}}%
\pgfpathlineto{\pgfqpoint{0.940608in}{1.360700in}}%
\pgfpathlineto{\pgfqpoint{0.960580in}{1.344659in}}%
\pgfpathlineto{\pgfqpoint{0.980553in}{1.319806in}}%
\pgfpathlineto{\pgfqpoint{1.000526in}{1.308910in}}%
\pgfpathlineto{\pgfqpoint{1.020498in}{1.293976in}}%
\pgfpathlineto{\pgfqpoint{1.040471in}{1.282749in}}%
\pgfpathlineto{\pgfqpoint{1.060443in}{1.275091in}}%
\pgfpathlineto{\pgfqpoint{1.080416in}{1.264528in}}%
\pgfpathlineto{\pgfqpoint{1.100388in}{1.246918in}}%
\pgfpathlineto{\pgfqpoint{1.120361in}{1.236678in}}%
\pgfpathlineto{\pgfqpoint{1.140333in}{1.146299in}}%
\pgfpathlineto{\pgfqpoint{1.160306in}{1.143040in}}%
\pgfpathlineto{\pgfqpoint{1.220224in}{1.117246in}}%
\pgfpathlineto{\pgfqpoint{1.240196in}{1.113690in}}%
\pgfpathlineto{\pgfqpoint{1.260169in}{1.105601in}}%
\pgfpathlineto{\pgfqpoint{1.280141in}{1.032741in}}%
\pgfpathlineto{\pgfqpoint{1.300114in}{1.000788in}}%
\pgfpathlineto{\pgfqpoint{1.320086in}{0.994582in}}%
\pgfpathlineto{\pgfqpoint{1.340059in}{0.993358in}}%
\pgfpathlineto{\pgfqpoint{1.360032in}{0.978879in}}%
\pgfpathlineto{\pgfqpoint{1.380004in}{0.975466in}}%
\pgfpathlineto{\pgfqpoint{1.399977in}{0.975447in}}%
\pgfpathlineto{\pgfqpoint{1.419949in}{0.973715in}}%
\pgfpathlineto{\pgfqpoint{1.439922in}{0.973715in}}%
\pgfpathlineto{\pgfqpoint{1.459894in}{0.970967in}}%
\pgfpathlineto{\pgfqpoint{1.479867in}{0.958129in}}%
\pgfpathlineto{\pgfqpoint{1.499839in}{0.958129in}}%
\pgfpathlineto{\pgfqpoint{1.519812in}{0.949113in}}%
\pgfpathlineto{\pgfqpoint{1.539785in}{0.947546in}}%
\pgfpathlineto{\pgfqpoint{1.559757in}{0.929692in}}%
\pgfpathlineto{\pgfqpoint{1.579730in}{0.929692in}}%
\pgfpathlineto{\pgfqpoint{1.599702in}{0.927882in}}%
\pgfpathlineto{\pgfqpoint{1.619675in}{0.912444in}}%
\pgfpathlineto{\pgfqpoint{1.739510in}{0.904262in}}%
\pgfpathlineto{\pgfqpoint{1.759483in}{0.903897in}}%
\pgfpathlineto{\pgfqpoint{1.779455in}{0.868436in}}%
\pgfpathlineto{\pgfqpoint{1.799428in}{0.865097in}}%
\pgfpathlineto{\pgfqpoint{1.819400in}{0.856896in}}%
\pgfpathlineto{\pgfqpoint{1.839373in}{0.851986in}}%
\pgfpathlineto{\pgfqpoint{1.859346in}{0.851985in}}%
\pgfpathlineto{\pgfqpoint{1.879318in}{0.819902in}}%
\pgfpathlineto{\pgfqpoint{1.919263in}{0.818160in}}%
\pgfpathlineto{\pgfqpoint{1.939236in}{0.784476in}}%
\pgfpathlineto{\pgfqpoint{1.959208in}{0.780387in}}%
\pgfpathlineto{\pgfqpoint{1.979181in}{0.780381in}}%
\pgfpathlineto{\pgfqpoint{1.999153in}{0.778382in}}%
\pgfpathlineto{\pgfqpoint{2.039099in}{0.771766in}}%
\pgfpathlineto{\pgfqpoint{2.059071in}{0.770499in}}%
\pgfpathlineto{\pgfqpoint{2.079044in}{0.762402in}}%
\pgfpathlineto{\pgfqpoint{2.099016in}{0.760096in}}%
\pgfpathlineto{\pgfqpoint{2.178906in}{0.756174in}}%
\pgfpathlineto{\pgfqpoint{2.218852in}{0.755998in}}%
\pgfpathlineto{\pgfqpoint{2.238824in}{0.754073in}}%
\pgfpathlineto{\pgfqpoint{2.258797in}{0.753831in}}%
\pgfpathlineto{\pgfqpoint{2.298742in}{0.744237in}}%
\pgfpathlineto{\pgfqpoint{2.318714in}{0.740989in}}%
\pgfpathlineto{\pgfqpoint{2.338687in}{0.715816in}}%
\pgfpathlineto{\pgfqpoint{2.398605in}{0.714878in}}%
\pgfpathlineto{\pgfqpoint{2.418577in}{0.713059in}}%
\pgfpathlineto{\pgfqpoint{2.458522in}{0.712405in}}%
\pgfpathlineto{\pgfqpoint{2.478495in}{0.710825in}}%
\pgfpathlineto{\pgfqpoint{2.538413in}{0.710357in}}%
\pgfpathlineto{\pgfqpoint{2.658248in}{0.710145in}}%
\pgfpathlineto{\pgfqpoint{2.718166in}{0.709840in}}%
\pgfpathlineto{\pgfqpoint{2.738138in}{0.698363in}}%
\pgfpathlineto{\pgfqpoint{2.798056in}{0.697489in}}%
\pgfpathlineto{\pgfqpoint{2.857973in}{0.697082in}}%
\pgfpathlineto{\pgfqpoint{3.017754in}{0.696709in}}%
\pgfpathlineto{\pgfqpoint{3.037727in}{0.689823in}}%
\pgfpathlineto{\pgfqpoint{3.077672in}{0.689771in}}%
\pgfpathlineto{\pgfqpoint{3.097644in}{0.688638in}}%
\pgfpathlineto{\pgfqpoint{3.177534in}{0.687884in}}%
\pgfpathlineto{\pgfqpoint{3.217480in}{0.687165in}}%
\pgfpathlineto{\pgfqpoint{5.494351in}{0.687163in}}%
\pgfpathlineto{\pgfqpoint{5.494351in}{0.687163in}}%
\pgfusepath{stroke}%
\end{pgfscope}%
\begin{pgfscope}%
\pgfpathrectangle{\pgfqpoint{0.565124in}{0.565123in}}{\pgfqpoint{5.184876in}{2.684877in}}%
\pgfusepath{clip}%
\pgfsetrectcap%
\pgfsetroundjoin%
\pgfsetlinewidth{2.007500pt}%
\definecolor{currentstroke}{rgb}{1.000000,0.498039,0.054902}%
\pgfsetstrokecolor{currentstroke}%
\pgfsetdash{}{0pt}%
\pgfpathmoveto{\pgfqpoint{0.820773in}{3.127957in}}%
\pgfpathlineto{\pgfqpoint{0.880690in}{3.127960in}}%
\pgfpathlineto{\pgfqpoint{0.900663in}{3.059568in}}%
\pgfpathlineto{\pgfqpoint{1.160306in}{3.059572in}}%
\pgfpathlineto{\pgfqpoint{1.180279in}{3.036434in}}%
\pgfpathlineto{\pgfqpoint{1.360032in}{3.036438in}}%
\pgfpathlineto{\pgfqpoint{1.380004in}{3.022542in}}%
\pgfpathlineto{\pgfqpoint{1.459894in}{3.022543in}}%
\pgfpathlineto{\pgfqpoint{1.479867in}{3.018213in}}%
\pgfpathlineto{\pgfqpoint{1.499839in}{3.009950in}}%
\pgfpathlineto{\pgfqpoint{1.519812in}{3.009771in}}%
\pgfpathlineto{\pgfqpoint{1.559757in}{2.968472in}}%
\pgfpathlineto{\pgfqpoint{1.639647in}{2.968468in}}%
\pgfpathlineto{\pgfqpoint{1.659620in}{2.956617in}}%
\pgfpathlineto{\pgfqpoint{1.919263in}{2.956600in}}%
\pgfpathlineto{\pgfqpoint{1.939236in}{2.952700in}}%
\pgfpathlineto{\pgfqpoint{2.019126in}{2.952686in}}%
\pgfpathlineto{\pgfqpoint{2.039099in}{2.652358in}}%
\pgfpathlineto{\pgfqpoint{2.138961in}{2.652349in}}%
\pgfpathlineto{\pgfqpoint{2.158934in}{2.596027in}}%
\pgfpathlineto{\pgfqpoint{2.258797in}{2.595358in}}%
\pgfpathlineto{\pgfqpoint{2.278769in}{2.593842in}}%
\pgfpathlineto{\pgfqpoint{2.298742in}{2.098955in}}%
\pgfpathlineto{\pgfqpoint{2.358660in}{2.098329in}}%
\pgfpathlineto{\pgfqpoint{2.378632in}{2.085753in}}%
\pgfpathlineto{\pgfqpoint{2.398605in}{2.081974in}}%
\pgfpathlineto{\pgfqpoint{2.418577in}{2.080183in}}%
\pgfpathlineto{\pgfqpoint{2.438550in}{2.052172in}}%
\pgfpathlineto{\pgfqpoint{2.458522in}{1.901880in}}%
\pgfpathlineto{\pgfqpoint{2.478495in}{1.885648in}}%
\pgfpathlineto{\pgfqpoint{2.498467in}{1.885647in}}%
\pgfpathlineto{\pgfqpoint{2.518440in}{1.884285in}}%
\pgfpathlineto{\pgfqpoint{2.538413in}{1.881313in}}%
\pgfpathlineto{\pgfqpoint{2.558385in}{1.876640in}}%
\pgfpathlineto{\pgfqpoint{2.578358in}{1.875545in}}%
\pgfpathlineto{\pgfqpoint{2.598330in}{1.869515in}}%
\pgfpathlineto{\pgfqpoint{2.618303in}{1.866576in}}%
\pgfpathlineto{\pgfqpoint{2.638275in}{1.356324in}}%
\pgfpathlineto{\pgfqpoint{2.658248in}{1.340066in}}%
\pgfpathlineto{\pgfqpoint{2.678220in}{1.327481in}}%
\pgfpathlineto{\pgfqpoint{2.718166in}{1.327482in}}%
\pgfpathlineto{\pgfqpoint{2.738138in}{1.020726in}}%
\pgfpathlineto{\pgfqpoint{2.778083in}{1.019960in}}%
\pgfpathlineto{\pgfqpoint{2.857973in}{1.019956in}}%
\pgfpathlineto{\pgfqpoint{2.877946in}{0.921188in}}%
\pgfpathlineto{\pgfqpoint{2.937864in}{0.921185in}}%
\pgfpathlineto{\pgfqpoint{2.957836in}{0.917182in}}%
\pgfpathlineto{\pgfqpoint{3.257425in}{0.917180in}}%
\pgfpathlineto{\pgfqpoint{3.277397in}{0.915183in}}%
\pgfpathlineto{\pgfqpoint{3.477123in}{0.915183in}}%
\pgfpathlineto{\pgfqpoint{3.497095in}{0.913301in}}%
\pgfpathlineto{\pgfqpoint{3.517068in}{0.913300in}}%
\pgfpathlineto{\pgfqpoint{3.537040in}{0.904603in}}%
\pgfpathlineto{\pgfqpoint{3.576986in}{0.903555in}}%
\pgfpathlineto{\pgfqpoint{3.596958in}{0.839626in}}%
\pgfpathlineto{\pgfqpoint{3.616931in}{0.839625in}}%
\pgfpathlineto{\pgfqpoint{3.636903in}{0.837011in}}%
\pgfpathlineto{\pgfqpoint{3.656876in}{0.810542in}}%
\pgfpathlineto{\pgfqpoint{3.676848in}{0.802567in}}%
\pgfpathlineto{\pgfqpoint{3.696821in}{0.765725in}}%
\pgfpathlineto{\pgfqpoint{3.796684in}{0.765570in}}%
\pgfpathlineto{\pgfqpoint{3.816656in}{0.754226in}}%
\pgfpathlineto{\pgfqpoint{3.836629in}{0.753059in}}%
\pgfpathlineto{\pgfqpoint{3.856601in}{0.746093in}}%
\pgfpathlineto{\pgfqpoint{3.876574in}{0.726755in}}%
\pgfpathlineto{\pgfqpoint{3.896547in}{0.726219in}}%
\pgfpathlineto{\pgfqpoint{3.916519in}{0.719411in}}%
\pgfpathlineto{\pgfqpoint{3.936492in}{0.715859in}}%
\pgfpathlineto{\pgfqpoint{3.956464in}{0.714092in}}%
\pgfpathlineto{\pgfqpoint{3.976437in}{0.711171in}}%
\pgfpathlineto{\pgfqpoint{4.375888in}{0.711165in}}%
\pgfpathlineto{\pgfqpoint{4.395861in}{0.687174in}}%
\pgfpathlineto{\pgfqpoint{5.514324in}{0.687163in}}%
\pgfpathlineto{\pgfqpoint{5.514324in}{0.687163in}}%
\pgfusepath{stroke}%
\end{pgfscope}%
\begin{pgfscope}%
\pgfsetrectcap%
\pgfsetmiterjoin%
\pgfsetlinewidth{0.803000pt}%
\definecolor{currentstroke}{rgb}{0.000000,0.000000,0.000000}%
\pgfsetstrokecolor{currentstroke}%
\pgfsetdash{}{0pt}%
\pgfpathmoveto{\pgfqpoint{0.565124in}{0.565123in}}%
\pgfpathlineto{\pgfqpoint{0.565124in}{3.250000in}}%
\pgfusepath{stroke}%
\end{pgfscope}%
\begin{pgfscope}%
\pgfsetrectcap%
\pgfsetmiterjoin%
\pgfsetlinewidth{0.803000pt}%
\definecolor{currentstroke}{rgb}{0.000000,0.000000,0.000000}%
\pgfsetstrokecolor{currentstroke}%
\pgfsetdash{}{0pt}%
\pgfpathmoveto{\pgfqpoint{5.750000in}{0.565123in}}%
\pgfpathlineto{\pgfqpoint{5.750000in}{3.250000in}}%
\pgfusepath{stroke}%
\end{pgfscope}%
\begin{pgfscope}%
\pgfsetrectcap%
\pgfsetmiterjoin%
\pgfsetlinewidth{0.803000pt}%
\definecolor{currentstroke}{rgb}{0.000000,0.000000,0.000000}%
\pgfsetstrokecolor{currentstroke}%
\pgfsetdash{}{0pt}%
\pgfpathmoveto{\pgfqpoint{0.565124in}{0.565123in}}%
\pgfpathlineto{\pgfqpoint{5.750000in}{0.565123in}}%
\pgfusepath{stroke}%
\end{pgfscope}%
\begin{pgfscope}%
\pgfsetrectcap%
\pgfsetmiterjoin%
\pgfsetlinewidth{0.803000pt}%
\definecolor{currentstroke}{rgb}{0.000000,0.000000,0.000000}%
\pgfsetstrokecolor{currentstroke}%
\pgfsetdash{}{0pt}%
\pgfpathmoveto{\pgfqpoint{0.565124in}{3.250000in}}%
\pgfpathlineto{\pgfqpoint{5.750000in}{3.250000in}}%
\pgfusepath{stroke}%
\end{pgfscope}%
\begin{pgfscope}%
\pgfsetbuttcap%
\pgfsetmiterjoin%
\definecolor{currentfill}{rgb}{1.000000,1.000000,1.000000}%
\pgfsetfillcolor{currentfill}%
\pgfsetfillopacity{0.800000}%
\pgfsetlinewidth{1.003750pt}%
\definecolor{currentstroke}{rgb}{0.800000,0.800000,0.800000}%
\pgfsetstrokecolor{currentstroke}%
\pgfsetstrokeopacity{0.800000}%
\pgfsetdash{}{0pt}%
\pgfpathmoveto{\pgfqpoint{4.273532in}{2.751543in}}%
\pgfpathlineto{\pgfqpoint{5.652778in}{2.751543in}}%
\pgfpathquadraticcurveto{\pgfqpoint{5.680556in}{2.751543in}}{\pgfqpoint{5.680556in}{2.779321in}}%
\pgfpathlineto{\pgfqpoint{5.680556in}{3.152778in}}%
\pgfpathquadraticcurveto{\pgfqpoint{5.680556in}{3.180556in}}{\pgfqpoint{5.652778in}{3.180556in}}%
\pgfpathlineto{\pgfqpoint{4.273532in}{3.180556in}}%
\pgfpathquadraticcurveto{\pgfqpoint{4.245755in}{3.180556in}}{\pgfqpoint{4.245755in}{3.152778in}}%
\pgfpathlineto{\pgfqpoint{4.245755in}{2.779321in}}%
\pgfpathquadraticcurveto{\pgfqpoint{4.245755in}{2.751543in}}{\pgfqpoint{4.273532in}{2.751543in}}%
\pgfpathclose%
\pgfusepath{stroke,fill}%
\end{pgfscope}%
\begin{pgfscope}%
\pgfsetrectcap%
\pgfsetroundjoin%
\pgfsetlinewidth{2.007500pt}%
\definecolor{currentstroke}{rgb}{0.121569,0.466667,0.705882}%
\pgfsetstrokecolor{currentstroke}%
\pgfsetdash{}{0pt}%
\pgfpathmoveto{\pgfqpoint{4.301310in}{3.076389in}}%
\pgfpathlineto{\pgfqpoint{4.579088in}{3.076389in}}%
\pgfusepath{stroke}%
\end{pgfscope}%
\begin{pgfscope}%
\definecolor{textcolor}{rgb}{0.000000,0.000000,0.000000}%
\pgfsetstrokecolor{textcolor}%
\pgfsetfillcolor{textcolor}%
\pgftext[x=4.690199in,y=3.027778in,left,base]{\color{textcolor}\rmfamily\fontsize{10.000000}{12.000000}\selectfont Scena fittizia}%
\end{pgfscope}%
\begin{pgfscope}%
\pgfsetrectcap%
\pgfsetroundjoin%
\pgfsetlinewidth{2.007500pt}%
\definecolor{currentstroke}{rgb}{1.000000,0.498039,0.054902}%
\pgfsetstrokecolor{currentstroke}%
\pgfsetdash{}{0pt}%
\pgfpathmoveto{\pgfqpoint{4.301310in}{2.882716in}}%
\pgfpathlineto{\pgfqpoint{4.579088in}{2.882716in}}%
\pgfusepath{stroke}%
\end{pgfscope}%
\begin{pgfscope}%
\definecolor{textcolor}{rgb}{0.000000,0.000000,0.000000}%
\pgfsetstrokecolor{textcolor}%
\pgfsetfillcolor{textcolor}%
\pgftext[x=4.690199in,y=2.834105in,left,base]{\color{textcolor}\rmfamily\fontsize{10.000000}{12.000000}\selectfont Scena realistica}%
\end{pgfscope}%
\end{pgfpicture}%
\makeatother%
\endgroup%

                \caption{Valutazione Closest-First on Camera}
                \label{fig:eval-cfv}
            \end{figure}
        
        \subsection{Closest-First on Camera}
            La seguente politica, fornendo priorità alle istanze di asset interne al view-frustum, elimina il problema che si presentava nella strategia analizzata al paragrafo precedente: la presenza di un grande numero di istanze di asset non in vista. Questa raffinatura presenta un miglioramento del 58.3\% per la scena fittizia e del 41.6\% per la scena realistica.

        \newpage            
        \subsection{Sphere Tracing}
            % sphere tracing: % evidenzia le differenze con l'angolo della luce
            \subsubsection{Valutazione della distanza}
            Considerare prioritari le istanze che compongono la vista, per quanto ragionevole, non porta apparenti miglioramenti nella scena realistica (figura \ref{fig:eval-stdist}). Questo comportamento può essere motivato dal fatto che l'inquadratura considerata è quasi priva di istanze occluse più vicine di altre istanze non occluse, eliminando il beneficio principale di questa politica. Nella scena fittizia invece, grazie alla presenza di istanze vicine e completamente occluse, si nota un miglioramento, benché leggero, del 6.1\%.

            È opportuno notare che i risultati di questa strategia si alterano al variare dell'angolo della luce. Variando l'angolo da $30^\circ$ a $5^\circ$ si nota una variazione del 5.6\% per la scena fittizia e del 7.1\% per la scena realistica rispetto alle valutazioni effettuate con la fonte luminosa posta a $30^\circ$. Questa politica fornisce priorità agli oggetti in vista, non considerando l'impatto percettivo delle ombre portate causate da istanze di asset non in vista. Questo implica che aumentare o diminuire la quantità di ombre portate generate da asset non in vista, causato da una modifica dell'angolo di provenienza della luce, implica una diversa efficacia della strategia.
            Si osserva che il caso ideale è quello in cui la fonte luminosa splenda verso la camera, cosicché gli asset generatori di ombre portate siano principalmente quelli in vista. 

            \begin{figure}[htb!]
                \centering
                %% Creator: Matplotlib, PGF backend
%%
%% To include the figure in your LaTeX document, write
%%   \input{<filename>.pgf}
%%
%% Make sure the required packages are loaded in your preamble
%%   \usepackage{pgf}
%%
%% Figures using additional raster images can only be included by \input if
%% they are in the same directory as the main LaTeX file. For loading figures
%% from other directories you can use the `import` package
%%   \usepackage{import}
%%
%% and then include the figures with
%%   \import{<path to file>}{<filename>.pgf}
%%
%% Matplotlib used the following preamble
%%
\begingroup%
\makeatletter%
\begin{pgfpicture}%
\pgfpathrectangle{\pgfpointorigin}{\pgfqpoint{5.900000in}{3.400000in}}%
\pgfusepath{use as bounding box, clip}%
\begin{pgfscope}%
\pgfsetbuttcap%
\pgfsetmiterjoin%
\definecolor{currentfill}{rgb}{1.000000,1.000000,1.000000}%
\pgfsetfillcolor{currentfill}%
\pgfsetlinewidth{0.000000pt}%
\definecolor{currentstroke}{rgb}{1.000000,1.000000,1.000000}%
\pgfsetstrokecolor{currentstroke}%
\pgfsetdash{}{0pt}%
\pgfpathmoveto{\pgfqpoint{0.000000in}{0.000000in}}%
\pgfpathlineto{\pgfqpoint{5.900000in}{0.000000in}}%
\pgfpathlineto{\pgfqpoint{5.900000in}{3.400000in}}%
\pgfpathlineto{\pgfqpoint{0.000000in}{3.400000in}}%
\pgfpathclose%
\pgfusepath{fill}%
\end{pgfscope}%
\begin{pgfscope}%
\pgfsetbuttcap%
\pgfsetmiterjoin%
\definecolor{currentfill}{rgb}{1.000000,1.000000,1.000000}%
\pgfsetfillcolor{currentfill}%
\pgfsetlinewidth{0.000000pt}%
\definecolor{currentstroke}{rgb}{0.000000,0.000000,0.000000}%
\pgfsetstrokecolor{currentstroke}%
\pgfsetstrokeopacity{0.000000}%
\pgfsetdash{}{0pt}%
\pgfpathmoveto{\pgfqpoint{0.565124in}{0.565123in}}%
\pgfpathlineto{\pgfqpoint{5.750000in}{0.565123in}}%
\pgfpathlineto{\pgfqpoint{5.750000in}{3.250000in}}%
\pgfpathlineto{\pgfqpoint{0.565124in}{3.250000in}}%
\pgfpathclose%
\pgfusepath{fill}%
\end{pgfscope}%
\begin{pgfscope}%
\pgfsetbuttcap%
\pgfsetroundjoin%
\definecolor{currentfill}{rgb}{0.000000,0.000000,0.000000}%
\pgfsetfillcolor{currentfill}%
\pgfsetlinewidth{0.803000pt}%
\definecolor{currentstroke}{rgb}{0.000000,0.000000,0.000000}%
\pgfsetstrokecolor{currentstroke}%
\pgfsetdash{}{0pt}%
\pgfsys@defobject{currentmarker}{\pgfqpoint{0.000000in}{-0.048611in}}{\pgfqpoint{0.000000in}{0.000000in}}{%
\pgfpathmoveto{\pgfqpoint{0.000000in}{0.000000in}}%
\pgfpathlineto{\pgfqpoint{0.000000in}{-0.048611in}}%
\pgfusepath{stroke,fill}%
}%
\begin{pgfscope}%
\pgfsys@transformshift{0.800800in}{0.565123in}%
\pgfsys@useobject{currentmarker}{}%
\end{pgfscope}%
\end{pgfscope}%
\begin{pgfscope}%
\definecolor{textcolor}{rgb}{0.000000,0.000000,0.000000}%
\pgfsetstrokecolor{textcolor}%
\pgfsetfillcolor{textcolor}%
\pgftext[x=0.800800in,y=0.467901in,,top]{\color{textcolor}\rmfamily\fontsize{10.000000}{12.000000}\selectfont \(\displaystyle {0}\)}%
\end{pgfscope}%
\begin{pgfscope}%
\pgfsetbuttcap%
\pgfsetroundjoin%
\definecolor{currentfill}{rgb}{0.000000,0.000000,0.000000}%
\pgfsetfillcolor{currentfill}%
\pgfsetlinewidth{0.803000pt}%
\definecolor{currentstroke}{rgb}{0.000000,0.000000,0.000000}%
\pgfsetstrokecolor{currentstroke}%
\pgfsetdash{}{0pt}%
\pgfsys@defobject{currentmarker}{\pgfqpoint{0.000000in}{-0.048611in}}{\pgfqpoint{0.000000in}{0.000000in}}{%
\pgfpathmoveto{\pgfqpoint{0.000000in}{0.000000in}}%
\pgfpathlineto{\pgfqpoint{0.000000in}{-0.048611in}}%
\pgfusepath{stroke,fill}%
}%
\begin{pgfscope}%
\pgfsys@transformshift{1.799428in}{0.565123in}%
\pgfsys@useobject{currentmarker}{}%
\end{pgfscope}%
\end{pgfscope}%
\begin{pgfscope}%
\definecolor{textcolor}{rgb}{0.000000,0.000000,0.000000}%
\pgfsetstrokecolor{textcolor}%
\pgfsetfillcolor{textcolor}%
\pgftext[x=1.799428in,y=0.467901in,,top]{\color{textcolor}\rmfamily\fontsize{10.000000}{12.000000}\selectfont \(\displaystyle {50}\)}%
\end{pgfscope}%
\begin{pgfscope}%
\pgfsetbuttcap%
\pgfsetroundjoin%
\definecolor{currentfill}{rgb}{0.000000,0.000000,0.000000}%
\pgfsetfillcolor{currentfill}%
\pgfsetlinewidth{0.803000pt}%
\definecolor{currentstroke}{rgb}{0.000000,0.000000,0.000000}%
\pgfsetstrokecolor{currentstroke}%
\pgfsetdash{}{0pt}%
\pgfsys@defobject{currentmarker}{\pgfqpoint{0.000000in}{-0.048611in}}{\pgfqpoint{0.000000in}{0.000000in}}{%
\pgfpathmoveto{\pgfqpoint{0.000000in}{0.000000in}}%
\pgfpathlineto{\pgfqpoint{0.000000in}{-0.048611in}}%
\pgfusepath{stroke,fill}%
}%
\begin{pgfscope}%
\pgfsys@transformshift{2.798056in}{0.565123in}%
\pgfsys@useobject{currentmarker}{}%
\end{pgfscope}%
\end{pgfscope}%
\begin{pgfscope}%
\definecolor{textcolor}{rgb}{0.000000,0.000000,0.000000}%
\pgfsetstrokecolor{textcolor}%
\pgfsetfillcolor{textcolor}%
\pgftext[x=2.798056in,y=0.467901in,,top]{\color{textcolor}\rmfamily\fontsize{10.000000}{12.000000}\selectfont \(\displaystyle {100}\)}%
\end{pgfscope}%
\begin{pgfscope}%
\pgfsetbuttcap%
\pgfsetroundjoin%
\definecolor{currentfill}{rgb}{0.000000,0.000000,0.000000}%
\pgfsetfillcolor{currentfill}%
\pgfsetlinewidth{0.803000pt}%
\definecolor{currentstroke}{rgb}{0.000000,0.000000,0.000000}%
\pgfsetstrokecolor{currentstroke}%
\pgfsetdash{}{0pt}%
\pgfsys@defobject{currentmarker}{\pgfqpoint{0.000000in}{-0.048611in}}{\pgfqpoint{0.000000in}{0.000000in}}{%
\pgfpathmoveto{\pgfqpoint{0.000000in}{0.000000in}}%
\pgfpathlineto{\pgfqpoint{0.000000in}{-0.048611in}}%
\pgfusepath{stroke,fill}%
}%
\begin{pgfscope}%
\pgfsys@transformshift{3.796684in}{0.565123in}%
\pgfsys@useobject{currentmarker}{}%
\end{pgfscope}%
\end{pgfscope}%
\begin{pgfscope}%
\definecolor{textcolor}{rgb}{0.000000,0.000000,0.000000}%
\pgfsetstrokecolor{textcolor}%
\pgfsetfillcolor{textcolor}%
\pgftext[x=3.796684in,y=0.467901in,,top]{\color{textcolor}\rmfamily\fontsize{10.000000}{12.000000}\selectfont \(\displaystyle {150}\)}%
\end{pgfscope}%
\begin{pgfscope}%
\pgfsetbuttcap%
\pgfsetroundjoin%
\definecolor{currentfill}{rgb}{0.000000,0.000000,0.000000}%
\pgfsetfillcolor{currentfill}%
\pgfsetlinewidth{0.803000pt}%
\definecolor{currentstroke}{rgb}{0.000000,0.000000,0.000000}%
\pgfsetstrokecolor{currentstroke}%
\pgfsetdash{}{0pt}%
\pgfsys@defobject{currentmarker}{\pgfqpoint{0.000000in}{-0.048611in}}{\pgfqpoint{0.000000in}{0.000000in}}{%
\pgfpathmoveto{\pgfqpoint{0.000000in}{0.000000in}}%
\pgfpathlineto{\pgfqpoint{0.000000in}{-0.048611in}}%
\pgfusepath{stroke,fill}%
}%
\begin{pgfscope}%
\pgfsys@transformshift{4.795312in}{0.565123in}%
\pgfsys@useobject{currentmarker}{}%
\end{pgfscope}%
\end{pgfscope}%
\begin{pgfscope}%
\definecolor{textcolor}{rgb}{0.000000,0.000000,0.000000}%
\pgfsetstrokecolor{textcolor}%
\pgfsetfillcolor{textcolor}%
\pgftext[x=4.795312in,y=0.467901in,,top]{\color{textcolor}\rmfamily\fontsize{10.000000}{12.000000}\selectfont \(\displaystyle {200}\)}%
\end{pgfscope}%
\begin{pgfscope}%
\definecolor{textcolor}{rgb}{0.000000,0.000000,0.000000}%
\pgfsetstrokecolor{textcolor}%
\pgfsetfillcolor{textcolor}%
\pgftext[x=3.157562in,y=0.288889in,,top]{\color{textcolor}\rmfamily\fontsize{10.000000}{12.000000}\selectfont Tempo (Frame)}%
\end{pgfscope}%
\begin{pgfscope}%
\pgfsetbuttcap%
\pgfsetroundjoin%
\definecolor{currentfill}{rgb}{0.000000,0.000000,0.000000}%
\pgfsetfillcolor{currentfill}%
\pgfsetlinewidth{0.803000pt}%
\definecolor{currentstroke}{rgb}{0.000000,0.000000,0.000000}%
\pgfsetstrokecolor{currentstroke}%
\pgfsetdash{}{0pt}%
\pgfsys@defobject{currentmarker}{\pgfqpoint{-0.048611in}{0.000000in}}{\pgfqpoint{-0.000000in}{0.000000in}}{%
\pgfpathmoveto{\pgfqpoint{-0.000000in}{0.000000in}}%
\pgfpathlineto{\pgfqpoint{-0.048611in}{0.000000in}}%
\pgfusepath{stroke,fill}%
}%
\begin{pgfscope}%
\pgfsys@transformshift{0.565124in}{0.687163in}%
\pgfsys@useobject{currentmarker}{}%
\end{pgfscope}%
\end{pgfscope}%
\begin{pgfscope}%
\definecolor{textcolor}{rgb}{0.000000,0.000000,0.000000}%
\pgfsetstrokecolor{textcolor}%
\pgfsetfillcolor{textcolor}%
\pgftext[x=0.398457in, y=0.638938in, left, base]{\color{textcolor}\rmfamily\fontsize{10.000000}{12.000000}\selectfont \(\displaystyle {0}\)}%
\end{pgfscope}%
\begin{pgfscope}%
\pgfsetbuttcap%
\pgfsetroundjoin%
\definecolor{currentfill}{rgb}{0.000000,0.000000,0.000000}%
\pgfsetfillcolor{currentfill}%
\pgfsetlinewidth{0.803000pt}%
\definecolor{currentstroke}{rgb}{0.000000,0.000000,0.000000}%
\pgfsetstrokecolor{currentstroke}%
\pgfsetdash{}{0pt}%
\pgfsys@defobject{currentmarker}{\pgfqpoint{-0.048611in}{0.000000in}}{\pgfqpoint{-0.000000in}{0.000000in}}{%
\pgfpathmoveto{\pgfqpoint{-0.000000in}{0.000000in}}%
\pgfpathlineto{\pgfqpoint{-0.048611in}{0.000000in}}%
\pgfusepath{stroke,fill}%
}%
\begin{pgfscope}%
\pgfsys@transformshift{0.565124in}{1.221913in}%
\pgfsys@useobject{currentmarker}{}%
\end{pgfscope}%
\end{pgfscope}%
\begin{pgfscope}%
\definecolor{textcolor}{rgb}{0.000000,0.000000,0.000000}%
\pgfsetstrokecolor{textcolor}%
\pgfsetfillcolor{textcolor}%
\pgftext[x=0.329012in, y=1.173687in, left, base]{\color{textcolor}\rmfamily\fontsize{10.000000}{12.000000}\selectfont \(\displaystyle {20}\)}%
\end{pgfscope}%
\begin{pgfscope}%
\pgfsetbuttcap%
\pgfsetroundjoin%
\definecolor{currentfill}{rgb}{0.000000,0.000000,0.000000}%
\pgfsetfillcolor{currentfill}%
\pgfsetlinewidth{0.803000pt}%
\definecolor{currentstroke}{rgb}{0.000000,0.000000,0.000000}%
\pgfsetstrokecolor{currentstroke}%
\pgfsetdash{}{0pt}%
\pgfsys@defobject{currentmarker}{\pgfqpoint{-0.048611in}{0.000000in}}{\pgfqpoint{-0.000000in}{0.000000in}}{%
\pgfpathmoveto{\pgfqpoint{-0.000000in}{0.000000in}}%
\pgfpathlineto{\pgfqpoint{-0.048611in}{0.000000in}}%
\pgfusepath{stroke,fill}%
}%
\begin{pgfscope}%
\pgfsys@transformshift{0.565124in}{1.756662in}%
\pgfsys@useobject{currentmarker}{}%
\end{pgfscope}%
\end{pgfscope}%
\begin{pgfscope}%
\definecolor{textcolor}{rgb}{0.000000,0.000000,0.000000}%
\pgfsetstrokecolor{textcolor}%
\pgfsetfillcolor{textcolor}%
\pgftext[x=0.329012in, y=1.708437in, left, base]{\color{textcolor}\rmfamily\fontsize{10.000000}{12.000000}\selectfont \(\displaystyle {40}\)}%
\end{pgfscope}%
\begin{pgfscope}%
\pgfsetbuttcap%
\pgfsetroundjoin%
\definecolor{currentfill}{rgb}{0.000000,0.000000,0.000000}%
\pgfsetfillcolor{currentfill}%
\pgfsetlinewidth{0.803000pt}%
\definecolor{currentstroke}{rgb}{0.000000,0.000000,0.000000}%
\pgfsetstrokecolor{currentstroke}%
\pgfsetdash{}{0pt}%
\pgfsys@defobject{currentmarker}{\pgfqpoint{-0.048611in}{0.000000in}}{\pgfqpoint{-0.000000in}{0.000000in}}{%
\pgfpathmoveto{\pgfqpoint{-0.000000in}{0.000000in}}%
\pgfpathlineto{\pgfqpoint{-0.048611in}{0.000000in}}%
\pgfusepath{stroke,fill}%
}%
\begin{pgfscope}%
\pgfsys@transformshift{0.565124in}{2.291411in}%
\pgfsys@useobject{currentmarker}{}%
\end{pgfscope}%
\end{pgfscope}%
\begin{pgfscope}%
\definecolor{textcolor}{rgb}{0.000000,0.000000,0.000000}%
\pgfsetstrokecolor{textcolor}%
\pgfsetfillcolor{textcolor}%
\pgftext[x=0.329012in, y=2.243186in, left, base]{\color{textcolor}\rmfamily\fontsize{10.000000}{12.000000}\selectfont \(\displaystyle {60}\)}%
\end{pgfscope}%
\begin{pgfscope}%
\pgfsetbuttcap%
\pgfsetroundjoin%
\definecolor{currentfill}{rgb}{0.000000,0.000000,0.000000}%
\pgfsetfillcolor{currentfill}%
\pgfsetlinewidth{0.803000pt}%
\definecolor{currentstroke}{rgb}{0.000000,0.000000,0.000000}%
\pgfsetstrokecolor{currentstroke}%
\pgfsetdash{}{0pt}%
\pgfsys@defobject{currentmarker}{\pgfqpoint{-0.048611in}{0.000000in}}{\pgfqpoint{-0.000000in}{0.000000in}}{%
\pgfpathmoveto{\pgfqpoint{-0.000000in}{0.000000in}}%
\pgfpathlineto{\pgfqpoint{-0.048611in}{0.000000in}}%
\pgfusepath{stroke,fill}%
}%
\begin{pgfscope}%
\pgfsys@transformshift{0.565124in}{2.826161in}%
\pgfsys@useobject{currentmarker}{}%
\end{pgfscope}%
\end{pgfscope}%
\begin{pgfscope}%
\definecolor{textcolor}{rgb}{0.000000,0.000000,0.000000}%
\pgfsetstrokecolor{textcolor}%
\pgfsetfillcolor{textcolor}%
\pgftext[x=0.329012in, y=2.777936in, left, base]{\color{textcolor}\rmfamily\fontsize{10.000000}{12.000000}\selectfont \(\displaystyle {80}\)}%
\end{pgfscope}%
\begin{pgfscope}%
\definecolor{textcolor}{rgb}{0.000000,0.000000,0.000000}%
\pgfsetstrokecolor{textcolor}%
\pgfsetfillcolor{textcolor}%
\pgftext[x=0.273457in,y=1.907562in,,bottom,rotate=90.000000]{\color{textcolor}\rmfamily\fontsize{10.000000}{12.000000}\selectfont Differenza percettiva}%
\end{pgfscope}%
\begin{pgfscope}%
\pgfpathrectangle{\pgfqpoint{0.565124in}{0.565123in}}{\pgfqpoint{5.184876in}{2.684877in}}%
\pgfusepath{clip}%
\pgfsetrectcap%
\pgfsetroundjoin%
\pgfsetlinewidth{2.007500pt}%
\definecolor{currentstroke}{rgb}{0.121569,0.466667,0.705882}%
\pgfsetstrokecolor{currentstroke}%
\pgfsetdash{}{0pt}%
\pgfpathmoveto{\pgfqpoint{0.800800in}{1.536914in}}%
\pgfpathlineto{\pgfqpoint{0.820773in}{1.498452in}}%
\pgfpathlineto{\pgfqpoint{0.840745in}{1.445509in}}%
\pgfpathlineto{\pgfqpoint{0.860718in}{1.416978in}}%
\pgfpathlineto{\pgfqpoint{0.900663in}{1.362714in}}%
\pgfpathlineto{\pgfqpoint{0.920635in}{1.335696in}}%
\pgfpathlineto{\pgfqpoint{0.940608in}{1.313683in}}%
\pgfpathlineto{\pgfqpoint{0.980553in}{1.286399in}}%
\pgfpathlineto{\pgfqpoint{1.000526in}{1.272161in}}%
\pgfpathlineto{\pgfqpoint{1.020498in}{1.250083in}}%
\pgfpathlineto{\pgfqpoint{1.040471in}{1.155334in}}%
\pgfpathlineto{\pgfqpoint{1.060443in}{1.143916in}}%
\pgfpathlineto{\pgfqpoint{1.080416in}{1.131025in}}%
\pgfpathlineto{\pgfqpoint{1.100388in}{1.123360in}}%
\pgfpathlineto{\pgfqpoint{1.120361in}{1.110800in}}%
\pgfpathlineto{\pgfqpoint{1.140333in}{1.109269in}}%
\pgfpathlineto{\pgfqpoint{1.160306in}{1.106322in}}%
\pgfpathlineto{\pgfqpoint{1.260169in}{1.106011in}}%
\pgfpathlineto{\pgfqpoint{1.280141in}{1.039467in}}%
\pgfpathlineto{\pgfqpoint{1.300114in}{0.987513in}}%
\pgfpathlineto{\pgfqpoint{1.320086in}{0.982331in}}%
\pgfpathlineto{\pgfqpoint{1.340059in}{0.981201in}}%
\pgfpathlineto{\pgfqpoint{1.360032in}{0.966721in}}%
\pgfpathlineto{\pgfqpoint{1.419949in}{0.963916in}}%
\pgfpathlineto{\pgfqpoint{1.439922in}{0.963916in}}%
\pgfpathlineto{\pgfqpoint{1.459894in}{0.955330in}}%
\pgfpathlineto{\pgfqpoint{1.479867in}{0.948330in}}%
\pgfpathlineto{\pgfqpoint{1.499839in}{0.948330in}}%
\pgfpathlineto{\pgfqpoint{1.519812in}{0.938640in}}%
\pgfpathlineto{\pgfqpoint{1.539785in}{0.937747in}}%
\pgfpathlineto{\pgfqpoint{1.559757in}{0.919893in}}%
\pgfpathlineto{\pgfqpoint{1.579730in}{0.919893in}}%
\pgfpathlineto{\pgfqpoint{1.599702in}{0.918083in}}%
\pgfpathlineto{\pgfqpoint{1.619675in}{0.902027in}}%
\pgfpathlineto{\pgfqpoint{1.659620in}{0.899402in}}%
\pgfpathlineto{\pgfqpoint{1.719538in}{0.895249in}}%
\pgfpathlineto{\pgfqpoint{1.759483in}{0.894098in}}%
\pgfpathlineto{\pgfqpoint{1.779455in}{0.867288in}}%
\pgfpathlineto{\pgfqpoint{1.799428in}{0.863949in}}%
\pgfpathlineto{\pgfqpoint{1.819400in}{0.855748in}}%
\pgfpathlineto{\pgfqpoint{1.839373in}{0.850837in}}%
\pgfpathlineto{\pgfqpoint{1.859346in}{0.850836in}}%
\pgfpathlineto{\pgfqpoint{1.879318in}{0.818754in}}%
\pgfpathlineto{\pgfqpoint{1.899291in}{0.818394in}}%
\pgfpathlineto{\pgfqpoint{1.919263in}{0.783566in}}%
\pgfpathlineto{\pgfqpoint{1.939236in}{0.783328in}}%
\pgfpathlineto{\pgfqpoint{1.959208in}{0.779239in}}%
\pgfpathlineto{\pgfqpoint{1.979181in}{0.779233in}}%
\pgfpathlineto{\pgfqpoint{1.999153in}{0.777234in}}%
\pgfpathlineto{\pgfqpoint{2.039099in}{0.770618in}}%
\pgfpathlineto{\pgfqpoint{2.059071in}{0.769351in}}%
\pgfpathlineto{\pgfqpoint{2.079044in}{0.761254in}}%
\pgfpathlineto{\pgfqpoint{2.099016in}{0.758948in}}%
\pgfpathlineto{\pgfqpoint{2.178906in}{0.755026in}}%
\pgfpathlineto{\pgfqpoint{2.218852in}{0.754850in}}%
\pgfpathlineto{\pgfqpoint{2.238824in}{0.752925in}}%
\pgfpathlineto{\pgfqpoint{2.258797in}{0.752683in}}%
\pgfpathlineto{\pgfqpoint{2.298742in}{0.743089in}}%
\pgfpathlineto{\pgfqpoint{2.318714in}{0.739841in}}%
\pgfpathlineto{\pgfqpoint{2.338687in}{0.714668in}}%
\pgfpathlineto{\pgfqpoint{2.398605in}{0.713730in}}%
\pgfpathlineto{\pgfqpoint{2.418577in}{0.711910in}}%
\pgfpathlineto{\pgfqpoint{2.458522in}{0.711256in}}%
\pgfpathlineto{\pgfqpoint{2.478495in}{0.709677in}}%
\pgfpathlineto{\pgfqpoint{2.538413in}{0.709209in}}%
\pgfpathlineto{\pgfqpoint{2.658248in}{0.708894in}}%
\pgfpathlineto{\pgfqpoint{2.718166in}{0.708692in}}%
\pgfpathlineto{\pgfqpoint{2.738138in}{0.698363in}}%
\pgfpathlineto{\pgfqpoint{2.798056in}{0.697489in}}%
\pgfpathlineto{\pgfqpoint{2.857973in}{0.697082in}}%
\pgfpathlineto{\pgfqpoint{3.017754in}{0.696709in}}%
\pgfpathlineto{\pgfqpoint{3.037727in}{0.689823in}}%
\pgfpathlineto{\pgfqpoint{3.077672in}{0.689771in}}%
\pgfpathlineto{\pgfqpoint{3.097644in}{0.688638in}}%
\pgfpathlineto{\pgfqpoint{3.177534in}{0.687884in}}%
\pgfpathlineto{\pgfqpoint{3.217480in}{0.687165in}}%
\pgfpathlineto{\pgfqpoint{5.494351in}{0.687163in}}%
\pgfpathlineto{\pgfqpoint{5.494351in}{0.687163in}}%
\pgfusepath{stroke}%
\end{pgfscope}%
\begin{pgfscope}%
\pgfpathrectangle{\pgfqpoint{0.565124in}{0.565123in}}{\pgfqpoint{5.184876in}{2.684877in}}%
\pgfusepath{clip}%
\pgfsetrectcap%
\pgfsetroundjoin%
\pgfsetlinewidth{2.007500pt}%
\definecolor{currentstroke}{rgb}{1.000000,0.498039,0.054902}%
\pgfsetstrokecolor{currentstroke}%
\pgfsetdash{}{0pt}%
\pgfpathmoveto{\pgfqpoint{0.820773in}{3.127957in}}%
\pgfpathlineto{\pgfqpoint{0.880690in}{3.127960in}}%
\pgfpathlineto{\pgfqpoint{0.900663in}{3.059568in}}%
\pgfpathlineto{\pgfqpoint{1.160306in}{3.059572in}}%
\pgfpathlineto{\pgfqpoint{1.180279in}{3.036434in}}%
\pgfpathlineto{\pgfqpoint{1.360032in}{3.036438in}}%
\pgfpathlineto{\pgfqpoint{1.380004in}{3.022542in}}%
\pgfpathlineto{\pgfqpoint{1.459894in}{3.022543in}}%
\pgfpathlineto{\pgfqpoint{1.479867in}{3.018213in}}%
\pgfpathlineto{\pgfqpoint{1.499839in}{3.009949in}}%
\pgfpathlineto{\pgfqpoint{1.519812in}{3.009771in}}%
\pgfpathlineto{\pgfqpoint{1.559757in}{2.968472in}}%
\pgfpathlineto{\pgfqpoint{1.639647in}{2.968468in}}%
\pgfpathlineto{\pgfqpoint{1.659620in}{2.956617in}}%
\pgfpathlineto{\pgfqpoint{1.919263in}{2.956599in}}%
\pgfpathlineto{\pgfqpoint{1.939236in}{2.952700in}}%
\pgfpathlineto{\pgfqpoint{2.019126in}{2.952685in}}%
\pgfpathlineto{\pgfqpoint{2.039099in}{2.652357in}}%
\pgfpathlineto{\pgfqpoint{2.138961in}{2.652347in}}%
\pgfpathlineto{\pgfqpoint{2.158934in}{2.596026in}}%
\pgfpathlineto{\pgfqpoint{2.258797in}{2.595356in}}%
\pgfpathlineto{\pgfqpoint{2.278769in}{2.593840in}}%
\pgfpathlineto{\pgfqpoint{2.298742in}{2.098954in}}%
\pgfpathlineto{\pgfqpoint{2.358660in}{2.098327in}}%
\pgfpathlineto{\pgfqpoint{2.378632in}{2.085750in}}%
\pgfpathlineto{\pgfqpoint{2.398605in}{2.081972in}}%
\pgfpathlineto{\pgfqpoint{2.418577in}{2.080180in}}%
\pgfpathlineto{\pgfqpoint{2.438550in}{2.052169in}}%
\pgfpathlineto{\pgfqpoint{2.458522in}{1.901880in}}%
\pgfpathlineto{\pgfqpoint{2.478495in}{1.885647in}}%
\pgfpathlineto{\pgfqpoint{2.498467in}{1.885646in}}%
\pgfpathlineto{\pgfqpoint{2.518440in}{1.884284in}}%
\pgfpathlineto{\pgfqpoint{2.538413in}{1.881311in}}%
\pgfpathlineto{\pgfqpoint{2.558385in}{1.876639in}}%
\pgfpathlineto{\pgfqpoint{2.578358in}{1.875546in}}%
\pgfpathlineto{\pgfqpoint{2.598330in}{1.869515in}}%
\pgfpathlineto{\pgfqpoint{2.618303in}{1.866576in}}%
\pgfpathlineto{\pgfqpoint{2.638275in}{1.356323in}}%
\pgfpathlineto{\pgfqpoint{2.658248in}{1.340063in}}%
\pgfpathlineto{\pgfqpoint{2.678220in}{1.327478in}}%
\pgfpathlineto{\pgfqpoint{2.718166in}{1.327480in}}%
\pgfpathlineto{\pgfqpoint{2.738138in}{1.020725in}}%
\pgfpathlineto{\pgfqpoint{2.778083in}{1.019958in}}%
\pgfpathlineto{\pgfqpoint{2.857973in}{1.019952in}}%
\pgfpathlineto{\pgfqpoint{2.877946in}{0.921185in}}%
\pgfpathlineto{\pgfqpoint{2.937864in}{0.921188in}}%
\pgfpathlineto{\pgfqpoint{2.957836in}{0.917184in}}%
\pgfpathlineto{\pgfqpoint{3.257425in}{0.917181in}}%
\pgfpathlineto{\pgfqpoint{3.277397in}{0.915182in}}%
\pgfpathlineto{\pgfqpoint{3.477123in}{0.915184in}}%
\pgfpathlineto{\pgfqpoint{3.497095in}{0.913304in}}%
\pgfpathlineto{\pgfqpoint{3.517068in}{0.913306in}}%
\pgfpathlineto{\pgfqpoint{3.537040in}{0.904606in}}%
\pgfpathlineto{\pgfqpoint{3.576986in}{0.903552in}}%
\pgfpathlineto{\pgfqpoint{3.596958in}{0.839624in}}%
\pgfpathlineto{\pgfqpoint{3.616931in}{0.839623in}}%
\pgfpathlineto{\pgfqpoint{3.636903in}{0.837014in}}%
\pgfpathlineto{\pgfqpoint{3.656876in}{0.810547in}}%
\pgfpathlineto{\pgfqpoint{3.676848in}{0.802568in}}%
\pgfpathlineto{\pgfqpoint{3.696821in}{0.765721in}}%
\pgfpathlineto{\pgfqpoint{3.796684in}{0.765572in}}%
\pgfpathlineto{\pgfqpoint{3.816656in}{0.754227in}}%
\pgfpathlineto{\pgfqpoint{3.836629in}{0.753055in}}%
\pgfpathlineto{\pgfqpoint{3.856601in}{0.746090in}}%
\pgfpathlineto{\pgfqpoint{3.876574in}{0.726753in}}%
\pgfpathlineto{\pgfqpoint{3.896547in}{0.726218in}}%
\pgfpathlineto{\pgfqpoint{3.916519in}{0.719408in}}%
\pgfpathlineto{\pgfqpoint{3.936492in}{0.715859in}}%
\pgfpathlineto{\pgfqpoint{3.956464in}{0.714093in}}%
\pgfpathlineto{\pgfqpoint{3.976437in}{0.711172in}}%
\pgfpathlineto{\pgfqpoint{4.375888in}{0.711163in}}%
\pgfpathlineto{\pgfqpoint{4.395861in}{0.687170in}}%
\pgfpathlineto{\pgfqpoint{5.514324in}{0.687163in}}%
\pgfpathlineto{\pgfqpoint{5.514324in}{0.687163in}}%
\pgfusepath{stroke}%
\end{pgfscope}%
\begin{pgfscope}%
\pgfsetrectcap%
\pgfsetmiterjoin%
\pgfsetlinewidth{0.803000pt}%
\definecolor{currentstroke}{rgb}{0.000000,0.000000,0.000000}%
\pgfsetstrokecolor{currentstroke}%
\pgfsetdash{}{0pt}%
\pgfpathmoveto{\pgfqpoint{0.565124in}{0.565123in}}%
\pgfpathlineto{\pgfqpoint{0.565124in}{3.250000in}}%
\pgfusepath{stroke}%
\end{pgfscope}%
\begin{pgfscope}%
\pgfsetrectcap%
\pgfsetmiterjoin%
\pgfsetlinewidth{0.803000pt}%
\definecolor{currentstroke}{rgb}{0.000000,0.000000,0.000000}%
\pgfsetstrokecolor{currentstroke}%
\pgfsetdash{}{0pt}%
\pgfpathmoveto{\pgfqpoint{5.750000in}{0.565123in}}%
\pgfpathlineto{\pgfqpoint{5.750000in}{3.250000in}}%
\pgfusepath{stroke}%
\end{pgfscope}%
\begin{pgfscope}%
\pgfsetrectcap%
\pgfsetmiterjoin%
\pgfsetlinewidth{0.803000pt}%
\definecolor{currentstroke}{rgb}{0.000000,0.000000,0.000000}%
\pgfsetstrokecolor{currentstroke}%
\pgfsetdash{}{0pt}%
\pgfpathmoveto{\pgfqpoint{0.565124in}{0.565123in}}%
\pgfpathlineto{\pgfqpoint{5.750000in}{0.565123in}}%
\pgfusepath{stroke}%
\end{pgfscope}%
\begin{pgfscope}%
\pgfsetrectcap%
\pgfsetmiterjoin%
\pgfsetlinewidth{0.803000pt}%
\definecolor{currentstroke}{rgb}{0.000000,0.000000,0.000000}%
\pgfsetstrokecolor{currentstroke}%
\pgfsetdash{}{0pt}%
\pgfpathmoveto{\pgfqpoint{0.565124in}{3.250000in}}%
\pgfpathlineto{\pgfqpoint{5.750000in}{3.250000in}}%
\pgfusepath{stroke}%
\end{pgfscope}%
\begin{pgfscope}%
\pgfsetbuttcap%
\pgfsetmiterjoin%
\definecolor{currentfill}{rgb}{1.000000,1.000000,1.000000}%
\pgfsetfillcolor{currentfill}%
\pgfsetfillopacity{0.800000}%
\pgfsetlinewidth{1.003750pt}%
\definecolor{currentstroke}{rgb}{0.800000,0.800000,0.800000}%
\pgfsetstrokecolor{currentstroke}%
\pgfsetstrokeopacity{0.800000}%
\pgfsetdash{}{0pt}%
\pgfpathmoveto{\pgfqpoint{4.273532in}{2.751543in}}%
\pgfpathlineto{\pgfqpoint{5.652778in}{2.751543in}}%
\pgfpathquadraticcurveto{\pgfqpoint{5.680556in}{2.751543in}}{\pgfqpoint{5.680556in}{2.779321in}}%
\pgfpathlineto{\pgfqpoint{5.680556in}{3.152778in}}%
\pgfpathquadraticcurveto{\pgfqpoint{5.680556in}{3.180556in}}{\pgfqpoint{5.652778in}{3.180556in}}%
\pgfpathlineto{\pgfqpoint{4.273532in}{3.180556in}}%
\pgfpathquadraticcurveto{\pgfqpoint{4.245755in}{3.180556in}}{\pgfqpoint{4.245755in}{3.152778in}}%
\pgfpathlineto{\pgfqpoint{4.245755in}{2.779321in}}%
\pgfpathquadraticcurveto{\pgfqpoint{4.245755in}{2.751543in}}{\pgfqpoint{4.273532in}{2.751543in}}%
\pgfpathclose%
\pgfusepath{stroke,fill}%
\end{pgfscope}%
\begin{pgfscope}%
\pgfsetrectcap%
\pgfsetroundjoin%
\pgfsetlinewidth{2.007500pt}%
\definecolor{currentstroke}{rgb}{0.121569,0.466667,0.705882}%
\pgfsetstrokecolor{currentstroke}%
\pgfsetdash{}{0pt}%
\pgfpathmoveto{\pgfqpoint{4.301310in}{3.076389in}}%
\pgfpathlineto{\pgfqpoint{4.579088in}{3.076389in}}%
\pgfusepath{stroke}%
\end{pgfscope}%
\begin{pgfscope}%
\definecolor{textcolor}{rgb}{0.000000,0.000000,0.000000}%
\pgfsetstrokecolor{textcolor}%
\pgfsetfillcolor{textcolor}%
\pgftext[x=4.690199in,y=3.027778in,left,base]{\color{textcolor}\rmfamily\fontsize{10.000000}{12.000000}\selectfont Scena fittizia}%
\end{pgfscope}%
\begin{pgfscope}%
\pgfsetrectcap%
\pgfsetroundjoin%
\pgfsetlinewidth{2.007500pt}%
\definecolor{currentstroke}{rgb}{1.000000,0.498039,0.054902}%
\pgfsetstrokecolor{currentstroke}%
\pgfsetdash{}{0pt}%
\pgfpathmoveto{\pgfqpoint{4.301310in}{2.882716in}}%
\pgfpathlineto{\pgfqpoint{4.579088in}{2.882716in}}%
\pgfusepath{stroke}%
\end{pgfscope}%
\begin{pgfscope}%
\definecolor{textcolor}{rgb}{0.000000,0.000000,0.000000}%
\pgfsetstrokecolor{textcolor}%
\pgfsetfillcolor{textcolor}%
\pgftext[x=4.690199in,y=2.834105in,left,base]{\color{textcolor}\rmfamily\fontsize{10.000000}{12.000000}\selectfont Scena realistica}%
\end{pgfscope}%
\end{pgfpicture}%
\makeatother%
\endgroup%

                \caption{Valutazione Sphere Tracing valutato per la distanza}
                \label{fig:eval-stdist}
            \end{figure}

            \newpage
            \subsubsection{Valutazione della dimensione}
            Valutare le istanze in vista per dimensione porta un notevole guadagno nella scena realistica (Figura \ref{fig:eval-stdim}): del 28.3\% rispetto alla valutazione per distanza (e alla valutazione Closest-First on Camera) e del 58.38\% da Closest-First. Questo miglioramento era prevedibile dato che la scena realistica è composta da elementi di grandi dimensioni che occupano un'importante sezione della vista anche se posti distanti dalla camera. 
            
            È presente un guadagno anche nella scena fittizia ma, dato che le istanze di asset che la popolano presentano una maggiore uniformità in dimensione, fatte poche eccezioni, il guadagno è minore, del 5.7\% rispetto alla valutazione per distanza, del 11.5\% dalla strategia Closest-First on camera e del 63.12\% da Closest-First.
            %discussione riguardo gli angoli della luce
                % \begin{figure}[htb!]
                %     \centering
                %     \includegraphics[scale=.5]{images/valutazioni/fittizia/28-04-23 04-20-50 SphereTracingSizePriority 809.5422230853329.png}
                %     \caption{Valutazione Sphere Tracing valutato per la dimensione sulla scena fittizia}
                %     \label{fig:eval-stdim-fit}
                % \end{figure}
    
                % \begin{figure}[htb!]
                %     \centering
                %     \includegraphics[scale=.5]{images/valutazioni/realistica/28-04-23 05-01-17 SphereTracingSizePriority 5462.929151987784.png}
                %     \caption{Valutazione Sphere Tracing valutato per la dimensione sulla scena realistica}
                %     \label{fig:eval-stdim-re}
                % \end{figure}
            % \begin{figure}[htbp]
            %     \centering
            %     \includegraphics[width=\textwidth]{images/sequences/sequence-stdim-re.png}
            %     \par
            %     \vspace{15pt}
            %     \centering
            %     \includegraphics[width=\textwidth]{images/sequences/sequence-stdim-fit.png}
            %     \caption{Sequenza di carimento di Sphere Tracing valutato per la dimensione}
            %     \label{fig:seq-stdim}
            % \end{figure}
                
            \begin{figure}[htb!]
                \centering
                %% Creator: Matplotlib, PGF backend
%%
%% To include the figure in your LaTeX document, write
%%   \input{<filename>.pgf}
%%
%% Make sure the required packages are loaded in your preamble
%%   \usepackage{pgf}
%%
%% Figures using additional raster images can only be included by \input if
%% they are in the same directory as the main LaTeX file. For loading figures
%% from other directories you can use the `import` package
%%   \usepackage{import}
%%
%% and then include the figures with
%%   \import{<path to file>}{<filename>.pgf}
%%
%% Matplotlib used the following preamble
%%
\begingroup%
\makeatletter%
\begin{pgfpicture}%
\pgfpathrectangle{\pgfpointorigin}{\pgfqpoint{5.900000in}{3.400000in}}%
\pgfusepath{use as bounding box, clip}%
\begin{pgfscope}%
\pgfsetbuttcap%
\pgfsetmiterjoin%
\definecolor{currentfill}{rgb}{1.000000,1.000000,1.000000}%
\pgfsetfillcolor{currentfill}%
\pgfsetlinewidth{0.000000pt}%
\definecolor{currentstroke}{rgb}{1.000000,1.000000,1.000000}%
\pgfsetstrokecolor{currentstroke}%
\pgfsetdash{}{0pt}%
\pgfpathmoveto{\pgfqpoint{0.000000in}{0.000000in}}%
\pgfpathlineto{\pgfqpoint{5.900000in}{0.000000in}}%
\pgfpathlineto{\pgfqpoint{5.900000in}{3.400000in}}%
\pgfpathlineto{\pgfqpoint{0.000000in}{3.400000in}}%
\pgfpathclose%
\pgfusepath{fill}%
\end{pgfscope}%
\begin{pgfscope}%
\pgfsetbuttcap%
\pgfsetmiterjoin%
\definecolor{currentfill}{rgb}{1.000000,1.000000,1.000000}%
\pgfsetfillcolor{currentfill}%
\pgfsetlinewidth{0.000000pt}%
\definecolor{currentstroke}{rgb}{0.000000,0.000000,0.000000}%
\pgfsetstrokecolor{currentstroke}%
\pgfsetstrokeopacity{0.000000}%
\pgfsetdash{}{0pt}%
\pgfpathmoveto{\pgfqpoint{0.565124in}{0.565123in}}%
\pgfpathlineto{\pgfqpoint{5.750000in}{0.565123in}}%
\pgfpathlineto{\pgfqpoint{5.750000in}{3.250000in}}%
\pgfpathlineto{\pgfqpoint{0.565124in}{3.250000in}}%
\pgfpathclose%
\pgfusepath{fill}%
\end{pgfscope}%
\begin{pgfscope}%
\pgfsetbuttcap%
\pgfsetroundjoin%
\definecolor{currentfill}{rgb}{0.000000,0.000000,0.000000}%
\pgfsetfillcolor{currentfill}%
\pgfsetlinewidth{0.803000pt}%
\definecolor{currentstroke}{rgb}{0.000000,0.000000,0.000000}%
\pgfsetstrokecolor{currentstroke}%
\pgfsetdash{}{0pt}%
\pgfsys@defobject{currentmarker}{\pgfqpoint{0.000000in}{-0.048611in}}{\pgfqpoint{0.000000in}{0.000000in}}{%
\pgfpathmoveto{\pgfqpoint{0.000000in}{0.000000in}}%
\pgfpathlineto{\pgfqpoint{0.000000in}{-0.048611in}}%
\pgfusepath{stroke,fill}%
}%
\begin{pgfscope}%
\pgfsys@transformshift{0.800800in}{0.565123in}%
\pgfsys@useobject{currentmarker}{}%
\end{pgfscope}%
\end{pgfscope}%
\begin{pgfscope}%
\definecolor{textcolor}{rgb}{0.000000,0.000000,0.000000}%
\pgfsetstrokecolor{textcolor}%
\pgfsetfillcolor{textcolor}%
\pgftext[x=0.800800in,y=0.467901in,,top]{\color{textcolor}\rmfamily\fontsize{10.000000}{12.000000}\selectfont \(\displaystyle {0}\)}%
\end{pgfscope}%
\begin{pgfscope}%
\pgfsetbuttcap%
\pgfsetroundjoin%
\definecolor{currentfill}{rgb}{0.000000,0.000000,0.000000}%
\pgfsetfillcolor{currentfill}%
\pgfsetlinewidth{0.803000pt}%
\definecolor{currentstroke}{rgb}{0.000000,0.000000,0.000000}%
\pgfsetstrokecolor{currentstroke}%
\pgfsetdash{}{0pt}%
\pgfsys@defobject{currentmarker}{\pgfqpoint{0.000000in}{-0.048611in}}{\pgfqpoint{0.000000in}{0.000000in}}{%
\pgfpathmoveto{\pgfqpoint{0.000000in}{0.000000in}}%
\pgfpathlineto{\pgfqpoint{0.000000in}{-0.048611in}}%
\pgfusepath{stroke,fill}%
}%
\begin{pgfscope}%
\pgfsys@transformshift{1.799428in}{0.565123in}%
\pgfsys@useobject{currentmarker}{}%
\end{pgfscope}%
\end{pgfscope}%
\begin{pgfscope}%
\definecolor{textcolor}{rgb}{0.000000,0.000000,0.000000}%
\pgfsetstrokecolor{textcolor}%
\pgfsetfillcolor{textcolor}%
\pgftext[x=1.799428in,y=0.467901in,,top]{\color{textcolor}\rmfamily\fontsize{10.000000}{12.000000}\selectfont \(\displaystyle {50}\)}%
\end{pgfscope}%
\begin{pgfscope}%
\pgfsetbuttcap%
\pgfsetroundjoin%
\definecolor{currentfill}{rgb}{0.000000,0.000000,0.000000}%
\pgfsetfillcolor{currentfill}%
\pgfsetlinewidth{0.803000pt}%
\definecolor{currentstroke}{rgb}{0.000000,0.000000,0.000000}%
\pgfsetstrokecolor{currentstroke}%
\pgfsetdash{}{0pt}%
\pgfsys@defobject{currentmarker}{\pgfqpoint{0.000000in}{-0.048611in}}{\pgfqpoint{0.000000in}{0.000000in}}{%
\pgfpathmoveto{\pgfqpoint{0.000000in}{0.000000in}}%
\pgfpathlineto{\pgfqpoint{0.000000in}{-0.048611in}}%
\pgfusepath{stroke,fill}%
}%
\begin{pgfscope}%
\pgfsys@transformshift{2.798056in}{0.565123in}%
\pgfsys@useobject{currentmarker}{}%
\end{pgfscope}%
\end{pgfscope}%
\begin{pgfscope}%
\definecolor{textcolor}{rgb}{0.000000,0.000000,0.000000}%
\pgfsetstrokecolor{textcolor}%
\pgfsetfillcolor{textcolor}%
\pgftext[x=2.798056in,y=0.467901in,,top]{\color{textcolor}\rmfamily\fontsize{10.000000}{12.000000}\selectfont \(\displaystyle {100}\)}%
\end{pgfscope}%
\begin{pgfscope}%
\pgfsetbuttcap%
\pgfsetroundjoin%
\definecolor{currentfill}{rgb}{0.000000,0.000000,0.000000}%
\pgfsetfillcolor{currentfill}%
\pgfsetlinewidth{0.803000pt}%
\definecolor{currentstroke}{rgb}{0.000000,0.000000,0.000000}%
\pgfsetstrokecolor{currentstroke}%
\pgfsetdash{}{0pt}%
\pgfsys@defobject{currentmarker}{\pgfqpoint{0.000000in}{-0.048611in}}{\pgfqpoint{0.000000in}{0.000000in}}{%
\pgfpathmoveto{\pgfqpoint{0.000000in}{0.000000in}}%
\pgfpathlineto{\pgfqpoint{0.000000in}{-0.048611in}}%
\pgfusepath{stroke,fill}%
}%
\begin{pgfscope}%
\pgfsys@transformshift{3.796684in}{0.565123in}%
\pgfsys@useobject{currentmarker}{}%
\end{pgfscope}%
\end{pgfscope}%
\begin{pgfscope}%
\definecolor{textcolor}{rgb}{0.000000,0.000000,0.000000}%
\pgfsetstrokecolor{textcolor}%
\pgfsetfillcolor{textcolor}%
\pgftext[x=3.796684in,y=0.467901in,,top]{\color{textcolor}\rmfamily\fontsize{10.000000}{12.000000}\selectfont \(\displaystyle {150}\)}%
\end{pgfscope}%
\begin{pgfscope}%
\pgfsetbuttcap%
\pgfsetroundjoin%
\definecolor{currentfill}{rgb}{0.000000,0.000000,0.000000}%
\pgfsetfillcolor{currentfill}%
\pgfsetlinewidth{0.803000pt}%
\definecolor{currentstroke}{rgb}{0.000000,0.000000,0.000000}%
\pgfsetstrokecolor{currentstroke}%
\pgfsetdash{}{0pt}%
\pgfsys@defobject{currentmarker}{\pgfqpoint{0.000000in}{-0.048611in}}{\pgfqpoint{0.000000in}{0.000000in}}{%
\pgfpathmoveto{\pgfqpoint{0.000000in}{0.000000in}}%
\pgfpathlineto{\pgfqpoint{0.000000in}{-0.048611in}}%
\pgfusepath{stroke,fill}%
}%
\begin{pgfscope}%
\pgfsys@transformshift{4.795312in}{0.565123in}%
\pgfsys@useobject{currentmarker}{}%
\end{pgfscope}%
\end{pgfscope}%
\begin{pgfscope}%
\definecolor{textcolor}{rgb}{0.000000,0.000000,0.000000}%
\pgfsetstrokecolor{textcolor}%
\pgfsetfillcolor{textcolor}%
\pgftext[x=4.795312in,y=0.467901in,,top]{\color{textcolor}\rmfamily\fontsize{10.000000}{12.000000}\selectfont \(\displaystyle {200}\)}%
\end{pgfscope}%
\begin{pgfscope}%
\definecolor{textcolor}{rgb}{0.000000,0.000000,0.000000}%
\pgfsetstrokecolor{textcolor}%
\pgfsetfillcolor{textcolor}%
\pgftext[x=3.157562in,y=0.288889in,,top]{\color{textcolor}\rmfamily\fontsize{10.000000}{12.000000}\selectfont Tempo (Frame)}%
\end{pgfscope}%
\begin{pgfscope}%
\pgfsetbuttcap%
\pgfsetroundjoin%
\definecolor{currentfill}{rgb}{0.000000,0.000000,0.000000}%
\pgfsetfillcolor{currentfill}%
\pgfsetlinewidth{0.803000pt}%
\definecolor{currentstroke}{rgb}{0.000000,0.000000,0.000000}%
\pgfsetstrokecolor{currentstroke}%
\pgfsetdash{}{0pt}%
\pgfsys@defobject{currentmarker}{\pgfqpoint{-0.048611in}{0.000000in}}{\pgfqpoint{-0.000000in}{0.000000in}}{%
\pgfpathmoveto{\pgfqpoint{-0.000000in}{0.000000in}}%
\pgfpathlineto{\pgfqpoint{-0.048611in}{0.000000in}}%
\pgfusepath{stroke,fill}%
}%
\begin{pgfscope}%
\pgfsys@transformshift{0.565124in}{0.687163in}%
\pgfsys@useobject{currentmarker}{}%
\end{pgfscope}%
\end{pgfscope}%
\begin{pgfscope}%
\definecolor{textcolor}{rgb}{0.000000,0.000000,0.000000}%
\pgfsetstrokecolor{textcolor}%
\pgfsetfillcolor{textcolor}%
\pgftext[x=0.398457in, y=0.638938in, left, base]{\color{textcolor}\rmfamily\fontsize{10.000000}{12.000000}\selectfont \(\displaystyle {0}\)}%
\end{pgfscope}%
\begin{pgfscope}%
\pgfsetbuttcap%
\pgfsetroundjoin%
\definecolor{currentfill}{rgb}{0.000000,0.000000,0.000000}%
\pgfsetfillcolor{currentfill}%
\pgfsetlinewidth{0.803000pt}%
\definecolor{currentstroke}{rgb}{0.000000,0.000000,0.000000}%
\pgfsetstrokecolor{currentstroke}%
\pgfsetdash{}{0pt}%
\pgfsys@defobject{currentmarker}{\pgfqpoint{-0.048611in}{0.000000in}}{\pgfqpoint{-0.000000in}{0.000000in}}{%
\pgfpathmoveto{\pgfqpoint{-0.000000in}{0.000000in}}%
\pgfpathlineto{\pgfqpoint{-0.048611in}{0.000000in}}%
\pgfusepath{stroke,fill}%
}%
\begin{pgfscope}%
\pgfsys@transformshift{0.565124in}{1.239248in}%
\pgfsys@useobject{currentmarker}{}%
\end{pgfscope}%
\end{pgfscope}%
\begin{pgfscope}%
\definecolor{textcolor}{rgb}{0.000000,0.000000,0.000000}%
\pgfsetstrokecolor{textcolor}%
\pgfsetfillcolor{textcolor}%
\pgftext[x=0.329012in, y=1.191023in, left, base]{\color{textcolor}\rmfamily\fontsize{10.000000}{12.000000}\selectfont \(\displaystyle {20}\)}%
\end{pgfscope}%
\begin{pgfscope}%
\pgfsetbuttcap%
\pgfsetroundjoin%
\definecolor{currentfill}{rgb}{0.000000,0.000000,0.000000}%
\pgfsetfillcolor{currentfill}%
\pgfsetlinewidth{0.803000pt}%
\definecolor{currentstroke}{rgb}{0.000000,0.000000,0.000000}%
\pgfsetstrokecolor{currentstroke}%
\pgfsetdash{}{0pt}%
\pgfsys@defobject{currentmarker}{\pgfqpoint{-0.048611in}{0.000000in}}{\pgfqpoint{-0.000000in}{0.000000in}}{%
\pgfpathmoveto{\pgfqpoint{-0.000000in}{0.000000in}}%
\pgfpathlineto{\pgfqpoint{-0.048611in}{0.000000in}}%
\pgfusepath{stroke,fill}%
}%
\begin{pgfscope}%
\pgfsys@transformshift{0.565124in}{1.791334in}%
\pgfsys@useobject{currentmarker}{}%
\end{pgfscope}%
\end{pgfscope}%
\begin{pgfscope}%
\definecolor{textcolor}{rgb}{0.000000,0.000000,0.000000}%
\pgfsetstrokecolor{textcolor}%
\pgfsetfillcolor{textcolor}%
\pgftext[x=0.329012in, y=1.743108in, left, base]{\color{textcolor}\rmfamily\fontsize{10.000000}{12.000000}\selectfont \(\displaystyle {40}\)}%
\end{pgfscope}%
\begin{pgfscope}%
\pgfsetbuttcap%
\pgfsetroundjoin%
\definecolor{currentfill}{rgb}{0.000000,0.000000,0.000000}%
\pgfsetfillcolor{currentfill}%
\pgfsetlinewidth{0.803000pt}%
\definecolor{currentstroke}{rgb}{0.000000,0.000000,0.000000}%
\pgfsetstrokecolor{currentstroke}%
\pgfsetdash{}{0pt}%
\pgfsys@defobject{currentmarker}{\pgfqpoint{-0.048611in}{0.000000in}}{\pgfqpoint{-0.000000in}{0.000000in}}{%
\pgfpathmoveto{\pgfqpoint{-0.000000in}{0.000000in}}%
\pgfpathlineto{\pgfqpoint{-0.048611in}{0.000000in}}%
\pgfusepath{stroke,fill}%
}%
\begin{pgfscope}%
\pgfsys@transformshift{0.565124in}{2.343419in}%
\pgfsys@useobject{currentmarker}{}%
\end{pgfscope}%
\end{pgfscope}%
\begin{pgfscope}%
\definecolor{textcolor}{rgb}{0.000000,0.000000,0.000000}%
\pgfsetstrokecolor{textcolor}%
\pgfsetfillcolor{textcolor}%
\pgftext[x=0.329012in, y=2.295194in, left, base]{\color{textcolor}\rmfamily\fontsize{10.000000}{12.000000}\selectfont \(\displaystyle {60}\)}%
\end{pgfscope}%
\begin{pgfscope}%
\pgfsetbuttcap%
\pgfsetroundjoin%
\definecolor{currentfill}{rgb}{0.000000,0.000000,0.000000}%
\pgfsetfillcolor{currentfill}%
\pgfsetlinewidth{0.803000pt}%
\definecolor{currentstroke}{rgb}{0.000000,0.000000,0.000000}%
\pgfsetstrokecolor{currentstroke}%
\pgfsetdash{}{0pt}%
\pgfsys@defobject{currentmarker}{\pgfqpoint{-0.048611in}{0.000000in}}{\pgfqpoint{-0.000000in}{0.000000in}}{%
\pgfpathmoveto{\pgfqpoint{-0.000000in}{0.000000in}}%
\pgfpathlineto{\pgfqpoint{-0.048611in}{0.000000in}}%
\pgfusepath{stroke,fill}%
}%
\begin{pgfscope}%
\pgfsys@transformshift{0.565124in}{2.895504in}%
\pgfsys@useobject{currentmarker}{}%
\end{pgfscope}%
\end{pgfscope}%
\begin{pgfscope}%
\definecolor{textcolor}{rgb}{0.000000,0.000000,0.000000}%
\pgfsetstrokecolor{textcolor}%
\pgfsetfillcolor{textcolor}%
\pgftext[x=0.329012in, y=2.847279in, left, base]{\color{textcolor}\rmfamily\fontsize{10.000000}{12.000000}\selectfont \(\displaystyle {80}\)}%
\end{pgfscope}%
\begin{pgfscope}%
\definecolor{textcolor}{rgb}{0.000000,0.000000,0.000000}%
\pgfsetstrokecolor{textcolor}%
\pgfsetfillcolor{textcolor}%
\pgftext[x=0.273457in,y=1.907562in,,bottom,rotate=90.000000]{\color{textcolor}\rmfamily\fontsize{10.000000}{12.000000}\selectfont Differenza percettiva}%
\end{pgfscope}%
\begin{pgfscope}%
\pgfpathrectangle{\pgfqpoint{0.565124in}{0.565123in}}{\pgfqpoint{5.184876in}{2.684877in}}%
\pgfusepath{clip}%
\pgfsetrectcap%
\pgfsetroundjoin%
\pgfsetlinewidth{2.007500pt}%
\definecolor{currentstroke}{rgb}{0.121569,0.466667,0.705882}%
\pgfsetstrokecolor{currentstroke}%
\pgfsetdash{}{0pt}%
\pgfpathmoveto{\pgfqpoint{0.800800in}{1.566187in}}%
\pgfpathlineto{\pgfqpoint{0.820773in}{1.436116in}}%
\pgfpathlineto{\pgfqpoint{0.840745in}{1.376603in}}%
\pgfpathlineto{\pgfqpoint{0.860718in}{1.336835in}}%
\pgfpathlineto{\pgfqpoint{0.880690in}{1.303095in}}%
\pgfpathlineto{\pgfqpoint{0.900663in}{1.259225in}}%
\pgfpathlineto{\pgfqpoint{0.940608in}{1.212086in}}%
\pgfpathlineto{\pgfqpoint{0.980553in}{1.175465in}}%
\pgfpathlineto{\pgfqpoint{1.020498in}{1.145133in}}%
\pgfpathlineto{\pgfqpoint{1.040471in}{1.134587in}}%
\pgfpathlineto{\pgfqpoint{1.060443in}{1.128409in}}%
\pgfpathlineto{\pgfqpoint{1.080416in}{1.126074in}}%
\pgfpathlineto{\pgfqpoint{1.160306in}{1.124565in}}%
\pgfpathlineto{\pgfqpoint{1.180279in}{1.120349in}}%
\pgfpathlineto{\pgfqpoint{1.240196in}{1.119922in}}%
\pgfpathlineto{\pgfqpoint{1.260169in}{1.119922in}}%
\pgfpathlineto{\pgfqpoint{1.280141in}{1.051221in}}%
\pgfpathlineto{\pgfqpoint{1.300114in}{0.998736in}}%
\pgfpathlineto{\pgfqpoint{1.320086in}{0.992329in}}%
\pgfpathlineto{\pgfqpoint{1.340059in}{0.991066in}}%
\pgfpathlineto{\pgfqpoint{1.360032in}{0.976117in}}%
\pgfpathlineto{\pgfqpoint{1.439922in}{0.971490in}}%
\pgfpathlineto{\pgfqpoint{1.459894in}{0.961494in}}%
\pgfpathlineto{\pgfqpoint{1.479867in}{0.957129in}}%
\pgfpathlineto{\pgfqpoint{1.499839in}{0.947951in}}%
\pgfpathlineto{\pgfqpoint{1.519812in}{0.946301in}}%
\pgfpathlineto{\pgfqpoint{1.539785in}{0.927931in}}%
\pgfpathlineto{\pgfqpoint{1.579730in}{0.927435in}}%
\pgfpathlineto{\pgfqpoint{1.599702in}{0.925567in}}%
\pgfpathlineto{\pgfqpoint{1.619675in}{0.909629in}}%
\pgfpathlineto{\pgfqpoint{1.739510in}{0.901181in}}%
\pgfpathlineto{\pgfqpoint{1.759483in}{0.900805in}}%
\pgfpathlineto{\pgfqpoint{1.779455in}{0.873125in}}%
\pgfpathlineto{\pgfqpoint{1.799428in}{0.869677in}}%
\pgfpathlineto{\pgfqpoint{1.819400in}{0.861213in}}%
\pgfpathlineto{\pgfqpoint{1.839373in}{0.856144in}}%
\pgfpathlineto{\pgfqpoint{1.859346in}{0.856143in}}%
\pgfpathlineto{\pgfqpoint{1.879318in}{0.823020in}}%
\pgfpathlineto{\pgfqpoint{1.919263in}{0.821221in}}%
\pgfpathlineto{\pgfqpoint{1.939236in}{0.786445in}}%
\pgfpathlineto{\pgfqpoint{1.959208in}{0.782224in}}%
\pgfpathlineto{\pgfqpoint{1.979181in}{0.782217in}}%
\pgfpathlineto{\pgfqpoint{1.999153in}{0.780154in}}%
\pgfpathlineto{\pgfqpoint{2.039099in}{0.773324in}}%
\pgfpathlineto{\pgfqpoint{2.059071in}{0.772015in}}%
\pgfpathlineto{\pgfqpoint{2.079044in}{0.763656in}}%
\pgfpathlineto{\pgfqpoint{2.099016in}{0.761275in}}%
\pgfpathlineto{\pgfqpoint{2.178906in}{0.757226in}}%
\pgfpathlineto{\pgfqpoint{2.218852in}{0.757045in}}%
\pgfpathlineto{\pgfqpoint{2.238824in}{0.755057in}}%
\pgfpathlineto{\pgfqpoint{2.258797in}{0.754807in}}%
\pgfpathlineto{\pgfqpoint{2.298742in}{0.744902in}}%
\pgfpathlineto{\pgfqpoint{2.318714in}{0.741549in}}%
\pgfpathlineto{\pgfqpoint{2.338687in}{0.715560in}}%
\pgfpathlineto{\pgfqpoint{2.398605in}{0.714591in}}%
\pgfpathlineto{\pgfqpoint{2.418577in}{0.712713in}}%
\pgfpathlineto{\pgfqpoint{2.458522in}{0.712037in}}%
\pgfpathlineto{\pgfqpoint{2.478495in}{0.710406in}}%
\pgfpathlineto{\pgfqpoint{2.538413in}{0.709924in}}%
\pgfpathlineto{\pgfqpoint{2.658248in}{0.709704in}}%
\pgfpathlineto{\pgfqpoint{2.718166in}{0.709390in}}%
\pgfpathlineto{\pgfqpoint{2.738138in}{0.698726in}}%
\pgfpathlineto{\pgfqpoint{2.798056in}{0.697824in}}%
\pgfpathlineto{\pgfqpoint{2.857973in}{0.697403in}}%
\pgfpathlineto{\pgfqpoint{3.017754in}{0.697019in}}%
\pgfpathlineto{\pgfqpoint{3.037727in}{0.689910in}}%
\pgfpathlineto{\pgfqpoint{3.077672in}{0.689855in}}%
\pgfpathlineto{\pgfqpoint{3.097644in}{0.688686in}}%
\pgfpathlineto{\pgfqpoint{3.177534in}{0.687907in}}%
\pgfpathlineto{\pgfqpoint{3.217480in}{0.687165in}}%
\pgfpathlineto{\pgfqpoint{5.494351in}{0.687163in}}%
\pgfpathlineto{\pgfqpoint{5.494351in}{0.687163in}}%
\pgfusepath{stroke}%
\end{pgfscope}%
\begin{pgfscope}%
\pgfpathrectangle{\pgfqpoint{0.565124in}{0.565123in}}{\pgfqpoint{5.184876in}{2.684877in}}%
\pgfusepath{clip}%
\pgfsetrectcap%
\pgfsetroundjoin%
\pgfsetlinewidth{2.007500pt}%
\definecolor{currentstroke}{rgb}{1.000000,0.498039,0.054902}%
\pgfsetstrokecolor{currentstroke}%
\pgfsetdash{}{0pt}%
\pgfpathmoveto{\pgfqpoint{0.820773in}{3.127958in}}%
\pgfpathlineto{\pgfqpoint{0.880690in}{3.127960in}}%
\pgfpathlineto{\pgfqpoint{0.900663in}{3.104083in}}%
\pgfpathlineto{\pgfqpoint{1.180279in}{3.104088in}}%
\pgfpathlineto{\pgfqpoint{1.200251in}{3.089740in}}%
\pgfpathlineto{\pgfqpoint{1.280141in}{3.089742in}}%
\pgfpathlineto{\pgfqpoint{1.300114in}{3.085256in}}%
\pgfpathlineto{\pgfqpoint{1.479867in}{3.085257in}}%
\pgfpathlineto{\pgfqpoint{1.499839in}{3.076688in}}%
\pgfpathlineto{\pgfqpoint{1.519812in}{3.076502in}}%
\pgfpathlineto{\pgfqpoint{1.539785in}{3.054427in}}%
\pgfpathlineto{\pgfqpoint{1.559757in}{2.707670in}}%
\pgfpathlineto{\pgfqpoint{1.579730in}{2.649023in}}%
\pgfpathlineto{\pgfqpoint{1.599702in}{2.159039in}}%
\pgfpathlineto{\pgfqpoint{1.619675in}{2.148198in}}%
\pgfpathlineto{\pgfqpoint{1.639647in}{2.143333in}}%
\pgfpathlineto{\pgfqpoint{1.659620in}{2.122208in}}%
\pgfpathlineto{\pgfqpoint{1.679593in}{1.976787in}}%
\pgfpathlineto{\pgfqpoint{1.699565in}{1.976787in}}%
\pgfpathlineto{\pgfqpoint{1.719538in}{1.963167in}}%
\pgfpathlineto{\pgfqpoint{1.739510in}{1.935682in}}%
\pgfpathlineto{\pgfqpoint{1.759483in}{1.935682in}}%
\pgfpathlineto{\pgfqpoint{1.779455in}{1.928859in}}%
\pgfpathlineto{\pgfqpoint{1.819400in}{1.928537in}}%
\pgfpathlineto{\pgfqpoint{1.839373in}{1.916278in}}%
\pgfpathlineto{\pgfqpoint{1.859346in}{1.915046in}}%
\pgfpathlineto{\pgfqpoint{1.919263in}{1.906885in}}%
\pgfpathlineto{\pgfqpoint{1.939236in}{1.901395in}}%
\pgfpathlineto{\pgfqpoint{1.959208in}{1.897524in}}%
\pgfpathlineto{\pgfqpoint{1.979181in}{1.897300in}}%
\pgfpathlineto{\pgfqpoint{1.999153in}{1.384465in}}%
\pgfpathlineto{\pgfqpoint{2.019126in}{1.380341in}}%
\pgfpathlineto{\pgfqpoint{2.039099in}{1.380076in}}%
\pgfpathlineto{\pgfqpoint{2.059071in}{1.361916in}}%
\pgfpathlineto{\pgfqpoint{2.079044in}{1.355952in}}%
\pgfpathlineto{\pgfqpoint{2.099016in}{1.342161in}}%
\pgfpathlineto{\pgfqpoint{2.138961in}{1.341356in}}%
\pgfpathlineto{\pgfqpoint{2.378632in}{1.341351in}}%
\pgfpathlineto{\pgfqpoint{2.398605in}{1.332163in}}%
\pgfpathlineto{\pgfqpoint{2.618303in}{1.332103in}}%
\pgfpathlineto{\pgfqpoint{2.638275in}{1.324071in}}%
\pgfpathlineto{\pgfqpoint{2.678220in}{1.324068in}}%
\pgfpathlineto{\pgfqpoint{2.698193in}{0.994400in}}%
\pgfpathlineto{\pgfqpoint{2.738138in}{0.993609in}}%
\pgfpathlineto{\pgfqpoint{2.758111in}{0.993608in}}%
\pgfpathlineto{\pgfqpoint{2.778083in}{0.887861in}}%
\pgfpathlineto{\pgfqpoint{2.798056in}{0.887863in}}%
\pgfpathlineto{\pgfqpoint{2.838001in}{0.882141in}}%
\pgfpathlineto{\pgfqpoint{2.857973in}{0.882138in}}%
\pgfpathlineto{\pgfqpoint{2.877946in}{0.857498in}}%
\pgfpathlineto{\pgfqpoint{2.937864in}{0.857500in}}%
\pgfpathlineto{\pgfqpoint{2.957836in}{0.853485in}}%
\pgfpathlineto{\pgfqpoint{3.077672in}{0.853485in}}%
\pgfpathlineto{\pgfqpoint{3.097644in}{0.849755in}}%
\pgfpathlineto{\pgfqpoint{3.117617in}{0.849752in}}%
\pgfpathlineto{\pgfqpoint{3.137589in}{0.845537in}}%
\pgfpathlineto{\pgfqpoint{3.197507in}{0.845544in}}%
\pgfpathlineto{\pgfqpoint{3.217480in}{0.837663in}}%
\pgfpathlineto{\pgfqpoint{3.237452in}{0.837661in}}%
\pgfpathlineto{\pgfqpoint{3.257425in}{0.833058in}}%
\pgfpathlineto{\pgfqpoint{3.437178in}{0.833058in}}%
\pgfpathlineto{\pgfqpoint{3.457150in}{0.830378in}}%
\pgfpathlineto{\pgfqpoint{3.477123in}{0.829110in}}%
\pgfpathlineto{\pgfqpoint{3.497095in}{0.818908in}}%
\pgfpathlineto{\pgfqpoint{3.537040in}{0.818905in}}%
\pgfpathlineto{\pgfqpoint{3.557013in}{0.815842in}}%
\pgfpathlineto{\pgfqpoint{3.576986in}{0.815845in}}%
\pgfpathlineto{\pgfqpoint{3.596958in}{0.809390in}}%
\pgfpathlineto{\pgfqpoint{3.616931in}{0.809388in}}%
\pgfpathlineto{\pgfqpoint{3.636903in}{0.804747in}}%
\pgfpathlineto{\pgfqpoint{3.656876in}{0.804746in}}%
\pgfpathlineto{\pgfqpoint{3.676848in}{0.768455in}}%
\pgfpathlineto{\pgfqpoint{3.696821in}{0.739299in}}%
\pgfpathlineto{\pgfqpoint{3.796684in}{0.739284in}}%
\pgfpathlineto{\pgfqpoint{3.816656in}{0.727582in}}%
\pgfpathlineto{\pgfqpoint{3.836629in}{0.727583in}}%
\pgfpathlineto{\pgfqpoint{3.856601in}{0.720642in}}%
\pgfpathlineto{\pgfqpoint{3.876574in}{0.703283in}}%
\pgfpathlineto{\pgfqpoint{4.355915in}{0.702912in}}%
\pgfpathlineto{\pgfqpoint{4.375888in}{0.687327in}}%
\pgfpathlineto{\pgfqpoint{5.514324in}{0.687163in}}%
\pgfpathlineto{\pgfqpoint{5.514324in}{0.687163in}}%
\pgfusepath{stroke}%
\end{pgfscope}%
\begin{pgfscope}%
\pgfsetrectcap%
\pgfsetmiterjoin%
\pgfsetlinewidth{0.803000pt}%
\definecolor{currentstroke}{rgb}{0.000000,0.000000,0.000000}%
\pgfsetstrokecolor{currentstroke}%
\pgfsetdash{}{0pt}%
\pgfpathmoveto{\pgfqpoint{0.565124in}{0.565123in}}%
\pgfpathlineto{\pgfqpoint{0.565124in}{3.250000in}}%
\pgfusepath{stroke}%
\end{pgfscope}%
\begin{pgfscope}%
\pgfsetrectcap%
\pgfsetmiterjoin%
\pgfsetlinewidth{0.803000pt}%
\definecolor{currentstroke}{rgb}{0.000000,0.000000,0.000000}%
\pgfsetstrokecolor{currentstroke}%
\pgfsetdash{}{0pt}%
\pgfpathmoveto{\pgfqpoint{5.750000in}{0.565123in}}%
\pgfpathlineto{\pgfqpoint{5.750000in}{3.250000in}}%
\pgfusepath{stroke}%
\end{pgfscope}%
\begin{pgfscope}%
\pgfsetrectcap%
\pgfsetmiterjoin%
\pgfsetlinewidth{0.803000pt}%
\definecolor{currentstroke}{rgb}{0.000000,0.000000,0.000000}%
\pgfsetstrokecolor{currentstroke}%
\pgfsetdash{}{0pt}%
\pgfpathmoveto{\pgfqpoint{0.565124in}{0.565123in}}%
\pgfpathlineto{\pgfqpoint{5.750000in}{0.565123in}}%
\pgfusepath{stroke}%
\end{pgfscope}%
\begin{pgfscope}%
\pgfsetrectcap%
\pgfsetmiterjoin%
\pgfsetlinewidth{0.803000pt}%
\definecolor{currentstroke}{rgb}{0.000000,0.000000,0.000000}%
\pgfsetstrokecolor{currentstroke}%
\pgfsetdash{}{0pt}%
\pgfpathmoveto{\pgfqpoint{0.565124in}{3.250000in}}%
\pgfpathlineto{\pgfqpoint{5.750000in}{3.250000in}}%
\pgfusepath{stroke}%
\end{pgfscope}%
\begin{pgfscope}%
\pgfsetbuttcap%
\pgfsetmiterjoin%
\definecolor{currentfill}{rgb}{1.000000,1.000000,1.000000}%
\pgfsetfillcolor{currentfill}%
\pgfsetfillopacity{0.800000}%
\pgfsetlinewidth{1.003750pt}%
\definecolor{currentstroke}{rgb}{0.800000,0.800000,0.800000}%
\pgfsetstrokecolor{currentstroke}%
\pgfsetstrokeopacity{0.800000}%
\pgfsetdash{}{0pt}%
\pgfpathmoveto{\pgfqpoint{4.273532in}{2.751543in}}%
\pgfpathlineto{\pgfqpoint{5.652778in}{2.751543in}}%
\pgfpathquadraticcurveto{\pgfqpoint{5.680556in}{2.751543in}}{\pgfqpoint{5.680556in}{2.779321in}}%
\pgfpathlineto{\pgfqpoint{5.680556in}{3.152778in}}%
\pgfpathquadraticcurveto{\pgfqpoint{5.680556in}{3.180556in}}{\pgfqpoint{5.652778in}{3.180556in}}%
\pgfpathlineto{\pgfqpoint{4.273532in}{3.180556in}}%
\pgfpathquadraticcurveto{\pgfqpoint{4.245755in}{3.180556in}}{\pgfqpoint{4.245755in}{3.152778in}}%
\pgfpathlineto{\pgfqpoint{4.245755in}{2.779321in}}%
\pgfpathquadraticcurveto{\pgfqpoint{4.245755in}{2.751543in}}{\pgfqpoint{4.273532in}{2.751543in}}%
\pgfpathclose%
\pgfusepath{stroke,fill}%
\end{pgfscope}%
\begin{pgfscope}%
\pgfsetrectcap%
\pgfsetroundjoin%
\pgfsetlinewidth{2.007500pt}%
\definecolor{currentstroke}{rgb}{0.121569,0.466667,0.705882}%
\pgfsetstrokecolor{currentstroke}%
\pgfsetdash{}{0pt}%
\pgfpathmoveto{\pgfqpoint{4.301310in}{3.076389in}}%
\pgfpathlineto{\pgfqpoint{4.579088in}{3.076389in}}%
\pgfusepath{stroke}%
\end{pgfscope}%
\begin{pgfscope}%
\definecolor{textcolor}{rgb}{0.000000,0.000000,0.000000}%
\pgfsetstrokecolor{textcolor}%
\pgfsetfillcolor{textcolor}%
\pgftext[x=4.690199in,y=3.027778in,left,base]{\color{textcolor}\rmfamily\fontsize{10.000000}{12.000000}\selectfont Scena fittizia}%
\end{pgfscope}%
\begin{pgfscope}%
\pgfsetrectcap%
\pgfsetroundjoin%
\pgfsetlinewidth{2.007500pt}%
\definecolor{currentstroke}{rgb}{1.000000,0.498039,0.054902}%
\pgfsetstrokecolor{currentstroke}%
\pgfsetdash{}{0pt}%
\pgfpathmoveto{\pgfqpoint{4.301310in}{2.882716in}}%
\pgfpathlineto{\pgfqpoint{4.579088in}{2.882716in}}%
\pgfusepath{stroke}%
\end{pgfscope}%
\begin{pgfscope}%
\definecolor{textcolor}{rgb}{0.000000,0.000000,0.000000}%
\pgfsetstrokecolor{textcolor}%
\pgfsetfillcolor{textcolor}%
\pgftext[x=4.690199in,y=2.834105in,left,base]{\color{textcolor}\rmfamily\fontsize{10.000000}{12.000000}\selectfont Scena realistica}%
\end{pgfscope}%
\end{pgfpicture}%
\makeatother%
\endgroup%

                \caption{Valutazione Sphere Tracing valutato per la dimensione}
                \label{fig:eval-stdim}
            \end{figure}             

        \newpage
        \subsection{Ray Tracing - Ombre portate}
        Combinare la determinazione delle istanze in vista e la previsione delle ombre portate porta un forte incremento di performance in entrambe le scene. Nella scena realistica si percepisce un miglioramento del 47\% rispetto alla strategia di valutazione per dimensione con lo Sphere Tracing, e del 62.6\% e 78.2\% per le strategie Closest-First on Camera e Closest-First rispettivamente. 
        
        I risultati sono promettenti anche per la scena fittizia data la presenza di un maggior numero di ombre portate provenienti da molte istanze non inquadrate. Si osserva un miglioramento del 75.9\% dalla valutazione per dimensione con lo Sphere Tracing e del 77.3\% per la valutazione per la distanza sempre dello stesso, del 78.6\% per il Closest-First on Camera e del 91.1\% per il Closest-First. È importante sottolineare che queste prestazioni sono mantenute indipendentemente dall'angolo di provenienza della luce a differenza delle politiche valutate nei paragrafi precedenti. Il parametro $\alpha$ è stato impostato ad $1$ per le valutazioni. Attraverso esperimenti empirici è il valore che ha portato i migliori risultati.
            % \begin{figure}[htb!]
            %     \centering
            %     % \includegraphics[scale=.55]{images/valutazioni/fittizia/28-04-23 03-31-24 RayTracing 195.07658568058338.png}
            %     %% Creator: Matplotlib, PGF backend
%%
%% To include the figure in your LaTeX document, write
%%   \input{<filename>.pgf}
%%
%% Make sure the required packages are loaded in your preamble
%%   \usepackage{pgf}
%%
%% Figures using additional raster images can only be included by \input if
%% they are in the same directory as the main LaTeX file. For loading figures
%% from other directories you can use the `import` package
%%   \usepackage{import}
%%
%% and then include the figures with
%%   \import{<path to file>}{<filename>.pgf}
%%
%% Matplotlib used the following preamble
%%
\begingroup%
\makeatletter%
\begin{pgfpicture}%
\pgfpathrectangle{\pgfpointorigin}{\pgfqpoint{5.900000in}{3.400000in}}%
\pgfusepath{use as bounding box, clip}%
\begin{pgfscope}%
\pgfsetbuttcap%
\pgfsetmiterjoin%
\definecolor{currentfill}{rgb}{1.000000,1.000000,1.000000}%
\pgfsetfillcolor{currentfill}%
\pgfsetlinewidth{0.000000pt}%
\definecolor{currentstroke}{rgb}{1.000000,1.000000,1.000000}%
\pgfsetstrokecolor{currentstroke}%
\pgfsetdash{}{0pt}%
\pgfpathmoveto{\pgfqpoint{0.000000in}{0.000000in}}%
\pgfpathlineto{\pgfqpoint{5.900000in}{0.000000in}}%
\pgfpathlineto{\pgfqpoint{5.900000in}{3.400000in}}%
\pgfpathlineto{\pgfqpoint{0.000000in}{3.400000in}}%
\pgfpathclose%
\pgfusepath{fill}%
\end{pgfscope}%
\begin{pgfscope}%
\pgfsetbuttcap%
\pgfsetmiterjoin%
\definecolor{currentfill}{rgb}{1.000000,1.000000,1.000000}%
\pgfsetfillcolor{currentfill}%
\pgfsetlinewidth{0.000000pt}%
\definecolor{currentstroke}{rgb}{0.000000,0.000000,0.000000}%
\pgfsetstrokecolor{currentstroke}%
\pgfsetstrokeopacity{0.000000}%
\pgfsetdash{}{0pt}%
\pgfpathmoveto{\pgfqpoint{0.565124in}{0.549691in}}%
\pgfpathlineto{\pgfqpoint{5.750000in}{0.549691in}}%
\pgfpathlineto{\pgfqpoint{5.750000in}{3.250000in}}%
\pgfpathlineto{\pgfqpoint{0.565124in}{3.250000in}}%
\pgfpathclose%
\pgfusepath{fill}%
\end{pgfscope}%
\begin{pgfscope}%
\pgfsetbuttcap%
\pgfsetroundjoin%
\definecolor{currentfill}{rgb}{0.000000,0.000000,0.000000}%
\pgfsetfillcolor{currentfill}%
\pgfsetlinewidth{0.803000pt}%
\definecolor{currentstroke}{rgb}{0.000000,0.000000,0.000000}%
\pgfsetstrokecolor{currentstroke}%
\pgfsetdash{}{0pt}%
\pgfsys@defobject{currentmarker}{\pgfqpoint{0.000000in}{-0.048611in}}{\pgfqpoint{0.000000in}{0.000000in}}{%
\pgfpathmoveto{\pgfqpoint{0.000000in}{0.000000in}}%
\pgfpathlineto{\pgfqpoint{0.000000in}{-0.048611in}}%
\pgfusepath{stroke,fill}%
}%
\begin{pgfscope}%
\pgfsys@transformshift{0.770968in}{0.549691in}%
\pgfsys@useobject{currentmarker}{}%
\end{pgfscope}%
\end{pgfscope}%
\begin{pgfscope}%
\definecolor{textcolor}{rgb}{0.000000,0.000000,0.000000}%
\pgfsetstrokecolor{textcolor}%
\pgfsetfillcolor{textcolor}%
\pgftext[x=0.770968in,y=0.452469in,,top]{\color{textcolor}\rmfamily\fontsize{10.000000}{12.000000}\selectfont \(\displaystyle {0}\)}%
\end{pgfscope}%
\begin{pgfscope}%
\pgfsetbuttcap%
\pgfsetroundjoin%
\definecolor{currentfill}{rgb}{0.000000,0.000000,0.000000}%
\pgfsetfillcolor{currentfill}%
\pgfsetlinewidth{0.803000pt}%
\definecolor{currentstroke}{rgb}{0.000000,0.000000,0.000000}%
\pgfsetstrokecolor{currentstroke}%
\pgfsetdash{}{0pt}%
\pgfsys@defobject{currentmarker}{\pgfqpoint{0.000000in}{-0.048611in}}{\pgfqpoint{0.000000in}{0.000000in}}{%
\pgfpathmoveto{\pgfqpoint{0.000000in}{0.000000in}}%
\pgfpathlineto{\pgfqpoint{0.000000in}{-0.048611in}}%
\pgfusepath{stroke,fill}%
}%
\begin{pgfscope}%
\pgfsys@transformshift{1.367616in}{0.549691in}%
\pgfsys@useobject{currentmarker}{}%
\end{pgfscope}%
\end{pgfscope}%
\begin{pgfscope}%
\definecolor{textcolor}{rgb}{0.000000,0.000000,0.000000}%
\pgfsetstrokecolor{textcolor}%
\pgfsetfillcolor{textcolor}%
\pgftext[x=1.367616in,y=0.452469in,,top]{\color{textcolor}\rmfamily\fontsize{10.000000}{12.000000}\selectfont \(\displaystyle {20}\)}%
\end{pgfscope}%
\begin{pgfscope}%
\pgfsetbuttcap%
\pgfsetroundjoin%
\definecolor{currentfill}{rgb}{0.000000,0.000000,0.000000}%
\pgfsetfillcolor{currentfill}%
\pgfsetlinewidth{0.803000pt}%
\definecolor{currentstroke}{rgb}{0.000000,0.000000,0.000000}%
\pgfsetstrokecolor{currentstroke}%
\pgfsetdash{}{0pt}%
\pgfsys@defobject{currentmarker}{\pgfqpoint{0.000000in}{-0.048611in}}{\pgfqpoint{0.000000in}{0.000000in}}{%
\pgfpathmoveto{\pgfqpoint{0.000000in}{0.000000in}}%
\pgfpathlineto{\pgfqpoint{0.000000in}{-0.048611in}}%
\pgfusepath{stroke,fill}%
}%
\begin{pgfscope}%
\pgfsys@transformshift{1.964265in}{0.549691in}%
\pgfsys@useobject{currentmarker}{}%
\end{pgfscope}%
\end{pgfscope}%
\begin{pgfscope}%
\definecolor{textcolor}{rgb}{0.000000,0.000000,0.000000}%
\pgfsetstrokecolor{textcolor}%
\pgfsetfillcolor{textcolor}%
\pgftext[x=1.964265in,y=0.452469in,,top]{\color{textcolor}\rmfamily\fontsize{10.000000}{12.000000}\selectfont \(\displaystyle {40}\)}%
\end{pgfscope}%
\begin{pgfscope}%
\pgfsetbuttcap%
\pgfsetroundjoin%
\definecolor{currentfill}{rgb}{0.000000,0.000000,0.000000}%
\pgfsetfillcolor{currentfill}%
\pgfsetlinewidth{0.803000pt}%
\definecolor{currentstroke}{rgb}{0.000000,0.000000,0.000000}%
\pgfsetstrokecolor{currentstroke}%
\pgfsetdash{}{0pt}%
\pgfsys@defobject{currentmarker}{\pgfqpoint{0.000000in}{-0.048611in}}{\pgfqpoint{0.000000in}{0.000000in}}{%
\pgfpathmoveto{\pgfqpoint{0.000000in}{0.000000in}}%
\pgfpathlineto{\pgfqpoint{0.000000in}{-0.048611in}}%
\pgfusepath{stroke,fill}%
}%
\begin{pgfscope}%
\pgfsys@transformshift{2.560913in}{0.549691in}%
\pgfsys@useobject{currentmarker}{}%
\end{pgfscope}%
\end{pgfscope}%
\begin{pgfscope}%
\definecolor{textcolor}{rgb}{0.000000,0.000000,0.000000}%
\pgfsetstrokecolor{textcolor}%
\pgfsetfillcolor{textcolor}%
\pgftext[x=2.560913in,y=0.452469in,,top]{\color{textcolor}\rmfamily\fontsize{10.000000}{12.000000}\selectfont \(\displaystyle {60}\)}%
\end{pgfscope}%
\begin{pgfscope}%
\pgfsetbuttcap%
\pgfsetroundjoin%
\definecolor{currentfill}{rgb}{0.000000,0.000000,0.000000}%
\pgfsetfillcolor{currentfill}%
\pgfsetlinewidth{0.803000pt}%
\definecolor{currentstroke}{rgb}{0.000000,0.000000,0.000000}%
\pgfsetstrokecolor{currentstroke}%
\pgfsetdash{}{0pt}%
\pgfsys@defobject{currentmarker}{\pgfqpoint{0.000000in}{-0.048611in}}{\pgfqpoint{0.000000in}{0.000000in}}{%
\pgfpathmoveto{\pgfqpoint{0.000000in}{0.000000in}}%
\pgfpathlineto{\pgfqpoint{0.000000in}{-0.048611in}}%
\pgfusepath{stroke,fill}%
}%
\begin{pgfscope}%
\pgfsys@transformshift{3.157562in}{0.549691in}%
\pgfsys@useobject{currentmarker}{}%
\end{pgfscope}%
\end{pgfscope}%
\begin{pgfscope}%
\definecolor{textcolor}{rgb}{0.000000,0.000000,0.000000}%
\pgfsetstrokecolor{textcolor}%
\pgfsetfillcolor{textcolor}%
\pgftext[x=3.157562in,y=0.452469in,,top]{\color{textcolor}\rmfamily\fontsize{10.000000}{12.000000}\selectfont \(\displaystyle {80}\)}%
\end{pgfscope}%
\begin{pgfscope}%
\pgfsetbuttcap%
\pgfsetroundjoin%
\definecolor{currentfill}{rgb}{0.000000,0.000000,0.000000}%
\pgfsetfillcolor{currentfill}%
\pgfsetlinewidth{0.803000pt}%
\definecolor{currentstroke}{rgb}{0.000000,0.000000,0.000000}%
\pgfsetstrokecolor{currentstroke}%
\pgfsetdash{}{0pt}%
\pgfsys@defobject{currentmarker}{\pgfqpoint{0.000000in}{-0.048611in}}{\pgfqpoint{0.000000in}{0.000000in}}{%
\pgfpathmoveto{\pgfqpoint{0.000000in}{0.000000in}}%
\pgfpathlineto{\pgfqpoint{0.000000in}{-0.048611in}}%
\pgfusepath{stroke,fill}%
}%
\begin{pgfscope}%
\pgfsys@transformshift{3.754210in}{0.549691in}%
\pgfsys@useobject{currentmarker}{}%
\end{pgfscope}%
\end{pgfscope}%
\begin{pgfscope}%
\definecolor{textcolor}{rgb}{0.000000,0.000000,0.000000}%
\pgfsetstrokecolor{textcolor}%
\pgfsetfillcolor{textcolor}%
\pgftext[x=3.754210in,y=0.452469in,,top]{\color{textcolor}\rmfamily\fontsize{10.000000}{12.000000}\selectfont \(\displaystyle {100}\)}%
\end{pgfscope}%
\begin{pgfscope}%
\pgfsetbuttcap%
\pgfsetroundjoin%
\definecolor{currentfill}{rgb}{0.000000,0.000000,0.000000}%
\pgfsetfillcolor{currentfill}%
\pgfsetlinewidth{0.803000pt}%
\definecolor{currentstroke}{rgb}{0.000000,0.000000,0.000000}%
\pgfsetstrokecolor{currentstroke}%
\pgfsetdash{}{0pt}%
\pgfsys@defobject{currentmarker}{\pgfqpoint{0.000000in}{-0.048611in}}{\pgfqpoint{0.000000in}{0.000000in}}{%
\pgfpathmoveto{\pgfqpoint{0.000000in}{0.000000in}}%
\pgfpathlineto{\pgfqpoint{0.000000in}{-0.048611in}}%
\pgfusepath{stroke,fill}%
}%
\begin{pgfscope}%
\pgfsys@transformshift{4.350859in}{0.549691in}%
\pgfsys@useobject{currentmarker}{}%
\end{pgfscope}%
\end{pgfscope}%
\begin{pgfscope}%
\definecolor{textcolor}{rgb}{0.000000,0.000000,0.000000}%
\pgfsetstrokecolor{textcolor}%
\pgfsetfillcolor{textcolor}%
\pgftext[x=4.350859in,y=0.452469in,,top]{\color{textcolor}\rmfamily\fontsize{10.000000}{12.000000}\selectfont \(\displaystyle {120}\)}%
\end{pgfscope}%
\begin{pgfscope}%
\pgfsetbuttcap%
\pgfsetroundjoin%
\definecolor{currentfill}{rgb}{0.000000,0.000000,0.000000}%
\pgfsetfillcolor{currentfill}%
\pgfsetlinewidth{0.803000pt}%
\definecolor{currentstroke}{rgb}{0.000000,0.000000,0.000000}%
\pgfsetstrokecolor{currentstroke}%
\pgfsetdash{}{0pt}%
\pgfsys@defobject{currentmarker}{\pgfqpoint{0.000000in}{-0.048611in}}{\pgfqpoint{0.000000in}{0.000000in}}{%
\pgfpathmoveto{\pgfqpoint{0.000000in}{0.000000in}}%
\pgfpathlineto{\pgfqpoint{0.000000in}{-0.048611in}}%
\pgfusepath{stroke,fill}%
}%
\begin{pgfscope}%
\pgfsys@transformshift{4.947508in}{0.549691in}%
\pgfsys@useobject{currentmarker}{}%
\end{pgfscope}%
\end{pgfscope}%
\begin{pgfscope}%
\definecolor{textcolor}{rgb}{0.000000,0.000000,0.000000}%
\pgfsetstrokecolor{textcolor}%
\pgfsetfillcolor{textcolor}%
\pgftext[x=4.947508in,y=0.452469in,,top]{\color{textcolor}\rmfamily\fontsize{10.000000}{12.000000}\selectfont \(\displaystyle {140}\)}%
\end{pgfscope}%
\begin{pgfscope}%
\pgfsetbuttcap%
\pgfsetroundjoin%
\definecolor{currentfill}{rgb}{0.000000,0.000000,0.000000}%
\pgfsetfillcolor{currentfill}%
\pgfsetlinewidth{0.803000pt}%
\definecolor{currentstroke}{rgb}{0.000000,0.000000,0.000000}%
\pgfsetstrokecolor{currentstroke}%
\pgfsetdash{}{0pt}%
\pgfsys@defobject{currentmarker}{\pgfqpoint{0.000000in}{-0.048611in}}{\pgfqpoint{0.000000in}{0.000000in}}{%
\pgfpathmoveto{\pgfqpoint{0.000000in}{0.000000in}}%
\pgfpathlineto{\pgfqpoint{0.000000in}{-0.048611in}}%
\pgfusepath{stroke,fill}%
}%
\begin{pgfscope}%
\pgfsys@transformshift{5.544156in}{0.549691in}%
\pgfsys@useobject{currentmarker}{}%
\end{pgfscope}%
\end{pgfscope}%
\begin{pgfscope}%
\definecolor{textcolor}{rgb}{0.000000,0.000000,0.000000}%
\pgfsetstrokecolor{textcolor}%
\pgfsetfillcolor{textcolor}%
\pgftext[x=5.544156in,y=0.452469in,,top]{\color{textcolor}\rmfamily\fontsize{10.000000}{12.000000}\selectfont \(\displaystyle {160}\)}%
\end{pgfscope}%
\begin{pgfscope}%
\definecolor{textcolor}{rgb}{0.000000,0.000000,0.000000}%
\pgfsetstrokecolor{textcolor}%
\pgfsetfillcolor{textcolor}%
\pgftext[x=3.157562in,y=0.273457in,,top]{\color{textcolor}\rmfamily\fontsize{10.000000}{12.000000}\selectfont Tempo}%
\end{pgfscope}%
\begin{pgfscope}%
\pgfsetbuttcap%
\pgfsetroundjoin%
\definecolor{currentfill}{rgb}{0.000000,0.000000,0.000000}%
\pgfsetfillcolor{currentfill}%
\pgfsetlinewidth{0.803000pt}%
\definecolor{currentstroke}{rgb}{0.000000,0.000000,0.000000}%
\pgfsetstrokecolor{currentstroke}%
\pgfsetdash{}{0pt}%
\pgfsys@defobject{currentmarker}{\pgfqpoint{-0.048611in}{0.000000in}}{\pgfqpoint{-0.000000in}{0.000000in}}{%
\pgfpathmoveto{\pgfqpoint{-0.000000in}{0.000000in}}%
\pgfpathlineto{\pgfqpoint{-0.048611in}{0.000000in}}%
\pgfusepath{stroke,fill}%
}%
\begin{pgfscope}%
\pgfsys@transformshift{0.565124in}{0.672432in}%
\pgfsys@useobject{currentmarker}{}%
\end{pgfscope}%
\end{pgfscope}%
\begin{pgfscope}%
\definecolor{textcolor}{rgb}{0.000000,0.000000,0.000000}%
\pgfsetstrokecolor{textcolor}%
\pgfsetfillcolor{textcolor}%
\pgftext[x=0.398457in, y=0.624207in, left, base]{\color{textcolor}\rmfamily\fontsize{10.000000}{12.000000}\selectfont \(\displaystyle {0}\)}%
\end{pgfscope}%
\begin{pgfscope}%
\pgfsetbuttcap%
\pgfsetroundjoin%
\definecolor{currentfill}{rgb}{0.000000,0.000000,0.000000}%
\pgfsetfillcolor{currentfill}%
\pgfsetlinewidth{0.803000pt}%
\definecolor{currentstroke}{rgb}{0.000000,0.000000,0.000000}%
\pgfsetstrokecolor{currentstroke}%
\pgfsetdash{}{0pt}%
\pgfsys@defobject{currentmarker}{\pgfqpoint{-0.048611in}{0.000000in}}{\pgfqpoint{-0.000000in}{0.000000in}}{%
\pgfpathmoveto{\pgfqpoint{-0.000000in}{0.000000in}}%
\pgfpathlineto{\pgfqpoint{-0.048611in}{0.000000in}}%
\pgfusepath{stroke,fill}%
}%
\begin{pgfscope}%
\pgfsys@transformshift{0.565124in}{1.089062in}%
\pgfsys@useobject{currentmarker}{}%
\end{pgfscope}%
\end{pgfscope}%
\begin{pgfscope}%
\definecolor{textcolor}{rgb}{0.000000,0.000000,0.000000}%
\pgfsetstrokecolor{textcolor}%
\pgfsetfillcolor{textcolor}%
\pgftext[x=0.398457in, y=1.040837in, left, base]{\color{textcolor}\rmfamily\fontsize{10.000000}{12.000000}\selectfont \(\displaystyle {5}\)}%
\end{pgfscope}%
\begin{pgfscope}%
\pgfsetbuttcap%
\pgfsetroundjoin%
\definecolor{currentfill}{rgb}{0.000000,0.000000,0.000000}%
\pgfsetfillcolor{currentfill}%
\pgfsetlinewidth{0.803000pt}%
\definecolor{currentstroke}{rgb}{0.000000,0.000000,0.000000}%
\pgfsetstrokecolor{currentstroke}%
\pgfsetdash{}{0pt}%
\pgfsys@defobject{currentmarker}{\pgfqpoint{-0.048611in}{0.000000in}}{\pgfqpoint{-0.000000in}{0.000000in}}{%
\pgfpathmoveto{\pgfqpoint{-0.000000in}{0.000000in}}%
\pgfpathlineto{\pgfqpoint{-0.048611in}{0.000000in}}%
\pgfusepath{stroke,fill}%
}%
\begin{pgfscope}%
\pgfsys@transformshift{0.565124in}{1.505692in}%
\pgfsys@useobject{currentmarker}{}%
\end{pgfscope}%
\end{pgfscope}%
\begin{pgfscope}%
\definecolor{textcolor}{rgb}{0.000000,0.000000,0.000000}%
\pgfsetstrokecolor{textcolor}%
\pgfsetfillcolor{textcolor}%
\pgftext[x=0.329012in, y=1.457467in, left, base]{\color{textcolor}\rmfamily\fontsize{10.000000}{12.000000}\selectfont \(\displaystyle {10}\)}%
\end{pgfscope}%
\begin{pgfscope}%
\pgfsetbuttcap%
\pgfsetroundjoin%
\definecolor{currentfill}{rgb}{0.000000,0.000000,0.000000}%
\pgfsetfillcolor{currentfill}%
\pgfsetlinewidth{0.803000pt}%
\definecolor{currentstroke}{rgb}{0.000000,0.000000,0.000000}%
\pgfsetstrokecolor{currentstroke}%
\pgfsetdash{}{0pt}%
\pgfsys@defobject{currentmarker}{\pgfqpoint{-0.048611in}{0.000000in}}{\pgfqpoint{-0.000000in}{0.000000in}}{%
\pgfpathmoveto{\pgfqpoint{-0.000000in}{0.000000in}}%
\pgfpathlineto{\pgfqpoint{-0.048611in}{0.000000in}}%
\pgfusepath{stroke,fill}%
}%
\begin{pgfscope}%
\pgfsys@transformshift{0.565124in}{1.922322in}%
\pgfsys@useobject{currentmarker}{}%
\end{pgfscope}%
\end{pgfscope}%
\begin{pgfscope}%
\definecolor{textcolor}{rgb}{0.000000,0.000000,0.000000}%
\pgfsetstrokecolor{textcolor}%
\pgfsetfillcolor{textcolor}%
\pgftext[x=0.329012in, y=1.874097in, left, base]{\color{textcolor}\rmfamily\fontsize{10.000000}{12.000000}\selectfont \(\displaystyle {15}\)}%
\end{pgfscope}%
\begin{pgfscope}%
\pgfsetbuttcap%
\pgfsetroundjoin%
\definecolor{currentfill}{rgb}{0.000000,0.000000,0.000000}%
\pgfsetfillcolor{currentfill}%
\pgfsetlinewidth{0.803000pt}%
\definecolor{currentstroke}{rgb}{0.000000,0.000000,0.000000}%
\pgfsetstrokecolor{currentstroke}%
\pgfsetdash{}{0pt}%
\pgfsys@defobject{currentmarker}{\pgfqpoint{-0.048611in}{0.000000in}}{\pgfqpoint{-0.000000in}{0.000000in}}{%
\pgfpathmoveto{\pgfqpoint{-0.000000in}{0.000000in}}%
\pgfpathlineto{\pgfqpoint{-0.048611in}{0.000000in}}%
\pgfusepath{stroke,fill}%
}%
\begin{pgfscope}%
\pgfsys@transformshift{0.565124in}{2.338952in}%
\pgfsys@useobject{currentmarker}{}%
\end{pgfscope}%
\end{pgfscope}%
\begin{pgfscope}%
\definecolor{textcolor}{rgb}{0.000000,0.000000,0.000000}%
\pgfsetstrokecolor{textcolor}%
\pgfsetfillcolor{textcolor}%
\pgftext[x=0.329012in, y=2.290727in, left, base]{\color{textcolor}\rmfamily\fontsize{10.000000}{12.000000}\selectfont \(\displaystyle {20}\)}%
\end{pgfscope}%
\begin{pgfscope}%
\pgfsetbuttcap%
\pgfsetroundjoin%
\definecolor{currentfill}{rgb}{0.000000,0.000000,0.000000}%
\pgfsetfillcolor{currentfill}%
\pgfsetlinewidth{0.803000pt}%
\definecolor{currentstroke}{rgb}{0.000000,0.000000,0.000000}%
\pgfsetstrokecolor{currentstroke}%
\pgfsetdash{}{0pt}%
\pgfsys@defobject{currentmarker}{\pgfqpoint{-0.048611in}{0.000000in}}{\pgfqpoint{-0.000000in}{0.000000in}}{%
\pgfpathmoveto{\pgfqpoint{-0.000000in}{0.000000in}}%
\pgfpathlineto{\pgfqpoint{-0.048611in}{0.000000in}}%
\pgfusepath{stroke,fill}%
}%
\begin{pgfscope}%
\pgfsys@transformshift{0.565124in}{2.755582in}%
\pgfsys@useobject{currentmarker}{}%
\end{pgfscope}%
\end{pgfscope}%
\begin{pgfscope}%
\definecolor{textcolor}{rgb}{0.000000,0.000000,0.000000}%
\pgfsetstrokecolor{textcolor}%
\pgfsetfillcolor{textcolor}%
\pgftext[x=0.329012in, y=2.707357in, left, base]{\color{textcolor}\rmfamily\fontsize{10.000000}{12.000000}\selectfont \(\displaystyle {25}\)}%
\end{pgfscope}%
\begin{pgfscope}%
\pgfsetbuttcap%
\pgfsetroundjoin%
\definecolor{currentfill}{rgb}{0.000000,0.000000,0.000000}%
\pgfsetfillcolor{currentfill}%
\pgfsetlinewidth{0.803000pt}%
\definecolor{currentstroke}{rgb}{0.000000,0.000000,0.000000}%
\pgfsetstrokecolor{currentstroke}%
\pgfsetdash{}{0pt}%
\pgfsys@defobject{currentmarker}{\pgfqpoint{-0.048611in}{0.000000in}}{\pgfqpoint{-0.000000in}{0.000000in}}{%
\pgfpathmoveto{\pgfqpoint{-0.000000in}{0.000000in}}%
\pgfpathlineto{\pgfqpoint{-0.048611in}{0.000000in}}%
\pgfusepath{stroke,fill}%
}%
\begin{pgfscope}%
\pgfsys@transformshift{0.565124in}{3.172212in}%
\pgfsys@useobject{currentmarker}{}%
\end{pgfscope}%
\end{pgfscope}%
\begin{pgfscope}%
\definecolor{textcolor}{rgb}{0.000000,0.000000,0.000000}%
\pgfsetstrokecolor{textcolor}%
\pgfsetfillcolor{textcolor}%
\pgftext[x=0.329012in, y=3.123986in, left, base]{\color{textcolor}\rmfamily\fontsize{10.000000}{12.000000}\selectfont \(\displaystyle {30}\)}%
\end{pgfscope}%
\begin{pgfscope}%
\definecolor{textcolor}{rgb}{0.000000,0.000000,0.000000}%
\pgfsetstrokecolor{textcolor}%
\pgfsetfillcolor{textcolor}%
\pgftext[x=0.273457in,y=1.899846in,,bottom,rotate=90.000000]{\color{textcolor}\rmfamily\fontsize{10.000000}{12.000000}\selectfont Differenza percettiva}%
\end{pgfscope}%
\begin{pgfscope}%
\pgfpathrectangle{\pgfqpoint{0.565124in}{0.549691in}}{\pgfqpoint{5.184876in}{2.700309in}}%
\pgfusepath{clip}%
\pgfsetrectcap%
\pgfsetroundjoin%
\pgfsetlinewidth{2.007500pt}%
\definecolor{currentstroke}{rgb}{0.121569,0.466667,0.705882}%
\pgfsetstrokecolor{currentstroke}%
\pgfsetdash{}{0pt}%
\pgfpathmoveto{\pgfqpoint{0.800800in}{3.127259in}}%
\pgfpathlineto{\pgfqpoint{0.830632in}{2.547361in}}%
\pgfpathlineto{\pgfqpoint{0.860465in}{2.301762in}}%
\pgfpathlineto{\pgfqpoint{0.890297in}{2.070902in}}%
\pgfpathlineto{\pgfqpoint{0.920130in}{1.779736in}}%
\pgfpathlineto{\pgfqpoint{0.949962in}{1.678325in}}%
\pgfpathlineto{\pgfqpoint{0.979795in}{1.514068in}}%
\pgfpathlineto{\pgfqpoint{1.009627in}{1.455498in}}%
\pgfpathlineto{\pgfqpoint{1.039459in}{1.376207in}}%
\pgfpathlineto{\pgfqpoint{1.069292in}{1.292514in}}%
\pgfpathlineto{\pgfqpoint{1.099124in}{1.251603in}}%
\pgfpathlineto{\pgfqpoint{1.128957in}{1.125702in}}%
\pgfpathlineto{\pgfqpoint{1.158789in}{1.065341in}}%
\pgfpathlineto{\pgfqpoint{1.188622in}{1.029981in}}%
\pgfpathlineto{\pgfqpoint{1.218454in}{0.989446in}}%
\pgfpathlineto{\pgfqpoint{1.248286in}{0.938726in}}%
\pgfpathlineto{\pgfqpoint{1.278119in}{0.907757in}}%
\pgfpathlineto{\pgfqpoint{1.307951in}{0.868069in}}%
\pgfpathlineto{\pgfqpoint{1.337784in}{0.833323in}}%
\pgfpathlineto{\pgfqpoint{1.367616in}{0.810448in}}%
\pgfpathlineto{\pgfqpoint{1.397449in}{0.762493in}}%
\pgfpathlineto{\pgfqpoint{1.427281in}{0.749261in}}%
\pgfpathlineto{\pgfqpoint{1.457113in}{0.732959in}}%
\pgfpathlineto{\pgfqpoint{1.546611in}{0.702302in}}%
\pgfpathlineto{\pgfqpoint{1.576443in}{0.698231in}}%
\pgfpathlineto{\pgfqpoint{1.606276in}{0.691983in}}%
\pgfpathlineto{\pgfqpoint{1.636108in}{0.687326in}}%
\pgfpathlineto{\pgfqpoint{1.725605in}{0.682748in}}%
\pgfpathlineto{\pgfqpoint{1.785270in}{0.681754in}}%
\pgfpathlineto{\pgfqpoint{1.815103in}{0.680166in}}%
\pgfpathlineto{\pgfqpoint{2.859238in}{0.680101in}}%
\pgfpathlineto{\pgfqpoint{2.889070in}{0.672444in}}%
\pgfpathlineto{\pgfqpoint{5.514324in}{0.672432in}}%
\pgfpathlineto{\pgfqpoint{5.514324in}{0.672432in}}%
\pgfusepath{stroke}%
\end{pgfscope}%
\begin{pgfscope}%
\pgfsetrectcap%
\pgfsetmiterjoin%
\pgfsetlinewidth{0.803000pt}%
\definecolor{currentstroke}{rgb}{0.000000,0.000000,0.000000}%
\pgfsetstrokecolor{currentstroke}%
\pgfsetdash{}{0pt}%
\pgfpathmoveto{\pgfqpoint{0.565124in}{0.549691in}}%
\pgfpathlineto{\pgfqpoint{0.565124in}{3.250000in}}%
\pgfusepath{stroke}%
\end{pgfscope}%
\begin{pgfscope}%
\pgfsetrectcap%
\pgfsetmiterjoin%
\pgfsetlinewidth{0.803000pt}%
\definecolor{currentstroke}{rgb}{0.000000,0.000000,0.000000}%
\pgfsetstrokecolor{currentstroke}%
\pgfsetdash{}{0pt}%
\pgfpathmoveto{\pgfqpoint{5.750000in}{0.549691in}}%
\pgfpathlineto{\pgfqpoint{5.750000in}{3.250000in}}%
\pgfusepath{stroke}%
\end{pgfscope}%
\begin{pgfscope}%
\pgfsetrectcap%
\pgfsetmiterjoin%
\pgfsetlinewidth{0.803000pt}%
\definecolor{currentstroke}{rgb}{0.000000,0.000000,0.000000}%
\pgfsetstrokecolor{currentstroke}%
\pgfsetdash{}{0pt}%
\pgfpathmoveto{\pgfqpoint{0.565124in}{0.549691in}}%
\pgfpathlineto{\pgfqpoint{5.750000in}{0.549691in}}%
\pgfusepath{stroke}%
\end{pgfscope}%
\begin{pgfscope}%
\pgfsetrectcap%
\pgfsetmiterjoin%
\pgfsetlinewidth{0.803000pt}%
\definecolor{currentstroke}{rgb}{0.000000,0.000000,0.000000}%
\pgfsetstrokecolor{currentstroke}%
\pgfsetdash{}{0pt}%
\pgfpathmoveto{\pgfqpoint{0.565124in}{3.250000in}}%
\pgfpathlineto{\pgfqpoint{5.750000in}{3.250000in}}%
\pgfusepath{stroke}%
\end{pgfscope}%
\end{pgfpicture}%
\makeatother%
\endgroup%

            %     \caption{Valutazione Ray Tracing - ombre portate sulla scena fittizia}
            %     \label{fig:eval-raytracing-fit}
            % \end{figure}

            % \begin{figure}[htb!]
            %     \centering
            %     \includegraphics[scale=.55]{images/valutazioni/realistica/28-04-23 03-32-23 RayTracing 2860.4245073910374.png}
            %     \caption{Valutazione Ray tracing - ombre portate sulla scena realistica}
            %     \label{fig:eval-raytracing-re}
            % \end{figure}

            % \begin{figure}[htbp]
            %     \centering
            %     \includegraphics[width=\textwidth]{images/sequences/sequence-rt-re.png}
            %     \par
            %     \vspace{15pt}
            %     \centering
            %     \includegraphics[width=\textwidth]{images/sequences/sequence-rt-fit.png}
            %     \caption{Sequenza di caricamento di Ray Tracing}
            %     \label{fig:seq-rt}
            % \end{figure}

            \begin{figure}[htb!]
                \centering
                %% Creator: Matplotlib, PGF backend
%%
%% To include the figure in your LaTeX document, write
%%   \input{<filename>.pgf}
%%
%% Make sure the required packages are loaded in your preamble
%%   \usepackage{pgf}
%%
%% Figures using additional raster images can only be included by \input if
%% they are in the same directory as the main LaTeX file. For loading figures
%% from other directories you can use the `import` package
%%   \usepackage{import}
%%
%% and then include the figures with
%%   \import{<path to file>}{<filename>.pgf}
%%
%% Matplotlib used the following preamble
%%
\begingroup%
\makeatletter%
\begin{pgfpicture}%
\pgfpathrectangle{\pgfpointorigin}{\pgfqpoint{5.900000in}{3.400000in}}%
\pgfusepath{use as bounding box, clip}%
\begin{pgfscope}%
\pgfsetbuttcap%
\pgfsetmiterjoin%
\definecolor{currentfill}{rgb}{1.000000,1.000000,1.000000}%
\pgfsetfillcolor{currentfill}%
\pgfsetlinewidth{0.000000pt}%
\definecolor{currentstroke}{rgb}{1.000000,1.000000,1.000000}%
\pgfsetstrokecolor{currentstroke}%
\pgfsetdash{}{0pt}%
\pgfpathmoveto{\pgfqpoint{0.000000in}{0.000000in}}%
\pgfpathlineto{\pgfqpoint{5.900000in}{0.000000in}}%
\pgfpathlineto{\pgfqpoint{5.900000in}{3.400000in}}%
\pgfpathlineto{\pgfqpoint{0.000000in}{3.400000in}}%
\pgfpathclose%
\pgfusepath{fill}%
\end{pgfscope}%
\begin{pgfscope}%
\pgfsetbuttcap%
\pgfsetmiterjoin%
\definecolor{currentfill}{rgb}{1.000000,1.000000,1.000000}%
\pgfsetfillcolor{currentfill}%
\pgfsetlinewidth{0.000000pt}%
\definecolor{currentstroke}{rgb}{0.000000,0.000000,0.000000}%
\pgfsetstrokecolor{currentstroke}%
\pgfsetstrokeopacity{0.000000}%
\pgfsetdash{}{0pt}%
\pgfpathmoveto{\pgfqpoint{0.565124in}{0.565123in}}%
\pgfpathlineto{\pgfqpoint{5.750000in}{0.565123in}}%
\pgfpathlineto{\pgfqpoint{5.750000in}{3.250000in}}%
\pgfpathlineto{\pgfqpoint{0.565124in}{3.250000in}}%
\pgfpathclose%
\pgfusepath{fill}%
\end{pgfscope}%
\begin{pgfscope}%
\pgfsetbuttcap%
\pgfsetroundjoin%
\definecolor{currentfill}{rgb}{0.000000,0.000000,0.000000}%
\pgfsetfillcolor{currentfill}%
\pgfsetlinewidth{0.803000pt}%
\definecolor{currentstroke}{rgb}{0.000000,0.000000,0.000000}%
\pgfsetstrokecolor{currentstroke}%
\pgfsetdash{}{0pt}%
\pgfsys@defobject{currentmarker}{\pgfqpoint{0.000000in}{-0.048611in}}{\pgfqpoint{0.000000in}{0.000000in}}{%
\pgfpathmoveto{\pgfqpoint{0.000000in}{0.000000in}}%
\pgfpathlineto{\pgfqpoint{0.000000in}{-0.048611in}}%
\pgfusepath{stroke,fill}%
}%
\begin{pgfscope}%
\pgfsys@transformshift{0.800800in}{0.565123in}%
\pgfsys@useobject{currentmarker}{}%
\end{pgfscope}%
\end{pgfscope}%
\begin{pgfscope}%
\definecolor{textcolor}{rgb}{0.000000,0.000000,0.000000}%
\pgfsetstrokecolor{textcolor}%
\pgfsetfillcolor{textcolor}%
\pgftext[x=0.800800in,y=0.467901in,,top]{\color{textcolor}\rmfamily\fontsize{10.000000}{12.000000}\selectfont \(\displaystyle {0}\)}%
\end{pgfscope}%
\begin{pgfscope}%
\pgfsetbuttcap%
\pgfsetroundjoin%
\definecolor{currentfill}{rgb}{0.000000,0.000000,0.000000}%
\pgfsetfillcolor{currentfill}%
\pgfsetlinewidth{0.803000pt}%
\definecolor{currentstroke}{rgb}{0.000000,0.000000,0.000000}%
\pgfsetstrokecolor{currentstroke}%
\pgfsetdash{}{0pt}%
\pgfsys@defobject{currentmarker}{\pgfqpoint{0.000000in}{-0.048611in}}{\pgfqpoint{0.000000in}{0.000000in}}{%
\pgfpathmoveto{\pgfqpoint{0.000000in}{0.000000in}}%
\pgfpathlineto{\pgfqpoint{0.000000in}{-0.048611in}}%
\pgfusepath{stroke,fill}%
}%
\begin{pgfscope}%
\pgfsys@transformshift{1.799428in}{0.565123in}%
\pgfsys@useobject{currentmarker}{}%
\end{pgfscope}%
\end{pgfscope}%
\begin{pgfscope}%
\definecolor{textcolor}{rgb}{0.000000,0.000000,0.000000}%
\pgfsetstrokecolor{textcolor}%
\pgfsetfillcolor{textcolor}%
\pgftext[x=1.799428in,y=0.467901in,,top]{\color{textcolor}\rmfamily\fontsize{10.000000}{12.000000}\selectfont \(\displaystyle {50}\)}%
\end{pgfscope}%
\begin{pgfscope}%
\pgfsetbuttcap%
\pgfsetroundjoin%
\definecolor{currentfill}{rgb}{0.000000,0.000000,0.000000}%
\pgfsetfillcolor{currentfill}%
\pgfsetlinewidth{0.803000pt}%
\definecolor{currentstroke}{rgb}{0.000000,0.000000,0.000000}%
\pgfsetstrokecolor{currentstroke}%
\pgfsetdash{}{0pt}%
\pgfsys@defobject{currentmarker}{\pgfqpoint{0.000000in}{-0.048611in}}{\pgfqpoint{0.000000in}{0.000000in}}{%
\pgfpathmoveto{\pgfqpoint{0.000000in}{0.000000in}}%
\pgfpathlineto{\pgfqpoint{0.000000in}{-0.048611in}}%
\pgfusepath{stroke,fill}%
}%
\begin{pgfscope}%
\pgfsys@transformshift{2.798056in}{0.565123in}%
\pgfsys@useobject{currentmarker}{}%
\end{pgfscope}%
\end{pgfscope}%
\begin{pgfscope}%
\definecolor{textcolor}{rgb}{0.000000,0.000000,0.000000}%
\pgfsetstrokecolor{textcolor}%
\pgfsetfillcolor{textcolor}%
\pgftext[x=2.798056in,y=0.467901in,,top]{\color{textcolor}\rmfamily\fontsize{10.000000}{12.000000}\selectfont \(\displaystyle {100}\)}%
\end{pgfscope}%
\begin{pgfscope}%
\pgfsetbuttcap%
\pgfsetroundjoin%
\definecolor{currentfill}{rgb}{0.000000,0.000000,0.000000}%
\pgfsetfillcolor{currentfill}%
\pgfsetlinewidth{0.803000pt}%
\definecolor{currentstroke}{rgb}{0.000000,0.000000,0.000000}%
\pgfsetstrokecolor{currentstroke}%
\pgfsetdash{}{0pt}%
\pgfsys@defobject{currentmarker}{\pgfqpoint{0.000000in}{-0.048611in}}{\pgfqpoint{0.000000in}{0.000000in}}{%
\pgfpathmoveto{\pgfqpoint{0.000000in}{0.000000in}}%
\pgfpathlineto{\pgfqpoint{0.000000in}{-0.048611in}}%
\pgfusepath{stroke,fill}%
}%
\begin{pgfscope}%
\pgfsys@transformshift{3.796684in}{0.565123in}%
\pgfsys@useobject{currentmarker}{}%
\end{pgfscope}%
\end{pgfscope}%
\begin{pgfscope}%
\definecolor{textcolor}{rgb}{0.000000,0.000000,0.000000}%
\pgfsetstrokecolor{textcolor}%
\pgfsetfillcolor{textcolor}%
\pgftext[x=3.796684in,y=0.467901in,,top]{\color{textcolor}\rmfamily\fontsize{10.000000}{12.000000}\selectfont \(\displaystyle {150}\)}%
\end{pgfscope}%
\begin{pgfscope}%
\pgfsetbuttcap%
\pgfsetroundjoin%
\definecolor{currentfill}{rgb}{0.000000,0.000000,0.000000}%
\pgfsetfillcolor{currentfill}%
\pgfsetlinewidth{0.803000pt}%
\definecolor{currentstroke}{rgb}{0.000000,0.000000,0.000000}%
\pgfsetstrokecolor{currentstroke}%
\pgfsetdash{}{0pt}%
\pgfsys@defobject{currentmarker}{\pgfqpoint{0.000000in}{-0.048611in}}{\pgfqpoint{0.000000in}{0.000000in}}{%
\pgfpathmoveto{\pgfqpoint{0.000000in}{0.000000in}}%
\pgfpathlineto{\pgfqpoint{0.000000in}{-0.048611in}}%
\pgfusepath{stroke,fill}%
}%
\begin{pgfscope}%
\pgfsys@transformshift{4.795312in}{0.565123in}%
\pgfsys@useobject{currentmarker}{}%
\end{pgfscope}%
\end{pgfscope}%
\begin{pgfscope}%
\definecolor{textcolor}{rgb}{0.000000,0.000000,0.000000}%
\pgfsetstrokecolor{textcolor}%
\pgfsetfillcolor{textcolor}%
\pgftext[x=4.795312in,y=0.467901in,,top]{\color{textcolor}\rmfamily\fontsize{10.000000}{12.000000}\selectfont \(\displaystyle {200}\)}%
\end{pgfscope}%
\begin{pgfscope}%
\definecolor{textcolor}{rgb}{0.000000,0.000000,0.000000}%
\pgfsetstrokecolor{textcolor}%
\pgfsetfillcolor{textcolor}%
\pgftext[x=3.157562in,y=0.288889in,,top]{\color{textcolor}\rmfamily\fontsize{10.000000}{12.000000}\selectfont Tempo (Frame)}%
\end{pgfscope}%
\begin{pgfscope}%
\pgfsetbuttcap%
\pgfsetroundjoin%
\definecolor{currentfill}{rgb}{0.000000,0.000000,0.000000}%
\pgfsetfillcolor{currentfill}%
\pgfsetlinewidth{0.803000pt}%
\definecolor{currentstroke}{rgb}{0.000000,0.000000,0.000000}%
\pgfsetstrokecolor{currentstroke}%
\pgfsetdash{}{0pt}%
\pgfsys@defobject{currentmarker}{\pgfqpoint{-0.048611in}{0.000000in}}{\pgfqpoint{-0.000000in}{0.000000in}}{%
\pgfpathmoveto{\pgfqpoint{-0.000000in}{0.000000in}}%
\pgfpathlineto{\pgfqpoint{-0.048611in}{0.000000in}}%
\pgfusepath{stroke,fill}%
}%
\begin{pgfscope}%
\pgfsys@transformshift{0.565124in}{0.687163in}%
\pgfsys@useobject{currentmarker}{}%
\end{pgfscope}%
\end{pgfscope}%
\begin{pgfscope}%
\definecolor{textcolor}{rgb}{0.000000,0.000000,0.000000}%
\pgfsetstrokecolor{textcolor}%
\pgfsetfillcolor{textcolor}%
\pgftext[x=0.398457in, y=0.638938in, left, base]{\color{textcolor}\rmfamily\fontsize{10.000000}{12.000000}\selectfont \(\displaystyle {0}\)}%
\end{pgfscope}%
\begin{pgfscope}%
\pgfsetbuttcap%
\pgfsetroundjoin%
\definecolor{currentfill}{rgb}{0.000000,0.000000,0.000000}%
\pgfsetfillcolor{currentfill}%
\pgfsetlinewidth{0.803000pt}%
\definecolor{currentstroke}{rgb}{0.000000,0.000000,0.000000}%
\pgfsetstrokecolor{currentstroke}%
\pgfsetdash{}{0pt}%
\pgfsys@defobject{currentmarker}{\pgfqpoint{-0.048611in}{0.000000in}}{\pgfqpoint{-0.000000in}{0.000000in}}{%
\pgfpathmoveto{\pgfqpoint{-0.000000in}{0.000000in}}%
\pgfpathlineto{\pgfqpoint{-0.048611in}{0.000000in}}%
\pgfusepath{stroke,fill}%
}%
\begin{pgfscope}%
\pgfsys@transformshift{0.565124in}{0.999067in}%
\pgfsys@useobject{currentmarker}{}%
\end{pgfscope}%
\end{pgfscope}%
\begin{pgfscope}%
\definecolor{textcolor}{rgb}{0.000000,0.000000,0.000000}%
\pgfsetstrokecolor{textcolor}%
\pgfsetfillcolor{textcolor}%
\pgftext[x=0.398457in, y=0.950842in, left, base]{\color{textcolor}\rmfamily\fontsize{10.000000}{12.000000}\selectfont \(\displaystyle {5}\)}%
\end{pgfscope}%
\begin{pgfscope}%
\pgfsetbuttcap%
\pgfsetroundjoin%
\definecolor{currentfill}{rgb}{0.000000,0.000000,0.000000}%
\pgfsetfillcolor{currentfill}%
\pgfsetlinewidth{0.803000pt}%
\definecolor{currentstroke}{rgb}{0.000000,0.000000,0.000000}%
\pgfsetstrokecolor{currentstroke}%
\pgfsetdash{}{0pt}%
\pgfsys@defobject{currentmarker}{\pgfqpoint{-0.048611in}{0.000000in}}{\pgfqpoint{-0.000000in}{0.000000in}}{%
\pgfpathmoveto{\pgfqpoint{-0.000000in}{0.000000in}}%
\pgfpathlineto{\pgfqpoint{-0.048611in}{0.000000in}}%
\pgfusepath{stroke,fill}%
}%
\begin{pgfscope}%
\pgfsys@transformshift{0.565124in}{1.310971in}%
\pgfsys@useobject{currentmarker}{}%
\end{pgfscope}%
\end{pgfscope}%
\begin{pgfscope}%
\definecolor{textcolor}{rgb}{0.000000,0.000000,0.000000}%
\pgfsetstrokecolor{textcolor}%
\pgfsetfillcolor{textcolor}%
\pgftext[x=0.329012in, y=1.262746in, left, base]{\color{textcolor}\rmfamily\fontsize{10.000000}{12.000000}\selectfont \(\displaystyle {10}\)}%
\end{pgfscope}%
\begin{pgfscope}%
\pgfsetbuttcap%
\pgfsetroundjoin%
\definecolor{currentfill}{rgb}{0.000000,0.000000,0.000000}%
\pgfsetfillcolor{currentfill}%
\pgfsetlinewidth{0.803000pt}%
\definecolor{currentstroke}{rgb}{0.000000,0.000000,0.000000}%
\pgfsetstrokecolor{currentstroke}%
\pgfsetdash{}{0pt}%
\pgfsys@defobject{currentmarker}{\pgfqpoint{-0.048611in}{0.000000in}}{\pgfqpoint{-0.000000in}{0.000000in}}{%
\pgfpathmoveto{\pgfqpoint{-0.000000in}{0.000000in}}%
\pgfpathlineto{\pgfqpoint{-0.048611in}{0.000000in}}%
\pgfusepath{stroke,fill}%
}%
\begin{pgfscope}%
\pgfsys@transformshift{0.565124in}{1.622875in}%
\pgfsys@useobject{currentmarker}{}%
\end{pgfscope}%
\end{pgfscope}%
\begin{pgfscope}%
\definecolor{textcolor}{rgb}{0.000000,0.000000,0.000000}%
\pgfsetstrokecolor{textcolor}%
\pgfsetfillcolor{textcolor}%
\pgftext[x=0.329012in, y=1.574650in, left, base]{\color{textcolor}\rmfamily\fontsize{10.000000}{12.000000}\selectfont \(\displaystyle {15}\)}%
\end{pgfscope}%
\begin{pgfscope}%
\pgfsetbuttcap%
\pgfsetroundjoin%
\definecolor{currentfill}{rgb}{0.000000,0.000000,0.000000}%
\pgfsetfillcolor{currentfill}%
\pgfsetlinewidth{0.803000pt}%
\definecolor{currentstroke}{rgb}{0.000000,0.000000,0.000000}%
\pgfsetstrokecolor{currentstroke}%
\pgfsetdash{}{0pt}%
\pgfsys@defobject{currentmarker}{\pgfqpoint{-0.048611in}{0.000000in}}{\pgfqpoint{-0.000000in}{0.000000in}}{%
\pgfpathmoveto{\pgfqpoint{-0.000000in}{0.000000in}}%
\pgfpathlineto{\pgfqpoint{-0.048611in}{0.000000in}}%
\pgfusepath{stroke,fill}%
}%
\begin{pgfscope}%
\pgfsys@transformshift{0.565124in}{1.934779in}%
\pgfsys@useobject{currentmarker}{}%
\end{pgfscope}%
\end{pgfscope}%
\begin{pgfscope}%
\definecolor{textcolor}{rgb}{0.000000,0.000000,0.000000}%
\pgfsetstrokecolor{textcolor}%
\pgfsetfillcolor{textcolor}%
\pgftext[x=0.329012in, y=1.886554in, left, base]{\color{textcolor}\rmfamily\fontsize{10.000000}{12.000000}\selectfont \(\displaystyle {20}\)}%
\end{pgfscope}%
\begin{pgfscope}%
\pgfsetbuttcap%
\pgfsetroundjoin%
\definecolor{currentfill}{rgb}{0.000000,0.000000,0.000000}%
\pgfsetfillcolor{currentfill}%
\pgfsetlinewidth{0.803000pt}%
\definecolor{currentstroke}{rgb}{0.000000,0.000000,0.000000}%
\pgfsetstrokecolor{currentstroke}%
\pgfsetdash{}{0pt}%
\pgfsys@defobject{currentmarker}{\pgfqpoint{-0.048611in}{0.000000in}}{\pgfqpoint{-0.000000in}{0.000000in}}{%
\pgfpathmoveto{\pgfqpoint{-0.000000in}{0.000000in}}%
\pgfpathlineto{\pgfqpoint{-0.048611in}{0.000000in}}%
\pgfusepath{stroke,fill}%
}%
\begin{pgfscope}%
\pgfsys@transformshift{0.565124in}{2.246684in}%
\pgfsys@useobject{currentmarker}{}%
\end{pgfscope}%
\end{pgfscope}%
\begin{pgfscope}%
\definecolor{textcolor}{rgb}{0.000000,0.000000,0.000000}%
\pgfsetstrokecolor{textcolor}%
\pgfsetfillcolor{textcolor}%
\pgftext[x=0.329012in, y=2.198458in, left, base]{\color{textcolor}\rmfamily\fontsize{10.000000}{12.000000}\selectfont \(\displaystyle {25}\)}%
\end{pgfscope}%
\begin{pgfscope}%
\pgfsetbuttcap%
\pgfsetroundjoin%
\definecolor{currentfill}{rgb}{0.000000,0.000000,0.000000}%
\pgfsetfillcolor{currentfill}%
\pgfsetlinewidth{0.803000pt}%
\definecolor{currentstroke}{rgb}{0.000000,0.000000,0.000000}%
\pgfsetstrokecolor{currentstroke}%
\pgfsetdash{}{0pt}%
\pgfsys@defobject{currentmarker}{\pgfqpoint{-0.048611in}{0.000000in}}{\pgfqpoint{-0.000000in}{0.000000in}}{%
\pgfpathmoveto{\pgfqpoint{-0.000000in}{0.000000in}}%
\pgfpathlineto{\pgfqpoint{-0.048611in}{0.000000in}}%
\pgfusepath{stroke,fill}%
}%
\begin{pgfscope}%
\pgfsys@transformshift{0.565124in}{2.558588in}%
\pgfsys@useobject{currentmarker}{}%
\end{pgfscope}%
\end{pgfscope}%
\begin{pgfscope}%
\definecolor{textcolor}{rgb}{0.000000,0.000000,0.000000}%
\pgfsetstrokecolor{textcolor}%
\pgfsetfillcolor{textcolor}%
\pgftext[x=0.329012in, y=2.510362in, left, base]{\color{textcolor}\rmfamily\fontsize{10.000000}{12.000000}\selectfont \(\displaystyle {30}\)}%
\end{pgfscope}%
\begin{pgfscope}%
\pgfsetbuttcap%
\pgfsetroundjoin%
\definecolor{currentfill}{rgb}{0.000000,0.000000,0.000000}%
\pgfsetfillcolor{currentfill}%
\pgfsetlinewidth{0.803000pt}%
\definecolor{currentstroke}{rgb}{0.000000,0.000000,0.000000}%
\pgfsetstrokecolor{currentstroke}%
\pgfsetdash{}{0pt}%
\pgfsys@defobject{currentmarker}{\pgfqpoint{-0.048611in}{0.000000in}}{\pgfqpoint{-0.000000in}{0.000000in}}{%
\pgfpathmoveto{\pgfqpoint{-0.000000in}{0.000000in}}%
\pgfpathlineto{\pgfqpoint{-0.048611in}{0.000000in}}%
\pgfusepath{stroke,fill}%
}%
\begin{pgfscope}%
\pgfsys@transformshift{0.565124in}{2.870492in}%
\pgfsys@useobject{currentmarker}{}%
\end{pgfscope}%
\end{pgfscope}%
\begin{pgfscope}%
\definecolor{textcolor}{rgb}{0.000000,0.000000,0.000000}%
\pgfsetstrokecolor{textcolor}%
\pgfsetfillcolor{textcolor}%
\pgftext[x=0.329012in, y=2.822266in, left, base]{\color{textcolor}\rmfamily\fontsize{10.000000}{12.000000}\selectfont \(\displaystyle {35}\)}%
\end{pgfscope}%
\begin{pgfscope}%
\pgfsetbuttcap%
\pgfsetroundjoin%
\definecolor{currentfill}{rgb}{0.000000,0.000000,0.000000}%
\pgfsetfillcolor{currentfill}%
\pgfsetlinewidth{0.803000pt}%
\definecolor{currentstroke}{rgb}{0.000000,0.000000,0.000000}%
\pgfsetstrokecolor{currentstroke}%
\pgfsetdash{}{0pt}%
\pgfsys@defobject{currentmarker}{\pgfqpoint{-0.048611in}{0.000000in}}{\pgfqpoint{-0.000000in}{0.000000in}}{%
\pgfpathmoveto{\pgfqpoint{-0.000000in}{0.000000in}}%
\pgfpathlineto{\pgfqpoint{-0.048611in}{0.000000in}}%
\pgfusepath{stroke,fill}%
}%
\begin{pgfscope}%
\pgfsys@transformshift{0.565124in}{3.182396in}%
\pgfsys@useobject{currentmarker}{}%
\end{pgfscope}%
\end{pgfscope}%
\begin{pgfscope}%
\definecolor{textcolor}{rgb}{0.000000,0.000000,0.000000}%
\pgfsetstrokecolor{textcolor}%
\pgfsetfillcolor{textcolor}%
\pgftext[x=0.329012in, y=3.134170in, left, base]{\color{textcolor}\rmfamily\fontsize{10.000000}{12.000000}\selectfont \(\displaystyle {40}\)}%
\end{pgfscope}%
\begin{pgfscope}%
\definecolor{textcolor}{rgb}{0.000000,0.000000,0.000000}%
\pgfsetstrokecolor{textcolor}%
\pgfsetfillcolor{textcolor}%
\pgftext[x=0.273457in,y=1.907562in,,bottom,rotate=90.000000]{\color{textcolor}\rmfamily\fontsize{10.000000}{12.000000}\selectfont Differenza percettiva}%
\end{pgfscope}%
\begin{pgfscope}%
\pgfpathrectangle{\pgfqpoint{0.565124in}{0.565123in}}{\pgfqpoint{5.184876in}{2.684877in}}%
\pgfusepath{clip}%
\pgfsetrectcap%
\pgfsetroundjoin%
\pgfsetlinewidth{2.007500pt}%
\definecolor{currentstroke}{rgb}{0.121569,0.466667,0.705882}%
\pgfsetstrokecolor{currentstroke}%
\pgfsetdash{}{0pt}%
\pgfpathmoveto{\pgfqpoint{0.800800in}{2.524934in}}%
\pgfpathlineto{\pgfqpoint{0.820773in}{2.090802in}}%
\pgfpathlineto{\pgfqpoint{0.840745in}{1.906938in}}%
\pgfpathlineto{\pgfqpoint{0.860718in}{1.734108in}}%
\pgfpathlineto{\pgfqpoint{0.880690in}{1.516130in}}%
\pgfpathlineto{\pgfqpoint{0.900663in}{1.440210in}}%
\pgfpathlineto{\pgfqpoint{0.920635in}{1.317242in}}%
\pgfpathlineto{\pgfqpoint{0.940608in}{1.273394in}}%
\pgfpathlineto{\pgfqpoint{0.980553in}{1.151378in}}%
\pgfpathlineto{\pgfqpoint{1.000526in}{1.120751in}}%
\pgfpathlineto{\pgfqpoint{1.020498in}{1.026497in}}%
\pgfpathlineto{\pgfqpoint{1.040471in}{0.981309in}}%
\pgfpathlineto{\pgfqpoint{1.060443in}{0.954837in}}%
\pgfpathlineto{\pgfqpoint{1.080416in}{0.924491in}}%
\pgfpathlineto{\pgfqpoint{1.100388in}{0.886520in}}%
\pgfpathlineto{\pgfqpoint{1.120361in}{0.863336in}}%
\pgfpathlineto{\pgfqpoint{1.140333in}{0.833624in}}%
\pgfpathlineto{\pgfqpoint{1.160306in}{0.807611in}}%
\pgfpathlineto{\pgfqpoint{1.180279in}{0.790487in}}%
\pgfpathlineto{\pgfqpoint{1.200251in}{0.754586in}}%
\pgfpathlineto{\pgfqpoint{1.220224in}{0.744680in}}%
\pgfpathlineto{\pgfqpoint{1.240196in}{0.732476in}}%
\pgfpathlineto{\pgfqpoint{1.300114in}{0.709525in}}%
\pgfpathlineto{\pgfqpoint{1.320086in}{0.706477in}}%
\pgfpathlineto{\pgfqpoint{1.340059in}{0.701799in}}%
\pgfpathlineto{\pgfqpoint{1.360032in}{0.698313in}}%
\pgfpathlineto{\pgfqpoint{1.419949in}{0.694886in}}%
\pgfpathlineto{\pgfqpoint{1.599702in}{0.692950in}}%
\pgfpathlineto{\pgfqpoint{2.178906in}{0.692904in}}%
\pgfpathlineto{\pgfqpoint{2.198879in}{0.687172in}}%
\pgfpathlineto{\pgfqpoint{5.494351in}{0.687163in}}%
\pgfpathlineto{\pgfqpoint{5.494351in}{0.687163in}}%
\pgfusepath{stroke}%
\end{pgfscope}%
\begin{pgfscope}%
\pgfpathrectangle{\pgfqpoint{0.565124in}{0.565123in}}{\pgfqpoint{5.184876in}{2.684877in}}%
\pgfusepath{clip}%
\pgfsetrectcap%
\pgfsetroundjoin%
\pgfsetlinewidth{2.007500pt}%
\definecolor{currentstroke}{rgb}{1.000000,0.498039,0.054902}%
\pgfsetstrokecolor{currentstroke}%
\pgfsetdash{}{0pt}%
\pgfpathmoveto{\pgfqpoint{0.820773in}{3.127960in}}%
\pgfpathlineto{\pgfqpoint{0.880690in}{3.127960in}}%
\pgfpathlineto{\pgfqpoint{0.900663in}{3.063141in}}%
\pgfpathlineto{\pgfqpoint{0.920635in}{2.574733in}}%
\pgfpathlineto{\pgfqpoint{0.940608in}{2.507245in}}%
\pgfpathlineto{\pgfqpoint{1.020498in}{2.507245in}}%
\pgfpathlineto{\pgfqpoint{1.040471in}{2.495573in}}%
\pgfpathlineto{\pgfqpoint{1.120361in}{2.495573in}}%
\pgfpathlineto{\pgfqpoint{1.140333in}{2.344408in}}%
\pgfpathlineto{\pgfqpoint{1.160306in}{2.344408in}}%
\pgfpathlineto{\pgfqpoint{1.180279in}{2.319286in}}%
\pgfpathlineto{\pgfqpoint{1.200251in}{2.298776in}}%
\pgfpathlineto{\pgfqpoint{1.280141in}{2.298776in}}%
\pgfpathlineto{\pgfqpoint{1.300114in}{2.292041in}}%
\pgfpathlineto{\pgfqpoint{1.320086in}{2.252226in}}%
\pgfpathlineto{\pgfqpoint{1.419949in}{2.252222in}}%
\pgfpathlineto{\pgfqpoint{1.439922in}{2.247107in}}%
\pgfpathlineto{\pgfqpoint{1.619675in}{2.247101in}}%
\pgfpathlineto{\pgfqpoint{1.639647in}{2.229765in}}%
\pgfpathlineto{\pgfqpoint{1.719538in}{2.229769in}}%
\pgfpathlineto{\pgfqpoint{1.739510in}{2.225207in}}%
\pgfpathlineto{\pgfqpoint{1.999153in}{2.225212in}}%
\pgfpathlineto{\pgfqpoint{2.019126in}{2.223867in}}%
\pgfpathlineto{\pgfqpoint{2.039099in}{2.218884in}}%
\pgfpathlineto{\pgfqpoint{2.099016in}{2.218879in}}%
\pgfpathlineto{\pgfqpoint{2.118989in}{2.196587in}}%
\pgfpathlineto{\pgfqpoint{2.318714in}{2.196594in}}%
\pgfpathlineto{\pgfqpoint{2.338687in}{2.193603in}}%
\pgfpathlineto{\pgfqpoint{2.358660in}{1.739626in}}%
\pgfpathlineto{\pgfqpoint{2.378632in}{1.736838in}}%
\pgfpathlineto{\pgfqpoint{2.398605in}{1.722055in}}%
\pgfpathlineto{\pgfqpoint{2.418577in}{1.721865in}}%
\pgfpathlineto{\pgfqpoint{2.438550in}{1.718159in}}%
\pgfpathlineto{\pgfqpoint{2.518440in}{1.718161in}}%
\pgfpathlineto{\pgfqpoint{2.538413in}{1.707136in}}%
\pgfpathlineto{\pgfqpoint{2.618303in}{1.707140in}}%
\pgfpathlineto{\pgfqpoint{2.638275in}{1.617504in}}%
\pgfpathlineto{\pgfqpoint{2.718166in}{1.617498in}}%
\pgfpathlineto{\pgfqpoint{2.758111in}{1.593837in}}%
\pgfpathlineto{\pgfqpoint{2.778083in}{1.560893in}}%
\pgfpathlineto{\pgfqpoint{2.798056in}{1.555603in}}%
\pgfpathlineto{\pgfqpoint{2.838001in}{1.548142in}}%
\pgfpathlineto{\pgfqpoint{2.857973in}{1.542353in}}%
\pgfpathlineto{\pgfqpoint{2.937864in}{1.542187in}}%
\pgfpathlineto{\pgfqpoint{2.957836in}{1.530936in}}%
\pgfpathlineto{\pgfqpoint{2.997781in}{1.510922in}}%
\pgfpathlineto{\pgfqpoint{3.017754in}{1.510355in}}%
\pgfpathlineto{\pgfqpoint{3.037727in}{1.491130in}}%
\pgfpathlineto{\pgfqpoint{3.057699in}{1.491129in}}%
\pgfpathlineto{\pgfqpoint{3.077672in}{1.475418in}}%
\pgfpathlineto{\pgfqpoint{3.097644in}{1.467319in}}%
\pgfpathlineto{\pgfqpoint{3.117617in}{1.463231in}}%
\pgfpathlineto{\pgfqpoint{3.137589in}{1.426648in}}%
\pgfpathlineto{\pgfqpoint{3.217480in}{1.426637in}}%
\pgfpathlineto{\pgfqpoint{3.237452in}{1.422627in}}%
\pgfpathlineto{\pgfqpoint{3.277397in}{1.422634in}}%
\pgfpathlineto{\pgfqpoint{3.297370in}{1.388010in}}%
\pgfpathlineto{\pgfqpoint{3.317342in}{1.379338in}}%
\pgfpathlineto{\pgfqpoint{3.377260in}{1.379320in}}%
\pgfpathlineto{\pgfqpoint{3.397233in}{1.130760in}}%
\pgfpathlineto{\pgfqpoint{3.417205in}{1.110742in}}%
\pgfpathlineto{\pgfqpoint{3.437178in}{1.110668in}}%
\pgfpathlineto{\pgfqpoint{3.457150in}{0.978853in}}%
\pgfpathlineto{\pgfqpoint{3.477123in}{0.963034in}}%
\pgfpathlineto{\pgfqpoint{3.616931in}{0.963028in}}%
\pgfpathlineto{\pgfqpoint{3.636903in}{0.952544in}}%
\pgfpathlineto{\pgfqpoint{3.656876in}{0.952543in}}%
\pgfpathlineto{\pgfqpoint{3.676848in}{0.870533in}}%
\pgfpathlineto{\pgfqpoint{3.696821in}{0.804642in}}%
\pgfpathlineto{\pgfqpoint{3.796684in}{0.804608in}}%
\pgfpathlineto{\pgfqpoint{3.816656in}{0.778153in}}%
\pgfpathlineto{\pgfqpoint{3.836629in}{0.778158in}}%
\pgfpathlineto{\pgfqpoint{3.856601in}{0.762484in}}%
\pgfpathlineto{\pgfqpoint{3.876574in}{0.723260in}}%
\pgfpathlineto{\pgfqpoint{4.355915in}{0.722413in}}%
\pgfpathlineto{\pgfqpoint{4.375888in}{0.687189in}}%
\pgfpathlineto{\pgfqpoint{5.514324in}{0.687163in}}%
\pgfpathlineto{\pgfqpoint{5.514324in}{0.687163in}}%
\pgfusepath{stroke}%
\end{pgfscope}%
\begin{pgfscope}%
\pgfsetrectcap%
\pgfsetmiterjoin%
\pgfsetlinewidth{0.803000pt}%
\definecolor{currentstroke}{rgb}{0.000000,0.000000,0.000000}%
\pgfsetstrokecolor{currentstroke}%
\pgfsetdash{}{0pt}%
\pgfpathmoveto{\pgfqpoint{0.565124in}{0.565123in}}%
\pgfpathlineto{\pgfqpoint{0.565124in}{3.250000in}}%
\pgfusepath{stroke}%
\end{pgfscope}%
\begin{pgfscope}%
\pgfsetrectcap%
\pgfsetmiterjoin%
\pgfsetlinewidth{0.803000pt}%
\definecolor{currentstroke}{rgb}{0.000000,0.000000,0.000000}%
\pgfsetstrokecolor{currentstroke}%
\pgfsetdash{}{0pt}%
\pgfpathmoveto{\pgfqpoint{5.750000in}{0.565123in}}%
\pgfpathlineto{\pgfqpoint{5.750000in}{3.250000in}}%
\pgfusepath{stroke}%
\end{pgfscope}%
\begin{pgfscope}%
\pgfsetrectcap%
\pgfsetmiterjoin%
\pgfsetlinewidth{0.803000pt}%
\definecolor{currentstroke}{rgb}{0.000000,0.000000,0.000000}%
\pgfsetstrokecolor{currentstroke}%
\pgfsetdash{}{0pt}%
\pgfpathmoveto{\pgfqpoint{0.565124in}{0.565123in}}%
\pgfpathlineto{\pgfqpoint{5.750000in}{0.565123in}}%
\pgfusepath{stroke}%
\end{pgfscope}%
\begin{pgfscope}%
\pgfsetrectcap%
\pgfsetmiterjoin%
\pgfsetlinewidth{0.803000pt}%
\definecolor{currentstroke}{rgb}{0.000000,0.000000,0.000000}%
\pgfsetstrokecolor{currentstroke}%
\pgfsetdash{}{0pt}%
\pgfpathmoveto{\pgfqpoint{0.565124in}{3.250000in}}%
\pgfpathlineto{\pgfqpoint{5.750000in}{3.250000in}}%
\pgfusepath{stroke}%
\end{pgfscope}%
\begin{pgfscope}%
\pgfsetbuttcap%
\pgfsetmiterjoin%
\definecolor{currentfill}{rgb}{1.000000,1.000000,1.000000}%
\pgfsetfillcolor{currentfill}%
\pgfsetfillopacity{0.800000}%
\pgfsetlinewidth{1.003750pt}%
\definecolor{currentstroke}{rgb}{0.800000,0.800000,0.800000}%
\pgfsetstrokecolor{currentstroke}%
\pgfsetstrokeopacity{0.800000}%
\pgfsetdash{}{0pt}%
\pgfpathmoveto{\pgfqpoint{4.273532in}{2.751543in}}%
\pgfpathlineto{\pgfqpoint{5.652778in}{2.751543in}}%
\pgfpathquadraticcurveto{\pgfqpoint{5.680556in}{2.751543in}}{\pgfqpoint{5.680556in}{2.779321in}}%
\pgfpathlineto{\pgfqpoint{5.680556in}{3.152778in}}%
\pgfpathquadraticcurveto{\pgfqpoint{5.680556in}{3.180556in}}{\pgfqpoint{5.652778in}{3.180556in}}%
\pgfpathlineto{\pgfqpoint{4.273532in}{3.180556in}}%
\pgfpathquadraticcurveto{\pgfqpoint{4.245755in}{3.180556in}}{\pgfqpoint{4.245755in}{3.152778in}}%
\pgfpathlineto{\pgfqpoint{4.245755in}{2.779321in}}%
\pgfpathquadraticcurveto{\pgfqpoint{4.245755in}{2.751543in}}{\pgfqpoint{4.273532in}{2.751543in}}%
\pgfpathclose%
\pgfusepath{stroke,fill}%
\end{pgfscope}%
\begin{pgfscope}%
\pgfsetrectcap%
\pgfsetroundjoin%
\pgfsetlinewidth{2.007500pt}%
\definecolor{currentstroke}{rgb}{0.121569,0.466667,0.705882}%
\pgfsetstrokecolor{currentstroke}%
\pgfsetdash{}{0pt}%
\pgfpathmoveto{\pgfqpoint{4.301310in}{3.076389in}}%
\pgfpathlineto{\pgfqpoint{4.579088in}{3.076389in}}%
\pgfusepath{stroke}%
\end{pgfscope}%
\begin{pgfscope}%
\definecolor{textcolor}{rgb}{0.000000,0.000000,0.000000}%
\pgfsetstrokecolor{textcolor}%
\pgfsetfillcolor{textcolor}%
\pgftext[x=4.690199in,y=3.027778in,left,base]{\color{textcolor}\rmfamily\fontsize{10.000000}{12.000000}\selectfont Scena fittizia}%
\end{pgfscope}%
\begin{pgfscope}%
\pgfsetrectcap%
\pgfsetroundjoin%
\pgfsetlinewidth{2.007500pt}%
\definecolor{currentstroke}{rgb}{1.000000,0.498039,0.054902}%
\pgfsetstrokecolor{currentstroke}%
\pgfsetdash{}{0pt}%
\pgfpathmoveto{\pgfqpoint{4.301310in}{2.882716in}}%
\pgfpathlineto{\pgfqpoint{4.579088in}{2.882716in}}%
\pgfusepath{stroke}%
\end{pgfscope}%
\begin{pgfscope}%
\definecolor{textcolor}{rgb}{0.000000,0.000000,0.000000}%
\pgfsetstrokecolor{textcolor}%
\pgfsetfillcolor{textcolor}%
\pgftext[x=4.690199in,y=2.834105in,left,base]{\color{textcolor}\rmfamily\fontsize{10.000000}{12.000000}\selectfont Scena realistica}%
\end{pgfscope}%
\end{pgfpicture}%
\makeatother%
\endgroup%

                \caption{Valutazione Ray Tracing - Ombre portate}
                \label{fig:eval-rt}
            \end{figure}         


    \begin{sidewaysfigure}
        \centering
        \includegraphics[width=\textheight,height=.70793\textwidth]{images/valutazioni/eval-fittizia-cluster.png}
        \caption{Sequenze di caricamento della scena fittizia utilizzando le strategie proposte}
        \label{fig:fittizia-cluster}
    \end{sidewaysfigure}

    \begin{sidewaysfigure}
        \centering
        \includegraphics[width=\textheight,height=.71123834149044233077846523224674\textwidth]{images/valutazioni/eval-realistica-cluster.png}
        \caption{Sequenze di caricamento della scena realistica utilizzando le strategie proposte}
        \label{fig:realistica-cluster}
    \end{sidewaysfigure}
    % le valutazioni messe a confronto: ray tracing wins