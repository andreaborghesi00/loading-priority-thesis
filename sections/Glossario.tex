\chapter*{Glossario}

\begin{itemize}[label={}]
  \item \textbf{Differenza (o distanza) percettiva}: la distanza euclidea tra due colori espressi da una tripla di valori, o la media tra la distanza euclidea tra due insiemi ordinati di triple di valori, in base al contesto.
  \item \textbf{Asset}: In termini generali, una rappresentazione digitale (tipicamente contenuta in uno o più file) che modella un elemento costitutivo dell'ambiente virtuale, quale una mesh, un materiale, una tessitura, un file audio, etc, pronto per essere utilizzato durante l'esecuzione dell'applicativo.
  \item \textbf{Istanza di un asset}: anche detta solo istanza, è l'istanziazione di un asset in una scena. Una modifica ad un'istanza non comporta alla modifica dell'asset né alla modifica di altre istanze dello stesso.
  \item \textbf{Frame}: immagine che compone un video in movimento. Come immagine può essere espressa come una matrice di pixel (o frammenti).
  \item \textbf{Vista (o View)}: Risultato della rendering pipeline
  \item \textbf{View Frustum (o frustum della camera)}: La regione di spazio che verrà renderizzata a schermo
  \item \textbf{Scena}: Spazio contenente istanze di asset  
\end{itemize}
