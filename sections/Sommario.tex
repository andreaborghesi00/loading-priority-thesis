\chapter*{Sommario}
    \thispagestyle{empty}
    % Nel contesto del rendering di scene 3D di grandi dimensioni i limiti imposti dalla memoria sono sempre un ostacolo da affrontare con cautela. Il passaggio dei dati dalla memoria di massa o da remoto fino alla VRAM è un problema recentemente affrontato a livello hardware. Questa tesi si pone l'obiettivo di proporre strategie per la gestione dell'ordine di caricamento delle istanze interne ad una scena 3D, approcciando il problema a livello software. Si pone particolare attenzione nel rendere l'operazione di caricamento il meno percettibile possibile dall'utente. Questa necessità nasce principalmente nel mondo videoludico, dove è uso comune strutturare un mondo virtuale in sotto-scene di cui, con gli standard odierni, si richiede che il passaggio tra queste sia impercettibile. Per fare ciò, le strategie tratteranno molto da vicino la gestione delle istanze che compongono la vista finale, con l'obiettivo di comporla in tempo utile comprensiva di asset critici al funzionamento dell'applicativo. La misura dell'efficacia di queste politiche viene stimata con uno strumento basato su immagini che approssima la distanza percettiva tra le due. Delle strategie implementate, le più efficaci secondo lo strumento valutativo progettato, sono quelle analizzano principalmente la sezione occupata nella vista da ogni istanza di asset e la provenienza della fonte luminosa per predire la presenza di ombre.
    
    Nel contesto del rendering di scene 3D complesse, le limitazioni della VRAM impongono che queste vengano caricate solo quando necessarie, anziché risiedere in memoria per l'intera durata dell'applicazione. Quando una nuova scena viene caricata, numerosi asset competono per la larghezza di banda nel trasferimento tra la memoria di massa e la VRAM. I ritardi percepiti dall'utente finale vengono ridotti visualizzando e interagendo con una scena parziale durante il caricamento. Questo risultato dipende fortemente dall'ordine in cui vengono caricati gli asset. In questo lavoro, (1) si dimostra l'impatto della scelta dell'ordine degli asset; (2) viene proposto un piccolo numero di strategie euristiche per determinare automaticamente un ordine degli asset, ottimizzato per una determinata scena e punto di inizio; (3) si presenta uno strumento agnostico basato su immagini per valutare l'efficacia di un determinato ordine di caricamento degli asset, che può essere utilizzato per classificare a posteriori ulteriori alternative; (4) si collauda lo strumento di valutazione prodotto sulle euristiche di ordinamento proposte su un piccolo set di scene di prova tratte da istanze di video giochi reali e non. Lo studio suggerisce che i criteri per guidare l'ordine del caricamento degli asset dovrebbero includere la previsione delle ombre portate e la dimensione della sezione dello schermo occupata da ciascuna istanza.